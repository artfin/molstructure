\documentclass[hyperref={pdfpagelabel=false},usepdftitle=false,xcolor=dvipsnames]{beamer}

\usepackage[T1]{fontenc}
\usepackage[utf8]{inputenc}
\usepackage[russian]{babel}

% page margin
\usepackage[top=2cm, bottom=2cm, left=2cm, right=2cm]{geometry}

% AMS packages
\usepackage{amsmath}
\usepackage{amssymb}
\usepackage{amsfonts}
\usepackage{amsthm}

% blackboard lettering
\usepackage{dsfont}
\usepackage{bbm}

\usepackage{fancyhdr}
\pagestyle{fancy}
% modifying page layout using fancyhdr
\fancyhf{}
\renewcommand{\sectionmark}[1]{\markright{\thesection\ #1}} % adding number to section name
\renewcommand{\subsectionmark}[1]{\markright{\thesubsection\ #1}} % adding number to subsection name

\rhead{\fancyplain{}{\rightmark }} % placing the section/subsection name in the right corner of the header
\cfoot{\fancyplain{}{\thepage}}  % placing a page number in the center of the footer   

% table packages
\usepackage{tabularx, ragged2e, booktabs, caption}
% includegraphics package
\usepackage{graphicx}

\usepackage{subfigure}

% commands
\newcommand{\lb}{\left(}
\newcommand{\rb}{\right)}
\newcommand{\mH}{\mathcal{H}}
\newcommand{\mL}{\mathcal{L}}
\newcommand{\mf}{\mathbf}

\newcommand{\bbB}{\mathbb{B}}
\newcommand{\bbG}{\mathbb{G}}
\newcommand{\bbS}{\mathbb{S}}
\newcommand{\bbI}{\mathbb{I}}
\newcommand{\bbA}{\mathbb{A}}
\newcommand{\bba}{\mathbbm{a}}

% positioning commands
\newcommand{\vpravo}{\hspace{0.63cm}}
\newcommand{\vverh}{\vspace*{-0.1cm}}

% table cell centering
\newcolumntype{C}[1]{>{\Centering}m{#1}}
\renewcommand\tabularxcolumn[1]{C{#1}}

\makeatletter
\providecommand\barcirc{\mathpalette\@barred\circ}%
\def\@barred#1#2{\ooalign{\hfil$#1-$\hfil\cr\hfil$#1#2$\hfil\cr}}%
\newcommand\stst{^{\protect\barcirc}}%
\makeatother

% tikz packages
\usepackage{tikz}
\usetikzlibrary{shapes.geometric, arrows, positioning, decorations.markings}
\usetikzlibrary{fit}
\usepackage{microtype}
\usepackage{framed}
\usetikzlibrary{decorations.pathmorphing,calc,backgrounds}

\usepackage[nottoc]{tocbibind}



\usepackage{graphicx}

\newcommand{\lb}{\left(}
\newcommand{\rb}{\right)}

\begin{document}

\begin{frame}{\normalsize Резонаторный спектрометр (Cavity Ring-down Experiment)}
\begin{block}{}
\begin{figure}
\includegraphics[width = 0.5\linewidth]{pictures/fabry-perot.png}
\caption{\tiny Схема открытого резонатора Фабри-Перо с диэлектрической пленкой связи}
\end{figure}
\end{block}
\begin{block}{}
	\begin{center}
	\scalebox{0.7}{$ \displaystyle Q = \frac{\omega_0 W}{P_d} = \frac{\omega_0}{\Delta \omega} $} \\
	\scalebox{0.7}{$ \displaystyle P = P_{res.} + P_{gas}, \quad P_{gas} = \frac{1}{2} \left( 1 - \exp \left( - 2 \alpha L \right) \right) \approx \alpha L$} \\
	\scalebox{0.7}{\textit{leakage ringdown}: $\displaystyle I = I_0 \exp \lb - \frac{t}{\tau_0} \rb, \quad \tau_0 = \frac{L}{c(1 - R)}$} \\
	\scalebox{0.7}{\textit{absorber presence}: $\displaystyle I = I_0 \exp \lb - \frac{t}{\tau} \rb, \quad A = \frac{L}{c} \lb \frac{1}{\tau} - \frac{1}{\tau_0} \rb$}
	\end{center}
\end{block}
\end{frame}

\begin{frame}{\normalsize CRDS Experimental set-up}
	\begin{figure}
		\includegraphics[width = 0.8\linewidth]{pictures/ringdown.png}
	\end{figure}
\end{frame}

\begin{frame}{Столкновительно-индуцированное поглощение}
	\begin{itemize}
		\item Молекулярный и "супермолекулярный" спектр
		\item Столкновение как источник линии поглощения
		\item Аналог вириального разложения для поглощения: \\
			\centering \scalebox{0.8}{$ \alpha = A n + B n^2 + C n^3 + \dots$} \\
\begin{center} A -- dipole allowed contribution, \\ 
	       B -- induced binary contribution, \\
	       C -- induced ternary contribution
\end{center}
			
	\end{itemize}
\end{frame}

\begin{frame}{Бинарный трансляционный спектр}
	\begin{block}{}
		\begin{figure}
			\includegraphics[width = \linewidth]{pictures/trans.png}
		\end{figure}
	\end{block}
\end{frame}

\end{document}
