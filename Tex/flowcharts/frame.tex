
% define styles for the normal border and the torn border
\tikzset{
  normal border/.style={orange!30!black!10, decorate, decoration={random steps, segment length=2.5cm, amplitude=.7mm}},
  torn border/.style={orange!30!black!5, decorate, decoration={random steps, segment length=.5cm, amplitude=1.7mm}},
}

% Macro to draw the shape behind the text, when it fits completly in the
% page
\newcommand*\parchmentframe[1]{%
  \tikz{%
    \node[inner sep=2em] (A) {#1};  % Draw the text of the node
    \begin{scope}[on background layer]  % Draw the shape behind
      \fill[normal border] (A.south east) -- (A.south west) -- (A.north west) -- (A.north east) -- cycle;
    \end{scope}
  }%
}

% Macro to draw the shape, when the text will continue in next page
\newcommand*\parchmentframetop[1]{%
  \tikz{%
    \node[inner sep=2em] (A) {#1};    % Draw the text of the node
    \begin{scope}[on background layer]
      \fill[normal border]              % Draw the ``complete shape'' behind
      (A.south east) -- (A.south west) -- (A.north west) -- (A.north east) -- cycle;
      \fill[torn border]                % Add the torn lower border
      ($(A.south east)-(0,.2)$) -- ($(A.south west)-(0,.2)$) -- ($(A.south west)+(0,.2)$) -- ($(A.south east)+(0,.2)$) -- cycle;
    \end{scope}
  }%
}

% Macro to draw the shape, when the text continues from previous page
\newcommand*\parchmentframebottom[1]{%
  \tikz{%
    \node[inner sep=2em] (A) {#1};   % Draw the text of the node
    \begin{scope}[on background layer]
      \fill[normal border]             % Draw the ``complete shape'' behind
      (A.south east) -- (A.south west) -- (A.north west) -- (A.north east) -- cycle;
      \fill[torn border]               % Add the torn upper border
      ($(A.north east)-(0,.2)$) -- ($(A.north west)-(0,.2)$) -- ($(A.north west)+(0,.2)$) -- ($(A.north east)+(0,.2)$) -- cycle;
    \end{scope}
  }%
}

% Macro to draw the shape, when both the text continues from previous page
% and it will continue in next page
\newcommand*\parchmentframemiddle[1]{%
  \tikz{%
    \node[inner sep=2em] (A) {#1};   % Draw the text of the node
    \begin{scope}[on background layer]
      \fill[normal border]             % Draw the ``complete shape'' behind
      (A.south east) -- (A.south west) -- (A.north west) -- (A.north east) -- cycle;
      \fill[torn border]               % Add the torn lower border
      ($(A.south east)-(0,.2)$) -- ($(A.south west)-(0,.2)$) -- ($(A.south west)+(0,.2)$) -- ($(A.south east)+(0,.2)$) -- cycle;
      \fill[torn border]               % Add the torn upper border
      ($(A.north east)-(0,.2)$) -- ($(A.north west)-(0,.2)$) -- ($(A.north west)+(0,.2)$) -- ($(A.north east)+(0,.2)$) -- cycle;
    \end{scope}
  }%
}

% Define the environment which puts the frame
% In this case, the environment also accepts an argument with an optional
% title (which defaults to ``Example'', which is typeset in a box overlaid
% on the top border
\newenvironment{parchment}[1][Example]{%
  \def\FrameCommand{\parchmentframe}%
  \def\FirstFrameCommand{\parchmentframetop}%
  \def\LastFrameCommand{\parchmentframebottom}%
  \def\MidFrameCommand{\parchmentframemiddle}%
  \vskip\baselineskip
  \MakeFramed {\advance\hsize-\width\FrameRestore}%
  \noindent\tikz{\node[inner sep=1ex, draw=black!20,fill=white, anchor=west, overlay] at (0em, 2em) {\sffamily#1};}\par}%
{\endMakeFramed}