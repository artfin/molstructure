\tikzstyle{arrow} = [thick, ->, >=stealth]
\tikzstyle{vecArrow} = [thick, decoration={markings,mark=at position
   1 with {\arrow[semithick]{open triangle 60}}},
   double distance=1.4pt, shorten >= 5.5pt,
   preaction = {decorate},
   postaction = {draw,line width=1.4pt, white,shorten >= 4.5pt}]

\tikzstyle{lagrange} = [rectangle, rounded corners, minimum width = 3cm, minimum height = 1cm, text centered, text width = 3cm, draw = black, fill=red!30]

\tikzstyle{hamilton} = [rectangle, rounded corners, minimum width = 3cm, minimum height = 1cm, text centered, text width = 3cm, draw = black, fill=yellow!30]

\tikzstyle{equations} = [rectangle, rounded corners, minimum width = 3cm, minimum height = 1cm, text centered, text width = 3cm, draw = black, fill=green!30]

\node (lag1) [lagrange] {$\mathcal{L} = \mathcal{L}(\vec{q}, \dot{\vec{q}})$};

\node (ham1) [hamilton, right = 2 cm of lag1] {$\mathcal{H} = \mathcal{H}(\vec{q}, \vec{p})$};

\node (eq1) [equations, below = 0.5 cm of lag1] {$\vec{p} = \frac{\partial \mathcal{L}}{\partial \dot{\vec{q}}}$};

\node (eq2) [equations, below = 0.5 cm of ham1] {$\dot{\vec{q}} = \frac{\partial \mathcal{L}}{\partial \vec{p}}$};

\node (lag2) [lagrange, below = 2 cm of lag1] {$\mathcal{L} = \mathcal{L}(\vec{\Omega})$};

\node (ham2) [hamilton, below = 2 cm of ham1] {$\mathcal{H} = \mathcal{H} (\vec{J})$};

\node (eq3) [equations, below = 0.5 cm of lag2] {$\vec{J} = \frac{\partial \mathcal{L}}{\partial \vec{\Omega}}$};

\node (eq4) [equations, below = 0.5 cm of ham2] {$\vec{\Omega} = \frac{\partial \mathcal{H}}{\partial \vec{J}}$};

\draw [vecArrow] (lag1) -- (ham1);

\draw [vecArrow] (lag2) -- (ham2);
