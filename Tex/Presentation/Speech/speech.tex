\documentclass[12pt]{article}

\usepackage[T3]{fontenc}
\usepackage[utf8]{inputenc}
\usepackage[russian]{babel}

\usepackage{amssymb, amsmath}

\usepackage{fullpage} % Package to use full page
\usepackage{parskip} % Package to tweak paragraph skipping

\usepackage{geometry}
 \geometry{
 a4paper,
 total={170mm,257mm},
 left=5mm,
 top=5mm,
 right=5mm
}

\begin{document}
В области молекулярной спектроскопии методы классической механики распространены достаточно ограниченно. К задачам, в которых классическая механика успешно конкурирует с методами молекулярной динамики и квантовыми вычислениями, относятся моделирование столкновительных спектров слабосвязанных систем или вращение молекулярных систем в условиях сильного колебательно-вращательного взаимодействия. Ко второй области и относится моя курсовая работа. \par
Первой задачей моей курсовой работы было разобраться с тем, как записать точный классический колебательно-вращательный гамильтониан. Кратко опишу схему получения в виде блок-схемы. Первый шаг -- отделение центра масс (переход в систему, связанную с центром масс), это позволяет уменьшить количество переменных на одну. Затем осуществим в молекулярную систему отсчета (подвижную) при помощи матрицы ортогонального преобразования S. Переходя к обобщенным координатам приходим к конечному виду лагранжиана в подвижной системе отсчета. Следующий шаг: использование теоремы Донкина для перехода к функции Гамильтона. \par
Трехатомные гидриды проявляют сильное колебательно-вращательное взаимодействие, т.к. в условиях высокого вращательного возбуждения легкие концевые атомы становятся подвижными, подвергаясь воздействию центробежных сил. \par
В качестве первой модельной системы рассмотрим одномерную модель трехатомного гидрида (т.е. модель, в которой фиксированы связи между центральным и концевыми атомами. Молекула совершает колебание деформационного типа). Единственной внутренней степенью свободы является деформационный угол. Ввиду того, что центральный атом значительно тяжелее концевых, поместим центр масс точно на центральный атом. Проведя вычисления по приведенной схеме, получаем классический колебательно-вращательный гамильтониан для нашей модельной системы. \par
Для описания вращательной динамики молекулярной системы обратимся к концепции вращательной энергии. ПВЭ представляет собой двумерную поверхность. Величина вращательной энергии откладывается в направлении вектора углового момента (относительно молекулярно фиксированной системы координат \textbf{при фиксированном угловом моменте}). Т.к. вращательная задача обладает центром инверсии (изменим направление вращения системы $\implies$ обращение вектора углового момента и сохранение вращательной энергии), то и ПВЭ является центральносимметричной поверхностью. Для получения поверхности вращательной энергии необходимо построить эффективный вращательный гамильтониан. Для этого находят равновесные значения обобщенных координат и импульсов из приведенной системы уравнений (при решении этой системы компоненты вектора углового момента принимается в качестве параметров). Подставляя полученные решения в колебательно-вращательный гамильтониан, приходим к эффективному вращательному гамильтониану. Этот подход может быть охарактеризован как модель "мягкого тела". При фиксированном направлении вектора углового момента внутренние координаты находятся в равновесном состоянии, определенные наличием центробежных сил (при этом всякие внутримолекулярные колебания отсутствуют и молекула вращается как твердое тело при фиксированном значении углового момента). Решая приведенную систему уравнений приходим к эффективному вращательному гамильтонину для нашей системы. \par
ПВЭ представляет собой функцию направления вектора углового момента при фиксированном модуле вектора. Рожки показывают примерное положение максимумов на поверхности. Наблюдаем перестройку поверхности вращательной энергии в районе $J=30$. Анализ стационарных точек показывает, что перестройка ПВЭ наступает при достижении критического значения модуля углового момента, который выражается через параметры потенциала Пешля-Теллера. При перестройке поверхности точка максимума теряет свою устойчивость и становится седловой точкой, в то время как возникает две соседних точки максимума.
Раз у нас есть 4 эквивалентных максимума, вокруг которых может прецессировать вектор углового момента, то во вращательном спектре следует ожидать возникновения четырехкратно вырожденных кластеров. \par
Поверхность вращательной энергии является мощным инструментом, позволяющим качественно анализировать вращательную динмику, однако с ростом колебательного возбуждения влияние колебательной подсистемы на вращение молекулы не может учтено в рамках эффективного гамильтониана. Рассмотрим так называемую полную систему динамических уравнений, состоящую из уравнений Гамильтона и так называемых обобщенных уравнений Эйлера, количество которых может сокращено на 1, учитывая то, что модуль вектора углового момента является интегралом движения. Эта система содержит всю информацию о колебательно-вращательном движении (не сделано никаких приближений), и она содержит мимнимальное возможное количество уравнений ($2s+2$). \par
Для нашей системы динамические уравнения выглядят так. Фазовым пространством вращательной подзадачи является двумерная сфера, радиус которой равен модулю вектора углового момента. Фазовые траектории представляют собой траектории конца вектора углового момента. На рисунке представлены фазовые траектории в основном колебательном состоянии ($n = 0$). \par 
Перейдем к полномерной модели трехатомного гидрида. В рамках этой модели мы отпускаем обе связи. Эта модель позволит уточнить результаты, полученные для одномерной модели.
Полученный по описанной схеме гамильтониан имеет несколько более сложный вид. Полагаем, что потенциальная энергия разбивается на три слагаемых, каждое из которых описывает свой тип колебаний. В качестве потенциала, описывающего колебание валетного типа использовался гармонический потенциал и потенциал Морзе. \par
Мы видим значительное усложнение профиля фазовых траекторий. Также мы видим, что полная локализация траектории вокруг новой устойчивой оси происходит при значительно более высоком значении углового момента, однако вид траекторий подсказывает, что устойчивость новая ось приобрела раньше. Использование потенциала Морзе еще выше подняло значение момента (и еще усложнила профиль фазовых траекторий), при котором происходит локализация углового момента.
\end{document}