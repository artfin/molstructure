Покажем истинность следующего результата:
\begin{gather}
\vec{J} = \bbA \dot{\vec{q}} + \bbI \, \vec{\Omega} \notag
\end{gather}

Рассмотрим угловой момент в лабораторной системе координат.

\begin{gather}
\vec{j} = \sum_{i=1}^{n} m_i \left[ \vec{r}_i \times \dot{\vec{r}}_i \right] \notag \\
\dot{\vec{r}}_i = \bbS^{-1} \lb \left[ \vec{\Omega} \times \vec{R}_i \right] + \dot{\vec{R}}_i \rb \notag \\
\vec{j} = \sum_{i=1}^{n} m_i \left[ \vec{r}_i \times \bbS^{-1} \lb \left[ \vec{\Omega} \times \vec{R}_i \right] + \dot{\vec{R}}_i \rb \right] =
\bbS^{-1} \sum_i m_i \left[ \vec{R}_i \times \dot{\vec{R}}_i \right] + \bbS^{-1} \sum_i m_i \left[ \vec{R}_i \times \left[ \vec{\Omega} \times \vec{R}_i \right] \right] \notag
\end{gather}

Внимательно посмотрим на слагаемое, содержащее двойное векторное произведение:
\begin{gather}
\sum_i m_i \left[ \vec{R}_i \times \left[ \vec{\Omega} \times \vec{R}_i \right] \right] = \sum_i \lb R_i^2 \vec{\Omega} - \lb \vec{R}_i , \vec{\Omega} \rb \vec{R}_i \rb = \bbI \vec{\Omega} \notag
\end{gather}

Используем результат, умножаем обе части на $\bbS$, учтем, что $\vec{J} = \bbS \vec{j}$: 
\begin{gather}
\vec{J} = \bbA \dot{\vec{q}} + \bbI \, \vec{\Omega} \notag 
\end{gather}