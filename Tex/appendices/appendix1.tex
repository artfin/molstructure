\vverh
\begin{gather}
\left\{
\begin{aligned}
\vec{\Omega} = \frac{\partial \mathcal{H}}{\partial \vec{J}} \\ 
\vec{J} = \frac{\partial \mathcal{L}}{\partial \vec{\Omega}}
\end{aligned}
\right. \notag
\end{gather}

Доказательство первого соотношения представляет собой чисто техническую процедуру, так что опустим его. Продемонстрируем один из возможных путей доказательства второго соотношения:
\vverh
\begin{gather}
\vec{J} = \frac{\partial \mathcal{L}}{\partial \vec{\Omega}} = \bbA \dot{\vec{q}} + \bbI \, \vec{\Omega} \notag
\end{gather}

Для этого рассмотрим вектора углового момента в лабораторной системе координат и, используя ортогональную матрицу $\bbS$, выразим его через угловую скорость $\vec{\Omega}$ и радиус-векторы частиц системы $\left\{ \vec{R}_i \right\}$, представленные в подвижной системе координат:
\vverh
\begin{gather}
\vec{j} = \sum_{i=1}^{n} m_i \left[ \vec{r}_i \times \dot{\vec{r}}_i \right] \notag \\
\dot{\vec{r}}_i = \bbS^{-1} \lb \left[ \vec{\Omega} \times \vec{R}_i \right] + \dot{\vec{R}}_i \rb \notag \\
\vec{j} = \sum_{i=1}^{n} m_i \left[ \vec{r}_i \times \bbS^{-1} \lb \left[ \vec{\Omega} \times \vec{R}_i \right] + \dot{\vec{R}}_i \rb \right] =
\bbS^{-1} \sum_i m_i \left[ \vec{R}_i \times \dot{\vec{R}}_i \right] + \bbS^{-1} \sum_i m_i \left[ \vec{R}_i \times \left[ \vec{\Omega} \times \vec{R}_i \right] \right] = \notag \\
= \bbS^{-1} \sum_i m_i \left[ \vec{R}_i \times \dot{\vec{R}}_i \right] + \bbS^{-1} \sum_i \lb R_i^2 \vec{\Omega} - \lb \vec{R}_i , \vec{\Omega} \rb \vec{R}_i \rb = \bbS^{-1} \bbA \dot{\vec{q}} + \bbI \, \vec{\Omega} \notag
\end{gather}

Умножая обе части на $\bbS$, приходим к искомому соотношению: 
\begin{gather}
\vec{J} = \bbS \vec{j} = \bbA \dot{\vec{q}} + \bbI \, \vec{\Omega} \notag 
\end{gather}