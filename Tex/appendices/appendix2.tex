Рассмотрим блочную матрицу $\bbU = \begin{bmatrix} \bbU_{11} & \bbU_{12} \\ \bbU_{21} & \bbU_{22} \end{bmatrix}$, т.ч. 
$\bbU_{11} = \lb \bbU_{11}^{jk} \rb_{\substack{j = 1 \dots m \\ k = 1 \dots m}}$,
$\bbU_{22} = \lb \bbU_{22}^{jk} \rb_{\substack{j = 1 \dots n \\ k = 1 \dots n}}$ -- обратимые матрицы; 
$\bbU_{12} = \lb \bbU_{12}^{jk} \rb_{\substack{j = 1 \dots m \\ k = 1 \dots n}}$, 
$\bbU_{21} = \lb \bbU_{21}^{jk} \rb_{\substack{j = 1 \dots n \\ k = 1 \dots m}}$.

\vlevo Положим $\bbV = \begin{bmatrix} \bbV_{11} & \bbV_{12} \\ \bbV_{21} & \bbV_{22} \end{bmatrix}$ -- это обратная (справа) матрица для матрицы $\bbU$, то есть, матрица, удовлетворяющая следующему соотношению:
\vverh
\begin{gather}
\bbU \bbV = \begin{bmatrix}
\bbE_{m \times m} & \bbO_{m \times n} \\
\bbO_{n \times m} & \bbE_{n \times n}
\end{bmatrix}, \label{eq1}
\end{gather}
\vspace*{-0.5cm}
\begin{leftalign*}
\text{где} \quad 
\begin{aligned}
&\bbE_{m \times m} - \text{единичная матрица, $\dim \bbE = m \times m$} \\
&\bbO_{m \times n} - \text{матрица, заполненная нулями, $\dim \bbO = m \times n$}.
\end{aligned} \notag
\end{leftalign*}

Соотношение  \eqref{eq1} эквивалентно следующей системе:
\vspace*{-0.25cm}
\begin{align*}
\bbU_{11} \bbV_{11} &+ \bbU_{12} \bbV_{21} = \bbE_{m \times m} \numberthis \label{eq2} \\
\bbU_{11} \bbV_{12} &+ \bbU_{12} \bbV_{22} = \bbO_{m \times n} \numberthis \label{eq3} \\
\bbU_{21} \bbV_{11} &+ \bbU_{22} \bbV_{21} = \bbO_{n \times m} \numberthis \label{eq4} \\
\bbU_{21} \bbV_{12} &+ \bbU_{22} \bbV_{22} = \bbE_{n \times n} \numberthis \label{eq5}
\end{align*}

Выразим из уравнений (\ref{eq3}) и (\ref{eq4}) $\bbV_{12}$ и $\bbV_{21}$ соответственно:
\begin{gather}
\bbV_{12} = - \bbU_{11}^{-1} \bbU_{12} \bbV_{22} \notag \\
\bbV_{21} = - \bbU_{22}^{-1} \bbU_{21} \bbV_{11} \notag
\end{gather}  

Подставляя полученные выражения в уравнения (\ref{eq2}) и (\ref{eq5}) и выражая $\bbV_{11}$ и $\bbV_{22}$ получаем:
\begin{gather}
\bbV_{11} = \lb \bbU_{11} - \bbU_{12} \bbU_{22}^{-1} \bbU_{21} \rb^{-1} \notag \\
\bbV_{22} = \lb \bbU_{22} - \bbU_{21} \bbU_{11}^{-1} \bbU_{12} \rb^{-1} \notag
\end{gather}

Приходим к следующему виду матрицы $\bbV$:
\begin{gather}
\bbV = \begin{bmatrix}
\lb \bbU_{11} - \bbU_{12} \bbU_{22}^{-1} \bbU_{21} \rb^{-1} & -\bbU_{11}^{-1} \bbU_{12} \lb \bbU_{22} - \bbU_{21} \bbU_{11}^{-1} \bbU_{12} \rb^{-1} \\
-\bbU_{22}^{-1} \bbU_{21} \lb \bbU_{11} - \bbU_{12} \bbU_{22}^{-1} \bbU_{21} \rb^{-1} &  \lb \bbU_{22} - \bbU_{21} \bbU_{11}^{-1} \bbU_{12} \rb^{-1} 
\end{bmatrix} \notag
\end{gather}

Интересный прием заключается в следующем.
Мы нашли выражение для обратной (справа) матрицы, теперь найдем выражение для обратной (слева) матрицы $\bbQ = \begin{bmatrix} \bbQ_{11} & \bbQ_{12} \\ \bbQ_{21} & \bbQ_{22} \end{bmatrix}$:
\begin{gather}
\bbQ \bbU = \begin{bmatrix}
\bbE_{m \times m} & \bbO_{m \times n} \\
\bbO_{n \times m} & \bbE_{n \times n} 
\end{bmatrix} \notag 
\end{gather} 

Проводя аналогичные вычисления приходим к следующему виду матрицы $\bbQ$:
\begin{gather}
\bbQ = \begin{bmatrix}
\lb \bbU_{11} - \bbU_{12} \bbU_{22}^{-1} \bbU_{21} \rb^{-1} & - \lb \bbU_{11} - \bbU_{12} \bbU_{22}^{-1} \bbU_{21} \rb^{-1} \bbU_{12} \bbU_{22}^{-1} \\
- \lb \bbU_{22} - \bbU_{21} \bbU_{11}^{-1} \bbU_{12} \rb^{-1} \bbU_{21} \bbU_{11}^{-1} & \lb \bbU_{22} - \bbU_{21} \bbU_{11}^{-1} \bbU_{12} \rb^{-1} 
\end{bmatrix} \notag
\end{gather}

Так как обратная слева матрица совпадает с обратной справа матрицей, то получаем следующие два тождества:
\begin{gather}
\left\{
\begin{aligned}
\bbV_{12} &= \bbQ_{12} \\
\bbV_{21} &= \bbQ_{21}
\end{aligned}
\right. 
\quad \implies \quad
\left\{
\begin{aligned}
\bbU_{11}^{-1} \bbU_{12} \lb \bbU_{22} - \bbU_{21} \bbU_{11}^{-1} \bbU_{12} \rb^{-1} = \lb \bbU_{11} - \bbU_{12} \bbU_{22}^{-1} \bbU_{21} \rb^{-1} \bbU_{12} \bbU_{22}^{-1} \\
\bbU_{22}^{-1} \bbU_{21} \lb \bbU_{11} - \bbU_{12} \bbU_{22}^{-1} \bbU_{21} \rb^{-1} = \lb \bbU_{22} - \bbU_{21} \bbU_{11}^{-1} \bbU_{12} \rb^{-1} \bbU_{21} \bbU_{11}^{-1}
\end{aligned}
\right. \notag
\end{gather}