\section*{Выводы}

\begin{itemize}
\item Описан метод полного колебательно-вращательного гамильтониана
\item Получены точные гамильтонианы для одномерной и полномерной моделей трехатомного гидрида
\item Описана бифуркация в рамках концепции поверхности вращательной энергии
\item Получены системы динамических уравнений для модельных систем
\item Получены решения систем динамических уравнений в различных колебательных состояниях 
\item Описано влияние уровней колебательного возбуждение на бифуркацию
\end{itemize}

Следует особо отметить скорость вычислений, основанных на классическом подходе. Решение систем динамических уравнений при помощи системы компьютерной алгебры \textit{Maple} даже на домашнем компьютере происходит в течение секунд.
В дальнейшей работе планируется использовать строгий квантовохимический потенциал. Как уже отмечалось, в работе \cite{petrov2002} описано протекание обратной бифуркации, которая ответственна за отсутствие кластеризации во вращательных спектрах воды.  

