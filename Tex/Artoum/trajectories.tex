\section{Решение полной системы динамических уравнений}
Поверхность вращательной энергии является мощным инструментом, позволяющим на качественном уровне анализировать вращательную динамику молекулярной системы. Однако с ростом полной колебательно-вращательной энергии влияние колебательной подсистемы на вращение молекулы не может быть точно учтено в рамках эффективного гамильтониана. Следовательно необходимо решать полную систему динамических уравнений. При решении полной системы динамических уравнений необходимо надлежащим образом задавать начальные условия.

\subsection{Фазовые траектории одномерной модели.}

Система динамических уравнений, полученная из гамильтониана \eqref{model_ham}, выглядит следующим образом:
\vverh
\begin{gather}
\left\{
\begin{aligned}
\dot{\Phi} = \left( \frac{J_x}{I_0 ( 1 - \cos q)} \cos \Phi + \frac{J_y}{2I_0} \sin \Phi \right) &\ctg \Theta - \frac{J_z}{I_0 (1 + \cos q)} \\
\dot{\Theta} = \frac{J_x}{I_0 (1 - \cos q)} &\sin \Phi - \frac{J_y}{2I_0} \cos \Phi \\
\dot{q} &= 2	\frac{p}{I_0} \\
\dot{p} = - \frac{\sin q}{2I_0} \left( \frac{J_z^2}{(1 + \cos q)^2} - \frac{J_x^2}{(1 - \cos q)^2} \right) &- \frac{1}{2I_0} \left( \frac{V_{+} \sin q}{(1 + \cos q)^2} - \frac{V_{-} \sin q}{(1 - \cos q)^2} \right)
\end{aligned}
\right. \notag
\end{gather}

Решение представленной системы дифференциальных уравнений производилось при помощи математической платформы \textit{Maple 2015}. Возвращаясь к вопросу начальных условий, в качестве начального значения деформационного угла удобно использовать $q_0 = q_e$ (равновесного значения $q$ при фиксированном значении модуля вектора углового момента $J$). Переписывая выражение модельного гамильтониана \eqref{model_ham} через эффективный потенциал $V_{eff.}$, несложно получить начальное значение имупльса:
\vverh
\begin{gather}
E_n = \frac{p_0^2}{I_0} + V_{eff.} (q_0) \quad \implies \quad p_0 = \sqrt{I_0 \cdot \left( E_n - V_{eff.} (q_0) \right) } \notag
\end{gather}

Отметим, что положение $E_n$ внутри энергетической полосы (рисунок \eqref{vib_levels}) с номером $n$ определяется углами $\Theta_0$, $\Phi_0$, характеризующими направление вектора углового момента. 
