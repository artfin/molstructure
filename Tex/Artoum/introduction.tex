\section{Введение}

Для подавляющего количества задач, решаемых в области теоретической молекулярной спектроскопии, в последнее время применяются методы, основанные на квантовом рассмотрении. Несмотря на значительные вычислительные мощности, доступные в наше время, существуют задачи, в которых квантовое рассмторении не представляется возможным. При решении таких задач распространены методы молекулярной динамики. Однако существует небольшой класс задач, при решеении которых методы классической механики успешно конкурируют как с квантовыми вычислениями, так и с методами молекулярной динамики. К таким задачам можно отнести моделирование столкновительных спектров слабосвязанных систем или вращение молекулярных систем в условиях сильного колебательно-вращательного взаимодействия. Помимо прочего, классическое рассотрение задачи может дать лучшее понимание квантовых явлений, происходящих в рассматриваемых задачах. В рамках данной работы изучается вращение трехатомных гидридов в условиях сильного колебательно-вращельного взаимодействия. Делается попытка объяснения кластеризации вращательных уровней при достаточно высоких значениях углового момента системы.   