\section{Введение}

\vpravo Для подавляющего количества задач, решаемых в области теоретической молекулярной спектроскопии, в последнее время применяются методы, основанные на квантовом рассмотрении. Несмотря на значительные вычислительные мощности, доступные в наше время, существуют задачи, в которых квантовое рассмторение не представляется возможным. При решении таких задач распространены методы молекулярной динамики. Однако существует небольшой класс задач, при решении которых методы классической механики успешно конкурируют как с квантовыми вычислениями, так и с методами молекулярной динамики. К таким задачам можно отнести моделирование столкновительных спектров слабосвязанных систем \cite{lok2004} или вращение молекулярных систем в условиях сильного колебательно-вращательного взаимодействия \cite{petrov2015}. Помимо прочего, классическое рассотрение задачи может дать лучшее понимание квантовых явлений, происходящих в рассматриваемых задачах. В рамках данной работы изучается вращение трехатомных гидридов в условиях сильного колебательно-вращельного взаимодействия. \par
С точки зрения исторической ретроспективы первые подходы к описанию колебательно-вращательного взаимодействия были осуществлены в рамках теории возмущений. В нулевом приближении молекулярное движение разбивается на невзаимодействующие колебательное и вращательное движения \cite{wilson}. Считается, что взаимодействие этих движений в небольшой степени искажает движение молекулярной системы, что и позволяет совершать описание этого взаимодействия в рамках ТВ. Однако такой подход совершенно неприменим в случае молекул, которые совершают высокоамплитудные внутренние колебания при высоких значениях углового момента. \par
Для объяснения структуры вращательных мультиплетов в молекулярной спектроскопии была введена концепция поверхности вращательной энергии (ПВЭ) \cite{king1947}, получившая дальнейшее развитие в работах \cite{har1984, makar1998}. ПВЭ представляет собой полную энергию молекулы как функцию направления вектора углового момента, причем угловой момент в рамках данной концепции рассматривается как чисто классическая величина. Феноменологически считается, что энергии вращательных уровней могут быть рассчитаны путем квантования траекторий вектора углового момента на ПВЭ. Однако в большей степени ПВЭ применяется для качественного описания структуры вращательных мультиплетов \cite{makar1998}. Характер траекторий вектора углового момента определяется количеством и характером стационарных точек ПВЭ \cite{pav1985, pav1987}. \par
Исследование типов стационарных точек обнаружило явление, названное бифуркацией, которое заключается в перестройке ПВЭ. В рамках данного подхода была создана общая классификация возможных типов бифуркации в молекулах. Однако концепция не позволяла с достаточной точностью предсказать значение критического углового момента $J_{cr.}$, при котором происходит бифуркация. \par
Альтернативным подходом к описанию явления бифуркации является решение полной системы динамических уравнений, содержащих полную информацию о колебательно-вращательном движении. В работе \cite{koz1996}  был описан подход к решению линеаризованных динамических уравнений в окрестности стационарных точек, что позволило получить более точные критические значения углового момента. \par
В работе \cite{petrov2015} рассматривалась полная система динамических уравнений для модельной системы с одной деформационной степенью свободы. Решения системы представлялись в форме колебательно-вращательных траекторий. В рамках нашей работы мы систематически опишем метод получения точного классического колебательно-вращательного гамильтониана; на примере модельной системы с деформационной степенью свободы опишем подход, позволяющий построить поверхность вращательной энергии, а также получим колебательно-вращательные траектории; проследим за изменениями в колебательно-вращательных траекториях, происходящими при переходе от одномерной к полномерной модельной системе (содержащей две валентных и одну деформационную степени свободы). 