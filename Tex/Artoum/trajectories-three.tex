\subsection{Фазовые траектории полномерной модели.}

Не уточняя вид потенциала в гамильтониане \eqref{model_ham2}, система динамических уравнений выглядит следующим образом:

\begin{gather}
\left\{
\begin{aligned}
& \dot{q} = \frac{\partial \mathcal{H}}{\partial p} = \frac{2p}{I_0} + \frac{r_1^2 - r_2^2}{2m r_1^2 r_2^2} J \sin \Phi \sin \Theta \\
& \dot{r_1} = \frac{\partial \mathcal{H}}{\partial p_1} = \frac{p_1}{m} \\
& \dot{r_2} = \frac{\partial \mathcal{H}}{\partial p_2} = \frac{p_2}{m}  \\
& \dot{p} = - \frac{\partial \mathcal{H}}{\partial q} = \frac{J^2 \cos^2 \Phi \sin^2 \Theta \sin q}{2 I_0 (1 - \cos q)^2} - \frac{J^2 \cos^2 \Theta \sin q}{2 I_0 (1 + \cos q)^2} - \frac{r_1^2 - r_2^2}{r_1^2 + r_2^2} \frac{J^2 \cos \Phi \sin \Theta \cos \Theta \cos q}{I_0 \sin^2 q} - \frac{\rmd U}{\rmd q}  \\
& \dot{p_1} = - \frac{\partial \mathcal{H}}{\partial r_1} =
\frac{1}{2} \left[ \frac{J^2 \cos^2 \Phi \sin^2 \Theta}{I_0 (1 - \cos q)} + \frac{J^2 \sin^2 \Phi \sin^2 \Theta}{2 I_0} + \frac{J^2 \cos^2 \Theta}{I_0 (1 + \cos q)} \right] \frac{1}{r_1} \left( 2 - \frac{I_0}{m r_2^2} \right) - \notag \\
& \hspace{1cm} - \frac{J^2 \cos \Phi \sin \Theta \cos \Theta}{I_0 \sin q} \left[ \frac{4 r_1 r_2^2}{(r_1^2 + r_2^2)^2} - \frac{r_1^2 - r_2^2}{r_1^2 + r_2^2} \frac{1}{r_1} \left( 2 - \frac{I_0}{m r_2^2} \right) \right] - \frac{p}{m r_1^3} J \sin \Phi \sin \Theta - \frac{\rmd U}{\rmd r_1} \\
& \dot{p_2} = - \frac{\partial \mathcal{H}}{\partial r_2} = 
\frac{1}{2} \left[ \frac{J^2 \cos^2 \Phi \sin^2 \Theta}{I_0 (1 - \cos q)} + \frac{J^2 \sin^2 \Phi \sin^2 \Theta}{2 I_0} + \frac{J^2 \cos^2 \Theta}{I_0 (1 + \cos q)} \right] \frac{1}{r_2} \left( 2 - \frac{I_0}{m r_1^2} \right) + \notag \\
& \hspace{1cm} + \frac{J^2 \cos \Phi \sin \Theta \cos \theta}{I_0 \sin q} \left[ \frac{4 r_1^2 r_2}{(r_1^2 + r_2^2)^2} + \frac{r_1^2 - r_2^2}{r_1^2 + r_2^2} \frac{1}{r_2} \left( 2 - \frac{I_0}{m r_1^2} \right) \right] + \frac{p}{m r_2^3} J \sin \Phi \sin \Theta - \frac{\rmd U}{\rmd r_2} \\
& \dot{\Phi} = \left[ \left( \frac{J \cos \Phi \sin \Theta}{I_0 (1 - \cos q)} +  \frac{r_1^2 - r_2^2}{r_1^2 + r_2^2} \frac{J \cos \Theta}{I_0 \sin q} \right) \cos \Phi + \left( \frac{J \sin \Phi \sin \Theta}{2 I_0} + \frac{r_1^2 - r_2^2}{2 m r_1^2 r_2^2} p \right) \sin \Phi \right] \ctg \Theta - \notag \\
& \hspace{3cm} - \left( \frac{J \cos \Theta}{I_0 ( 1 + \cos q)} + \frac{r_1^2 - r_2^2}{r_1^2 + r_2^2} \frac{J \cos \Phi \sin \Theta}{I_0 \sin q} \right) \\
& \dot{\Theta} = \left( \frac{J \cos \Phi \sin \Theta}{I_0 ( 1 - \cos q)} + \frac{r_1^2 - r_2^2}{r_1^2 + r_2^2} \frac{J \cos \Theta}{I_0 \sin q} \right) \sin \Phi - \left( \frac{J \sin \Phi \sin \Theta}{2 I_0} + \frac{r_1^2 - r_2^2}{2m r_1^2 r_2^2} p \right) \cos \Phi 
\end{aligned}
\right. \notag
\end{gather}
 
Решение представленной системы дифференциальных уравнений с потенциалами, описанными выше, производилось при помощи математической платформы \textit{Maple 2015}. 

Построим схему определения начальных условий, аналогичную описанной для более простой модельной системы. Первым шагом необходимо задаться значениями энергетических уровней соответствующих потенциалов. Обозначим $n$ -- уровень возбуждения деформационного колебания, $E_n$ -- энергия соответствующего уровня; $n_1$, $n_2$ -- уровни возбуждения валентных колебаний, $E_{n_1}$, $E_{n_2}$ -- энергии соответствующих уровней. Найдем эффективный потенциал и используем тот факт, что гамильтониан является интегралом движения:
\vverh
\begin{gather}
V_{eff.} = \frac{1}{2} \vec{J}^{\, \top} \bbI^{-1} \vec{J} + U(q, r_1, r_2) \notag \\
V_{eff.} = \frac{J_x^2}{2 I_0 (1 - \cos q)} + \frac{J_y^2}{m (r_1^2 + r_2^2)} + \frac{J_z^2}{I_0 (1 + \cos q)} + \frac{r_1^2 - r_2^2}{r_1^2 + r_2^2} \frac{J_x J_z}{2I_0 \sin q} + U(r_1, r_2, q) \notag \\
E_n + E_{n_1} + E_{n_2} = V_{eff.}(r_{1,0}, r_{2,0}, q_0, \Theta_0, \Phi_0) + \frac{r_{1,0}^2 - r_{2,0}^2}{r_{1,0}^2 + r_{2,0}^2} \frac{J_{y,0}^2}{4 m r_{1,0}^2 r_{2,0}^2} + \frac{r_{1,0}^2 - r_{2,0}^2}{2m r_{1,0}^2 r_{2,0}^2} J_{y,0} p_0 + \frac{p_{1,0}^2}{2m} + \frac{p_{2,0}^2}{2m} + \frac{p_0^2}{I_0} \notag \\
\left\{
\begin{aligned}
J_{x,0} &= J \cos \Phi_0 \sin \Theta_0 \\
J_{y,0} &= J \sin \Phi_0 \sin \Theta_0 \\
J_{z,0} &= J \cos \Theta_0
\end{aligned}
\right. \notag
\end{gather}

Заметим, что если мы возьмем $\Phi_0 = 0$, то $J_{y, 0} = 0$, и полученное выражение значительно упростится. Также положим $p_{1,0} = p_{2,0} = 0$, что дополнительно упростит выражение.
Приходим к выражению, аналогичному тому, что было выписано для предыдущей модельной системы:
\vverh
\begin{gather}
E_n + E_{n_1} + E_{n_2} = V_{eff.}(r_{1,0}, r_{2,0}, q_0, \Theta_0, \Phi_0) + \frac{p_0^2}{I_0}
\label{three_energy}
\end{gather}

Используя полуклассический подход, определим начальные значения $r_{1, 0}$, $r_{2, 0}$, при которых энергии валентных колебаний равны $E_{n_1}$ и $E_{n_2}$, соответственно. В случае гармонического потенциала начальное значение длины связи может быть получено в аналитической форме:  $r_1 = r_0 \pm \sqrt{\ddfrac{E_{n_1}}{k}}$, но в случае потенциала Морзе -- только численными методами. 
В качестве начального значения угла $q_0$ возьмем $q_e$, которое определяется модулем вектора углового момента и его направлением (выражение \eqref{eq_angle}). Таким образом, нам осталось определить начальное значение импульса $p$, которое мы находим исходя из соотношения \eqref{three_energy}, предварительно рассчитав значение эффективного потенциала при заданных условиях (не забывая про то, что $I_0 = I_0 (r_{1,0}, r_{2,0})$).
\vverh
\begin{gather}
p = \sqrt{I_0 (E_n + E_{n_1} + E_{n_2} - V_{eff.} (q_0, r_{1,0}, r_{2,0}, \Theta_0, \Phi_0))} \notag
\end{gather}

\vverh
\begin{figure}[H]
  \begin{center}
    \begin{tikzpicture}[framed, ->, >=stealth', auto, thick]
    
      \tikzstyle{type1} = [rectangle, rounded corners, minimum width = 3cm, minimum height = 1cm, text centered, text width = 3cm ,draw = black, fill = red!30]

\tikzstyle{type2} = [rectangle, rounded corners, minimum width = 3cm, minimum height = 1cm, text centered, text width = 3cm, draw = black, fill = green!30]

\tikzstyle{type3} = [rectangle, rounded corners, minimum width = 3cm, minimum height = 1.5cm, text centered, text width = 5 cm, draw = black, fill = yellow!30]

\tikzstyle{arrow} = [thick, ->, >=stealth]

\node (n) [type2] {$n$};
\node (n1) [type2, right = 0.5 cm of n] {$n_1$};
\node (n2) [type2, right = 0.5 cm of n1] {$n_2$};
\node (n_container) [type3, fit = (n) (n1) (n2)] {};
\node (n) [type2] {$n$};
\node (n1) [type2, right = 0.5 cm of n] {$n_1$};
\node (n2) [type2, right = 0.5 cm of n1] {$n_2$};

\node (r2) [type1, below right = 1 cm and 2 cm of n_container] {$r_{2,0}$};
\node (r1) [type1, below = 0.5 cm of r2] {$r_{1,0}$};
\node (r_container) [type3, fit = (r1) (r2)] {};	
\node (r2) [type1, below right = 1 cm and 2 cm of n_container] {$r_{2,0}$};
\node (r1) [type1, below = 0.5 cm of r2] {$r_{1,0}$};

\node (theta) [type2, below = 5.5 cm of n_container] {$\Theta_0$};
\node (phi) [type2, right = 1 cm of theta] {$\Phi_0 = 0$};
\node (angle_container) [type3, fit = (theta) (phi)] {}; 
\node (theta) [type2, below = 5.5 cm of n_container] {$\Theta_0$};
\node (phi) [type2, right = 1 cm of theta] {$\Phi_0 = 0$};

\node (energy) [type2, below left = 5.5 cm and 0.5 cm of n_container] {$E = E_n + E_{n_1} + E_{n_2}$};

\path (n1) edge[bend right] node [left] {} (r1);
\path (n2) edge[bend right] node [left] {} (r2); 
      
	  \begin{scope}[on background layer]
  		\node (outer) [draw,fill = gray!30, fit = (n_container) (r_container) (angle_container) (energy) (q) (J)] {};
  	  \end{scope}
    \end{tikzpicture}
    \caption{Схема определения начальных параметров системы.}
  \end{center}
\end{figure}

