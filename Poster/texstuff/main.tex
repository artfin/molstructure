% we don't want ae.sty
\expandafter\def\csname ver@ae.sty\endcsname{}

\documentclass[
  20pt,
  a1paper,
  portrait,
  margin=0mm,
  innermargin=15mm,
  blockverticalspace=15mm,
  colspace=15mm,
  subcolspace=8mm
]{tikzposter}

\usetheme{Simple}
%\usecolorstyle[colorPalette=BrownBlueOrange]{Germany}

\usepackage[T3]{fontenc}
\usepackage[utf8]{inputenc}
\usepackage[russian]{babel}

\usepackage{mhchem}
\usepackage{amssymb, amsmath}

\usepackage{tikz}
\usetikzlibrary{shapes.geometric, arrows, positioning, decorations.markings}
\usetikzlibrary{fit}
\usepackage{microtype}
\usepackage{framed}
\usetikzlibrary{decorations.pathmorphing,calc,backgrounds}

\usepackage{animate}

\usepackage{fixltx2e}
\usepackage{hyperref}

%\usetheme{Berkeley}
%\usetheme{Madrid} -- неплохо
%\usetheme{CambridgeUS}
%\usetheme{Singapore}
\usetheme{Warsaw}

\pdfmapfile{+sansmathaccent.map}

\title{Исследование бифуркаций в трехатомных гидридах методом классических траекторий}

\author{\small 
Финенко Артем \\[1ex] 
Научный руководитель: Петров С.В.}

\institute[MSU] % (optional, but mostly needed)
{
  МГУ им. М.В.Ломоносова \\
  Химический факультет
}

\date{23/12/2016}

\pgfdeclareimage[height=0.5cm]{university-logo}{../pictures/logo.jpg}
\logo{\pgfuseimage{university-logo}}

\newcommand\Fontvi{\fontsize{6}{7.2}\selectfont}

\beamertemplatenavigationsymbolsempty

\setbeamerfont{page number in head/foot}{size=\large}
\setbeamertemplate{footline}[frame number]
\setbeamertemplate{frametitle}[default][center]

% change font
\usefonttheme[onlymath]{serif}

% custom block environment
\newenvironment<>{varblock}[2][.9\textwidth]{%
  \setlength{\textwidth}{#1}
  \begin{actionenv}#3%
    \def\insertblocktitle{#2}%
    \par%
    \usebeamertemplate{block begin}}
  {\par%
    \usebeamertemplate{block end}%
  \end{actionenv}}

\tikzstyle{lagrange} = [rectangle, rounded corners, minimum width = 3cm, minimum height = 1cm, text centered, text width = 5cm, draw = black, fill=DarkOrchid!40]

\tikzstyle{equations} = [rectangle, rounded corners, text centered, draw = black, fill=green!30]

\tikzstyle{hamilton} = [rectangle, rounded corners, minimum width = 3cm, minimum height = 1cm, text centered, text width = 5 cm, draw = black, fill = Goldenrod!50]

\tikzstyle{result} = [rectangle, rounded corners, text centered, draw = black, fill = blue!30]

\tikzstyle{arrow} = [thick, ->, >=stealth]

\tikzstyle{vecArrow} = [thick, decoration={markings,mark=at position
   1 with {\arrow[semithick]{open triangle 60}}},
   double distance=1.4pt, shorten >= 5.5pt,
   preaction = {decorate},
   postaction = {draw,line width=1.4pt, white,shorten >= 4.5pt}]

\usepackage{caption}
\usepackage{subcaption}

% tikz libraries
\usepackage{tikz}
\usetikzlibrary{shapes.geometric, arrows, positioning, decorations.markings}
\usetikzlibrary{fit}
\usepackage{framed}
\usetikzlibrary{decorations.pathmorphing, calc, backgrounds}

\begin{document}
\maketitle

\block{Введение}{
    Для подавляющего количества задач, решаемых в области теоретической молекулярной спектроскопии, в последнее время применяются методы, основанные на квантовом рассмотрении. Однако, несмотря на значительные вычислительные мощности, доступные в наше время, существуют задачи, в которых квантовое рассмотрение не представляется возможным. Существует небольшой класс задач, при решении которых методы классической механики успешн конкурируют как с квантовыми вычислениями, так и с методами молекулярной динамики. К этому классу относят вращение молекулярных систем в условиях сильного колебательно-вращательного взаимодействия [!]. Помимо прочего, классическое рассмотрение может дать лучшее понимание квантовых явлений, происходящих в рамках рассматриваемой задачи. В данной работе изучается вращение трехатомных гидридов в условиях сильного колебательного взаимодействия.
}
\begin{columns}
\column{0.5}
\block{Метод анализа колебательно-вращательной динамики}{
    \begin{tikzpicture}
        \tikzstyle{lagrange} = [rectangle, rounded corners, minimum height = 1cm, text centered, draw = black, fill = red!30]
    
        \tikzstyle{hamilton} = [rectangle, rounded corners, minimum height = 1cm, text centered, draw = black, fill=yellow!30]

        \tikzstyle{equations} = [rectangle, rounded corners, minimum height = 1cm, text centered, draw = black, fill=green!30]

        \tikzstyle{equations_big} = [rectangle, rounded corners, minimum height = 1cm, text centered, draw = black, fill = green!30]
        
        \tikzstyle{arrow} = [thick, ->, >=stealth]
        \tikzstyle{arrow_comes_above} = [thick, ->, >=stealth, yshift=10pt]

        \tikzstyle{vecArrow} = [thick, decoration={markings,mark=at position
    1 with {\arrow[semithick]{open triangle 60}}},
    double distance=1.4pt, shorten >= 5.5pt,
    preaction = {decorate},
    postaction = {draw,line width=1.4pt, white,shorten >= 4.5pt}]

        \node (lag1) [lagrange] {$\mathcal{L} = \mathcal{L}(\vec{r}_1^{\ \prime},  \cdots, \vec{r}_n^{\ \prime}, \dot{\vec{r}}_1^{\ \prime}, \cdots, \dot{\vec{r}}_n^{\ \prime})$};
        
        \node (lag2) [lagrange, below = 2.5 cm of lag1] {$\mathcal{L} = \mathcal{L} (\vec{r}_1, \cdots, \vec{r}_{n-1}, \dot{\vec{r}}_1, \cdots, \dot{\vec{r}}_{n-1})$};

        \node (lag3) [lagrange, below = 2.0 cm of lag2] {$\mathcal{L} = \mathcal{L}(\vec{R}_1, \cdots, \vec{R}_n, \dot{\vec{R}}_1, \cdots, \dot{\vec{R}}_{n-1}, \vec{\Omega})$};

        \node (lag4) [lagrange, below = 2.0 cm of lag3] {$\mathcal{L} = \mathcal{L}(\vec{q}_1, \cdots, \vec{q}_s, \dot{\vec{q}}_1, \cdots, \dot{\vec{q}}_s, \vec{\Omega})$};

\node (ham1) [hamilton, below = 2.5 cm of lag4] {$\mathcal{H} = \mathcal{H}(\vec{q}_1, \cdots, \vec{q}_s, \vec{p}_1, \cdots, \vec{p}_s, \vec{J})$};


\title{\parbox{\linewidth}{\centering Исследование бифуркаций в трехатомных гидридах \\ методом классических траекторий}}
\node (eq5) [equations, below = 2.0 cm of ham1] {$
        \dot{\vec{p}} = \frac{\partial \mathcal{H}}{\partial \vec{q}} \\[1ex]
        \dot{\vec{q}} = - \frac{\partial \mathcal{H}}{\partial \vec{p}} \\
        \dot{\vec{J}} + [ \frac{\partial \mathcal{H}}{\partial \vec{J}} \times \vec{J} ] = 0 
$};

        \draw [vecArrow] (lag1) -- node[anchor = east, text width = 14cm] {\vspace*{-0.8cm} \begin{center} Переход в систему отсчета, связанную с центром масс\end{center}} node[anchor = west] {\hspace{0.3cm} $\vec{r}_i^{\ \prime} = \vec{R} + \vec{r}_i$} (lag2);

        \draw [vecArrow] (lag2) -- node[anchor = east] {Переход в подвижную систему отсчета}  node[anchor = west] {\hspace{0.3cm} $\vec{r}_i = \bbS \vec{R}_i$} (lag3);

        \draw [vecArrow] (lag3) -- node[anchor = east] {Переход к обобщенным координатам} node[anchor = west] {\hspace{0.3cm} $\vec{R}_i = \vec{R}_i(q_1, \cdots, q_s)$} (lag4);

        \draw [vecArrow] (lag4) -- node[anchor = east] {Применение теоремы Донкина} node[anchor = west] {\hspace{0.3cm} $J = \frac{\partial \mathcal{L}}{\partial \Omega}, \ p = - \frac{\partial \mathcal{L}}{\partial \dot{q}}
            $}(ham1);



%\node (eq6) [equations_big, right = 3 cm of eq5] {$
%\left\{
%\begin{aligned}
%\dot{\vec{p}} &= \frac{\partial \mathcal{H}}{\partial \vec{q}} \\
%\dot{\vec{q}} &= - \frac{\partial \mathcal{H}}{\partial \vec{p}} \\
%\dot{\varphi} &= \left( \frac{\partial \mathcal{H}}{\partial J_x} \cos \varphi + \frac{\partial \mathcal{H}}{\partial J_y} \sin \varphi \right) \ctg \theta - \frac{\partial \mathcal{H}}{\partial J_z} \\
%\dot{\theta} &= \frac{\partial \mathcal{H}}{\partial J_x} \sin \varphi - \frac{\partial \mathcal{H}}{\partial J_y} \cos \varphi
%\end{aligned}
%\right.
%$};

    \end{tikzpicture}

}
\column{0.5}
\block{BlocktitleC}{Blocktext}
\end{columns}
\end{document}
