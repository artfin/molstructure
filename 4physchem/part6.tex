\section{Расчет константы равновесия системы $Ar-CO_2$.}

\subsection{Константа равновесия в каноническом ансамбле}

Канонический ансамбль представляет собой модель термодинамической системы, погруженной в тепловой резервуар постоянной температуры. В представленной системе резервуар играет роль термостата, поддерживая температуру исследуемой системы постоянной независимо от направления теплового потока. Канонический ансамбль (также называемый $NVT$-ансамблем) представляет собой систему из $N$ частиц, находящихся в фиксированном объеме $V$ при фиксированной температуре $T$. Отметим, что энергия $NVT$-ансамбля может принимать любые значения, согласующиеся с заданными условиями. Рассматривая совокупность резервуара и погруженной в него исследуемой системы как микроканонический ансамбль, получают каноническое распределение (распределение Гиббса) -- вероятность нахождения системы в состоянии с энергией $E_n$:
\vverh
\begin{gather}
	\omega_n = A \times \exp \lb - \frac{E_n}{kT} \rb \quad \Leftrightarrow \quad \rho \lb p, q \rb = A \times \exp \lb - \frac{E \lb p, q \rb}{kT} \rb \notag
\end{gather}
Энтропия может быть выражена как среднее значение функции распределения:
\vverh
\begin{gather}
	S = - \langle \ln \omega_n \rangle \notag \\
	S = - \ln A + \frac{1}{T} \langle E_n \rangle = - \ln A + \frac{\overline{E}}{kT} \notag \\
	\ln A = \frac{\overline{E} - TS}{kT} = \frac{F}{kT} \notag 
\end{gather}
Средняя энергия $\overline{E}$ есть как раз то, что понимается в термодинамике под энергией $U$, таким образом получаем следующие выражение для функции распределения Гиббса через свободную энергию Гельмгольца (в квантовой и классической статистиках, соответственно):
\vverh
\begin{gather}
	\omega_n = \exp \lb - \frac{F - E_n}{kT} \rb \quad \Leftrightarrow \quad \rho(p, q) = \frac{1}{\lb 2 \pi \hbar \rb^{s}} \exp \lb \frac{F - E(p, q)}{kT} \rb \notag
\end{gather}

Условие нормировки для распределения $\omega_n$:
\vverh
\begin{gather}
	\sum_n \omega_n = \exp \lb \frac{F}{kT} \rb \sum_n \exp \lb - \frac{E_n}{kT} \rb = 1 \notag \\
	F = - kT \ln \sum_n \exp \lb - \frac{E_n}{kT} \rb = - kT \ln Z \notag
\end{gather}

При записи аналогичной формулы в классической термодинамике следует учесть, что если, например, переменить местами два одинаковых атома, то после такой перестановки микросостояние тела будет изображаться другой фазовой точкой, получающейся из первоначальной заменой и импульсов одного атома координатами и импульсами другого. С другой стороны, ввиду одинаковости переставляемых атомов, оба состояния тела физически тождественны. Таким образом, одному и тому же физическому микросостоянию тела в фазовом пространстве соответствует целый ряд точек. При интегрировании распределения каждое состояние должно учитываться лишь однократно (статистический интеграл можно рассмотреть как предел квантовой статистической суммы, суммирование в которой производится по всем различным квантовым состояниям), то есть, интегрирование производится лишь по тем областям фазового пространства, которые соответсвуют физически различным состояниям тела (штрих над интегралом будет подчеркивать эту особенность классической статистической суммы).
\vverh
\begin{gather}
	F = -k T \ln \int^{\prime} \exp \lb - \frac{E(p, q)}{kT} \rb d \Gamma, \quad d \Gamma = \frac{dp \, dq}{\lb 2 \pi \hbar \rb^s} \notag
\end{gather}

Запишем свободную энергию Гельмгольца, используя молекулярную статистическую сумму $q$. (Слагаемое $F(0)$ призвано подправить значение в правой части равенства, принимая во внимание свободу выбора начала отсчета для молекулярной суммы по состояниям.)  
\vverh
\begin{gather}
	F = - k T \ln Q = F(0) - kT \ln \frac{q^N}{N!} = F(0) - NkT \ln q + NkT \lb \ln N - 1 \rb \notag 
\end{gather}

Полагая $N = N_A$, приходим к мольному значению свободной энергии Гельмгольца:
\vverh
\begin{gather}
	F_m = F_m(0) - RT \ln \frac{q}{N_A} - RT \notag
\end{gather}

В приближении идеального газа имеем $G_m = F_m + p V_m = F_m + RT$. Заметим, что $F_m(0) = U_m(0)$. Под $q^{\barcirc} = q_{NVT}^{\barcirc}$ понимается значение суммы по состояниям, вычисленное при $N = N_A$, фиксированной температуре $T$ и при фиксированном объеме $V = \frac{RT}{p^{\barcirc}}$.
\vverh
\begin{gather} 
	G_m = F_m(0) - RT \ln \frac{q}{N_A} \notag \\
	G^{\barcirc} = U_m(0) - RT \ln \frac{q^{\barcirc}}{N_A} \notag \\
	\Delta_r G^{\barcirc} = \Delta_r U_m(0) - RT \ln \prod_{i} \lb \frac{q_i^{\barcirc}}{N_A} \rb ^{\nu_i} \notag
\end{gather}

Используя равенство $\Delta_r G^{\barcirc} = - RT \ln K$, получаем выражение для термодинамической константы равновесия:
\vverh
\begin{gather}
	K = \prod_{i = 1}^{r} \lb \frac{q_i^{\barcirc}}{N_A} \rb^{\nu_i} \exp \lb - \frac{\Delta_r U_m (0)}{RT} \rb \notag
\end{gather}

Термодинамическая и газовая константы равновесия связаны между собой следующим соотношением: 
\vverh
\begin{gather}
	K_p =  K \times \lb p^{\barcirc} \rb^{\sum_i \nu_i} \notag 
\end{gather}

Получим выражение для газовой константы равновесия $K_p$ слабосвязанного комплекса $Ar-CO_2$. В условиях совпадающего начала отсчета молекулярных сумм по состояниям для всех участников реакции, получаем следующее выражение:
\vverh
\begin{gather}
	K_p = \frac{N_A}{p^{\barcirc}} \frac{q_{Ar-CO_2}^{\barcirc}}{q_{Ar}^{\barcirc} q_{CO_2}^{\barcirc}} \label{eqconst}
\end{gather}

\subsection{Статистическая сумма связанного димера} 

Связанным димером называют пару мономеров, энергия которых меньше чем у пары мономеров на бесконечно большом расстоянии друг от друга. Следовательно, классическая сумма по состояниям связанного димера представляет собой следующий фазовый интеграл
\vverh
\begin{gather}
	Q_{bound}^{pair} = \frac{1}{h^8} \int\limits_{H - \ddfrac{\strut P_{cm}^2}{\strut 2 M} < 0} \exp \lb - \frac{H}{k T} \rb d x_{cm} d y_{cm} d z_{cm} d P_{x} d P_{y} d P_{z} d q_i d p_i, \label{ci1}
\end{gather}

где $q_i$, $p_i$ -- набор внутримолекулярных координат и импуьсов, $H$ - гамильтониан, записанный в лабораторной системе координат, $\mathbf{R} = \lb x_{cm}, y_{cm}, z_{cm} \rb$, $\mathbf{P}_R = \lb P_x, P_y, P_z \rb$ -- векторы координат и импульсов центра масс, а $M$ -- его масса. \par
Отметим, что гамильтониан, фигурирующий в выражении \eqref{ci1}, связан с гамильтонианом в молекулярно-фиксированной системе отсчета соотношением
\vverh
\begin{gather}
	H = \mH + \frac{P_{cm}^2}{2 M} \notag
\end{gather}

Интегрирование по векторам центра масс $\mf{R}$, $\mf{P}_R$ в выражении \eqref{ci1} дает трансляционную статистическую сумму димера:
\vverh
\begin{gather}
	\lb Q_{bound}^{pair} \rb_{tr} = \lb \frac{2 \pi M k T}{h^2} \rb^{\frac{3}{2}} V \notag
\end{gather}

\subsection{Эйлеровы углы и сопряженные им импульсы}
Рассмотрим кинетическую энергию в лагранжевой форме
\vverh
\begin{gather}
	T_\mL = \frac{1}{2} \dot{\mf{q}}^\top \bba \, \dot{\mf{q}} + \mf{\Omega}^\top \bbA \dot{\mf{q}} + \frac{1}{2} \mf{\Omega}^\top \bbI \, \mf{\Omega} \notag
\end{gather}
Вектор угловой скорости $\mf{\Omega}$ связан с вектором эйлеровых скоростей $\dot{\mf{e}}$ при помощи матрицы $\bbV$ \cite{goldstein}:
\begin{gather}
\mathbf{\Omega} = \mathbb{V} \dot{\mathbf{e}} = 
\begin{bmatrix}
\sin \theta \sin \psi & \cos \varphi & 0 \\
\sin \theta \cos \psi & - \sin \psi & 0 \\
\cos \theta & 0 & 1
\end{bmatrix}
\begin{bmatrix}
\dot{\varphi} \\
\dot{\theta} \\
\dot{\psi}
\end{bmatrix} \notag 
\end{gather}

По определению построим вектор эйлеровых импульсов:
\vverh
\begin{gather}
	\mf{p}_e = \frac{\partial T_\mL}{\partial \dot{\mf{e}}} = \bbV^\top \bbA \dot{\mf{q}} + \bbV^\top \bbI \, \bbV \dot{\mf{e}} = \bbV^\top \bbA \dot{\mf{q}} + \bbV^\top \bbI \, \mf{\Omega} \notag
\end{gather}

Несложно показать, что вектор углового момента в подвижной системе отсчета равен
\vverh
\begin{gather}
	\mf{J} = \frac{\partial \mL}{\partial \mf{\Omega}} = \bbA \dot{\mf{q}} + \bbI \, \mf{\Omega} \notag
\end{gather}

Таким образом, получаем следующую связь между вектором эйлеровых импульсов и вектором углового момента:
\vverh
\begin{gather}
	\mf{J} = \lb \bbV^\top \rb^{-1} \mf{p}_e \notag
\end{gather}

Интегрирование в \eqref{ci1} в том числе ведется по эйлеровым углам и импульсам. В связи с этим рассмотрим следующий кратный интеграл и осуществим в нем замену переменных $p_\varphi, p_\theta, p_\psi \longrightarrow J_x, J_y, J_z$. (В этом подразделе $\theta$ -- один из эйлеровых углов, а не внутренняя система координата системы $Ar-CO_2$. Эта замена нам интересна по той причине, что в выражение гамильтониана входят именно компоненты углового момента, а не эйлеровы импульсы; так выражение получается намного более компактным.)
\vverh
\begin{gather}
	\int\limits_{0}^{2 \pi} d \varphi \int\limits_{0}^{\pi} d \theta \int\limits_{0}^{2 \pi} d \psi \int d p_\varphi \int d p_\theta \int d p_\psi = \int \, \left[ Jac \right] d J_x \int d J_y \int d J_z \notag
\end{gather}

Якобиан замены переменных равен
\vverh
\begin{gather}
	\left[ Jac \, \right] = \Bigg{|} \frac{\partial \mathbf{p}}{\partial \mathbf{J}} \Bigg{|} = \Bigg{|} \det \lb \bbV^\top \rb \Bigg{|} = \sin \theta \notag
\end{gather}

Если подынтегральное выражение не зависит от эйлеровых углов (как в случае со стастической суммой связанного димера), то интегральное выражение преобразуется к виду
\vverh
\begin{gather}
	\int\limits_{0}^{2 \pi} d \varphi \int\limits_{0}^{\pi} d \theta \int\limits_{0}^{2 \pi} d \psi \int d p_\varphi \int d p_\theta \int d p_\psi = 8 \pi^2 \int d J_x \int d J_y \int d J_z \notag 
\end{gather}

Однако, в случае системы $Ar-CO_2$ вращение на углы $\psi$ большие чем $\pi$ не приводит к новым состояниям системы (из-за симметричности молекулы $CO_2$), следовательно для рассматриваемой системы интегральное выражение имеет вид
\vverh
\begin{gather}
	\int\limits_{0}^{2 \pi} d \varphi \int\limits_{0}^{\pi} d \theta \int\limits_{0}^{\pi} d \psi \int d p_\varphi \int d p_\theta \int d p_\psi = 4 \pi^2 \int d J_x \int d J_y \int d J_z \notag 
\end{gather}

\subsection{Упрощенный вид стастической суммы по состояниям связанного димера}

Заменяя эйлеровы импульсы на компоненты углового момента и интегрируя по переменным центра масс, сводим интеграл \eqref{ci1} к следующему:
\vverh
\begin{gather}
	Q_{bound}^{pair} = \lb Q_{bound}^{pair} \rb_{tr} \frac{4 \pi^2}{h^5} \int\limits_{\mH < 0} \exp \lb - \frac{\mH}{kT} \rb d R \, d \theta \, d p_R \, d p_\theta \, d J_x \, d J_y \, d J_z. \notag
\end{gather}

Рассмотрим гамильтониан системы $Ar-CO_2$ в молекулярно-фиксированной системе координат:
\begin{gather}
	\mH = \frac{1}{2 \mu_2} p_R^2 + \lb \frac{1}{2 \mu_2 R^2} + \frac{1}{2 \mu_1 l^2} \rb p_\theta^2 - \frac{1}{\mu_2 R^2} p_\theta J_y + \frac{1}{2 \mu_2 R^2} J_y^2 + \frac{1}{2 \mu_2 R^2} J_x^2 + \frac{1}{2 \sin^2 \theta} \lb \frac{\cos^2 \theta}{\mu_2 R^2} + \frac{1}{\mu_1 l^2} \rb J_z^2 + \notag \\
+ \frac{\ctg \theta}{\mu_2 R^2} J_x J_z + U(R, \theta) \notag
\end{gather}

Заметим, что форма гамильтониана $\mH$ позволяет представить его в виде положительно определенной квадратичной формы:
\begin{gather}
	\mH = \frac{p_R^2}{2 \mu_2} + \frac{p_\theta^2}{2 \mu_1 l^2} + \frac{1}{2 \mu_2 R^2} \lb p_\theta - J_y \rb^2 + \frac{1}{2 \mu_2 R^2} \lb J_x + J_z \ctg \theta \rb^2 + \frac{J_z^2}{2 \mu_1 l^2 \sin^2 \theta} + U(R, \theta) \notag
\end{gather}

Дальнейшие преобразования основаны на анализе фазового интеграла, представленном в работе \cite{vigasin2015}. Подготовим следующую линейную замену переменных $p_R, p_\theta, J_x, J_y, J_z \rightarrow x_1, x_2, x_3, x_4, x_5$ (причем $R, \theta$ считаем постоянными при осуществлении замены), упрощающую подэкспоненциальное выражение (впервые предложена в \cite{ozaki}):
\vverh
\begin{gather}
	\hspace*{-0.5cm}
	\everymath{\scriptstyle}
	\footnotesize
	\left\{
	\begin{aligned}
	x_1^2 &= \frac{p_R^2}{2 \mu_2 kT} \\
	x_2^2 &= \frac{p_\theta^2}{2 \mu_1 l^2 kT} \\
	x_3^2 &= \frac{ \lb p_\theta - J_y \rb^2}{2 \mu_2 R^2 k T} \\
	x_4^2 &= \frac{ \lb J_x + J_z \ctg \theta \rb^2}{2 \mu_2 R^2 k T} \\
	x_5^2 &= \frac{J_z^2}{2 \mu_1 l^2 \sin^2 \theta kT}
	\end{aligned}
	\right. \quad \implies \quad 
	\left\{
	\begin{aligned}
	dx_1 &= \frac{dp_R}{\sqrt{ 2 \mu_2 k T}} \\
	dx_2 &= \frac{dp_\theta}{\sqrt{2 \mu_1 l^2 k T}} \\
	dx_3 &= \frac{dp_\theta - dJ_y}{\sqrt{2 \mu_2 R^2 k T}} \\
	dx_4 &= \frac{dJ_x + \ctg \theta dJ_z}{ \sqrt{2 \mu_2 R^2 k T}} \\
	dx_5 &= \frac{dJ_z}{\sqrt{2 \mu_1 l^2 \sin^2 \theta k T}}
	\end{aligned}
	\right. \quad \implies \quad
	\left\{
	\begin{aligned}
		dp_R &= \sqrt{2 \mu_2 k T} dx_1 \\
		dp_\theta &= \sqrt{2 \mu_1 l^2 kT} dx_2 \\
		dJ_y &= \sqrt{2 \mu_1 l^2 kT} dx_2 - \sqrt{2 \mu_2 R^2 kT} dx_3 \\
		dJ_x &= \sqrt{2 \mu_2 R^2 kT} dx_4 - \sqrt{2 \mu_1 l^2 \cos^2 \theta kT} dx_5 \\
		dJ_z &= \sqrt{2 \mu_1 l^2 \sin^2 \theta kT} dx_5
	\end{aligned}
	\right.
	\notag
\end{gather}

Для упрощения выражений откинем на время трансляционную статистическую сумму димера и $4 \pi^2$. Основываясь на теореме Фубини, интеграл \eqref{ci1} может быть представлен в виде повторного интеграла, в котором сначала интегрирование ведется по переменным $p_R$, $p_\theta$, $J_x$, $J_y$, $J_z$, а затем  -- по переменным $R, \theta$. Таким образом, во внутреннем интеграле переменные $R, \theta$ являются постоянными, что позволяет осуществить приготовленную замену:
\vverh
\begin{gather}
	\frac{1}{h^5} \int_{H < 0} \exp \lb -\frac{\mH}{kT} \rb dR \, dp_R \, d \theta \, dp_\theta \, d J_x \, d J_y \, d J_z = \frac{1}{h^5} \iint dR \, d \theta \int \exp \lb -\frac{\mH}{kT} \rb d p_R \, d p_\theta \, d J_x \, d J_y \, d J_z = \notag \\
	= \frac{1}{h^5} \iint  \left[ Jac \, \right] \exp \lb -\frac{U}{kT} \rb d R \, d \theta \times \idotsint\limits_{x_1^2 + \dots + x_5^2 + \frac{U}{kT} < 0} \exp \lb - x_1^2 - \dots - x_5^2 \rb dx_1 \dots dx_5, \notag 
\end{gather}

\vlevo где якобиан замены переменных $\left[ Jac \, \right]$ равен следующему произведению радикалов: 
\begin{gather}
	\hspace*{-1.0cm}
	\left[ Jac \, \right] = \Bigg{|} \frac{\partial [p_R, p_\theta, J_x, J_y, J_z]}{\partial [x_1, x_2, x_3, x_4, x_5]} \Bigg{|} = \det  
	\begin{bmatrix}
		\sqrt{2 \mu_2 k T} & 0 & 0 & 0 & 0 \\
		0 & \sqrt{2 \mu_1 l^2 k T} & 0 & 0 & 0 \\
		0 & 0 & 0 & \sqrt{2 \mu_2 R^2 k T} & -\sqrt{2 \mu_1 l^2 \cos^2 \theta} \\
		0 & \sqrt{2 \mu_1 l^2 k T} & -\sqrt{2 \mu_2 R^2 k T} & 0 & 0 \\
		0 & 0 & 0 & 0 & \sqrt{2 \mu_1 l^2 \sin^2 \theta k T}
	\end{bmatrix} = \notag \\
	= \sqrt{2 \mu_2 k T} \sqrt{2 \mu_1 l^2 k T} \sqrt{2 \mu_2 R^2 kT} \sqrt{2 \mu_2 R^2 kT} \sqrt{2 \mu_1 l^2 \sin^2 \theta kT} = (2 \mu_2 kT)^\frac{3}{2} 2\mu_1 l^2 k T R^2 \sin \theta \notag
\end{gather}

Интеграл функции $\exp \lb - x_1^2  - x_2^2 - \dots - x_n^2 \rb$ по объему $n$-мерного шара с радиусом $R$ есть (доказательство этого соотношения приведено в приложении \eqref{appendix:multint})
\vverh
\begin{gather}
	\idotsint\limits_{x_1^2 + \dots + x_n^2 \leqslant R} \exp \lb -x_1^2 - x_2^2 - \dots - x_n^2 \rb d x_1 \dots d x_n = \pi^\frac{n}{2} \frac{\gamma \lb \frac{n}{2}, R^2 \rb}{\Gamma \lb \frac{n}{2} \rb}, \notag
\end{gather}
где $\gamma(a, b)$ -- неполная гамма-функция:
\vverh
\begin{gather}
	\gamma \lb a, b \rb = \int\limits_0^b \omega^{a - 1} \exp \lb - \omega \rb d \omega \notag
\end{gather}

Применяя общее соотношение для 5-мерного случая, получаем:
\vverh
\begin{gather}
	\idotsint\limits_{x_1^2 + \dots  + x_5^2 \leqslant -\frac{U}{k T}} \exp \lb - x_1^2 - \dots - x_5^2 \rb d x_1 \dots d x_5 = \pi^\frac{5}{2} \frac{\gamma \lb \frac{5}{2}, - \frac{U}{k T} \rb}{\Gamma \lb \frac{5}{2} \rb} \notag
\end{gather}

Итак, фазовый интеграл представлен в виде интеграла по переменным $R, \theta$ и интеграла по объему 5-мерного шара, радиус которого определяется значением потенциальной функции $U(R, \theta)$. Заметим, что радиус шара является ненулевым только при отрицательных значениях потенциальной функции $U(R, \theta)$, следовательно фазовый интеграл принимает ненулевые значения только в том случае, если переменные $R, \theta$ лежат в области $U < 0$.
\vverh
\begin{gather}
	\hspace*{-9cm} \frac{1}{h^5} \iint d R \, d \theta \int \exp \lb -\frac{\mH}{k T} \rb d p_R  \, dp_\theta \, dJ_x \, dJ_y \, dJ_z = \notag \\
	\hspace{5cm} =\lb \frac{2 \pi \mu_2 k T}{h^2} \rb^{\frac{3}{2}} \frac{2 \mu_1 l^2 \pi k T}{h^2} \iint\limits_{U < 0} \exp \lb -\frac{U}{k T} \rb \frac{\gamma \lb \frac{5}{2}, - \frac{U}{k T} \rb}{\Gamma \lb \frac{5}{2} \rb} R^2 \sin \theta dR \, d\theta \notag 
\end{gather}

Итак, выражение для статистической суммы свзяанного димера принимает вид:
\vverh
\begin{gather}
	Q_{bound}^{pair} = 4 \pi^2 \lb \frac{2 \pi M k T}{h^2} \rb^{\frac{3}{2}} V \lb \frac{2 \pi \mu_2 k T}{h^2} \rb^{\frac{3}{2}} \frac{2 \mu_1 l^2 \pi k T}{h^2} \iint\limits_{U < 0} \exp \lb -\frac{U}{k T} \rb \frac{\gamma \lb \frac{5}{2}, - \frac{U}{k T} \rb}{\Gamma \lb \frac{5}{2} \rb} R^2 \sin \theta dR \, d\theta \notag
\end{gather}
Заметим, что предынтегральные множители могут быть выражены через статистические суммы мономеров:
\vverh
\begin{gather}
	Q_{CO_2} = Q_{CO_2}^{tr} Q_{CO_2}^{rot} = \lb \frac{2 \pi m_{CO_2} k T}{h^2} \rb^{\frac{3}{2}} V \frac{4 \pi^2 k T}{h^2} \mu_1 l^2 \notag \\
	Q_{Ar} = Q_{Ar}^{tr} = \lb \frac{2 \pi m_{Ar} k T}{h^2} \rb^{\frac{3}{2}} \notag \\
	Q_{bound}^{pair} = 2 \pi Q_{Ar} Q_{CO_2} \iint\limits_{U < 0} \exp \lb - \frac{U}{kT} \rb \frac{\gamma \lb \frac{5}{2}, - \frac{U}{k T} \rb}{\Gamma \lb \frac{5}{2} \rb} R^2 \sin \theta d R \, d \theta \notag
\end{gather}

Подставляя полученное выражение для статистической суммы димера в \eqref{eqconst}, получаем
\vverh
\begin{gather}
	K_p = \frac{2 \pi N_A}{R T} \iint\limits_{U < 0} \exp \lb - \frac{U}{k T} \rb \frac{\gamma \lb \frac{5}{2}, - \frac{U}{kT} \rb}{\Gamma \lb \frac{5}{2} \rb} R^2 \sin \theta d R \, d \theta \label{eqconstsimple}
\end{gather}


