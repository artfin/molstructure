\documentclass[hyperref={pdfpagelabel=false},usepdftitle=false,xcolor=dvipsnames]{beamer}

\usepackage[T1]{fontenc}
\usepackage[utf8]{inputenc}
\usepackage[russian]{babel}

% page margin
\usepackage[top=2cm, bottom=2cm, left=2cm, right=2cm]{geometry}

% AMS packages
\usepackage{amsmath}
\usepackage{amssymb}
\usepackage{amsfonts}
\usepackage{amsthm}

% blackboard lettering
\usepackage{dsfont}
\usepackage{bbm}

\usepackage{fancyhdr}
\pagestyle{fancy}
% modifying page layout using fancyhdr
\fancyhf{}
\renewcommand{\sectionmark}[1]{\markright{\thesection\ #1}} % adding number to section name
\renewcommand{\subsectionmark}[1]{\markright{\thesubsection\ #1}} % adding number to subsection name

\rhead{\fancyplain{}{\rightmark }} % placing the section/subsection name in the right corner of the header
\cfoot{\fancyplain{}{\thepage}}  % placing a page number in the center of the footer   

% table packages
\usepackage{tabularx, ragged2e, booktabs, caption}
% includegraphics package
\usepackage{graphicx}

\usepackage{subfigure}

% commands
\newcommand{\lb}{\left(}
\newcommand{\rb}{\right)}
\newcommand{\mH}{\mathcal{H}}
\newcommand{\mL}{\mathcal{L}}
\newcommand{\mf}{\mathbf}

\newcommand{\bbB}{\mathbb{B}}
\newcommand{\bbG}{\mathbb{G}}
\newcommand{\bbS}{\mathbb{S}}
\newcommand{\bbI}{\mathbb{I}}
\newcommand{\bbA}{\mathbb{A}}
\newcommand{\bba}{\mathbbm{a}}

% positioning commands
\newcommand{\vpravo}{\hspace{0.63cm}}
\newcommand{\vverh}{\vspace*{-0.1cm}}

% table cell centering
\newcolumntype{C}[1]{>{\Centering}m{#1}}
\renewcommand\tabularxcolumn[1]{C{#1}}

\makeatletter
\providecommand\barcirc{\mathpalette\@barred\circ}%
\def\@barred#1#2{\ooalign{\hfil$#1-$\hfil\cr\hfil$#1#2$\hfil\cr}}%
\newcommand\stst{^{\protect\barcirc}}%
\makeatother

% tikz packages
\usepackage{tikz}
\usetikzlibrary{shapes.geometric, arrows, positioning, decorations.markings}
\usetikzlibrary{fit}
\usepackage{microtype}
\usepackage{framed}
\usetikzlibrary{decorations.pathmorphing,calc,backgrounds}

\usepackage[nottoc]{tocbibind}



\newcommand{\mL}{\mathcal{L}}
\newcommand{\mH}{\mathcal{H}}
\newcommand{\mf}{\mathbf}
\newcommand{\lb}{\left(}
\newcommand{\rb}{\right)}

\usepackage{dsfont}
\usepackage{bbm}
\newcommand{\bbI}{\mathds{I}}
\newcommand{\bba}{\mathbbm{a}}
\newcommand{\bbA}{\mathds{A}}
\newcommand{\bbB}{\mathds{B}}
\newcommand{\bbG}{\mathds{G}}

\usepackage{graphicx}

% custom block environment
\newenvironment<>{varblock}[2][.9\textwidth]{%
  \setlength{\textwidth}{#1}
  \begin{actionenv}#3%
    \def\insertblocktitle{#2}%
    \par%
    \usebeamertemplate{block begin}}
  {\par%
    \usebeamertemplate{block end}%
  \end{actionenv}}

\begin{document}

\begin{frame}
{
\vspace*{-0.75cm}
\large 
\begin{center}
Кафедра физической химии \\
Лаборатория строения и квантовой механики молекул
\vspace*{-0.75cm}
\end{center}}
 \titlepage
\end{frame}

\begin{frame}{\normalsize Классический колебательно-вращательный гамильтониан \rom{1}}
\begin{tikzpicture}[framed, font=\sffamily\small]

\node (lag1) [lagrange, text width = 2.5cm] {\scalebox{.8}{$T_\mL = \displaystyle\frac{1}{2}\sum_{i=1}^{n} m_i \dot{\mf{r}}_i^2$}};

\node (lag2) [lagrange, text width = 7cm, below = 1 cm of lag1] {\scalebox{.8}{$T_\mL = \displaystyle\frac{1}{2} \sum_{i=1}^{n} m_i \dot{\mf{R}}_i^2 + \mf{\Omega}^{\top} \sum_{i=1}^{n} m_i \left[ \mf{R}_i \times \dot{\mf{R}}_i \right] + \frac{1}{2} \mf{\Omega}^\top \bbI \, \mf{\Omega}$}};

\node (lag3) [lagrange, text width = 7 cm, below = 1.5 cm of lag2] {\scalebox{.8}{$
\begin{aligned}
	& \quad \quad T_\mL = \displaystyle\frac{1}{2} \dot{\mf{q}}^\top \bba \, \dot{\mf{q}} + \mf{\Omega}^\top \bbA \, \mf{\dot{q}} + \frac{1}{2} \mf{\Omega}^\top \bbI \, \mf{\Omega} \\
	\bba_{jk} &= \sum_{i=1}^{n} m_i \frac{\partial \mf{R}_i}{\partial q_j} \frac{\partial \mf{R}_i}{\partial q_k}, \quad \bbA_{jk} = \sum_{i=1}^{n} m_i \hspace*{-0.1cm} \left[ \mf{R}_i \times \frac{\partial \mf{R}_i}{\partial q_k} \right]_j 
\end{aligned}$
}};

\node (eq1) [equations, minimum width = 2cm, text width = 2cm, below right = 0.1 cm and 2 cm of lag1] {$\mf{R}_i = \mathbb{S} \mf{r}_i$};

\node (eq2) [equations, minimum width = 3cm, text width = 3cm, below right = 0.1 cm and -0.5 cm of lag2] {$
\begin{aligned}
	\mf{R}_i &= \mf{R}_i \lb \mf{q} \rb \\
	\mf{q} &= \lb q_1, \dots, q_n \rb
\end{aligned}$};

\draw [vecArrow] (lag1) -- node (text1) [anchor = east, text width = 5cm] {\vspace*{-0.5cm} \begin{center}Переход в подвижную систему отсчета \end{center}} (lag2);

\draw [vecArrow] (lag2) -- node (text2) [anchor = east, text width = 5cm] {\vspace*{-0.5cm} \begin{center}Переход к обобщенным координатам\end{center}} (lag3);

\draw [arrow] (lag1) -| (eq1);
\draw [arrow] (eq1) |- ([yshift=4pt]lag2);
\draw [arrow] ([yshift=-4pt]lag2) -| (eq2);
\draw [arrow] (eq2) |- ([yshift=4pt]lag3);

\begin{scope}[on background layer]
  \node [fill = BurntOrange!20,fit= (lag1) (lag2) (lag3) (text1) (text2) (eq1) (eq2)] {};
\end{scope}
\end{tikzpicture}
\end{frame}

\begin{frame}{\normalsize Классический колебательно-вращательный гамильтониан \rom{2}}
\begin{tikzpicture}[framed, font=\sffamily\small]

\node (lag1) [lagrange, text width = 6cm] {\scalebox{.8}{$
T_\mL = \displaystyle\frac{1}{2}
\begin{bmatrix}
	\mf{\Omega}^\top & \hspace*{-0.3cm} \dot{\mf{q}}^\top
\end{bmatrix}
\bbB
\begin{bmatrix}
	\mf{\Omega} \\ \dot{\mf{q}}
\end{bmatrix},
\quad 
\bbB =
\begin{bmatrix}
	\bbI & \hspace*{-0.2cm} \bbA \\
	\hspace*{0.15cm} \bbA^\top & \hspace*{-0.25cm} \bba
\end{bmatrix}
$}};

\node (eq1) [equations, text width = 5cm, below right = 0.3cm and -2cm of lag1] {\scalebox{.8}{$
\begin{aligned}
	\mf{J} &= \displaystyle\frac{\partial T_\mL}{\partial \mf{\Omega}} = \bbI \, \mf{\Omega} + \bbA \, \dot{\mf{q}} \\
	\mf{p} &= \frac{\partial T_\mL}{\partial \dot{\mf{q}}} = \bbA^\top \Omega + \bba \, \dot{\mf{q}}
\end{aligned}
\quad
\begin{bmatrix}
	\mf{J} \\
	\mf{p}
\end{bmatrix}
= \bbB
\begin{bmatrix}
	\mf{\Omega} \\
	\dot{\mf{q}}
\end{bmatrix}
$}};

\node (ham1) [hamilton, text width = 7.5cm, below = 2.2cm of lag1] {\scalebox{.8}{$
\begin{aligned}
&T_\mH = 
\begin{bmatrix}
	\mf{\Omega}^\top & \hspace*{-.25cm} \dot{\mf{q}}^\top
\end{bmatrix}
\begin{bmatrix}
	\mf{J} \\
	\mf{p}
\end{bmatrix}
- T_\mL = \displaystyle\frac{1}{2}
\begin{bmatrix}
	\mf{J}^\top & \hspace*{-.25cm} \mf{p}^\top
\end{bmatrix}
\bbG
\begin{bmatrix}
	\mf{J} \\
	\mf{p}
\end{bmatrix},
\quad
\bbG = \bbB^{-1} \\
& \quad \quad T_\mH = \frac{1}{2} \mf{J}^\top \bbG_{11} \, \mf{J} + \mf{J}^\top \bbG_{12} \, \mf{p} + \frac{1}{2} \mf{p}^\top \bbG_{22} \, \mf{p} \\
& \hspace*{2.5cm} \bbG_{11} = \lb \bbI - \bbA \bba^{-1} \bbA^\top \rb^{-1} \\
& \hspace*{2.5cm} \bbG_{12} = - \bbI^{-1} \bbA \, \bbG_{22} = - \bbG_{11} \bbA \bba^{-1} \\
& \hspace*{2.5cm} \bbG_{21} = - \bba^{-1} \bbA^\top \bbG_{11} = \bbG_{22} \, \bbA^\top \bbI^{-1} \\
& \hspace*{2.5cm} \bbG_{22} = \lb \bba - \bbA^\top \bbI^{-1} \bbA \rb^{-1}
\end{aligned}
$}};
	
\draw [vecArrow] (lag1) -- node (text1) [anchor = east, text width = 5cm] {\vspace*{-0.5cm} \begin{center} Применение \\ теоремы Донкина \end{center}} (ham1);

	\draw [arrow] (lag1) -| ([xshift=30pt]eq1);
	\draw [arrow] ([xshift=60pt]eq1) |- (ham1); 

\begin{scope}[on background layer]
	\node [fill = BurntOrange!20, fit = (lag1) (eq1) (ham1) (text1)] {};	
\end{scope}
\end{tikzpicture}
\end{frame}

\begin{frame}{\normalsize Точный классический колебательно-вращательный гамильтониан системы $Ar-CO_2$ \rom{1}}
\begin{figure}
\begin{tikzpicture}
	[oxygen/.style={ball color = red, circle, text = white}, 
	carbon/.style={ball color = black!30, circle, text = white},
	argon/.style={ball color = blue, circle, text = white}]
	\node (z1) at (4.5, 3) {OZ};
	\node (z2) at (-1.5, -1) {}
		edge [->, thick] (z1);

	\node (x1) at (-1, 3.5) {OX};
	\node (x2) at (3.5, -1) {}
		edge [->, thick] (x1); 

	\node (carbon) [carbon] {C};
	\node (oxygen1) [oxygen, below right of = carbon] {O}
		edge [double, thick] (carbon);
	\node (oxygen2) [oxygen, above left of = carbon] {O}
		edge [double, thick] (carbon);
	\node[argon] (argon) at (3, 2) {Ar};
\begin{scope}[on background layer]
	\node [fill=BurntOrange!20,fit=(z1) (z2) (x1) (x2) (carbon) (oxygen1) (oxygen2)] {};
\end{scope}
\end{tikzpicture}
\caption{\centering Молекулярная система координат для системы $Ar-CO2$}
\end{figure}
\end{frame} 

\begin{frame}{\normalsize Точный классический колебательно-вращательный гамильтониан системы $Ar-CO_2$ \rom{2}}

	Обозначим массы атомов: кислорода -- $m_1$, аргона -- $m_2$, углерода -- $m_3$, обозначим через $l$ расстояние $O-O$ в молекуле $CO_2$, расстояние от атома $C$ до атома $Ar$ -- через $R$, угол между вектором $C-Ar$ и $CO_2$ -- через $\theta$. Пара переменных $R, \theta$ образуют систему внутренних координат.  
	
\begin{block}{}
\begin{gather}
	\scalebox{.8}{$\hspace*{-2cm} \mH = \displaystyle\frac{1}{2 \mu_2} p_R^2 + \lb \frac{1}{2 \mu_2 R^2} + \frac{1}{2 \mu_1 l^2} \rb p_\theta^2 - \frac{1}{\mu_2 R^2} p_\theta J_y + \frac{1}{2 \mu_2 R^2} J_y^2 + \frac{1}{2 \mu_2 R^2} J_x^2 + $} \notag \\
	\scalebox{.8}{$\hspace*{5cm} \displaystyle + \frac{1}{2 \sin^2 \theta} \lb \frac{\cos^2 \theta}{\mu_2 R^2} + \frac{1}{\mu_1 l^2} \rb J_z^2 + \frac{\ctg \theta}{\mu_2 R^2} J_x J_z + U(R, \theta) $} \notag \\
	\scalebox{.8}{$ \displaystyle \mu_1 = \frac{m_1}{2}, \quad \mu_2 = \frac{m_2 \lb 2 m_1 + m_3 \rb}{2 m_1 + m_2 + m_3}$} \notag
\end{gather}
\end{block}
\end{frame}

\begin{frame}{\normalsize Поверхность потенциальной энергии межмолекулярного взаимодействия}
\vspace*{-0.30cm}
\begin{figure}
	\hspace*{-1cm}
\includegraphics[width = \linewidth]{../pictures/potential_well.png}
\vspace*{-0.1cm}
\caption{\centering \scriptsize Сечения поверхностей потенциальной энергии при разных углах $\theta$ в области потенциальной ямы}
\end{figure}
\end{frame}

%\begin{frame}{\normalsize Расчет температурной зависимости второго вириального коэффициента для системы $Ar-CO_2$ \rom{1}}
	%\scalebox{0.8}{
%$\displaystyle B_2 = - \frac{N_A}{2 V} \displaystyle\dfrac{\int_{\lb \tau_1 \rb}}{\int d \Omega_1}$} 
%\end{frame}

\begin{frame}{\normalsize Расчет температурной зависимости второго вириального коэффициента для системы $Ar-CO_2$ \rom{2}}
%\vspace*{-0.3cm}
%\begin{block}{}
%\begin{gather}
%\scalebox{0.7}{$B_2 = \displaystyle \pi N_0 \int_{0}^{\infty} \int_{0}^{\pi} \lb 1 - \exp \lb - \frac{U \lb R, \theta \rb}{k T} \rb \rb R^2 \sin \theta d R \, d \theta$} \notag
%\end{gather}
%\end{block}
\vspace*{-0.35cm}
\begin{figure}
\includegraphics[width=\linewidth]{../pictures/virexp.png}
\vspace*{-0.35cm}
\caption{\centering \scriptsize Температурные зависимости вириальных коэффициентов для разных ППЭ}
\end{figure}
\end{frame}

 \end{document}
