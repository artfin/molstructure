\usepackage[T1]{fontenc}
\usepackage[utf8]{inputenc}
\usepackage[russian]{babel}

\usepackage{amssymb, amsmath}

\usepackage{tikz}
\usetikzlibrary{shapes.geometric, arrows, positioning, decorations.markings}
\usetikzlibrary{fit}
\usepackage{microtype}
\usepackage{framed}
\usetikzlibrary{decorations.pathmorphing,calc,backgrounds}

\usetheme{CambridgeUS}
\useinnertheme{rectangles}
\useoutertheme{infolines}

\setbeamercolor{frametitle}{fg=Brown, bg = Brown!20}

\title{\textbf{Расчет константы равновесия слабосвязанного комплекса $Ar-CO_2$}}

\author{\small
\vspace*{-0.75cm}
\begin{flushright}
Выполнил: Финенко А. А. \\[1ex]
\underline{Научные руководители:} \\ [1ex]
к.ф.-м.н., с.н.с. Петров С. В. \\ 
м.н.с. Локштанов С. Е. \\
д.ф.-м.н., в.н.с. Вигасин А. А.
\end{flushright}
}

\pdfmapfile{+sansmathaccent.map}

\institute[MSU] % (optional, but mostly needed)
{\small МГУ им. М.В.Ломоносова, Химический факультет, 2017}
\date{}

\newcommand\Fontvi{\fontsize{6}{7.2}\selectfont}

\beamertemplatenavigationsymbolsempty

\setbeamerfont{page number in head/foot}{size=\large}
\setbeamertemplate{footline}[frame number]
\setbeamertemplate{frametitle}[default][center]

% change font
\usefonttheme[onlymath]{serif}

\makeatother
\setbeamertemplate{headline}
{}
\makeatletter

\makeatother
\setbeamertemplate{footline}
{
  \leavevmode%
  \hbox{%
  \begin{beamercolorbox}[wd=.7\paperwidth,ht=2.25ex,dp=1ex,center]{author in head/foot}%
    \usebeamerfont{author in head/foot} Расчет константы равновесия слабосвязанного комплекса $Ar-CO_2$
  \end{beamercolorbox}%
	\begin{beamercolorbox}[wd=.3\paperwidth,ht=2.25ex,dp=1ex,center]{title in head/foot}%
    \usebeamerfont{title in head/foot}Финенко А. \hspace*{3em}
    \insertframenumber{} / \inserttotalframenumber\hspace*{1ex}
  \end{beamercolorbox}}%
  \vskip0pt%
}
\makeatletter
\setbeamertemplate{navigation symbols}{}

% barcirc command
\makeatletter
\providecommand\barcirc{\mathpalette\@barred\circ}%
\def\@barred#1#2{\ooalign{\hfil$#1-$\hfil\cr\hfil$#1#2$\hfil\cr}}%
\newcommand\stst{^{\protect\barcirc}}%
\makeatother

% adding roman numerals
\makeatletter
\newcommand*{\rom}[1]{\expandafter\@slowromancap\romannumeral #1@}
\makeatother

\tikzstyle{lagrange} = [rectangle, rounded corners, minimum width = 3cm, minimum height = 1cm, text centered, text width = 9cm, draw = black, fill=DarkOrchid!40]

\tikzstyle{equations} = [rectangle, rounded corners, text centered, draw = black, fill=green!30]

\tikzstyle{hamilton} = [rectangle, rounded corners, minimum width = 3cm, minimum height = 1cm, text centered, text width = 5 cm, draw = black, fill = Goldenrod!50]

\tikzstyle{result} = [rectangle, rounded corners, text centered, draw = black, fill = blue!30]

\tikzstyle{arrow} = [thick, ->, >=stealth]

\tikzstyle{vecArrow} = [thick, decoration={markings,mark=at position
   1 with {\arrow[semithick]{open triangle 60}}},
   double distance=1.4pt, shorten >= 5.5pt,
   preaction = {decorate},
   postaction = {draw,line width=1.4pt, white,shorten >= 4.5pt}]



