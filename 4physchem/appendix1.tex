Исходя из соображений размерности объем $n$-мерного гипершара должен быть пропорционален $n$-ой степени радиуса $R$:
\vverh
\begin{gather}
	V_n (R) = \idotsint\limits_{x_1^2 + \dots x_n^2 \leqslant R^2} dx_1 \dots dx_n = C_n R^n \label{int1}
\end{gather}

Объем шара $V_n(R)$ может быть получен путем интегрирования площади сферических слоев $S_{n-1}(R)$ по радиусу шара:
\vverh
\begin{gather}
	V_n(R) = \int\limits_{0}^{R} S_{n-1}(r) dr \quad \implies \quad S_{n-1}(R) = \frac{d V_n(R)}{dR} = nC_n R^{n-1} \label{surf1} 
\end{gather}

В $n$-мерном пространстве могут быть введены гиперсферические координаты аналогично сферическим координатам в 3-мерном случае. Набор гиперсферических координат состоит из радиальной переменной $r$ и $n-1$ угловых координат $\phi_1,\phi_2, \dots ,\phi_{n-1}$, где углы $\phi_1, \phi_2, \dots, \phi_{n-2}$ принимают значения на отрезке $\left[ 0, \pi \right]$, а $\phi_{n-1}$ -- на $\left[ 0, 2 \pi \right]$. Декартовы координаты связаны с гиперсферическими следующими соотношениями:
\vverh
\begin{gather}
\begin{aligned}
x_1 &= r \cos \phi_1 \\
x_2 &= r \sin \phi_1 \cos \phi_2 \\
x_3 &= r \sin \phi_1 \sin \phi_2 \cos \phi_3 \\
&\dots \\
x_{n-1} &= r \sin \phi_1 \cdots \sin \phi_{n-2} \cos \phi_{n-1} \\
x_n &= r \sin \phi_1 \cdots \sin \phi_{n-2} \sin \phi_{n-1}
\end{aligned} \notag
\end{gather}

Элемент $n$-мерного объема в гиперсферических координатах может быть выражен через якобиан замены переменных:
\vverh
\begin{gather}
	d^n V = \Bigg{|} \frac{\partial \left[ x_1, x_2, \dots, x_n \right]}{\partial \left[ r, \phi_1, \dots, \phi_n \right]} \Bigg{|} d r \, d \phi_1 \, \dots d \phi_{n-1} = r^{n-1} \sin^{n-2} \phi_1 \sin^{n-3} \phi_2 \dots \sin \phi_{n-2} \, d r \, d \phi_1, \dots d \phi_{n-1} = \notag \\
= r^{n-1} dr \, d \Omega_{n-1}, \notag
\end{gather} 
где за $d \Omega_{n-1}$ обозначено произведение дифференциалов угловых переменных. \par

Осуществим гиперсферическую замену переменных в \eqref{int1}:
\vverh
\begin{gather}
	\idotsint\limits_{x_1^2 + \dots x_n^2 \leqslant R^2} dx_1 \dots dx_n = \idotsint d \Omega_{n-1} \int_{0}^{R} r^{n-1} dr \notag
\end{gather}
Учитывая \eqref{surf1}, очевидно:
\vverh
\begin{gather}
	\idotsint d \Omega_{n-1} = n C_n \notag
\end{gather}

Для того, чтобы получить численное выражение для $C_n$ рассмотрим интеграл функции $f(x_1, \dots x_n) = \exp \lb -x_1^2 - \dots x_n^2 \rb$ по объему $n$-мерного пространства. Осуществим гиперсферическую замену переменных: 
\vverh
\begin{gather}
	\int_{-\infty}^{\infty} \dots \int_{-\infty}^{\infty} \exp \lb -x_1^2 - \dots -x_n^2 \rb dx_1 \dots dx_n = \int_{0}^{\infty} \exp \lb -r^2 \rb r^{n-1} dr \int d \Omega_{n-1}  = \notag \\
	= n C_n \int_{0}^{R} r^{n-1} \exp \lb -r^2 \rb dr = \frac{1}{2} \Gamma \lb \frac{n}{2} \rb n C_n \notag
\end{gather}

Одновременно с этим, многомерный интеграл представим в виде произведения одномерных интегралов Пуассона:
\vverh
\begin{gather}
	\int_{-\infty}^{\infty} \dots \int_{-\infty}^{\infty} \exp \lb -x_1^2 - \dots - x_n^2 \rb d x_1 \dots dx_n = \left[ \int_{-\infty}^{\infty} \exp \lb -x_1^2 \rb dx_1 \right]^n  = \pi^\frac{n}{2} \notag 
\end{gather}

Получаем следующее выражение для $C_n$ 
\vverh
\begin{gather}
	C_n = \frac{\pi^\frac{n}{2}}{\frac{n}{2} \Gamma \lb \frac{n}{2} \rb} \notag
\end{gather}

Теперь рассмотрим интеграл функции $f(x_1, \dots x_n) = \exp \lb -x_1^2 - \dots x_n^2 \rb$ по объему $n$-мерной сферы:  
\begin{gather}
	\idotsint\limits_{x_1^2 + \dots + x_n^2 \leqslant R^2} \exp \lb -x_1^2 - \dots - x_n^2 \rb dx_1 \dots dx_n = \int_{0}^{R} r^{n-1} \exp \lb -r^2 \rb dr \idotsint d \Omega_{n-1} = \notag \\
	= \frac{2 \pi^{\frac{n}{2}}}{\Gamma \lb \frac{n}{2} \rb} \int_{0}^{R} r^{n-1} \exp \lb -r^2 \rb dr = \left[ t = r^2 \right] = \frac{\pi^{\frac{n}{2}}}{\Gamma \lb \frac{n}{2} \rb} \int_{0}^{R^2} t^{\frac{n}{2} - 1} \exp \lb -t \rb dt = \frac{\pi^{\frac{n}{2}}}{\Gamma \lb \frac{n}{2} \rb} \gamma \lb \frac{n}{2}, R^2 \rb, \notag
\end{gather}
где $\gamma$ -- неполная гамма-функция:
\vverh
\begin{gather}
	\gamma \lb a, b \rb = \int\limits_{0}^{b} \omega^{a - 1} \exp \lb - \omega \rb d \omega \notag
\end{gather}



