\section{Колебательно-вращательный гамильтониан системы $Ar-CO_2$}

Введем молекулярную систему отсчета: направим ось $OZ$ вдоль линии $C-Ar$, ось $OX$ через центр масс перпендикулярно ей в плоскости системы, ось $OY$ перпендикулярно плоскости системы. Обозначим массы атомов: кислорода -- через $m_1$, аргона -- через $m_2$, углерода -- через $m_3$; обозначим через $l$ расстояние $O-O$ в молекуле $CO_2$. Обозначим расстояние от атома $C$ до атома $Ar$ через $R$, угол между вектором $C-Ar$ и $CO_2$ -- через $\theta$. Пара переменных $R, \theta$ образуют систему внутренних координат (вектор $\mf{q}$ при формировании матриц, определяющих кинетическую энергию в лагранжевом и гамильтоновом представлениях, используется в форме $\begin{bmatrix} R & \theta \end{bmatrix}$). Внутримолекулярные колебания $CO_2$ происходят существенно быстрее молекулярных движений рассматриваемой системы $Ar-CO_2$ и в среднем не взаимодействуют с ними. Поэтому будем считать молекулу $CO_2$ жесткой в рамках дальнейшего рассмотрения.  

\begin{figure}[h]
\centering
\begin{tikzpicture}
	[oxygen/.style={ball color = red, circle, text = white}, 
	carbon/.style={ball color = black!30, circle, text = white},
	argon/.style={ball color = blue, circle, text = white}]
	\node (z1) at (4.5, 3) {OZ};
	\node (z2) at (-1.5, -1) {}
		edge [->, thick] (z1);

	\node (x1) at (-1, 3.5) {OX};
	\node (x2) at (3.5, -1) {}
		edge [->, thick] (x1); 

	\node (carbon) [carbon] {C};
	\node (oxygen1) [oxygen, below right of = carbon] {O}
		edge [double, thick] (carbon);
	\node (oxygen2) [oxygen, above left of = carbon] {O}
		edge [double, thick] (carbon);
	\node[argon] (argon) at (3, 2) {Ar};

	\begin{scope}[on background layer]
	\node [fill=yellow!30,fit=(z1) (z2) (x1) (x2) (carbon) (oxygen1) (oxygen2)] {};
\end{scope}
\end{tikzpicture}
\caption{Молекулярная система отсчета для системы $Ar-CO_2$.}
\end{figure}

Введем следующие обозначения:
\vverh
\begin{gather}
	\mu_1 = \frac{m_1}{2}, \quad 
	\mu_2 = \frac{m_2 \lb 2 m_1 + m_3 \rb}{2 m_1 + m_2 + m_3} \notag
\end{gather}

Матрицы $\bba$, $\bbA$, $\bbI$, определяющие вид кинетической энергии в лагранжевой форме: 
\vverh
\begin{gather}
	\bba =
	\begin{bmatrix}
		\mu_2 & 0 \\
		0 & \mu_1 l^2
	\end{bmatrix} \quad 
	\bbA = 
	\begin{bmatrix}
		0 & 0 \\
		0 & \mu_1 l^2 \\
		0 & 0 
	\end{bmatrix} \quad
	\bbI = 
	\begin{bmatrix}
		\mu_1 l^2 \cos^2 \theta + \mu_2 R^2 & 0 & -\mu_1 l^2 \sin \theta \cos \theta \\
		0 & \mu_1 l^2 + \mu_2 R^2 & 0 \\
		- \mu_1 l^2 \sin \theta \cos \theta & 0 & \mu_1 l^2 \sin^2 \theta
	\end{bmatrix} \notag
\end{gather}

Используя формулы Фробениуса, получаем матрицы $\bbG_{11}$, $\bbG_{12}$, $\bbG_{22}$, определяющие кинетическую энергию в гамильтоновой форме:
\vverh
\begin{gather}
	\bbG_{11} =
	\begin{bmatrix}
		\dfrac{1}{\mu_2 R^2} & 0 & \dfrac{\ctg \theta}{\mu_2 R^2} \\
		0 & \dfrac{1}{\mu_2 R^2} & 0 \\
		\dfrac{\ctg \theta}{\mu_2 R^2} & 0 & \dfrac{\ctg^2 \theta}{\mu_2 R^2} + \dfrac{1}{\mu_1 l^2 \sin^2 \theta}
	\end{bmatrix} \quad
	\bbG_{12} =
	\begin{bmatrix}
		0 & 0 \\
		0 & - \dfrac{1}{\mu_2 R^2} \\
		0 & 0
	\end{bmatrix} \quad 
	\bbG_{22} = 
	\begin{bmatrix}
		\dfrac{1}{\mu_2} & 0 \\
		0 & \dfrac{1}{\mu_2 R^2} + \dfrac{1}{\mu_1 l^2}
	\end{bmatrix} \notag
\end{gather}

Получаем гамильтониан системы $Ar-CO_2$ в заданной молекулярной системе координат:
\begin{gather}
\mH = \frac{1}{2 \mu_2} p_R^2 + \lb \frac{1}{2 \mu_2 R^2} + \frac{1}{2 \mu_1 l^2} \rb p_\theta^2 - \frac{1}{\mu_2 R^2} p_\theta J_y + \frac{1}{2 \mu_2 R^2} J_y^2 + \frac{1}{2 \mu_2 R^2} J_x^2 + \frac{1}{2 \sin^2 \theta} \lb \frac{\cos^2 \theta}{\mu_2 R^2} + \frac{1}{\mu_1 l^2} \rb J_z^2 + \notag \\
+ \frac{\ctg \theta}{\mu_2 R^2} J_x J_z + U(R, \theta) \notag
\end{gather}



