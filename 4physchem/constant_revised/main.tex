\documentclass[12pt]{article}

\usepackage[T1]{fontenc}
\usepackage[utf8]{inputenc}
\usepackage[russian]{babel}

% page margin
\usepackage[top=2cm, bottom=2cm, left=2cm, right=2cm]{geometry}

% AMS packages
\usepackage{amsmath}
\usepackage{amssymb}
\usepackage{amsfonts}
\usepackage{amsthm}

\usepackage{graphicx}

\usepackage{fancyhdr}
\pagestyle{fancy}
% modifying page layout using fancyhdr
\fancyhf{}
\renewcommand{\sectionmark}[1]{\markright{\thesection\ #1}}
\renewcommand{\subsectionmark}[1]{\markright{\thesubsection\ #1}}

\rhead{\fancyplain{}{\rightmark }}
\cfoot{\fancyplain{}{\thepage }}

\usepackage{titlesec}
\titleformat{\section}{\bfseries}{\thesection.}{1em}{}
\titleformat{\subsection}{\normalfont\itshape\bfseries}{\thesubsection.}{0.5em}{}

\newcommand{\mf}{\mathbf}

\newcommand{\lb}{\left(}
\newcommand{\rb}{\right)}

\newcommand{\mH}{\mathcal{H}}

\newcommand{\intl}{\int\limits}
\newcommand{\idotsintl}{\idotsint\limits}

\begin{document}

\section*{Константа равновесия K$_p$(CO$_2$-Ar)}

Классическая сумма по состояниям связанного димера представляет собой следующий фазовый интеграл
\begin{gather}
	Q_{bound}^{pair} = \frac{1}{h^{5}} \lb \frac{2 \pi M k T}{h^2} \rb^{3/2} V \intl_{\mH < 0} \exp \lb - \frac{\mH}{k T} \rb d q_i \, d p_i, \label{full1} 
\end{gather}

где $q_i, p_i$ -- набор внутримолекулярных координат и импульсов, $\mH$ -- гамильтониан в молекулярной системе отсчета. \par

Интегрирование в $\eqref{full1}$ в том числе производится по эйлеровым углам и импульсам. Т.к. подынтегральное выражение не зависит от эйлеровых углов, то интегрирование по ним сводится к умножению на величину отрезка интегрирования. Осуществим замену от эйлеровых импульсов к компонентам углового момента. 
\begin{gather}
	\intl_{0}^{2 \pi} d \varphi \intl_{0}^{\pi} d \theta \intl_{0}^{2 \pi} d \psi \int d p_\varphi \int d p_\theta \int d p_\psi = 8 \pi^2 \int d J_x \int d J_y \int d J_z \notag
\end{gather}

\begin{gather}
		Q_{bound}^{pair} = \frac{8 \pi^2}{h^5} \lb \frac{2 \pi M k T}{h^2} \rb^{3/2} V \intl_{\mH < 0} \exp \lb - \frac{\mH}{k T} \rb d R \, d \Theta \, d p_R \, d p_\Theta \, d J_x \, d J_y \, d J_z \label{full2}
\end{gather}

Гамильтониан CO$2-$Ar может быть разложен на сумму квадратов следующим образом: 
\begin{gather}
	\mH = \frac{p_R^2}{2 \mu_2} + \frac{p_\theta^2}{2 \mu_1 l^2} + \frac{1}{2 \mu_2 R^2} \lb p_\theta - J_y \rb^2 + \frac{1}{2 \mu_2 R^2} \lb J_x + J_z \ctg \theta \rb^2 + \frac{J_z^2}{2 \mu_1 l^2 \sin^2 \theta} + U(R, \theta) \notag
\end{gather}

Рассмотрим замену переменных, приводящую кинетическую энергию в гамильтониане к сумме квадратов:
\begin{gather}
		\mH = x_1^2 + x_2^2 + x_3^2 + x_4^2 + x_5^2 + U(R, \Theta) \notag
\end{gather}

\vspace*{-0.3cm}
\begin{gather}
	\hspace*{-0.5cm}
	\everymath{\scriptstyle}
	\footnotesize
	\left\{
	\begin{aligned}
	x_1^2 &= \frac{p_R^2}{2 \mu_2 kT} \\
	x_2^2 &= \frac{p_\theta^2}{2 \mu_1 l^2 kT} \\
	x_3^2 &= \frac{ \lb p_\theta - J_y \rb^2}{2 \mu_2 R^2 k T} \\
	x_4^2 &= \frac{ \lb J_x + J_z \ctg \theta \rb^2}{2 \mu_2 R^2 k T} \\
	x_5^2 &= \frac{J_z^2}{2 \mu_1 l^2 \sin^2 \theta kT}
	\end{aligned}
	\right. \quad \implies \quad 
	\left\{
	\begin{aligned}
	dx_1 &= \frac{dp_R}{\sqrt{ 2 \mu_2 k T}} \\
	dx_2 &= \frac{dp_\theta}{\sqrt{2 \mu_1 l^2 k T}} \\
	dx_3 &= \frac{dp_\theta - dJ_y}{\sqrt{2 \mu_2 R^2 k T}} \\
	dx_4 &= \frac{dJ_x + \ctg \theta dJ_z}{ \sqrt{2 \mu_2 R^2 k T}} \\
	dx_5 &= \frac{dJ_z}{\sqrt{2 \mu_1 l^2 \sin^2 \theta k T}}
	\end{aligned}
	\right. \quad \implies \quad
	\left\{
	\begin{aligned}
		dp_R &= \sqrt{2 \mu_2 k T} dx_1 \\
		dp_\theta &= \sqrt{2 \mu_1 l^2 kT} dx_2 \\
		dJ_y &= \sqrt{2 \mu_1 l^2 kT} dx_2 - \sqrt{2 \mu_2 R^2 kT} dx_3 \\
		dJ_x &= \sqrt{2 \mu_2 R^2 kT} dx_4 - \sqrt{2 \mu_1 l^2 \cos^2 \theta kT} dx_5 \\
		dJ_z &= \sqrt{2 \mu_1 l^2 \sin^2 \theta kT} dx_5
	\end{aligned}
	\right.
	\notag
\end{gather}

\begin{gather}
	\hspace*{-1.0cm}
	\left[ Jac \, \right] = \Bigg{|} \frac{\partial [p_R, p_\theta, J_x, J_y, J_z]}{\partial [x_1, x_2, x_3, x_4, x_5]} \Bigg{|} = \det  
	\begin{bmatrix}
		\sqrt{2 \mu_2 k T} & 0 & 0 & 0 & 0 \\
		0 & \sqrt{2 \mu_1 l^2 k T} & 0 & 0 & 0 \\
		0 & 0 & 0 & \sqrt{2 \mu_2 R^2 k T} & -\sqrt{2 \mu_1 l^2 \cos^2 \theta} \\
		0 & \sqrt{2 \mu_1 l^2 k T} & -\sqrt{2 \mu_2 R^2 k T} & 0 & 0 \\
		0 & 0 & 0 & 0 & \sqrt{2 \mu_1 l^2 \sin^2 \theta k T}
	\end{bmatrix} = \notag \\
	= \sqrt{2 \mu_2 k T} \sqrt{2 \mu_1 l^2 k T} \sqrt{2 \mu_2 R^2 kT} \sqrt{2 \mu_2 R^2 kT} \sqrt{2 \mu_1 l^2 \sin^2 \theta kT} = (2 \mu_2 kT)^\frac{3}{2} 2\mu_1 l^2 k T R^2 \sin \theta \notag
\end{gather}

С учётом теоремы Фубини интеграл \eqref{full2} может быть представлен в виде повторного интеграла, в котором сначала интегрирование ведется по переменным $p_R$, $p_\theta$, $J_x$, $J_y$, $J_z$, а затем  -- по переменным $R, \theta$. Таким образом, во внутреннем интеграле переменные $R, \theta$ являются постоянными, что позволяет осуществить приготовленную замену:

\begin{gather}
		Q_{bound}^{pair} = \frac{8 \pi^2}{h^5} \lb \frac{2 \pi M k T}{h^2} \rb^{3/2} V \intl_{\mH < 0} \exp \lb -\frac{\mH}{kT} \rb dR \, dp_R \, d \Theta \, dp_\Theta \, d J_x \, d J_y \, d J_z = \notag \\
		= \frac{8 \pi^2}{h^5} \lb \frac{2 \pi M k T}{h^2} \rb^{3/2} V \iint dR \, d \Theta \int \exp \lb -\frac{\mH}{kT} \rb d p_R \, d p_\Theta \, d J_x \, d J_y \, d J_z = \notag \\
		= \frac{8 \pi^2}{h^5} \lb \frac{2 \pi M k T}{h^2} \rb^{3/2} V \iint  \left[ Jac \, \right] \exp \lb -\frac{U}{kT} \rb d R \, d \Theta \times \idotsint\limits_{x_1^2 + \dots + x_5^2 + \frac{U}{kT} < 0} \exp \lb - x_1^2 - \dots - x_5^2 \rb dx_1 \dots dx_5. \label{full3}
\end{gather}

Интеграл функции $\exp \lb - x_1^2  - x_2^2 - \dots - x_n^2 \rb$ по объему $n$-мерного шара с радиусом $R$ равен 
\begin{gather}
	\idotsint\limits_{x_1^2 + \dots + x_n^2 \leqslant R} \exp \lb -x_1^2 - x_2^2 - \dots - x_n^2 \rb d x_1 \dots d x_n = \pi^{n/2} \frac{\gamma \lb \displaystyle \frac{n}{2}, R^2 \rb}{\Gamma \lb \displaystyle \frac{n}{2} \rb}, \notag
\end{gather}

Подставляя выражение якобиана и интеграл по объему $5$-мерного шара в $\eqref{full3}$, получаем:
\begin{gather}
		Q_{bound}^{pair} = \frac{8 \pi^2}{h^5} \lb \frac{2 \pi M k T}{h^2} \rb^{3/2} V \lb 2 \mu_2 k T \rb^{3/2} 2 \mu_1 l^2 k T \pi^{5/2} \iint\limits_{U < 0} \exp \lb - \frac{U}{k T} \rb \frac{\gamma \lb \displaystyle \frac{5}{2}, -U/kT \rb}{\Gamma \lb \displaystyle \frac{5}{2} \rb} R^2 \sin \Theta \, d R \, d \Theta \notag
\end{gather}

Отдельно рассмотрим множитель $C$ перед интегралом, перераспределим $\pi$ между множителями следующим образом:
\begin{gather}
	C = \frac{8 \pi}{h^5} \lb \frac{2 \pi M k T}{h^2} \rb^{3/2} V \lb 2 \pi \mu_2 k T \rb^{3/2} \lb 2 \pi^2 \mu_1 l^2 k T \rb \notag
\end{gather}

Распределим $h^5$ между вторым и третьим множителями, кроме того из $8 \pi$ сделаем $4 \pi$, перенеся множитель $2$ в третью скобку:
\begin{gather}
	C = 4 \pi \lb \frac{2 \pi M k T}{h^2} \rb^{3/2} V \lb \frac{2 \pi \mu_2 k T}{h^2} \rb^{3/2} \lb \frac{4 \pi^2 k T}{h^2} \mu_1 l^2 \rb \notag
\end{gather}

Заметим, что $\mu_2 = \displaystyle \frac{1}{M} m_{Ar} m_{CO_2}$, следовательно произведение второй и третьей скобки даст части трансляционных сумм Ar и CO$_2$:
\begin{gather}
		C = 4 \pi \lb \frac{2 \pi m_{Ar} k T}{h^2} \rb^{3/2} V \lb \frac{2 \pi m_{CO_2} k T}{h^2} \rb^{3/2} \lb \frac{4 \pi^2 k T}{h^2} \mu_1 l^2 \rb \notag
\end{gather}

Классическая вращательная сумма по состояниям для молекулы CO$_2$ равна  
\begin{gather}
	Q_{rot} = \frac{8 \pi^2 I k T}{\sigma h^2} = \frac{4 \pi^2 I k T}{h^2} = \frac{4 \pi^2 k T}{h^2} \mu_1 l^2 \notag
\end{gather}

Итак, коэффициент $C$ перед интегралом следующим образом связан с классическими суммами по состояниям мономеров:
\begin{gather}
	C = \frac{4 \pi}{V} Q_{tr}^{Ar} Q_{tr}^{CO_2} Q_{rot}^{CO_2} = \frac{4 \pi}{V} Q^{Ar} Q^{CO_2} \notag \\
	Q_{bound}^{pair} = \frac{4 \pi}{V} Q^{Ar} Q^{CO_2} \iint\limits_{U < 0} \exp \lb - \frac{U}{k T} \rb \frac{\gamma \lb \displaystyle \frac{5}{2}, -U/kT \rb}{\Gamma \lb \displaystyle \frac{5}{2} \rb} R^2 \sin \Theta \, d R \, d \Theta \notag 
\end{gather}

Подставляя полученную сумму по состояниям для связанных состояний в выражение для константы равновесия, получаем:
\begin{gather}
		K_p = \frac{N_0}{p} \frac{Q_{bound}^{pair}}{Q^{Ar} Q^{CO_2}} = \frac{4 \pi N_0}{p V} \iint\limits_{U < 0} \exp \lb - \frac{U}{k T} \rb \frac{\gamma \lb \displaystyle \frac{5}{2}, -U/kT \rb}{\Gamma \lb \displaystyle \frac{5}{2} \rb} R^2 \sin \Theta \, d R \, d \Theta = \notag \\
		= \frac{4 \pi N_0}{R T} \iint\limits_{U < 0} \exp \lb - \frac{U}{k T} \rb \frac{\gamma \lb \displaystyle \frac{5}{2}, -U/kT \rb}{\Gamma \lb \displaystyle \frac{5}{2} \rb} R^2 \sin \Theta \, d R \, d \Theta \notag  
\end{gather}

\newpage

\section*{Константа равновесия N$_2-$N$_2$}

Классическая сумма по состояниям связанного димера двух палочек имеет следующий вид
\begin{gather}
		Q_{bound}^{pair} = \frac{8 \pi^2}{s h^7} \lb \frac{2 \pi M k T}{h^2} \rb^{3/2} V \int_{\mH < 0} \exp \lb - \frac{\mH}{kT} \rb dR \, d \Theta_1 d \Theta_2 \, d \varphi \, d p_R \, d p_{\Theta_1} d p_{\Theta_2} d p_\varphi \, d J_x \, d J_y \, d J_z, \notag 
\end{gather}
где $s$ -- число симметрии димера, $M$ -- совокупная масса димера. \\
Применяя теорему Фубини, разделяем кратный интеграл на два, где первый -- по координатам, а второй -- по импульсам: 
\begin{gather}
	Q_{bound}^{pair} = \frac{8 \pi^2}{s h^7} \lb \frac{2 \pi M k T}{h^2} \rb^{3/2} V \idotsint dR \, d \Theta_1 d \Theta_2 \, d \varphi \idotsint \exp \lb - \frac{\mH}{kT} \rb d p_R \, d p_{\Theta_1} d p_{\Theta_2} d p_\varphi \, d J_x \, d J_y \, d J_z \notag  
\end{gather}

Переходим в импульсном интеграле к переменным $x_1, x_2 \ldots x_7$. Якобиан этого перехода, как было показано в другом документе, равен:
\begin{gather}
		\left[ \, Jac \, \right]_{ham} = 2^{7/2} \lb k T \rb^{7/2} \mu_1 l_1^2 \, \mu_2 l_2^2 \, \mu_3^{3/2} R^2 \sin \Theta_1 \sin \Theta_2 \notag \\
	Q_{bound}^{pair} = \frac{8 \pi^2}{s h^7} \lb \frac{2 \pi M k T}{h^2} \rb^{3/2} V \idotsint\limits_{U < 0} \left[ \, Jac \, \right]_{ham} \exp \lb -\frac{U}{kT} \rb d R \, d \Theta_1 d \Theta_2 \, d \varphi \notag \\
	\hspace{8cm} \times \idotsint\limits_{x_1^2 + \ldots + x_7^2 + U/kT < 0} \exp \lb - x_1^2 - \ldots -x_7^2 \rb d x_1 \, d x_2 \ldots d x_7  = \notag \\
	= \frac{8 \pi^2}{s h^7} \lb \frac{2 \pi M k T}{h^2} \rb^{3/2} V \idotsint\limits_{U < 0} \left[ \, Jac \, \right]_{ham} \frac{\gamma \lb 7/2, - U/kT \rb}{\Gamma \lb 7/2 \rb} \exp \lb - \frac{U}{kT} \rb dR \, d \Theta_1 d \Theta_2 \, d \varphi \notag 
\end{gather}

Рассмотрим отдельно коэффициент $C$ перед интегралом:
\begin{gather}
		C = \frac{8 \pi^2}{s} \lb \frac{2 \pi M k T}{h^2} \rb^{3/2} \lb \frac{2 \pi \mu_3 k T}{h^2} \rb^{3/2} \frac{2 \pi \mu_1 l_1^2 k T}{h^2} \frac{2 \pi \mu_2 l_2^2 k T}{h^2} V \notag
\end{gather}

Приведенная масса $\mu_3$ равна произведению масс палочек, деленная на их сумму, поэтому $\mu_3 \cdot M = m_{mon_1} \cdot m_{mon_2}$, где $m_{mon_1}, m_{mon_2}$ -- массы отдельных палочек. Следовательно, произведение первых двух скобок даст скобки, фигурирующие в трансляционных статсуммах мономеров. Затем распределим $4 \pi^2$ (из первой дроби) по $2 \pi$ между третьей и четвертой дробями.
\begin{gather}
		C = \frac{2}{s} \lb \frac{2 \pi m_{mon_1} k T}{h^2} \rb^{3/2} \lb \frac{2 \pi m_{mon_2} k T}{h^2} \rb^{3/2} \frac{4 \pi \mu_1 l_1^2 k T}{h^2} \frac{4 \pi \mu_2 l_2^2 k T}{h^2} V = \frac{2}{s} \frac{Q_{mon_1} Q_{mon_2}}{V} \notag
\end{gather}

Итак, получаем следующие выражения для статсуммы связанных димеров и константы равновесия 
\begin{gather}
		Q_{bound}^{pair} = \frac{2}{s} \frac{Q_{mon_1} Q_{mon_2}}{V} \idotsint\limits_{U < 0} \frac{\gamma \lb 7/2, - U/kT \rb}{\Gamma \lb 7/2 \rb} \exp \lb - \frac{U}{kT} \rb d R \, d \Theta_1 d \Theta_2 \, d \varphi, \notag \\     
		K_p = \frac{2}{s} \frac{N_A}{R T} \idotsint\limits_{U < 0} \frac{\gamma \lb 7/2, - U/kT \rb}{\Gamma \lb 7/2 \rb} \exp \lb - \frac{U}{kT} \rb d R \, d \Theta_1 d \Theta_2 \, d \varphi. \notag 
\end{gather}

В случае N$_2-$N$_2$ число симметрии димера $s$ равно 8, т.к. в фазовом пространстве существует по 2 эквивалентных области, отвечающих вращениям каждого из мономеров, а также 2 эквивалентных области, отвечающих повороту, меняющему местами мономеры местами (итого: $s = 2 \times 2 \times 2 = 8$).  

\begin{gather}
		K_p (N_2-N_2) = \frac{N_A}{4 R T} \idotsint\limits_{U < 0} \frac{\gamma \lb 7/2, - U/kT \rb}{\Gamma \lb 7/2 \rb} \exp \lb - \frac{U}{kT} \rb d R \, d \Theta_1 d \Theta_2 \, d \varphi. \notag 
\end{gather}


\end{document}
