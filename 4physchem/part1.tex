\section{Введение}
Развитие квантовой химии в последние десятилетия позволяет проводить \textit{ab-initio} расчеты по построению надежных поверхностей потенциальной энергии (ППЭ) и дипольного момента (ПДМ) для слабосвязанных молекулярных пар. Подобные квантовохимические данные позволяют с использованием методов статистической термодинамики оценить термодинамические характеристики слабосвязанных комплексов, такие как второй вириальный коэффициент, который часто используется для оценки достоверности построенной ППЭ. Второй вириальный коэффициент проявляет высокую чувствительность к различным элементам ППЭ, что и обуславливает его использование в качестве тестовой характеристики. \par
 Помимо исследования влияний межмолекулярных сил на отклонения от идеальности в поведении газов, существенную ценность имеет рассмотрение эффекта ассоциации молекулярных пар. В спектроскопии для оценки влияния вклада слабосвязанных комплексов необходимо иметь данные об их количестве. Это обуславливает интерес к расчету термодинамических констант равновесия для таких систем. Традиционные методы расчета констант равновесия в рамках статистической термодинамики, основанные на приближении ``жесткий волчок - гармонический осциллятор``, могут быть применены лишь с большой осторожностью \cite{camyvigasin}. \par
В работе \cite{vigasin2015} был предложен систематический подход к рассмотрению равновесия мономер-димер в рамках классической механики. В рамках предложенного подхода используются определенные допущения при классическом анализе сумм по состояниям, что делает результаты приближенными. Однако следует иметь в виду, что любой подход, основанный на классическом рассмотрении константы равновесия, несет приближенный характер. \par
В рамках настоящей работы рассматривается слабосвязанная молекулярная пара $Ar-CO_2$. Для анализа термодинамических характеристик используется ППЭ, рассчитанная Ю. Н. Калугиной \cite{kalugina2017}. На основе метода, предложенного в \cite{vigasin2015}, рассматривается константа равновесия системы $Ar-CO_2$. Кроме аналитического пути, мы развиваем вычислительную схему, основанную на адаптивном методе Монте-Карло \cite{lepage1978, vegas}, для расчета константы равновесия интегрированием по фазовому пространству системы. 
