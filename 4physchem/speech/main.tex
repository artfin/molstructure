\documentclass[12pt]{article}

% page margin
\usepackage[top=2cm, bottom=2cm, left=2cm, right=2cm]{geometry}

\usepackage[T1]{fontenc}
\usepackage[utf8]{inputenc}
\usepackage[russian]{babel}

\usepackage{fancyhdr}
\pagestyle{fancy}

\begin{document}
Развитие квантовохимических расчетов в последние десятилетия позволяет проводить \textit{ab initio} расчеты по построению надежных поверхностей потенциальной энергии и поверхностей дипольного момента. Подобные данные с использованием методов статистической термодинамики позволяют оценить термодинамические характериски слабосвязанных комплексов, имеющих ценность в атмосферных исследованиях. К таким системам и относится рассматриваемая молекулярная пара $Ar-CO_2$. Интерес к константе равновесия обусловлен возможностью оценки вклада слабосвязанных молекулярных пар в спектральных исследованиях. \par
О поверхностям ППЭ: 2 поверхности взяты из литературы, PSP -- первая представленная в литературе ППЭ для этой системы, обе в существенной степени опираются на экспериментальные данные (спектроскопические данные высокого разрешения и значения второго вириального коэффициента). \textit{ab initio} была рассчитана Ю.Калугиной на уровне CCSD(T) -- т.возмущений. Аналитическая форма потенциалов реализована разложением по полиномам Лежандра. \par 
Температурная зависимость второго вириального коэффициента часто используется для оценки достоверности построенной ППЭ, т.к. он проявляет высокую чувствительность к различным его элементам. (в качестве тестовой характеристики) Следует отметить тот факт, что в значение вириального коэффициента вносят далекие значения потенциала (до 40 бор). $f_{12}$ -- функция Майера, $\Omega_i$ -- набор внутримолекулярных угловых координат.  Расчет двумерного интеграла производился при помощи адпативного метода Монте-Карло, реализованного в рамках пакета vegas на языке \textit{Cython} с питоновским интерфейсом. \par
График: \textit{ab initio} наилучшим образом описывает экспериментальные данные. \par
В работах Кирквуда и Вигнера был разработан подход к сведению квантовой суммы по состояниям к фазовому интегралу, который может быть разложен в ряд по степеням приведенной постоянной Планка, причем первым членом этого ряда является классический интеграл Гиббса. Было получено аналогичное разложение для второго вириального коэффициента, где первым членом стал классический вириальный коэффициент. Строго разложение Вигнера-Кирквуда применимо к взаимодействию линейных молекул. \par
Вверху представлено выражение, связывающее газовую константу равновесия с суммами по состояниям димера и мономеров. При этом сумма по состояниям димера представляет собой следующий интеграл по области фазового пространства. $H$ -- гамильтониан в лабораторной системе координат. Существенным фактом для дальнейших ввыкладок, стал тот факт, что гамильтониан представляет собой положительно определенную квадратичную форму (точнее говоря кин.энергия). Откинем трансл. статсумму, рассмотрим интеграл по преобразованному фазовому пространству. Осуществляем следующую замену переменных, максимально упрощающую вид подэкспоненциального выражения. $U < 0$ --> радиус сферы $>0$. Здесь особенно эффективен метод Монте-Карло, т.к. интегрирование осуществляется по области $U<0$.  \par
В рамках следуещего этапа работы мы воспользовались мощностью вычислительного адаптивного метода МК, разработанного для вычисления интегралов высокой размерности. Константа равновесия была вычислена по общему выражению, здесь осуществлено интегрирование по 7-мерному преобразованному фазовому пространству. Наконец, последней стадией работы, стало вычисление константы по этому же выражению, но без использования аналитического вида гамильтониана. Метод МК не накладывает ограничений на вид подынтегрального выражения, поэтому вместо аналитического гамильтониана была использована схема расчета, начинающаяся с матриц лагранжевой формы. 
\end{document}

