\documentclass[12pt]{article}

\usepackage[T3]{fontenc}
\usepackage[utf8]{inputenc}
\usepackage[russian]{babel}

\usepackage[top=2cm, bottom=2cm, left=2cm, right=2cm]{geometry}

\usepackage{amsmath}
\usepackage{amsfonts}

\usepackage{mathbbol} % for lower case blackboard letters

\usepackage{fancyhdr}
\pagestyle{fancy}

\usepackage{amsthm}
\newtheorem{mytheorem}{Теорема}[]

\newcommand{\bbA}{\mathbb{A}}
\newcommand{\bba}{\mathbb{a}}
\newcommand{\bbb}{\mathbb{b}}
\newcommand{\bbI}{\mathbb{I}}

\newcommand{\vlevo}{\hspace*{-0.63cm}}
\newcommand{\vverh}{\vspace*{-0.1cm}}

\newcommand{\lb}{\left(}
\newcommand{\rb}{\right)}

\begin{document}
\section{Введение}

\section{Схема получения полного колебательно-вращательного гамильтониана}

\hspace{0.35cm} Рассмотрим систему $n$ материальных точек. Обозначим их массы через $m_i$, их радиус-векторы в лабораторной системе координат через $\vec{r}_i$, в подвижной системе координат -- через $\vec{R}_i$ ($i = 1 \dots n$). Разделим движение системы на движение центра масс и движение вокруг центра масс:
\vspace*{-0.1cm}
\begin{gather}
\left\{
\begin{aligned}
\vec{r}_1 &= \vec{r} + \vec{r}_1^{\ \prime}, \\
&\cdots \\
\vec{r}_n &= \vec{r} + \vec{r}_n^{\ \prime},
\end{aligned}
\right. \notag
\end{gather}

\hspace*{-0.75cm} где $\vec{r}$ -- радиус-вектор центра масс в лабораторной системе координат и $\vec{r}_i^{\ \prime}$ -- радиус-векторы рассматриваемых точек в системе отсчёта, связанной с центром масс.

Кинетическая энергия $T$ системы принимает вид: 
\vspace*{-0.1cm}
\begin{gather}
T = \frac{1}{2} \sum_{i=1}^{n} m_i \dot{\vec{r}}_i^{\ 2} = \frac{1}{2} \sum_{i=1}^{n} m_i (\dot{\vec{r}} + \dot{\vec{r}}_i^{\ \prime})^2  = \frac{1}{2} M \dot{r}^2 + \frac{1}{2} \sum_{i=1}^{n} m_i ( \dot{r}_i^\prime)^2 + \dot{\vec{r}} \sum_{i=1}^{n} m_i \dot{\vec{r}}_i^{\ \prime}, \notag
\end{gather}

\hspace*{-0.75cm} где $M = \sum_{i=1}^{n} m_i$.

Заметим, что последняя сумма является производной следующей суммы, которая равна нулю: 
\vspace*{-0.1cm}
\begin{gather}
\sum_{i=1}^{n} m_i \dot{\vec{r}}_i^{\ \prime} = \frac{d}{dt} \sum_{i=1}^{n} m_i \vec{r}_i^{\ \prime} = 0. \notag
\end{gather}

Итак, мы перешли в систему координат, связанную с центром масс, и отделили энергию движения центра масс:
\vspace*{-0.1cm}
\begin{gather}
T = \frac{1}{2} M \dot{r}^2 + \frac{1}{2} \sum_{i=1}^{n} m_i( \dot{r}_i^\prime)^2. \notag
\end{gather}

Забудем про слагаемое, отвечающее центру масс; откинем штрихи, чтобы упростить запись.
Перейдём в подвижную систему координат при помощи ортогональной матрицы $\mathbb{S}$:
\vspace*{-0.1cm}
\begin{gather}
\vec{r}_i = \mathbb{S} \vec{R}_i, \quad i = 1 \dots n. \notag
\end{gather}

Введем матрицу $\mathbb{A}$ следующим образом: $\mathbb{A} = \dot{\mathbb{S}} \mathbb{S}^{-1}$. Покажем, что она является кососимметрической матрицей; для этого продифференцируем единичную матрицу:
\vspace*{-0.1cm}
\begin{gather}
\frac{d}{dt} \mathbb{E} = \frac{d}{dt} \left( \mathbb{S} \mathbb{S}^{-1}\right) = \dot{\mathbb{S}} \mathbb{S}^{-1} + \mathbb{S} \dot{\mathbb{S}}^{-1} = 0. \notag
\end{gather}

Заметим, что первое слагаемое и есть матрица $\mathbb{A}$, а второе -- транспонированная матрица $\mathbb{A}$ (т.к. $\mathbb{S}^\top = \mathbb{S}^{-1}$ в силу ортогональности). Следовательно,
\vspace*{-0.1cm}
\begin{gather}
\mathbb{A} + \mathbb{A}^\top = 0, \notag
\end{gather}

\hspace*{-0.75cm} т.е. по определению матрица $\mathbb{A}$ является кососимметрической.

Так как размерность пространства кососимметрических матриц равна 3, то существует естественный изоморфизм, позволяющий сопоставить каждой кососимметрической матрице единственный псевдовектор:
\vspace*{-0.1cm}
\begin{gather}
\mathbb{A} = 
\begin{pmatrix}
0 & -\omega_3 & \omega_2 \\
\omega_3 & 0 & -\omega_1 \\
-\omega_2 & \omega_1 & 0
\end{pmatrix}
\quad
\longleftrightarrow
\quad
\vec{\omega} = 
\begin{pmatrix}
\omega_1 \\
\omega_2 \\
\omega_3
\end{pmatrix}
,
\notag
\end{gather}

\hspace*{-0.75cm} причем для любого вектора $\vec{x} \in \mathbf{R}^3$ имеем $\mathbb{A} \vec{x} = [ \vec{\omega} \times \vec{x} ]$, где $\vec{\omega}$ -- вектор угловой скорости в лабораторной системе координат.

Получим выражение для квадратов скоростей рассматриваемых точек в лабораторной системе координат через координаты и скорости в подвижной системе координат:
\vspace*{-0.1cm}
\begin{gather}
\dot{\vec{r}}_i = \mathbb{S} \dot{\vec{R}}_i + \dot{\mathbb{S}} \vec{R}_i = \dot{\mathbb{S}} \mathbb{S}^{-1} \vec{r}_i + \mathbb{S} \dot{\vec{R}}_i  = \mathbb{A} \vec{r}_i + \mathbb{S} \dot{\vec{R}}_i = [ \vec{\omega} \times \vec{r}_i ] + \mathbb{S} \dot{\vec{R}}_i = [\mathbb{S} \vec{\Omega} \times \mathbb{S} \vec{R}_i ] + \mathbb{S} \dot{\vec{R}}_i = \notag \\
= \mathbb{S} \left( [\vec{\Omega} \times \vec{R}_i ] + \dot{\vec{R}}_i \right)  \notag , \\
\dot{r}_i^2 = \dot{\vec{r}}_i^{\ \top} \dot{\vec{r}}_i = \left( \dot{\vec{R}}_i + [ \vec{\Omega} \times \vec{R}_i] \right)^\top \mathbb{S}^\top \mathbb{S} \left( \dot{\vec{R}}_i + [ \vec{\Omega} \times \vec{R}_i ] \right) = \dot{R}_i^2 + 2 \dot{\vec{R}}_i \ [ \vec{\Omega} \times \vec{R}_i] + [ \vec{\Omega} \times \vec{R}_i ]^2 \notag ,
\end{gather}

\hspace*{-0.75cm} где $\vec{\Omega}$ -- вектор угловой скорости в подвижной системе координат.

Рассмотрим последнее слагаемое как смешанное произведение и применим правило Лагранжа:
\vspace*{-0.1cm}
\begin{gather}
([\vec{\Omega} \times \vec{R}_i] , [\vec{\Omega} \times \vec{R}_i]) = \vec{\Omega}^\top [ \vec{R}_i \times [ \vec{\Omega} \times \vec{R}_i ]] = \vec{\Omega}^\top \left( \vec{\Omega} (\vec{R}_i, \vec{R}_i ) - \vec{R}_i (\vec{R}_i, \vec{\Omega} ) \right)
\notag .
\end{gather}

Итак, с учётом выполненных преобразований имеем:
\vspace*{-0.1cm}
\begin{gather}
T = \frac{1}{2} \sum_{i=1}^{n} m_i \dot{r}_i^2 = \frac{1}{2} \sum_{i=1}^{n} m_i \dot{R}_i^2 + \vec{\Omega} ^\top \sum_{i=1}^{n} m_i[ \vec{R}_i \times \dot{\vec{R}}_i ] + \frac{1}{2} \vec{\Omega}^\top \sum_{i=1}^{n} m_i \left( \vec{\Omega} (\vec{R}_i, \vec{R}_i ) - \vec{R}_i (\vec{R}_i, \vec{\Omega} ) \right) = 
\notag \\
= \frac{1}{2} \sum_{i=1}^{n} m_i \dot{R}_i^2 + \vec{\Omega}^\top \sum_{i=1}^{n} m_i [ \vec{R}_i \times \dot{\vec{R}}_i ] + \vec{\Omega}^\top \mathbb{I} \ \vec{\Omega} \notag .
\end{gather}

\vlevo где $\mathbb{I}$ -- матрица тензора инерции в подвижной системе координат.

Пусть исследуемая система содержит $s$ внутренних степеней свободы. Осуществим переход от векторов в подвижной системе к внутренним координатам $q_j, j=1 \dots s$:
\vverh
\begin{gather}
\left\{
\begin{aligned}
\vec{R}_1 &= \vec{R}_1 (q_1, \dots, q_s), \\
&\cdots \\
\vec{R}_n &= \vec{R}_n (q_1, \dots, q_s);
\end{aligned}
\right. \notag \\
\frac{d}{dt} \vec{R}_i = \sum_{j=1}^{s} \frac{\partial \vec{R}_i}{\partial q_j} \ \dot{q}_j \notag .
\end{gather}

Подставляя $\dot{\vec{R}}_i$ в выражение для кинетической энергии, получим:
\vverh
\begin{gather}
T = \frac{1}{2} \sum_{i=1}^{n} m_i \sum_{j=1}^{s} \frac{\partial \vec{R}_i}{\partial q_j} \dot{q}_j \sum_{k=1}^{s} \frac{\partial \vec{R}_i}{\partial q_k} \dot{q}_k + \vec{\Omega}^\top \sum_{i=1}^{n} m_i \left[ \vec{R}_i \times \sum_{j=1}^{s} \frac{\partial \vec{R}_i}{\partial q_j} \ \dot{q}_j \right] + \vec{\Omega}^\top \bbI \ \vec{\Omega} = \notag \\
= \frac{1}{2} \sum_{j=1}^{s} \sum_{k=1}^{s} \left( \sum_{i=1}^{n} m_i \frac{\partial \vec{R}_i}{\partial q_j} \frac{\partial \vec{R}_i}{\partial q_k} \right) \dot{q}_j \dot{q}_k + \vec{\Omega}^\top \sum_{j=1}^{s} \left( \sum_{i=1}^{n} m_i \left[ \vec{R}_i \times \frac{\partial \vec{R}_i}{\partial q_j} \right] \right) \dot{q}_j + \frac{1}{2} \vec{\Omega}^\top \bbI \ \vec{\Omega} \notag .
\end{gather}

Обозначая $a_{jk} = \sum_{i=1}^{n} m_i \frac{\partial \vec{R}_i}{\partial q_j} \frac{\partial \vec{R}_i}{\partial q_k}$, $A_{jk} = \sum_{i=1}^{n} m_i \left[ \vec{R}_i \times \frac{\partial \vec{R}_i}{\partial q_k} \right]_{\alpha}$ (здесь $\alpha = x,y,z$ соответствуют $j=1,2,3$), представим кинетиическую энергию в виде:
\vverh
\begin{gather}
T = \frac{1}{2} \dot{\vec{q}}^{\ \top} \mathbb{a} \ \dot{\vec{q}} + \vec{\Omega}^\top \mathbb{A} \ \dot{\vec{q}} + \frac{1}{2} \vec{\Omega}^\top \mathbb{I} \ \vec{\Omega} \notag ,
\end{gather}

\vlevo где $\bba = (a_{jk})_{j=1 \dots s, \ k=1 \dots s}$, $\bbA = (A_{jk})_{j=1 \dots 3, \ k=1 \dots s}$.

Воспользуемся следующей теоремой:

\begin{mytheorem}
Преобразование $\vec{y} = \frac{\partial X(\vec{x}, \vec{\alpha})}{\partial \vec{x}}$, определяемое производящей функцией $X = X(\vec{x}, \vec{\alpha})$, гессиан которой отличен от 0 ($\det \bigl( \frac{\partial ^2 X}{\partial x_i \partial x_j} \bigr) \neq 0$), имеет обратное преобразование $\vec{x} = \frac{\partial Y(\vec{y}, \vec{\alpha})}{\partial \vec{y}}$, определяемое производящей функцией $Y(\vec{y}, \vec{\alpha})$, гессиан которой также отличен от 0 ($\det \bigl( \frac{\partial ^2 Y}{\partial y_i \partial y_j} \bigr) \neq 0$). Производящая функция $Y(\vec{y}, \vec{\alpha})$ определяется следующим образом: $Y(\vec{y}, \vec{\alpha}) = \vec{x} \cdot \vec{y} - X(\vec{x}(\vec{y}, \vec{\alpha}), \vec{\alpha})$, причем $\frac{\partial X}{\partial \vec{\alpha}} + \frac{\partial Y}{\partial \vec{\alpha}} = 0$.
\end{mytheorem}

\end{document}