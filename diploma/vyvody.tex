\begin{enumerate}
    \item Развита методика расчета столкновительно-индуцированных спектров из первых принципов методом классических траекторий с применением обобщенных внутренних координат и подвижной системы осей. Ключевым элементом, обеспечивающим масштабируемость методики при рассмотрении многоатомных систем, является расчет гамильтониана и его производных \enquote{на лету} через матрицы лагранжиана и их производные (пункт \ref{section:dynamic-equations}).
    \item Выполнен \textit{ab initio} расчет столкновительно-индуцированных спектров системы N$_2-$N$_2$ в температурном диапазоне от 90К до 300К  и спектров системы CO$_2-$Ar в температурном диапазоне от 230К до 350К. Сравнение с экспериментальными данными, доступными при при отдельных температурах, подтверждает работоспособность развиваемой расчетной методики и ее применимость для моделирования континуальных столкновительно-индуцированных спектров.
    \item Развиваемая методика позволяет рассчитать спектры столкновительно-индуци\-рован\-ного поглощения с небольшим шагом по температуре для использования в базах данных по поглощению значимыми атмосферными молекулами. Однако определение границы применимости метода для получения надежных результатов требует дальнейшего анализа и накопления расчетных результатов для б\'{о}льшего числа молекулярных пар.  
\end{enumerate}
