\documentclass[10pt,pdf,hyperref={unicode},xcolor=dvipsnames]{beamer}

\usepackage{float}
\usepackage{amsmath}
\usepackage{amsfonts}
\usepackage{bbm}

% русский текст в формулах?
\usepackage{mathtext}

% русский
\usepackage[T2A]{fontenc}
\usepackage[english]{babel}
\usepackage[utf8]{inputenc}

% рисунки
\usepackage{graphicx, caption, subcaption}

\usetheme{CambridgeUS}

\usepackage{csquotes}
\usepackage[backend=bibtex]{biblatex}
\bibliography{biblio}

\usepackage{physics}
\usepackage{braket}
 
\usepackage{tabularx}
\usepackage{multirow}
\usepackage{booktabs}
\usepackage{array}

% ?
\setbeamertemplate{frametitle}[default][center]

\addtobeamertemplate{navigation symbols}{}{%
\usebeamerfont{footline}%
\usebeamercolor[fg]{footline}%
\hspace{1em}%
\large\insertframenumber/\inserttotalframenumber
}

\newcommand{\mytitle}[1]{\color{blue}{\textbf{#1}}}

\newcommand{\lb}{\left(}
\newcommand{\rb}{\right)}
\newcommand{\lsq}{\left[}
\newcommand{\rsq}{\right]}
\newcommand{\mean}[1]{\langle #1 \rangle}
\newcommand{\bOmega}{\boldsymbol{\Omega}}
\newcommand{\bbS}{\mathds{S}}
\newcommand{\bbM}{\mathds{M}}
\newcommand{\Jx}{J_\text{X}}
\newcommand{\Jy}{J_\text{Y}}
\newcommand{\Jz}{J_\text{Z}}
\newcommand{\pe}{\mathbf{p}_e}
\newcommand{\bUpsilon}{\boldsymbol{\Upsilon}_e}

\newcommand{\mf}{\mathbf}
\newcommand{\EOmega}{\boldsymbol{\Upsilon}_\mathbf{e}}
\newcommand{\bmu}{\boldsymbol{\mu}}

\usepackage{tikz}
\usetikzlibrary{shapes.geometric, arrows, positioning, decorations.markings, arrows.meta, snakes}
\usetikzlibrary{fit}
\usepackage{microtype}
\usepackage{framed}
\usetikzlibrary{decorations.pathmorphing,calc,backgrounds}
\usepackage{pgfplots}
\usepackage{animate}
\usepackage{xcolor}

\usepackage{varwidth}

\usepackage{tabularx}
\usepackage{booktabs}
\usepackage{siunitx}
\sisetup{output-exponent-marker=\ensuremath{\mathrm{e}}}

\tikzset{
	pics/carc/.style args={#1:#2:#3}{
		code = {
			\draw[pic actions] (#1:#3) arc(#1:#2:#3);
		}
	}
}

\tikzstyle{blackBoxStyle} = [rectangle, draw=black!90, fill=black!50, thick, inner sep=5pt, minimum size=1.5cm]
\tikzstyle{derivativeBoxStyle} = [rectangle, draw=BurntOrange!90, fill=BurntOrange!30, thick, inner sep = 5pt, minimum size = 1cm]

\usepackage{array}
\makeatletter
\renewcommand*\env@matrix[1][\arraystretch]{%
  \edef\arraystretch{#1}%
  \hskip -\arraycolsep
  \let\@ifnextchar\new@ifnextchar
  \array{*\c@MaxMatrixCols c}}
\makeatother

\pgfplotsset{compat=1.14}

\newcommand*{\Scale}[2][4]{\scalebox{#1}{$#2$}}%
\newcommand*{\rttensor}[1]{\underline{\underline{#1}}}
\newcommand*{\rttensortwo}[1]{\bar{\bar{#1}}}

\tikzset{math/.style={draw, execute at begin node={$\displaystyle}, execute at end node={$}}}

\newcommand{\boxandcomment}[4][]{%
    \tikz[baseline=(#2.base), remember picture]{%
        \node[math, label=below:{#3}, #1] (#2) {#4};}}


\usepackage{dsfont}
\usepackage{bbm}

\newcommand{\bba}{\mathbbm{a}}
\newcommand{\bbA}{\mathds{A}}
\newcommand{\bbI}{\mathds{I}}
\newcommand{\bbG}{\mathds{G}}
\newcommand{\bbW}{\mathds{W}}

%\usepackage{algorithm, algorithmic}
%\usepackage[noend]{algpseudocode}
%\usepackage{float}
\usepackage{algorithm,algpseudocode}

\usepackage{csquotes}
\usepackage[backend=bibtex]{biblatex}
\bibliography{biblio}

\newcommand{\hcancel}[1]{%
    \tikz[baseline=(tocancel.base)]{
        \node[inner sep=0pt,outer sep=0pt] (tocancel) {#1};
        \draw[red] (tocancel.south west) -- (tocancel.north east);
        \draw[red] (tocancel.south east) -- (tocancel.north west);
    }%
}%

\newcommand{\hcancelonce}[1]{%
    \tikz[baseline=(tocancel.base)]{
        \node[inner sep=0pt,outer sep=0pt] (tocancel) {#1};
        \draw[red] (tocancel.south west) -- (tocancel.north east);
    }%
}%

\definecolor{darkolivegreen}{rgb}{0.33, 0.42, 0.18}
\definecolor{darkpastelgreen}{rgb}{0.01, 0.75, 0.24}

\begin{document}

\begin{frame}{\center\mytitle{\Large Моделирование спектров столкновительно-индуцированного поглощения в дальней ИК области методом классических траекторий}}
\begin{table}[]
\flushright
\begin{tabular}{r}
\large \underline{Докладчик:}
\large Финенко Артем \\[1ex]
\large \underline{Научные руководители:} \\[1ex]
\large Петров С. В. \\
\large Локштанов С. Е. \\
\end{tabular}
\end{table}
\vfill
\center
\today
\end{frame}

\begin{frame}{Столкновительно-индуцированное поглощение}
    \begin{block}{Вращательный переход запрещен в мономере}
        \vspace*{-0.5cm}
        \begin{gather}
            \hcancel{$N_2(j_A) + h \nu \rightarrow N_2(j_B)$} \notag
        \end{gather}
    \end{block}
    \begin{block}{Переход в столкновительном комплексе}
        \vspace*{-0.5cm}
        \begin{gather}
            \left\{ N_2 + N_2\right\}(J) + h \nu \rightarrow \left\{ N_2 + N_2 \right\}(J^\prime) \notag
        \end{gather}
    \end{block}
    \begin{block}{Состояния молекулярных пар}
        \begin{enumerate}
            \item \color{red}{Связанные состояния}
            \item \color{darkpastelgreen}{Континуальные свободные состояния} 
            \item \color{darkpastelgreen}{Метастабильные состояния} 
        \end{enumerate}
    \end{block}
\end{frame}

\begin{frame}{Существующие подходы}
\end{frame}

\begin{frame}{Схема расчета}
    \begin{block}{Предварительная работа}
        \begin{enumerate}
            \item Аналитические аппроксимации \textit{ab initio} ППЭ и ПДМ 
        \end{enumerate}
    \end{block}
    \begin{block}{Компоненты расчета}
        \begin{enumerate}
        \item Распределение начальных условий
        \item Интегрирование уравнений движения -- получение столкновительных траекторий
        \item Преобразование Фурье функции дипольного момента вдоль каждой стокновительной траектории
        \item Расчет классической спектральной функции как среднее по ансамблю траекторий рассеяния 
        \item Десимметризация спектральной функции и расчет бинарного коэффициента поглощения
        \end{enumerate}
    \end{block}
\end{frame}

\end{document}
