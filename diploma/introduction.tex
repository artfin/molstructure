Homonuclear diatomic molecules, which includes several astrophysically and atmo-
spherically relevant species, have zero electric dipole moment due to their inversion
symmetry. This means that these molecules are in principle infrared inactive. How-
ever, gases of apolar molecules exhibit collision-induced absorption, as was first shown
experimentally for a gas of oxygen molecules.[10] This absorption originates from col-
lision complexes present in the gas. The collision complex can be considered to be a
short-lived species, which may have a lower symmetry than the individual molecules
and hence an electric dipole moment. Since these collision complexes are short-lived,
the resulting absorption spectra are typically broad in frequency.

\iffalse
Поглощение света описывается законом Бугера-Ламберта-Бера 
\begin{gather}
    I = I_0 \exp \lb - \alpha l \rb. 
\end{gather}
По закону Бера коэффицициент поглощения $\alpha$ связан линейно с плотностью поглощающего газа $n$
\begin{gather}
    \alpha = \sigma n,
\end{gather}
где через $\sigma$ обозначено сечение поглощения. Зависимость предполагает, что поглощение происходит индивидуальными молекулами. В общем случае, для описания этой зависимости может быть использовано вириальное разложение
\begin{gather}
    \alpha = \sigma n + \alpha_\text{binary} n^2 + \alpha_\text{ternary} n^3 + O(n^4), 
\end{gather}
в котором квадратичное слагаемое описывает столкновительно-индуцированное поглощение бинарными комплексами, кубическое слагаемое -- тройными комплексами, и т.д. Стоит отметить, что сечение поглощения мономера $\sigma$ зависит от давления (проявляется это, например, в сдвигах и уширениях линий), поэтому при каждой фиксированной частоте поглощение мономера не является линейным по плотности $n$, однако интегральная интенсивность линейна по плотности. \par
Таким образом, для моделирования суммарного профиля поглощения необходимо знать сечение поглощения $\sigma$, коэффициент бинарного поглощения $\alpha_\text{binary}$, и т.д., и как эти эти величины зависят от атмосферных характеристик, таких как температура. Данная работа посвящена расчет коэффициентов бинарного поглощения $\alpha_\text{binary}$ для различных систем, имеющих практическую значимость в атмосферных приложениях. \par
Спектры плотных газов и газовых смесей существенно отличаются от спектров, зарегистрированных при низких плотностях. По мере увеличения плотности, наблюдается линейное увеличение интенсивностей разрешенных колебательно-вращательных и электронных полос. При промежуточных значениях плотности могут возникнуть новые полосы поглощения, которые не наблюдались при более низких плотностях, причем интенсивность этих полос (по крайней мере в первом приближении) может быть описана квадратичным или кубическим законом. За появление этих полос отвественны ван дер Ваальсовы комплексы двух или большего количества молекул. Подобные полосы поглощения найдены во многих молекулярных газах, даже молекулы которых не обладают постоянным дипольным моментом.  

Состояния ван дер Ваальсовых комплексов классифицируют на связанные, полная колебательно-вращательная энергия меньше энергии диссоциации, и свободные. 

Впервые квадратичную зависимость коэффициента поглощения от плотности наблюдал Йенсен (Janssen) в 1885 году на полосах поглощениях кислорода \cite{janssen1885}.
\fi
