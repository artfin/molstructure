Начало активному изучению явления столкновительно-индуцированного поглощения (СИП) в газовых смесях положили исследования, относящиеся к середине прошлого века. Авторы \cite{crawford1949} наблюдали в спектрах азота при высоких давлениях полосы поглощения в области фундаментального перехода. Появление запрещенной полосы в спектрах поглощения азота -- одно из проявлений межмолекулярных взаимодействий, ответственных за индуцирование дипольного момента. Наблюдаемые спектры поглощения вызваны столкновительными комплексами, присутствующими в газе. Столкновительные комплексы обладают малыми временами жизни и низкой симметрией, а следовательно, и дипольным моментом. Вследствие малых времен жизни, вызванные взаимодействиями спектры протяженны по частоте и в широком диапазоне условий имеют континуальный характер. \par 
Столкновительно-индуцированное поглощение в инфракрасной области вносит существенный вклад в суммарное поглощение планетных атмосфер. Газы, состоящие из молекул, не обладающих постоянным дипольным моментом, поглощают излучение при достаточно высоких плотностях. Поглощение комплексами N$_2-$N$_2$ имеет существенное значение для радиационного баланса земной атмосферы, вследствие чего оно рутинно учитывается при проведении наземных и спутниковых наблюдений \cite{sioris2014}. Кроме того, данные о столкновительно-индуцированном поглощении комплексов с участием N$_2$ востребованы при моделировании атмосферы Титана, а данные о комплексах с участием CO$_2$ -- экзопланет с вулканической активностью. \par
Современные теоретические исследования, направленные на моделирование и описание столкновительно-индуцированного поглощения используют методы как квантовой, так и классической механики. При помощи квантово-механического подхода были смоделированы спектры поглощения столкновительных комплексов гетероатомных систем \cite{sharma1975, meyer1986}. Долгое время эти расчеты были направлены на извлечение параметров межмолекулярного взаимодействия, прежде чем точность квантово-химических данных стала достаточно высокой. Моделирование спектров многоатомных систем с применением квантово-механического подхода является чрезвычайно сложной вычислительной задачей. Многие квантово-механические расчеты опираются на серьезные допущения, такие как приближение изотропного потенциала, которые неприменимы ко многим представляющим интерес системам. До последнего времени выполнение расчетов с использованием меньших приближений было невозможным, однако развитие комьютерной техники и вычислительных методов позволило осуществить их для набора систем, включая N$_2-$N$_2$ \cite{karman2015}. \par
Вычислительные трудности, связанные с квантовыми расчетами, вынуждают исследователей обратиться к классическим и полуклассическим методам. Одним из методов моделирования спектральных профилей СИП, основанных на примении классического формализма, является метод классических траекторий. В недавней работе \cite{oparin2017} продемонстрирована эффективность траекторного подхода к моделированию спектра СИП на примере системы CO$_2-$Ar. \par
В этой работе приведены результаты применения разработанной траекторной методики расчета спектров СИП с использованием подвижной системы координат для описания классической динамики столкновения. 

\iffalse
Homonuclear diatomic molecules, which includes several astrophysically and atmo-
spherically relevant species, have zero electric dipole moment due to their inversion
symmetry. This means that these molecules are in principle infrared inactive. How-
ever, gases of apolar molecules exhibit collision-induced absorption, as was first shown
experimentally for a gas of oxygen molecules.[10] This absorption originates from col-
lision complexes present in the gas. The collision complex can be considered to be a
short-lived species, which may have a lower symmetry than the individual molecules
and hence an electric dipole moment. Since these collision complexes are short-lived,
the resulting absorption spectra are typically broad in frequency.

Поглощение света описывается законом Бугера-Ламберта-Бера 
\begin{gather}
    I = I_0 \exp \lb - \alpha l \rb. 
\end{gather}
По закону Бера коэффицициент поглощения $\alpha$ связан линейно с плотностью поглощающего газа $n$
\begin{gather}
    \alpha = \sigma n,
\end{gather}
где через $\sigma$ обозначено сечение поглощения. Зависимость предполагает, что поглощение происходит индивидуальными молекулами. В общем случае, для описания этой зависимости может быть использовано вириальное разложение
\begin{gather}
    \alpha = \sigma n + \alpha_\text{binary} n^2 + \alpha_\text{ternary} n^3 + O(n^4), 
\end{gather}
в котором квадратичное слагаемое описывает столкновительно-индуцированное поглощение бинарными комплексами, кубическое слагаемое -- тройными комплексами, и т.д. Стоит отметить, что сечение поглощения мономера $\sigma$ зависит от давления (проявляется это, например, в сдвигах и уширениях линий), поэтому при каждой фиксированной частоте поглощение мономера не является линейным по плотности $n$, однако интегральная интенсивность линейна по плотности. \par
Таким образом, для моделирования суммарного профиля поглощения необходимо знать сечение поглощения $\sigma$, коэффициент бинарного поглощения $\alpha_\text{binary}$, и т.д., и как эти эти величины зависят от атмосферных характеристик, таких как температура. Данная работа посвящена расчет коэффициентов бинарного поглощения $\alpha_\text{binary}$ для различных систем, имеющих практическую значимость в атмосферных приложениях. \par
Спектры плотных газов и газовых смесей существенно отличаются от спектров, зарегистрированных при низких плотностях. По мере увеличения плотности, наблюдается линейное увеличение интенсивностей разрешенных колебательно-вращательных и электронных полос. При промежуточных значениях плотности могут возникнуть новые полосы поглощения, которые не наблюдались при более низких плотностях, причем интенсивность этих полос (по крайней мере в первом приближении) может быть описана квадратичным или кубическим законом. За появление этих полос отвественны ван дер Ваальсовы комплексы двух или большего количества молекул. Подобные полосы поглощения найдены во многих молекулярных газах, даже молекулы которых не обладают постоянным дипольным моментом.  

Состояния ван дер Ваальсовых комплексов классифицируют на связанные, полная колебательно-вращательная энергия меньше энергии диссоциации, и свободные. 

Впервые квадратичную зависимость коэффициента поглощения от плотности наблюдал Йенсен (Janssen) в 1885 году на полосах поглощениях кислорода \cite{janssen1885}.
\fi
