\documentclass[12pt]{article}

\usepackage[T1]{fontenc}
\usepackage[utf8]{inputenc}
\usepackage[russian]{babel}

% page margin
\usepackage[top=2cm, bottom=2cm, left=2cm, right=2cm]{geometry}

\usepackage{graphicx}

% AMS packages
\usepackage{amsmath}
\usepackage{amssymb}
\usepackage{amsfonts}
\usepackage{amsthm}
\usepackage{csquotes}

% blackboar lettering
\usepackage{dsfont}
\usepackage{bbm}

\usepackage{fancyhdr}
\pagestyle{fancy}
% modifying page layout using fancyhdr
\fancyhf{}
\renewcommand{\sectionmark}[1]{\markright{\thesection\ #1}}
\renewcommand{\subsectionmark}[1]{\markright{\thesubsection\ #1}}

\rhead{\fancyplain{}{\rightmark }}
\cfoot{\fancyplain{}{\thepage }}

\newcommand{\lb}{\left(}
\newcommand{\rb}{\right)}

\title{Список сведений}
\date{}
\author{}
\usepackage{titling}

\setlength{\droptitle}{-8em}   % This is your set screw
\begin{document}
\maketitle

\begin{enumerate}
    \item Финенко Артем Андреевич
    \item Тема дипломной работы: \enquote{Моделирование спектров столкновительно-индуцированного поглощения в дальней инфракрасной области методом классических траекторий}
    \item Работа выполнялась в лаборатории строения и квантовой механики молекул кафедры физической химии химического факультета МГУ.
    \item Научные руководители -- кандидат физико-математических наук Петров Сергей Владимирович, старший научный сотрудник лаборатории строения и квантовой механики молекул кафедры физической химии химического факультета МГУ и Локштанов Сергей Евгеньевич, младший научный сотрудник лаборатории лазерной спектроскопии кафедры лазерной химии химического факультета МГУ.
    \item Рецензент -- доктор физико-математических наук Сурин Леонид Аркадьевич, заместитель директора по научной работе и заведующий отделом института спектроскопии РАН (ИСАН).
\end{enumerate}

\end{document}

