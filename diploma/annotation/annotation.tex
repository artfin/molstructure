\documentclass[12pt]{article}

\usepackage[T1]{fontenc}
\usepackage[utf8]{inputenc}
\usepackage[russian]{babel}

% page margin
\usepackage[top=2cm, bottom=2cm, left=2cm, right=2cm]{geometry}

\usepackage{graphicx}

% AMS packages
\usepackage{amsmath}
\usepackage{amssymb}
\usepackage{amsfonts}
\usepackage{amsthm}
\usepackage{csquotes}

% blackboar lettering
\usepackage{dsfont}
\usepackage{bbm}

\usepackage{fancyhdr}
\pagestyle{fancy}
% modifying page layout using fancyhdr
\fancyhf{}
\renewcommand{\sectionmark}[1]{\markright{\thesection\ #1}}
\renewcommand{\subsectionmark}[1]{\markright{\thesubsection\ #1}}

\rhead{\fancyplain{}{\rightmark }}
\cfoot{\fancyplain{}{\thepage }}

\newcommand{\lb}{\left(}
\newcommand{\rb}{\right)}

\title{Моделирование спектров столк\-но\-ви\-тель\-но-ин\-ду\-ци\-ро\-ван\-ного поглощения в дальней инфракрасной области методом классических траекторий}
\date{}
\author{Финенко Артем Андреевич \\ Лаборатория строения и квантовой механиики молекул \\ Научные руководители: к.ф.-м.н., с.н.с. Петров С.В., м.н.с. Локштанов С.Е.}
\usepackage{titling}

\setlength{\droptitle}{-8em}   % This is your set screw
\begin{document}
\maketitle
\vspace*{-1.5cm}
Эффект столкновительно-индуцированного поглощения (СИП) в дальней инфракрасной области имеет существенное значение для радиационного баланса достаточно плотных планетных атмосфер. Экспериментальные исследования спектров СИП проводились для набора систем при отдельных температурах, однако в приложениях востребованы данные о поглощении в широком диапазоне температуре и частот, в связи с чем возникает необходимость детального и всеобъемлющего теоретического моделирования спектров СИП. \par
Квантово-механические расчеты спектров СИП с полным учетом анизотропии межмолекулярных взаимодействий были проведены для ограниченного набора молекулярных систем. Расчеты, применяющие классический формализм, позволяют моделировать спектры СИП значительно быстрее и дешевле. \par  
В данной работе развивается методика моделирования спектров СИП в области рототрансляционной полосы с использованием классических траекторий. В первой части работы рассмотрены теоретические основы взаимодействия излучения с молекулярными системами. Во второй и третьей частях работы представлены основные элементы оригинальной методики траекторного расчета. Методика применяется, по мере усложнения, к двухатомной системе (He$-$Ar), системе, состоящей из атома и линейной молекулы (Ar$-$CO$_2$), и системе, состоящей из двух линейных молекул (N$_2-$N$_2$). Проведенные расчеты основываются на интегральном преобразовании, позволяющем получать спектр поглощения как среднее по ансамблю траекторий рассеяния. Рассмотрение многоатомных молекулярных пар производится в подвижной системе осей, что позволяет напрямую использовать поверхности потенциальной энергии и индуцированного дипольного момента, полученные при аппроксимации квантово-химических данных высокого уровня. Классическая динамика столкновения рассматривается в гамильтоновом формализме с полным учетом анизотропии межмолекулярных взаимодействий. Ключевым элементом разработанной вычислительной процедуры, обеспечивающим масштабируемость методики при распостранении на многоатомные системы, является расчет производных гамильтониана \enquote{на лету} с использованием лагранжиана и его производных, получение которых в значительной степени автоматизировано при помощи систем компьютерной алгебры. \par 
Сравнение с экспериментальными данными для систем N$_2-$N$_2$ и Ar$-$CO$_2$ и данными, полученными в результате кван\-тово-механических и молекулярно-динамических расчетов, показывает высокую эффективность траекторной методики. При помощи развиваемой методики могут быть расчитаны спектры СИП систем произвольных многоатомных молекулярных пар, реализация квантовомеханических расчетов для которых не представляется возможной в настоящее время. \par
Материалы дипломной работы частично опубликованы в Journal of Chemical Physics и представлены на международных конференциях. Исследования, представленные в работе, проводились при поддержке проектов РФФИ 18-05-00119, 18-32-20156, 18-55-16006 и программы №28 фундаментальных исследований Президиума РАН.


\end{document}
