\documentclass[12pt]{article}

\usepackage[T1]{fontenc}
\usepackage[utf8]{inputenc}
\usepackage[russian]{babel}

% page margin
\usepackage[top=2cm, bottom=2cm, left=2cm, right=2cm]{geometry}

\usepackage{graphicx}

% AMS packages
\usepackage{amsmath}
\usepackage{amssymb}
\usepackage{amsfonts}
\usepackage{amsthm}

% blackboar lettering
\usepackage{dsfont}
\usepackage{bbm}

\usepackage{fancyhdr}
\pagestyle{fancy}
% modifying page layout using fancyhdr
\fancyhf{}
\renewcommand{\sectionmark}[1]{\markright{\thesection\ #1}}
\renewcommand{\subsectionmark}[1]{\markright{\thesubsection\ #1}}

\rhead{\fancyplain{}{\rightmark }}
\cfoot{\fancyplain{}{\thepage }}

\newcommand{\lb}{\left(}
\newcommand{\rb}{\right)}

\title{Моделирование спектров столкновительно-индуцированного поглощения в дальней инфракрасной области методом классических траекторий}
\date{\today}
\author{Финенко Артем Андреевич \\ Научные руководители: Петров С.В., Локштанов С.Е.}
\usepackage{titling}

\setlength{\droptitle}{-4em}   % This is your set screw
\begin{document}
\maketitle

Столкновительно-индуцированное поглощение (СИП) в инфракрасной области имеет существенное значение для радиационного баланса планетных атмосфер. Моделирование столк\-новительно-индуцированных спектров имеет важное прикладное значение для уточнения существующих климатических моделей земной атмосферы и атмосфер экзопланет. \par
Полномерные квантово-механические расчеты спектров СИП оказываются вычислительно очень трудоемкими. Континуальная природа СИП открывает возможность теоретического моделирования с позиций классической механики. Подходы с применением классической механики можно подразделить на метод молекулярной динамики и метод классических траекторий. Последний представляется как наиболее простой метод с точки зрения вычислительной сложности. Кроме того, метод классических траекторий позволяет выделить вклады разных состояний молекулярных пар -- свободных пар, связанных и метастабильных димеров -- в суммарное поглощение, что позволяет в большей степени контролировать расчет и интерпретировать экспериментальные данные. \par  
В данной работе предлагается формализм моделирования спектров СИП в области рототрансляционной полосы методом классических траекторий. В первой части работы рассмотрены теоретические основы взаимодействия излучения с молекулярными системами. Во второй части рассмотрено СИП в двухатомной системе, получено выражение для спектральной функции, позволяющее производить расчеты с полным учетом симметрии задачи. Построена методика расчета столкновительно-индуцированного спектра свободных и метастабильных состояний. В расчете используются аппроксимации \textit{ab initio} поверхностей потенциальной энергии и индуцированного дипольного момента, построенных на квантово-химических данных высокого уровня. Эффективность траекторного расчета для двухатомных систем продемонстрирована на примере He$-$Ar. В третьей части работы рассматриваются молекулярные пары, состоящие из линейной молекулы и атома (на примере CO$_2-$Ar) и двух линейных молекул (на примере N$_2-$N$_2$). Для описания динамики столкновения используется подвижная система отсчета, позволяющая напрямую использовать аппроксимации квантово-химических поверхностей потенциальной энергии и индуцированного дипольного момента. В расчете используются точные классические гамильтонианы в подвижной системе отсчета, полученные методами компьютерной алгебры. Сравнение с квантовомеханическими данными для системы N$_2-$N$_2$ показало эффективность метода классических траекторий в задаче моделирования столкновительно-индуцированных спектров. Разработанный метод может быть обобщен на системы с б\'{о}льшим количеством вращательных степеней свободы, реализация квантовомеханического расчета для которых не представляется возможной в настоящее время. 


\end{document}
