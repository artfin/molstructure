\section{Взаимодействие электромагнитного излучения с молекулярными системами}
Рассмотрим систему из $N$ взаимодействующих частиц (молекул) в квантовом состоянии $\ket{j}$. Обозначим гамильтониан системы через $\hat{H}_0$. Пусть система подвергается воздействию электрического поля частоты $\omega$, которое индуцирует переходы в другие состояния $\ket{k}$ системы при условии, что частота поля близка к частотам Бора системы
%
\begin{gather}
    \omega_{jk} = (E_j - E_k) / \hbar.
\end{gather}

Будем считать, что длина волны рассматриваемого поля $\lambda$ много больше размеров молекул в системе, поэтому в локальной окрестности молекул поле можно считать однородным в пространстве.  Электрическая составляющая плоской волны может быть записана в виде суммы
%
\begin{gather}
    \mf{E}(t) = E_0 \boldsymbol{\varepsilon} \cos \omega t = \frac{E_0 \boldsymbol{\varepsilon}}{2} \lb e^{i \omega t} + e^{- i\omega t} \rb,
\end{gather}
%
где $E_0$ -- амплитуда волны, а $\boldsymbol{\varepsilon}$ -- единичный вектор, ориентированный вдоль направления распространения волны. Энергия взаимодействия молекулярной системы с электрическим полем в дипольном приближении равна
%
\begin{gather}
    W(t) = - (\boldsymbol{\mu} \cdot \mf{E}) = - \frac{E_0}{2} \lb \boldsymbol{\mu} \cdot \boldsymbol{\varepsilon} \rb \lsq e^{i \omega t} + e^{-i \omega t} \rsq, \label{part1-dipole-approximation} 
\end{gather}
%
где через $\boldsymbol{\mu}$ обозначен полный дипольный момент системы. Взаимодействие молекулярных систем с электромагнитным полем часто рассматривают в этом приближении, считая поле классическим объектом. \par  
Взаимодействие молекулярных систем с электрическим полем традиционно рассматривается в рамках временной теории возмущений \cite{cohentanuji, greiner}. Согласно приложению \ref{appendix:perturbation-theory}, вероятность индуцированного возмущением $W(t)$ перехода между состояниями невозмущенной системы $\ket{j} \rightarrow \ket{k}$ в первом порядке временной теории возмущения равна
%
\begin{gather}
    \mathcal{P}_{jk}(t) = \frac{1}{\hbar^2} \abs{ \int\limits_0^t W_{jk}(t^\prime) e^{i \omega_{jk} t^\prime} dt^\prime }^2,
\end{gather}
%
где через $W_{jk}(t)$ обозначен матричный элемент возмущения на состояниях невозмущенной системы, равный в данном случае
%
\begin{gather}
    W_{jk} = -\frac{E_0}{2} \bra{j} \boldsymbol{\mu} \cdot \boldsymbol{\varepsilon} \ket{k} \lsq e^{i \omega t} + e^{-i \omega t} \rsq.
\end{gather}

Коэффициенты разложения первого порядка $b_n^{(1)}(t)$ возмущенной волновой функции $\ket{\psi(t)}$ в базисе собственных функций невозмущенного гамильтониана равны (см. соотношение \eqref{app-expansion})
\begin{gather}
    b_{n}^{(1)}(t; \omega) = -\frac{E_0}{2 \hbar} \bra{j} \boldsymbol{\mu} \cdot \boldsymbol{\varepsilon} \ket{k} \lsq \frac{e^{i (\omega_{jk} + \omega) t} - 1}{\omega_{jk} + \omega} + \frac{e^{i (\omega_{jk} - \omega) t} - 1}{\omega_{jk} - \omega} \rsq. \label{part1-bn-expression}
\end{gather}
%
Квадрат модуля коэффициента $b_{n}^{(1)}$ определяет вероятность перехода в $n$-ое стационарное состояние невозмущенного гамильтониана. Возводя в квадрат выражение \eqref{part1-bn-expression}, после алгебраических преобразований приходим к
%
\begin{gather}
    \hspace*{-0.5cm}
    \abs{b_{n}^{(1)}(t)}^2 = \frac{E_0}{\hbar^2} \lsq \frac{\sin^2 \lb \frac{1}{2} \lb \omega_{jk} + \omega \rb t \rb}{\lb \omega_{jk} + \omega \rb^2} + \frac{\sin^2 \lb \frac{1}{2} \lb \omega_{jk} - \omega \rb t \rb}{\lb \omega_{jk} - \omega \rb^2} + \frac{ 8 \cos \lb \omega t \rb \sin \lb \frac{1}{2} \lb \omega_{jk} + \omega \rb t \rb \sin \lb \frac{1}{2} \lb \omega_{jk} - \omega \rb t \rb}{\lb \omega_{jk} + \omega \rb \lb \omega_{jk} - \omega \rb} \rsq. \label{part1-bn-expression2}
\end{gather}
 
В данном контексте нас интересуют не вероятности переходов, а скорости переходов $\Gamma_{jk}$ (иными словами, вероятности $P_{jk}$, отнесенные к единице времени) при больших значениях $t$
%
\begin{gather}
    \Gamma_{jk} = \lim_{t \rightarrow \infty} \frac{P_{jk}}{t}.
\end{gather}
%
При предельном переходе в \eqref{part1-bn-expression2} первые два слагаемые преобразуются к дельта-функ\-циям, в то время как последнее слагаемое оказывается нулевым \cite{baym-quantum-mechanics}. Итак, выражение для скорости переходов оказывается следующим \cite{mcquarrie-statistical-mechanics}  
%
\begin{gather}
    \Gamma_{jk} = \frac{\pi E_0^2}{2 \hbar^2} \abs{ \bra{j} \boldsymbol{\mu} \cdot \boldsymbol{\varepsilon} \ket{k} }^2 \Big[ \delta \lb \omega_{jk} - \omega \rb + \delta \lb \omega_{jk} + \omega \rb \Big]. \label{part1-transition-rate} 
\end{gather}

Выражение \eqref{part1-transition-rate} определяет скорость переходов между конкретными состояниями невозмущенной системы $\ket{j}$ и $\ket{k}$. Скорость поглощения энергии в ходе переходов между этими состояниями равна $\hbar \omega_{jk} \Gamma_{jk}$, т.к. в процессе одного акта поглощения система поглощает энергию, равную $\hbar \omega_{jk}$. Скорость поглощения энергии в ходе переходов с заданного уровня $\ket{j}$ может быть получена при суммировании по всем состояниям $\ket{k}$, которые доступны системе для перехода. Наконец, суммарная скорость поглощения энергии излучения системой $-\dot{E}_\text{rad}$ получается в результате суммирования по всем возможным начальным состояниям $\ket{j}$ с соответствующими заселенностями $\rho_j$
\begin{gather}
    -\dot{E}_\text{rad} = \sum_j \sum_k \rho_j \hbar \omega_{jk} \Gamma_{jk} = \frac{\pi E_0^2}{2 \hbar} \sum_{j, k} \omega_{jk} \rho_j \abs{ \bra{j} \boldsymbol{\mu} \cdot \boldsymbol{\varepsilon} \ket{k} }^2 \Big[ \delta \lb \omega_{jk} - \omega \rb + \delta \lb \omega_{jk} + \omega \rb \Big]. \label{part1-energy-absorption-rate}
\end{gather}

Для получения более симметричной формы уравнения \eqref{part1-energy-absorption-rate} осуществим алгебраические преобразования. Рассмотрим отдельно вторую сумму, получающуюся при раскрытии скобок в уравнении \eqref{part1-energy-absorption-rate}. Поменяем местами индексы $j \leftrightarrow k$, что обосновывается тем, что оба индекса пробегают по всем квантовым состояниям системы,
\begin{gather}
    \frac{\pi E_0^2}{2 \hbar} \sum_{j, k} \omega_{jk} \rho_j \abs{ \bra{j} \boldsymbol{\mu} \cdot \boldsymbol{\varepsilon} \ket{k} }^2 \delta \lb \omega_{jk} + \omega \rb = -\frac{\pi E_0^2}{2 \hbar} \sum_{j, k} \omega_{jk} \rho_j \abs{ \bra{j} \boldsymbol{\mu} \cdot \boldsymbol{\varepsilon} \ket{k} }^2 \delta \lb \omega_{jk} - \omega \rb. \label{part1-index-change}
\end{gather}

Подстановка \eqref{part1-index-change} в \eqref{part1-energy-absorption-rate} приводит к выражению, в котором индексы $j$ и $k$ входят симметричным образом 
%
\begin{gather}
    -\dot{E}_\text{rad} = \frac{\pi E_0^2}{2 \hbar} \sum_{j, k} \omega_{j k} \lb \rho_j - \rho_k \rb \abs{ \bra{j} \boldsymbol{\mu} \cdot \boldsymbol{\varepsilon} \ket{k} }^2 \delta \lb \omega_{jk} - \omega \rb.
\end{gather}

Т.к. мы предполагаем, что возмущение достаточно слабо и действует на протяжении малого промежутка времени, то будем считать, что в любой момент система находится в состоянии теплового равновесия при температуре $T$. Используя это предположение, выразим заселенность $k$-ого состояния через заселенность $j$-го состояния (понятно, что можно воспользоваться и обратной связью, т.к. мы специально привели формулу к симметричному относительно замены индексов виду)
%
\begin{gather}
    \rho_k = \rho_j \exp \lb - \beta \hbar \omega_{jk} \rb,
\end{gather}
%
где $\beta = \boltz T$ и $\boltz$ -- постоянная Больцмана. Кроме того, вследствие того, что внутри суммы находятся дельта-функции, центрированные на $\omega_{jk}$ , функции от $\omega$, вычисленные при частоте $\omega_{jk}$, могут быть вынесены из под знака суммы: 
\begin{align}
    -\dot{E}_\text{rad} &= \frac{\pi E_0^2}{2 \hbar} \sum_{j, k} \omega_{jk} \rho_j \lb 1 - \exp \lb - \beta \hbar \omega_{jk} \rb \rb \abs{ \bra{j} \boldsymbol{\mu} \cdot \boldsymbol{\varepsilon} \ket{k} }^2 \delta \lb \omega_{jk} - \omega \rb = \\
    &= \frac{\pi E_0^2}{2 \hbar} \omega \lb 1 - \exp \lb - \beta \hbar \omega \rb \rb \sum_{j, k} \rho_j \abs{ \bra{j} \boldsymbol{\mu} \cdot \boldsymbol{\varepsilon} \ket{k} }^2 \delta \lb \omega_{jk} - \omega \rb. 
\end{align}

Суммарный поток энергии $I$, переносимой электромагнитной волной через среду с показателем преломления $n$, равен усредненному по времени модулю вектора Пойнтинга $\langle S \rangle$ и равен \cite{mcquarrie-statistical-mechanics}
\begin{gather}
    I = \langle S \rangle = \frac{c}{8 \pi} n E_0^2,
\end{gather}
%
где $c$ -- скорость света в вакууме. Показатель поглощения среды $\alpha(\omega)$ определяют как отношение энергии, поглощаемой средой в единицу времени при частоте $\omega$, к энергии, переносимой электромагнитной волной в единицу времени \cite{mcquarrie-statistical-mechanics}
\begin{gather}
    \alpha(\omega) = \frac{-\dot{E}_\text{rad}}{I} = \frac{4 \pi^2}{\hbar c n} \omega \lb 1 - e^{- \beta \hbar \omega} \rb \sum_{j, k} \rho_j \abs{ \bra{j} \boldsymbol{\mu} \cdot \boldsymbol{\varepsilon} \ket{k}}^2 \delta \lb \omega_{jk} - \omega \rb. \label{part1-absorption-coefficient-definition}
\end{gather}

На основании выражения \eqref{part1-absorption-coefficient-definition} определяют спектральную функцию $J(\omega)$ \cite{gordon1968}
\begin{gather}
    J(\omega) = \frac{3 \hbar c n \alpha(\omega)}{4 \pi^2 \omega \lb 1 - e^{-\beta \hbar \omega} \rb} = 3 \sum_{j, k} \rho_j \abs{ \bra{j} \boldsymbol{\mu} \cdot \boldsymbol{\varepsilon} \ket{k} }^2 \delta \lb \omega_{jk} - \omega \rb. \label{part1-spectral-function-definition}
\end{gather}

Отметим, что обычно спектральную функцию определяют таким образом, чтобы ее интеграл по частотному диапазону был равен единице. Здесь принято несколько иное определение, эта функция не нормирована. Кроме того, спектральная функция может быть определена и для отрицательных частот, в этом случае она относится к испусканию излучения. \par
Альтернативную форму выражения \eqref{part1-spectral-function-definition} получают, осуществляя смену Шредингеровского представления квантовой механики на представление Гейзенберга. Состояния в представлении Гейзенберга не зависят от времени -- временная эволюция заложена в операторах. Эволюция оператора $\hat{A}(t)$ описывается оператором эволюции $\hat{U}(t)$
\begin{gather}
    \hat{A}(t) = \hat{U}^{+}(t) \hat{A}(0) \hat{U}(t) = e^{\frac{i}{\hbar} \hat{H} t} \hat{A}(0) e^{-\frac{i}{\hbar} \hat{H} t}. 
\end{gather}

Удобно перейти в выражении \eqref{part1-spectral-function-definition} к представлению Гейзенберга, представив дельта-функцию как Фурье-образ мнимой экспоненты
\begin{gather}
    \delta (\omega) = \frac{1}{2\pi} \int\limits_{-\infty}^\infty e^{i \omega t} dt,
\end{gather}
получаем
\begin{gather}
    J(\omega) = \frac{3}{2\pi} \sum_{j,k} \rho_j \bra{j} \boldsymbol{\mu} \cdot \bs{\varepsilon} \ket{k} \bra{k} \boldsymbol{\mu} \cdot \boldsymbol{\varepsilon} \ket{j} \int\limits_{-\infty}^\infty \exp \lsq \lb \frac{E_j - E_k}{\hbar} - \omega \rb t \rsq dt. \label{part1-spectral-function-2}
\end{gather}

Т.к. состояния $\ket{k}$ и $\ket{j}$ являются собственными состояниями гамильтониана $\hat{H}_0$, то
\begin{gather}
    \exp \lb -\frac{i}{\hbar} E_j t \rb \ket{j} = \exp \lb -\frac{i}{\hbar} \hat{H}_0 t \rb \ket{j}, \quad \exp \lb \frac{i}{\hbar} E_k t \rb \bra{k} = \exp \lb \frac{i}{\hbar} \hat{H}_0 t \rb \bra{k}. 
\end{gather}

Произведение матричного элемента и экспоненты с Боровской частотой, получаемое при внесении матричного элемента под интеграл в \eqref{part1-spectral-function-2}, легко переводится в представление Гейзенберга
\begin{gather}
    \exp \lb \frac{i}{\hbar} \lb E_k - E_j \rb t \rb \bra{k} \boldsymbol{\mu} \cdot \boldsymbol{\varepsilon} \ket{j} = \bra{k} \exp \lb \frac{i}{\hbar} \hat{H}_0 t \rb \boldsymbol{\mu} \cdot \boldsymbol{\varepsilon} \exp \lb -\frac{i}{\hbar} \hat{H}_0 t \rb \ket{j} = \bra{k} \boldsymbol{\mu}(t) \cdot \boldsymbol{\varepsilon} \ket{j}. \label{part1-matrix-element}
\end{gather}

Подставляя \eqref{part1-matrix-element} в \eqref{part1-spectral-function-2}, приходим к
\begin{gather}
    J(\omega) = \frac{3}{2\pi} \int\limits_{-\infty}^\infty \sum_{j,k} \rho_j \bra{j} \boldsymbol{\mu}(0) \cdot \boldsymbol{\varepsilon} \ket{k} \bra{k} \boldsymbol{\mu}(t) \cdot \boldsymbol{\varepsilon} \ket{j} e^{-i \omega t} dt.
\end{gather}

Учитывая соотношение замкнутости \eqref{app-completeness}, получим
\begin{gather}
    J(\omega) = \frac{3}{2\pi} \int\limits_{-\infty}^\infty \sum_j  \rho_{j} \bra{j} \boldsymbol{\mu}(0) \cdot \boldsymbol{\varepsilon} \cdot \boldsymbol{\mu}(t) \cdot \boldsymbol{\varepsilon} \ket{j} e^{-i\omega t} dt.
\end{gather}

Сумма в подынтегральном выражении является квантово-механическим средним по ансамблю значением оператора, которое в дальнейшем будем обозначать через $\langle \cdots \rangle$. Считая среду изотропной, проинтегрируем по всем возможным ориентациям $\boldsymbol{\varepsilon}$:
\begin{gather}
    J(\omega) = \frac{1}{2\pi} \intty \mean{ \bs{\mu}(0) \cdot \bs{\mu}(t) } e^{-i \omega t} dt. \label{litreview-spectral-function1}
\end{gather}

Итак, спектральная функция является Фурье-образом автокорреляционной функции оператора дипольного момента поглощающих молекул \cite{gordon1968}. При работе в единицах СИ в выражении \eqref{litreview-spectral-function} следует добавить переводной множитель
\begin{gather}
    J(\omega) = \frac{1}{4\pi \varepsilon_0} \frac{1}{2\pi} \intty \mean{ \bs{\mu}(0) \cdot \bs{\mu}(t) } e^{-i \omega t} dt. \label{litreview-spectral-function} 
\end{gather}
Следует подчеркнуть, что никаких приближений, касающихся характера движения диполей, в этом рассмотрении сделано не было. Движение системы полностью обусловлено уравнениями движения, определяемыми гамильтонианом системы $\hat{H}_0$. \par
Будем использовать следующее выражение для нормированного коэффициента поглощения через спектральную функцию 
\begin{gather}
    \alpha(\nu) = \frac{(2 \pi)^3 N_L^2}{3 \hbar c} \nu \lsq 1 - \exp \lb -\frac{h c \nu}{k T} \rb \rsq V J(\nu), \label{part1-absorption-coefficient-definition1}  
\end{gather}
%
где через $N_L$ обозначена постоянная Лошмидта, для частоты в обратных сантиметрах применяем обозначение $\nu$. 

\section{Теория временных функций корреляции и спектральных моментов} \label{section:correlation_functions}

Теория корреляционных функций получила широкое развитие для описания неравновесных систем \cite{zwanzig1965}, однако ее применение к равновесным системам также является эффективным. В системах, находящихся в термодинамическом равновесии, макроскопические параметры не претерпевают эволюции во времени, таким образом для них не имеет смысла вводить какой-то точки отсчета времени. Часто рассматривают условные вероятности, такие как $P(B, t_2 \vert A, t_1) dB$ -- вероятность того, что динамическая переменная $B$ примет значение в диапазоне $(B, \dots, B + dB)$ в момент времени $t_2$ при условии, что другая динамическая переменная имела значение $A$ в момент времени $t_1$ \cite{nitzan2006}. Также можно рассмотреть совместную вероятность $P(B, t_2; A, t_1) dB dA$ -- вероятность того, что переменная $A$ примет значение в диапазоне $(A, \dots, A+dA)$ в момент времени $t_1$ и переменная $B$ примет значение в диапазоне $(B, \dots B+dB)$ в момент времени $t_2$. Эти две вероятности связаны соотношением (формулой полной вероятности)
\begin{gather}
    P(B, t_2; A, t_1) = P(B, t_2 \vert A, t_1) P(A, t_1), 
\end{gather}
где $P(A, t_1) dA$ --вероятность того, что переменная $A$ примет значение в диапазоне $(A, \dots, A+dA)$ в момент времени $t_1$. В стационарных системах последняя вероятность, очевидно, не зависит от времени $P(A, t_1) = P(A)$; условная и совместная вероятности зависят только от разности времён
\begin{gather}
    P(B, t_2; A, t_1) = P(B, t_2 - t_1; A, 0), \quad P(B, t_2 \vert A, t_1) = P(B, t_2 - t_1 \vert A, 0),
\end{gather}
%
где $t = 0$ было положено произвольным образом.\par
Временную корреляционную функцию двух динамических переменных $A$ и $B$ определяют, как интеграл следующего вида 
\begin{gather}
    C_{AB}(t_1, t_2) = \mean{ A(t_1) B(t_2) } = \iint dA dB AB P(B, t_2; A, t_1). \label{part1-correlation-function-definition}
\end{gather}

В стационарных системах функция корреляции есть функция разности времен
\begin{gather}
    \mean{ A(t_1) B(t_2) } = \mean{ A(0) B(t) } = \mean{ A(-t) B(0) }, \quad t = t_2 - t_1.
\end{gather}

С точки зрения классической механики динамические переменные $A$, $B$ являются функциями координат $\mf{r}(t)$ и импульсов $\mf{p}(t)$ всех частиц системы
\begin{gather}
    A(t) = A \lb \mf{r}(t), \mf{p}(t) \rb, \quad B(t) = B \lb \mf{r}(t), \mf{p}(t) \rb.
\end{gather}

Фазовая траектория $\mf{r}(t)$, $\mf{p}(t)$ однозначно определена начальными условиями $\mf{r}(0)$, $\mf{p}(0)$. Таким образом, совместная вероятность в \eqref{part1-correlation-function-definition} определяется функцией распределения $f(\mf{r}, \mf{p})$ начальных условий для фазовых траекторий
\begin{gather}
    C_{AB}(t_1, t_2) = \int d\mf{r} \, d\mf{p} \, f \lb \mf{r}, \mf{p} \rb A\lb t_1; \mf{r}, \mf{p}, t = 0 \rb B \lb t_2; \mf{r}, \mf{p}, t = 0 \rb;
\end{gather}
%
обозначение $A(t_1; \mf{r}, \mf{p}, t = 0)$ означает, что динамическая переменная $A$ в момент времени $t_1$ вычисляется как функция координат и импульсов $A(\mf{r}(t_1), \mf{p}(t_1))$, вычисленных в момент времени $t_1$. \par
    При $t \rightarrow 0$ корреляционная функция $C_{AB}(t)$ становится средним значением произведения динамических переменных $A$ и $B$
\begin{gather}
    C_{AB}(0) = \mean{AB}.
\end{gather}
%
В другом пределе $t \rightarrow \infty$ можно предположить, что корреляция между переменными исчезает, то есть
\begin{gather}
    \lim_{t \rightarrow \infty} C_{AB}(t) = \mean{A} \mean{B}.
\end{gather}

Часто обьем системы, появляющийся в формуле \eqref{part1-absorption-coefficient-definition1}, рассматривают как часть спектральной функции $J(\omega)$ или рассматривают их произведение совместно. Поэтому в соответствии с обозначениями, используемыми в \cite{frommhold}, определим автокорреляционную функцию дипольного момента $C(t)$ как
\begin{gather}
    C(t) = \frac{1}{4 \pi \varepsilon_0} \frac{V}{2\pi} \mean{\bs{\mu}(0) \cdot \bs{\mu}(t)}. \label{litreview-autocorrelation-function-definition}
\end{gather}

Спектральная функция \eqref{litreview-spectral-function} принимает действительные значения, следовательно
\begin{gather}
    V J^+(\omega) = \intty C(t)^+ e^{i \omega t} dt = \intty C(-t)^+ e^{-i\omega t} dt = VJ(\omega), \label{litreview-time-inversion}
\end{gather}
%
где была сделана замена переменной $t \rightarrow -t$, а индекс $+$ обозначает комплексное сопряжение. Сранивая \eqref{litreview-time-inversion} с \eqref{litreview-spectral-function}, получаем
\begin{gather}
    C(t) = C(-t)^+.
\end{gather}

Если мы разложим корреляционную функцию на действительную и мнимую часть, то получим следующие соотношения 
\begin{gather}
    \text{Re} \, C(t) = \text{Re} \, C(-t), \quad \text{Im} \, C(t) = -\text{Im} \, C(t).
\end{gather}

То есть, действительная часть корреляционной функции является четной функции времени, а мнимая часть -- нечетной. Так как классические корреляционные функции являются действительными функциями, то они должны быть четными функциями времени. Квантово-механические же корреляционные функции, как правило, являются комплексными функциями, мнимая часть является исключительно квантово-механическим вкладом, отсутствующим при классическом рассмотрении.


Как уже отмечалось, корреляционные функции в равновесных системах зависят только от разности времени $t_1 - t_2$. Следовательно,
\begin{gather}
    0 = \frac{d}{ds} \mean{ A(t+s) B(s) } = \mean{ \dot{A}(t+s) B(s) } + \mean{ A(t+s) \dot{B}(s) } = \mean{ \dot{A}(t) B(0) } + \mean{ A(t) \dot{B}(0) }.
\end{gather}

Получаем следующее соотношение 
\begin{gather}
    \mean{\dot{A}(t)B(0)} = -\mean{A(t)\dot{B}(0)}, 
\end{gather}
которое для автокорреляционных функций переходит в 
\begin{gather}
    \mean{A \dot{A}} = 0. \label{litreview-temp1}
\end{gather}

Разложим автокорреляционную функцию в ряд по степеням времени $t$
\begin{gather}
    \mean{A(0) A(t)} = \Bigg\langle A(0) \lsq A(0) + t \dot{A}(0) + \frac{t^2}{2!} \ddot{A}(0) + \dots \rsq \Bigg\rangle = \mean{A(0) A(0)} + t \mean{A(0) \dot{A}(0)} + \frac{t^2}{2!} \mean{A(0) \ddot{A}(0)} + \dots \label{litreview-correlation-function-series}
\end{gather}

Отметим, что коэффициенты перед степенями $t^n$ не требуют знания динамики $A(t)$, а являются средними по ансамблю, т.к. временные производные могут быть записаны через скобку Пуассона
\begin{gather}
    \frac{dA}{dt} = \lsq A, H \rsq = \sum_k \lsq \frac{\partial A}{\partial x_k} \frac{\partial H}{\partial p_k} - \frac{\partial A}{\partial p_k} \frac{\partial H}{\partial x_k} \rsq. \label{litreview-poisson-bracket}
\end{gather}

Соотношение \eqref{litreview-temp1} соотносится с тем, что корреляционная функция является четной функцией. Из четности функции следует, что все коэффициенты перед нечетными степенями $t$ в \eqref{litreview-correlation-function-series} обращаются в нуль. В квантово-механической корреляционной функции нечетные степени не исчезают, и, более того, именно за счет них корреляционная функция обладает мнимой частью. \par
Коэффициенты в ряду по степеням времени \eqref{litreview-correlation-function-series} имеют физический смысл моментов соответствующего частотного спектра. Разрешим соотношение \eqref{litreview-spectral-function} относительно автокорреляционной функции дипольного момента и разложим комплексную экспоненту в подынтегральной функции в ряд
\begin{gather}
    C(t) = 2\pi V \intty J(\omega) e^{i \omega t} d \omega = 2 \pi V \intty J(\nu) e^{i 2\pi \nu c t} d\nu = 2 \pi \sum_{n = 0}^\infty \frac{\lb 2\pi i c t\rb^n}{n!} \intty \nu^n V J(\nu) d\nu. 
\end{gather}
%
Величины
\begin{gather}
    M_n = \intty \nu^n V J(\nu) d\nu \label{litreview-spectral-function-moments}
\end{gather}
%
называют $n$-ыми спектральными моментами. Теоретически знание всех спектральных моментов эквивалентно знанию спектральной функции (так называемая проблема моментов), однако на практике моменты выше второго находят редко. \par
Итак, спектральные моменты являются моментами спектральной функции и пропорциональны производным автокорреляционной функции дипольного момента в точке $t = 0$:
\begin{gather}
    \frac{1}{2 \pi} \frac{d^n C}{dt^n} \Bigg\vert_{t = 0} = \lb 2 \pi ic \rb^n M_n,
\end{gather}
%
что позволяет вычислить их как средние значения по фазовому пространству. Подробно рассматривать вопрос вычисления спектральных моментов не будем, приведем лишь выражения, по которым могут быть рассчитаны первые два спектральных момента:
\begin{gather}
    M_0 = 2 \pi \frac{\displaystyle \int \bs{\mu}^2 \exp \lsq -H(\mf{q}, \mf{p}) / \kb T \rsq d\mf{q} \, d\mf{p}}{\displaystyle \int \exp \lsq -H(\mf{q}, \mf{p}) / \kb T \rsq d\mf{q} \, d\mf{p}}, \label{litreview-m0-phase-space} \\
    M_2 = 2 \pi (2 \pi c)^2 \frac{\displaystyle \int \dot{\bs{\mu}}^2 \exp \lsq -H(\mf{q}, \mf{p}) / \kb T \rsq d\mf{q} \, d\mf{p}}{\displaystyle \int \exp \lsq -H(\mf{q}, \mf{p}) / \kb T \rsq d\mf{q} \, d\mf{p}}, \label{litreview-m2-phase-space}
\end{gather}
%
где производную по времени дипольного момента $\dot{\bs{\mu}}$ в выражении для второго спектрального момента следует преобразовать в скобку Пуассона \eqref{litreview-poisson-bracket}. Затем полученные значения интегралов следует привести к размерности см$^{-1} \cdot$Амага$^{-2}$ и  см$^{-3} \cdot$Амага$^{-2}$, соответственно. \par
В данной работе мы применяем спектральные моменты, полученные по формулам \eqref{litreview-m0-phase-space}, \eqref{litreview-m2-phase-space}, для контроля сходимости траекторного расчета. В траекторном расчете мы получаем спектральную функцию $J(\omega)$, моменты которой \eqref{litreview-spectral-function-moments} в пределе должны совпасть с моментами по фазовому пространству.

\chapter*{Приложение 2.A}
\addcontentsline{toc}{chapter}{Приложение 2.A. Временная теория возмущений}
{\Large\textbf{Временная теория возмущений}} \label{appendix:perturbation-theory}
\vspace{0.5cm}

Представленное ниже изложение основано на \cite{cohentanuji}. Рассмотрим физическую систему, описываемую гамильтонианом $\hat{H}_0$; пусть $E_n$ и $\ket{n}$ -- собственные значения и собственные векторы гамильтониана $\hat{H}_0$:
\begin{gather}
    \hat{H}_0 \ket{n} = E_n \ket{n}.
\end{gather}

Для простоты будем считать, что спектр гамильтониана $\hat{H}_0$ является дискретным и невырожденным. Дополнительно будем считать, что $\hat{H}_0$ не зависит явно от времени, и его собственные состояния являются стационарными. \par
В течении конечного интервала времени от $t = 0$ до $t = T$ к физической системе прикладывается возмущение, зависящее явно от времени, и гамильтониан принимает вид
\begin{gather}
    \hat{H}(t) = \hat{H}_0 + \lambda \hat{W}(t),
\end{gather}
где $\lambda$ -- малый вещественный безразмерный параметр, а $\hat{W}(t)$ -- оператор, равный нулю при $t < 0$. \par
Предполагаем, что в начальный момент времени система находится в стационарном состоянии $\ket{i}$, являющемся собственным состоянием оператора $\hat{H}_0$ с собственным значением $E_i$. В момент времени $t = 0$ приложения возмущения система начинает претерпевать эволюцию, т.к. состояние $\ket{i}$ в общем случае уже не будет собственным состоянием возмущенного гамильтониана. Нашей целью является вычисление вероятности $\mathcal{P}_{if}(t)$ найти систему в момент времени $t$ в другом собственном состоянии $\ket{f}$ гамильтониана $\hat{H_0}$. \par
Между моментами времени $0$ и $t$ система эволюционирует в соответствии с временным уравнением Шредингера:
\begin{gather}
    i \hbar \frac{d}{dt} \ket{\psi(t)} = \lsq \hat{H}_0 + \lambda \hat{W}(t) \rsq \ket{\psi(t)}, \quad \ket{\psi(t = 0)} = \ket{i}. \label{app-time-schroedinger}
\end{gather}

Искомая вероятность может быть записана в форме:
\begin{gather}
    \mathcal{P}_{if}(t) = \abs{ \braket{f | \, \psi(t)} }^2. \label{app-probability-definition} 
\end{gather}

Пусть $c_n(t)$ -- компоненты разложения кет-вектора $\ket{\psi(t)}$ в базисе $\big\{ \ket{n} \big\}$:
\begin{gather}
    \ket{\psi(t)} = \sum_n c_n(t) \ket{n}, \quad c_n(t) = \braket{n | \psi(t) }, \label{app-expansion}
\end{gather}
и $W_{nk}(t)$ -- матричные элементы оператора $\hat{W}(t)$ в том же базисе
\begin{gather}
    W_{nk}(t) \equiv \bra{n} \hat{W}(t) \ket{k}.
\end{gather}

Используя соотношение замкнутости
\begin{gather}
    \sum_k \ket{k} \bra{k} = 1, \label{app-completeness}
\end{gather}
умножим слева обе части временного уравнения Шредингера \eqref{app-time-schroedinger} на вектор состояния $\ket{n}$
\begin{gather}
    i \hbar \frac{d}{dt} c_n(t) = E_n c_n(t) + \sum_k \lambda W_{nk}(t) c_k(t). \label{app-time-dependent-eq1} 
\end{gather}

Уравнения \eqref{app-time-dependent-eq1}, записанные для разных $n$, образуют систему связанных дифференциальных уравнений, позволяющую определить компоненты $c_n(t)$ вектора $\ket{\psi(t)}$. \par
Если возмущение $\lambda \hat{W}(t)$ равно нулю, то уравнения \eqref{app-time-dependent-eq1} не связаны друг с другом, и их решение имеет форму
\begin{gather}
    c_n(t) = b_n e^{-i E_n t / \hbar}, \label{app-time-dependent-sol-nonperturb}
\end{gather}
где $b_n$ -- постоянные, зависящие от начальных условий. Это решение традиционно называют стационарным решением. \par
Если теперь рассмотреть систему с малым возмущением $\lambda \hat{W}(t) \neq 0$, то можно ожидать, что решение $c_n(t)$ уравнений \eqref{app-time-dependent-eq1} будет близким к решению \eqref{app-time-dependent-sol-nonperturb}. Таким образом, если выполнить замену функций
\begin{gather}
    c_n(t) = b_n(t) e^{-i E_n t / \hbar}, \label{app-function-change}
\end{gather}
то в случае малого возмущения мы ожидаем, что $b_n(t)$ будут медленно меняющимися функциями времени. Подставим \eqref{app-function-change} в уравнение \eqref{app-time-dependent-eq1} и получим
\begin{gather}
    i \hbar \frac{d}{dt} b_n(t) = \lambda \sum_k e^{i \omega_{nk} t} W_{nk}(t) b_k(t), \label{app-time-dependent-eq2} 
\end{gather}
где через $\omega_{nk}$ обозначены частоты Бора
\begin{gather}
    \omega_{nk} = \frac{E_n - E_k}{\hbar}.
\end{gather}

Система уравнений \eqref{app-time-dependent-eq2} абсолютно эквивалентна уравнению Шредингера \eqref{app-time-schroedinger}. Применим теорию возмущений для решения системы \eqref{app-time-dependent-eq2}. Будем искать решение в форме ряда по степеням $\lambda$
\begin{gather}
    b_n(t) = \bnp{0}(t) + \lambda \bnp{1}(t) + \lambda^2 \bnp{2}(t) + O(\lambda^3). \label{app-series}
\end{gather}

Подставив разложение \eqref{app-series} в \eqref{app-time-dependent-eq2} и приравняв коэффициенты при $\lambda^r$, находим
\begin{gather}
    \begin{aligned}
        i \hbar \frac{d}{dt} \bnp{0}(t) &= 0, \hspace{4.82cm} r = 0,  \\
        i \hbar \frac{d}{dt} \bnp{r}(t) &= \sum_k e^{i \omega_{nk} t} W_{nk}(t) b_k^{(r-1)}(t), \qquad r \neq 0.
    \end{aligned} \label{app-equations-for-coefficients}
\end{gather}

В соответствии с предположением, при $t < 0$ система находится в состоянии $\ket{i}$, следовательно, среди коэффициентов 
$b_n(t)$ отличен от нуля только $b_i(t)$
\begin{gather}
    b_n(t = 0) = \delta_{ni},
\end{gather}
и это равенство должно оставаться справедливым при любых значениях $\lambda$. Коэффициенты разложения \eqref{app-series} должны удовлетворять условиям:
\begin{gather}
    \begin{aligned}
        \bnp{0}(t = 0) &= \delta_{ni}, \\
        \bnp{r}(t = 0) &= 0, \quad r \geq 1.
    \end{aligned}  \label{app-initial-conditions-for-coefficients}
\end{gather}

Таким образом, решение нулевого порядка получается при $t > 0$:
\begin{gather}
    \bnp{0}(t) = \delta_{ni}.
\end{gather}

Этот результат позволяет переписать уравнение \eqref{app-equations-for-coefficients} для $r = 1$
\begin{gather}
    i \hbar \frac{d}{dt} \bnp{1}(t) = \sum_k e^{i \omega_{nk} t} W_{nk}(t) \delta_{ki} = W_{ni}(t) e^{i \omega_{ni} t}.
\end{gather}

С учетом начального условия \eqref{app-initial-conditions-for-coefficients} находим коэффициенты разложения первого порядка
\begin{gather}
    \bnp{1}(t) = \frac{1}{i \hbar} \int\limits_0^t W_{ni}(t^\prime) e^{i \omega_{ni} t^\prime} d t^\prime.
\end{gather}

Согласно выражению \eqref{app-probability-definition} вероятность перехода $\mathcal{P}_{if}(t)$ равна 
\begin{gather}
    \mathcal{P}_{if}(t) = \abs{ c_f(t) }^2 = \abs{ b_f(t) }^2.
\end{gather}

Допустим теперь, что состояния $\ket{i}$ и $\ket{f}$ являются различными, то есть, будем интересоваться переходами, индуцированными возмущением $\lambda \hat{W}(t)$, между двумя различными стационарными состояниями гамильтониана $\hat{H}_0$. Тогда $b_f^{(0)}(t) = 0$ и получим окончательно выражение для вероятности перехода (выполнена подстановка $\lambda = 1$) 
\begin{gather}
    \mathcal{P}_{if}(t) = \lambda^2 \abs{ b_f^{(1)}(t) }^2 = \frac{1}{\hbar^2} \abs{ \int\limits_0^t e^{i \omega_{fi} t^\prime} W_{fi}(t^\prime) dt^\prime }^2. \label{app-probability-general}
\end{gather}

Выражение \eqref{app-probability-general} показывает, что вероятность $\mathcal{P}_{if}(t)$ пропорциональна квадрату модуля преобразования Фурье матричного элемента возмущения $W_{fi}(t)$, взятого на частоте Бора, соответствующей рассматриваемому переходу. Т.к. возмущение действует в течение конечного интервала времени до $t = T$, то при $t \geq T$ коэффициент $b_m^{(1)}(t)$ становится постоянным:
\begin{gather}
    b_m^{(1)}(t) = \frac{1}{i \hbar} \int\limits_0^T W_{if}(t^\prime) e^{i \omega_{if} t^\prime} dt^\prime = \frac{1}{i \hbar} \int\limits_{-\infty}^\infty W_{if}(t^\prime) e^{i \omega_{if} t^\prime} dt^\prime.
\end{gather}

Используя Фурье преобразование матричного элемента $W_{if}(t)$
\begin{gather}
    W_{if}(\omega) = \frac{1}{2 \pi} \int\limits_{-\infty}^\infty W_{if}(t) e^{i \omega t} dt,
\end{gather}
приходим к следующему выражениям для коэффициента
\begin{gather}
    b_m^{(1)}(t) = \frac{2 \pi}{i \hbar} W_{if}(\omega_{if})
\end{gather}
и вероятности перехода
\begin{gather}
    \mathcal{P}_{if}(t) = \frac{4 \pi^2}{\hbar^2} \abs{ W_{if}(\omega_{if}) }^2, \quad t \geq T.
\end{gather}

В данном параграфе мы предполагали, что переход происходит между состояниями дискретного спектра невозмущенного оператора $\hat{H}_0$. Более того, мы предполагали, что невозмущенный оператор $\hat{H}_0$ обладает исключительно дискретным спектром. Если оператор $\hat{H}_0$ обладает и непрерывным спектром, то полный набор собственных функций состоит из
\begin{gather}
    \hat{H}_0 \ket{n} = E_n \ket{n}, \quad \hat{H}_0 \ket{\varphi, \alpha} = E(\alpha) \ket{\varphi, \alpha},
\end{gather}
где $\alpha$ -- непрерывный индекс, нумерующий состояния непрерывного спектра $\ket{\varphi, \alpha}$. Решение возмущенной задачи $\ket{\psi(t)}$ разложимо по полному набору собственных функций -- как дискретного, так и непрерывного спектра:
\begin{gather}
    \ket{\psi(t)} = \sum_n c_n(t) \ket{n} + \int c_\alpha(t) \ket{\varphi, \alpha} d\alpha.
\end{gather}

Изложенное выше рассмотрение может быть дополнено для учета непрерывной составляющей спектра невозмущенного оператора $\hat{H}_0$ \cite{greiner}. 

\iffalse
Рассмотрим гамильтониан $\hat{H}$, представимый в виде суммы разрешимого, независящего от времени гамильтониана $\hat{H}_0$ и возмущения $\hat{V}(t)$. В Шредингеровском представлении эволюция вектора состояния во времени определяется уравнением Шредингера
\begin{gather}
    i \hbar \frac{d}{dt} \ket{\alpha, t}_S = \hat{H} \ket{\alpha, t}_S,
\end{gather}
в то время как наблюдаемы стационарны во времени 
\begin{gather}
    i \hbar \frac{d}{dt} \hat{O} = 0.
\end{gather}

Исключим эволюцию состояния во времени, связанную с невозмущенным гамильтонианом $\hat{H}_0$, определив вектор состояния в \textit{представлении взаимодействия} 
\begin{gather}
    \ket{\alpha, t}_I = \exp \lb \frac{i \hat{H}_0 t}{\hbar} \rb \ket{\alpha, t}_S. 
\end{gather}

Рассморим временную эволюцию введенного состояния
\begin{gather}
    i \hbar \frac{d}{dt} \ket{\alpha, t}_I = i \hbar \frac{d}{dt} \exp \lb \frac{i \hat{H_0} t}{\hbar} \rb \ket{\alpha, t}_S = \lb i \hbar \frac{d}{dt} \exp \lb \frac{i \hat{H}_0 t}{\hbar}  \rb \rb \ket{\alpha, t}_S + \exp \lb \frac{i \hat{H}_0 t}{\hbar} \rb \lb i \hbar \frac{d}{dt} \ket{\alpha, t}_S \rb = \notag \\
    = -\hat{H}_0 \exp \lb \frac{i \hat{H}_0 t}{\hbar} \rb \ket{\alpha, t}_S + \exp \lb \frac{i \hat{H}_0 t}{\hbar} \rb \lb \hat{H}_0 + \hat{V} \rb \ket{\alpha, t}_S = \exp \lb \frac{i \hat{H_0} t}{\hbar} \rb \hat{V} \ket{\alpha, t}_S = \hat{V}_I \ket{\alpha, t}_I, \label{app1-time-evolution} 
\end{gather}
где было введено обозначение
\begin{gather}
    \hat{V}_I \equiv \exp \lb \frac{i \hat{H}_0 t}{\hbar} \rb \hat{V} \exp \lb - \frac{i \hat{H}_0 t}{\hbar} \rb.
\end{gather}

В отсутствие возмущения состояние в представлении взаимодействия является стационарным. В присутствии возмущения динамика состояния определяется уравнением \eqref{app1-time-evolution}, причем оператором временной эволюции становится $\hat{V}_I(t)$.
\fi
