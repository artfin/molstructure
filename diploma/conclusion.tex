\begin{enumerate}
    \item Выведены соотношения между координатами и импульсами сферической и связанной с плоскостью системами координат (пункт \ref{section:two-atom-coordinate-systems}). Полученные соотношения использованы для вывода  аналитических выражений для плотностей распределений координат и импульсов в условиях канонического ансамбля, при помощи которых сгенерированы начальные условия для траекторного расчета (пункт \ref{section:two-atom-distributions}).
    \item Получено и обосновано интегральное выражение для спектральной функции в гамильтоновых переменных в системе координат, связанной с плоскостью столкновения \eqref{two-atom-spectral-function4}. Полученное выражение обобщено на многоатомные системы \eqref{polyatom-spectral-function1}. 
    \item Рассчитаны спектры СИП для системы He$-$Ar, продемонстрировано согласие с экспериментальными данными. 
    \item Выведены точные классические лагранжианы для систем типа атом$-$линейная молекула и линейная молекула$-$линейная молекула в подвижной системе координат.
    \item Создана вычислительная техника, позволяющая производить численное интегрирование динамических систем уравнений с использованием матриц кинетической энергии в форме Лагранжа и их производных (пункт \ref{section:dynamic-equations}).
    \item Выведены динамические уравнения, описывающие относительное вращение подвижной системы координат с учетом сохранения модуля вектора углового момента (пункт \ref{section:rotational-dynamics}).
    \item Произведены траекторные расчеты спектров СИП для систем N$_2-$N$_2$ и CO$_2-$Ar. Продемонстрировано согласие с экспериметальными данными и данными, полученными в результате квантово-механических и молекулярно-динамических расчетов.
\end{enumerate}

\iffalse
Из проделанной работы были сделаны следующие выводы
\begin{enumerate}
    \item Построена методика расчета столкновительно-индуцированных спектров методом классических траекторий. При помощи развитой методики проведены расчеты спектров систем He$-$Ar, CO$_2-$Ar и N$_2-$N$_2$, которые находятся в хорошем согласии с экспериментальными и теоретическими данными.
\end{enumerate}
\fi
