\section*{Основные результаты}
\begin{enumerate}
    \item Выведены соотношения между координатами и импульсами сферической и связанной с плоскостью системами координат (раздел \ref{section:two-atom-coordinate-systems}). Полученные соотношения использованы для вывода  аналитических выражений для плотностей распределений координат и импульсов в условиях канонического ансамбля, при помощи которых сгенерированы начальные условия для траекторного расчета (раздел \ref{section:two-atom-distributions}).
    \item Получено и обосновано интегральное выражение для спектральной функции в гамильтоновых переменных в системе координат, связанной с плоскостью столкновения \eqref{two-atom-spectral-function4}. Полученное выражение обобщено на многоатомные системы \eqref{polyatom-spectral-function1}. 
    \item Рассчитаны спектры СИП для системы He$-$Ar методом классических траекторий с использованием поверхностей потенциальной энергии и индуцированного дипольного момента, построенных на квантово-химических данных высокого уровня. Продемонстрировано согласие с экспериментальными данными и результатами квантово-механического расчета. 
    \item Выведены точные классические лагранжианы для систем типа атом$-$линейная молекула и линейная молекула$-$линейная молекула в подвижной системе координат.
    \item Создана вычислительная процедура, позволяющая производить численное интегрирование динамических систем уравнений с использованием матриц кинетической энергии в форме Лагранжа и их производных (раздел \ref{section:dynamic-equations}).
    \item Выведены динамические уравнения, описывающие молекулярное движение в подвижной системе отсчета с учетом сохранения модуля вектора углового момента (раздел \ref{section:rotational-dynamics}).
    \item Произведены траекторные расчеты спектров СИП для систем N$_2-$N$_2$ и CO$_2-$Ar методом классических траекторий с использованием поверхностей потенциальной энергии и индуцированного дипольного момента, построенных на квантово-химических данных высокого уровня. Продемонстрировано согласие с экспериментальными данными и данными, полученными в результате квантово-механических и молекуляр\-но-динамических расчетов.
\end{enumerate}

\section*{Выводы}
\begin{enumerate}
    \item Развита методика расчета столкновительно-индуцированных спектров из первых принципов методом классических траекторий с применением обобщенных внутренних координат и подвижной системы осей. Ключевым элементом, обеспечивающим масштабируемость методики при рассмотрении многоатомных систем, является расчет гамильтониана и его производных \enquote{на лету} через матрицы лагранжиана и их производные (раздел \ref{section:dynamic-equations}).
    \item Выполнен расчет столкновительно-индуцированных спектров системы N$_2-$N$_2$ из первых принципов в температурном диапазоне от 90К до 300К  и спектров системы CO$_2-$Ar в температурном диапазоне от 230К до 350К. Сравнение с экспериментальными данными, доступными при отдельных температурах, подтверждает работоспособность развиваемой расчетной методики и ее применимость для моделирования континуальных столкновительно-индуцированных спектров.
    \item Развиваемая методика позволяет рассчитывать спектры СИП для различных молекулярных пар в широком интервале температур. Рассчитанные спектры применимы в  ряде атмосферных и астрофизических приложений и могут быть включены в специализированный раздел спектроскопических баз данных, например, HITRAN-CIA. Определение границы применимости метода для получения надежных результатов требует дальнейшего анализа и накопления расчетных данных для б\'{о}льшего числа молекулярных пар.  
\end{enumerate}

\iffalse
Из проделанной работы были сделаны следующие выводы
\begin{enumerate}
    \item Построена методика расчета столкновительно-индуцированных спектров методом классических траекторий. При помощи развитой методики проведены расчеты спектров систем He$-$Ar, CO$_2-$Ar и N$_2-$N$_2$, которые находятся в хорошем согласии с экспериментальными и теоретическими данными.
\end{enumerate}
\fi
