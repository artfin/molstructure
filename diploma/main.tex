\documentclass[12pt, a4paper, oneside]{extbook}

%\fontsize{14}{1.25}
\usepackage[T1]{fontenc}
%\usepackage[T2A]{fontenc}
\usepackage[utf8]{inputenc}
%\usepackage[main=russian, english]{babel}
\usepackage[russian]{babel}

% margins
%\setlength{\textwidth}{360pt}
%\setlength{\oddsidemargin}{22pt}
%\setlength{\evensidemargin}{70pt}
%\setlength{\marginparwidth}{106pt}

% page margin
%\usepackage[top=2cm, bottom=2cm, left=2cm, right=2cm]{geometry}
\usepackage[left=2cm,right=2cm,top=2cm,bottom=2cm,bindingoffset=0cm]{geometry}

% AMS packages
\usepackage{amsmath}
\usepackage{amssymb}
\usepackage{amsfonts}
\usepackage{amsthm}

% floating option for figures
\usepackage{float}

\usepackage{xcolor}

\usepackage{bbm}

\usepackage{fancyhdr}
\pagestyle{fancy}
% modifying page layout using fancyhdr
\fancyhf{}
\renewcommand{\sectionmark}[1]{\markright{\thesection\ #1}}
\renewcommand{\subsectionmark}[1]{\markright{\thesubsection\ #1}}

\rhead{\fancyplain{}{\rightmark }}
\cfoot{\fancyplain{}{\thepage }}

\usepackage{array}
\usepackage{braket}

\makeatletter
\def\env@dmatrix{\hskip -\arraycolsep
  \let\@ifnextchar\new@ifnextchar
  \extrarowheight=2ex
  \array{*\c@MaxMatrixCols{>{\displaystyle}c}}}

\newenvironment{dmatrix}
  {\env@dmatrix}
  {\endarray\hskip-\arraycolsep}

\newenvironment{bdmatrix}
  {\left[\env@dmatrix}
  {\endmatrix\right]}
\makeatother

\usepackage[autostyle]{csquotes}

\newcommand{\mf}{\mathbf}

\newcommand{\lb}{\left(}
\newcommand{\rb}{\right)}
\newcommand{\lsq}{\left[}
\newcommand{\rsq}{\right]}
\newcommand{\lc}{\left\{}
\newcommand{\rc}{\right\}}

\DeclareMathAlphabet{\mymathbb}{U}{BOONDOX-ds}{m}{n}
\newcommand{\bbzero}{\mymathbb{0}}
\newcommand{\bbone}{\mymathbb{1}}

\def\hh{_{\rm H}}

\usepackage[toc, page]{appendix}

\usepackage{chngcntr}
\usepackage{etoolbox}

\AtBeginEnvironment{subappendices}{%
\chapter*{Приложение}
\renewcommand\appendixname{STUFF}
%\renewcommand\appendixpagename{APPENDIXPAGENAME}
%\renewcommand{\appendixtocname}{\appendixname}
\addcontentsline{toc}{chapter}{Приложение}
\counterwithin{figure}{section}
\counterwithin{table}{section}
}

\usepackage{mathtools}

\DeclarePairedDelimiter\abs{\lvert}{\rvert}%
\DeclarePairedDelimiter\norm{\lVert}{\rVert}%

% Swap the definition of \abs* and \norm*, so that \abs
% and \norm resizes the size of the brackets, and the 
% starred version does not.
\makeatletter
\let\oldabs\abs
\def\abs{\@ifstar{\oldabs}{\oldabs*}}
%
\let\oldnorm\norm
\def\norm{\@ifstar{\oldnorm}{\oldnorm*}}
\makeatother

\newcommand{\bnp}[1]{b_n^{(#1)}}

% hyperlinks for eqref and cite
\usepackage{hyperref}

% indentation in the first paragraph of section or paragraph
\usepackage{indentfirst}

\usepackage{dsfont}
\newcommand{\boltz}{k_\text{b}}
\newcommand{\kb}{k_\text{b}}
\newcommand{\intty}{\int\limits_{-\infty}^\infty}
\newcommand{\mean}[1]{\langle #1 \rangle}
\newcommand{\bs}{\boldsymbol}
\newcommand{\mL}{\mathcal{L}}
\newcommand{\mH}{\mathcal{H}}
\newcommand{\mN}{\mathcal{N}}
\newcommand{\mHplane}{\mathcal{H}_\text{plane}}
\newcommand{\bbS}{\mathds{S}}
\newcommand{\bxi}{\boldsymbol{\xi}}
\newcommand{\rfixed}{r_\text{fixed}}
\newcommand{\Tl}{T_\text{L}}
\newcommand{\Th}{T_\text{H}}
\newcommand{\bOmega}{\boldsymbol{\Omega}}
\newcommand{\bbI}{\mathds{I}}
\newcommand{\bba}{\mathbbm{a}}
\newcommand{\bbA}{\mathds{A}}
\newcommand{\bbB}{\mathds{B}}
\newcommand{\bbG}{\mathds{G}}
\newcommand{\bbF}{\mathds{F}}
\newcommand{\bbM}{\mathds{M}}
\newcommand{\bbV}{\mathds{V}}
\newcommand{\bbW}{\mathds{W}}
\newcommand{\EOmega}{\boldsymbol{\Omega}_{\rm \textbf{e}}}
\newcommand{\mfpe}{\mf{p_e}}
\newcommand{\mfpet}{\mf{p_e^+}}
\newcommand{\mfq}{\mf{q}}
\newcommand{\mfp}{\mf{p}}
\newcommand{\mfJ}{\mf{J}}
\newcommand{\Jx}{J_\text{X}}
\newcommand{\Jy}{J_\text{Y}}
\newcommand{\Jz}{J_\text{Z}}
\newcommand{\dJx}{\dot{\Jx}}
\newcommand{\dJy}{\dot{\Jy}}
\newcommand{\dJz}{\dot{\Jz}}
\newcommand{\bUpsilon}{\boldsymbol{\Upsilon}_e}
\newcommand{\pe}{\mathbf{p}_e}

\usepackage[raggedright]{titlesec}
\usepackage{fourier, erewhon}

\usepackage{microtype}
 \SetTracking[no ligatures={f}]{encoding=*}{100}
% changing fontsize of chapter titles
 \titleformat{\chapter}[display]
 {\bfseries\Large\lsstyle\SetTracking[no ligatures = {f}]{encoding = *}{50}\filleft}
 {\MakeUppercase{\chaptername}\enspace\thechapter}
 {2ex}
 {\titlerule[1pt]\vspace{2ex}\MakeUppercase}%
\titlespacing*{\chapter}{0pt}{-60pt}{10ex}

% changing fontsize of section titles
\titleformat{\section}
 {\normalfont\fontsize{16}{15}\bfseries}{\thesection}{1em}{}

 \usepackage{tabularx}
 \usepackage{multirow}
 \usepackage{booktabs}
 \usepackage{array}

% \usepackage[backend=biber, bibencoding=auto]{biblatex}
% \usepackage[backend=biber, 
%             bibencoding=auto, 
%             style=gost-numeric,
%             language=auto,
%             sorting=none]{biblatex}
% \addbibresource{biblio.bib}

% disable per-chapter figure counter
% \usepackage{chngcntr}
% \counterwithout{figure}{chapter}
% \usepackage[figurewithin=none]{caption}


\begin{document}

\tableofcontents

\chapter{Введение}
Homonuclear diatomic molecules, which includes several astrophysically and atmo-
spherically relevant species, have zero electric dipole moment due to their inversion
symmetry. This means that these molecules are in principle infrared inactive. How-
ever, gases of apolar molecules exhibit collision-induced absorption, as was first shown
experimentally for a gas of oxygen molecules.[10] This absorption originates from col-
lision complexes present in the gas. The collision complex can be considered to be a
short-lived species, which may have a lower symmetry than the individual molecules
and hence an electric dipole moment. Since these collision complexes are short-lived,
the resulting absorption spectra are typically broad in frequency.

\iffalse
Поглощение света описывается законом Бугера-Ламберта-Бера 
\begin{gather}
    I = I_0 \exp \lb - \alpha l \rb. 
\end{gather}
По закону Бера коэффицициент поглощения $\alpha$ связан линейно с плотностью поглощающего газа $n$
\begin{gather}
    \alpha = \sigma n,
\end{gather}
где через $\sigma$ обозначено сечение поглощения. Зависимость предполагает, что поглощение происходит индивидуальными молекулами. В общем случае, для описания этой зависимости может быть использовано вириальное разложение
\begin{gather}
    \alpha = \sigma n + \alpha_\text{binary} n^2 + \alpha_\text{ternary} n^3 + O(n^4), 
\end{gather}
в котором квадратичное слагаемое описывает столкновительно-индуцированное поглощение бинарными комплексами, кубическое слагаемое -- тройными комплексами, и т.д. Стоит отметить, что сечение поглощения мономера $\sigma$ зависит от давления (проявляется это, например, в сдвигах и уширениях линий), поэтому при каждой фиксированной частоте поглощение мономера не является линейным по плотности $n$, однако интегральная интенсивность линейна по плотности. \par
Таким образом, для моделирования суммарного профиля поглощения необходимо знать сечение поглощения $\sigma$, коэффициент бинарного поглощения $\alpha_\text{binary}$, и т.д., и как эти эти величины зависят от атмосферных характеристик, таких как температура. Данная работа посвящена расчет коэффициентов бинарного поглощения $\alpha_\text{binary}$ для различных систем, имеющих практическую значимость в атмосферных приложениях. \par
Спектры плотных газов и газовых смесей существенно отличаются от спектров, зарегистрированных при низких плотностях. По мере увеличения плотности, наблюдается линейное увеличение интенсивностей разрешенных колебательно-вращательных и электронных полос. При промежуточных значениях плотности могут возникнуть новые полосы поглощения, которые не наблюдались при более низких плотностях, причем интенсивность этих полос (по крайней мере в первом приближении) может быть описана квадратичным или кубическим законом. За появление этих полос отвественны ван дер Ваальсовы комплексы двух или большего количества молекул. Подобные полосы поглощения найдены во многих молекулярных газах, даже молекулы которых не обладают постоянным дипольным моментом.  

Состояния ван дер Ваальсовых комплексов классифицируют на связанные, полная колебательно-вращательная энергия меньше энергии диссоциации, и свободные. 

Впервые квадратичную зависимость коэффициента поглощения от плотности наблюдал Йенсен (Janssen) в 1885 году на полосах поглощениях кислорода \cite{janssen1885}.
\fi


\chapter{Теоретическое введение}
\section{Взаимодействие электромагнитного излучения с молекулярными системами}
Рассмотрим систему из $N$ взаимодействующих частиц (молекул) в квантовом состоянии $\ket{j}$. Обозначим гамильтониан системы через $\hat{H}_0$. Пусть система подвергается воздействию электрического поля частоты $\omega$, которое индуцирует переходы в другие состояния $\ket{k}$ системы при условии, что частота поля близка к частотам Бора системы
%
\begin{gather}
    \omega_{jk} = (E_j - E_k) / \hbar.
\end{gather}

Будем считать, что длина волны рассматриваемого поля $\lambda$ много больше размеров молекул в системе, поэтому в локальной окрестности молекул поле можно считать однородным в пространстве.  Электрическая составляющая плоской волны может быть записана в виде суммы
%
\begin{gather}
    \mf{E}(t) = E_0 \boldsymbol{\varepsilon} \cos \omega t = \frac{E_0 \boldsymbol{\varepsilon}}{2} \lb e^{i \omega t} + e^{- i\omega t} \rb,
\end{gather}
%
где $E_0$ -- амплитуда волны, а $\boldsymbol{\varepsilon}$ -- единичный вектор, ориентированный вдоль направления распространения волны. Энергия взаимодействия молекулярной системы с электрическим полем в дипольном приближении равна
%
\begin{gather}
    W(t) = - (\boldsymbol{\mu} \cdot \mf{E}) = - \frac{E_0}{2} \lb \boldsymbol{\mu} \cdot \boldsymbol{\varepsilon} \rb \lsq e^{i \omega t} + e^{-i \omega t} \rsq, \label{part1-dipole-approximation} 
\end{gather}
%
где через $\boldsymbol{\mu}$ обозначен полный дипольный момент системы. Взаимодействие молекулярных систем с электромагнитным полем часто рассматривают в этом приближении, считая поле классическим объектом. \par  
Взаимодействие молекулярных систем с электрическим полем традиционно рассматривается в рамках временной теории возмущений \cite{cohentanuji, greiner}. Согласно приложению \ref{appendix:perturbation-theory}, вероятность индуцированного возмущением $W(t)$ перехода между состояниями невозмущенной системы $\ket{j} \rightarrow \ket{k}$ в первом порядке временной теории возмущения равна
%
\begin{gather}
    \mathcal{P}_{jk}(t) = \frac{1}{\hbar^2} \abs{ \int\limits_0^t W_{jk}(t^\prime) e^{i \omega_{jk} t^\prime} dt^\prime }^2,
\end{gather}
%
где через $W_{jk}(t)$ обозначен матричный элемент возмущения на состояниях невозмущенной системы, равный в данном случае
%
\begin{gather}
    W_{jk} = -\frac{E_0}{2} \bra{j} \boldsymbol{\mu} \cdot \boldsymbol{\varepsilon} \ket{k} \lsq e^{i \omega t} + e^{-i \omega t} \rsq.
\end{gather}

Коэффициенты разложения первого порядка $b_n^{(1)}(t)$ возмущенной волновой функции $\ket{\psi(t)}$ в базисе собственных функций невозмущенного гамильтониана равны (см. соотношение \eqref{app-expansion})
\begin{gather}
    b_{n}^{(1)}(t; \omega) = -\frac{E_0}{2 \hbar} \bra{j} \boldsymbol{\mu} \cdot \boldsymbol{\varepsilon} \ket{k} \lsq \frac{e^{i (\omega_{jk} + \omega) t} - 1}{\omega_{jk} + \omega} + \frac{e^{i (\omega_{jk} - \omega) t} - 1}{\omega_{jk} - \omega} \rsq. \label{part1-bn-expression}
\end{gather}
%
Квадрат модуля коэффициента $b_{n}^{(1)}$ определяет вероятность перехода в $n$-ое стационарное состояние невозмущенного гамильтониана. Возводя в квадрат выражение \eqref{part1-bn-expression}, после алгебраических преобразований приходим к
%
\begin{gather}
    \hspace*{-0.5cm}
    \abs{b_{n}^{(1)}(t)}^2 = \frac{E_0}{\hbar^2} \lsq \frac{\sin^2 \lb \frac{1}{2} \lb \omega_{jk} + \omega \rb t \rb}{\lb \omega_{jk} + \omega \rb^2} + \frac{\sin^2 \lb \frac{1}{2} \lb \omega_{jk} - \omega \rb t \rb}{\lb \omega_{jk} - \omega \rb^2} + \frac{ 8 \cos \lb \omega t \rb \sin \lb \frac{1}{2} \lb \omega_{jk} + \omega \rb t \rb \sin \lb \frac{1}{2} \lb \omega_{jk} - \omega \rb t \rb}{\lb \omega_{jk} + \omega \rb \lb \omega_{jk} - \omega \rb} \rsq. \label{part1-bn-expression2}
\end{gather}
 
В данном контексте нас интересуют не вероятности переходов, а скорости переходов $\Gamma_{jk}$ (иными словами, вероятности $P_{jk}$, отнесенные к единице времени) при больших значениях $t$
%
\begin{gather}
    \Gamma_{jk} = \lim_{t \rightarrow \infty} \frac{P_{jk}}{t}.
\end{gather}
%
При предельном переходе в \eqref{part1-bn-expression2} первые два слагаемые преобразуются к дельта-функ\-циям, в то время как последнее слагаемое оказывается нулевым \cite{baym-quantum-mechanics}. Итак, выражение для скорости переходов оказывается следующим \cite{mcquarrie-statistical-mechanics}  
%
\begin{gather}
    \Gamma_{jk} = \frac{\pi E_0^2}{2 \hbar^2} \abs{ \bra{j} \boldsymbol{\mu} \cdot \boldsymbol{\varepsilon} \ket{k} }^2 \Big[ \delta \lb \omega_{jk} - \omega \rb + \delta \lb \omega_{jk} + \omega \rb \Big]. \label{part1-transition-rate} 
\end{gather}

Выражение \eqref{part1-transition-rate} определяет скорость переходов между конкретными состояниями невозмущенной системы $\ket{j}$ и $\ket{k}$. Скорость поглощения энергии в ходе переходов между этими состояниями равна $\hbar \omega_{jk} \Gamma_{jk}$, т.к. в процессе одного акта поглощения система поглощает энергию, равную $\hbar \omega_{jk}$. Скорость поглощения энергии в ходе переходов с заданного уровня $\ket{j}$ может быть получена при суммировании по всем состояниям $\ket{k}$, которые доступны системе для перехода. Наконец, суммарная скорость поглощения энергии излучения системой $-\dot{E}_\text{rad}$ получается в результате суммирования по всем возможным начальным состояниям $\ket{j}$ с соответствующими заселенностями $\rho_j$
\begin{gather}
    -\dot{E}_\text{rad} = \sum_j \sum_k \rho_j \hbar \omega_{jk} \Gamma_{jk} = \frac{\pi E_0^2}{2 \hbar} \sum_{j, k} \omega_{jk} \rho_j \abs{ \bra{j} \boldsymbol{\mu} \cdot \boldsymbol{\varepsilon} \ket{k} }^2 \Big[ \delta \lb \omega_{jk} - \omega \rb + \delta \lb \omega_{jk} + \omega \rb \Big]. \label{part1-energy-absorption-rate}
\end{gather}

Для получения более симметричной формы уравнения \eqref{part1-energy-absorption-rate} осуществим алгебраические преобразования. Рассмотрим отдельно вторую сумму, получающуюся при раскрытии скобок в уравнении \eqref{part1-energy-absorption-rate}. Поменяем местами индексы $j \leftrightarrow k$, что обосновывается тем, что оба индекса пробегают по всем квантовым состояниям системы,
\begin{gather}
    \frac{\pi E_0^2}{2 \hbar} \sum_{j, k} \omega_{jk} \rho_j \abs{ \bra{j} \boldsymbol{\mu} \cdot \boldsymbol{\varepsilon} \ket{k} }^2 \delta \lb \omega_{jk} + \omega \rb = -\frac{\pi E_0^2}{2 \hbar} \sum_{j, k} \omega_{jk} \rho_j \abs{ \bra{j} \boldsymbol{\mu} \cdot \boldsymbol{\varepsilon} \ket{k} }^2 \delta \lb \omega_{jk} - \omega \rb. \label{part1-index-change}
\end{gather}

Подстановка \eqref{part1-index-change} в \eqref{part1-energy-absorption-rate} приводит к выражению, в котором индексы $j$ и $k$ входят симметричным образом 
%
\begin{gather}
    -\dot{E}_\text{rad} = \frac{\pi E_0^2}{2 \hbar} \sum_{j, k} \omega_{j k} \lb \rho_j - \rho_k \rb \abs{ \bra{j} \boldsymbol{\mu} \cdot \boldsymbol{\varepsilon} \ket{k} }^2 \delta \lb \omega_{jk} - \omega \rb.
\end{gather}

Т.к. мы предполагаем, что возмущение достаточно слабо и действует на протяжении малого промежутка времени, то будем считать, что в любой момент система находится в состоянии теплового равновесия при температуре $T$. Используя это предположение, выразим заселенность $k$-ого состояния через заселенность $j$-го состояния (понятно, что можно воспользоваться и обратной связью, т.к. мы специально привели формулу к симметричному относительно замены индексов виду)
%
\begin{gather}
    \rho_k = \rho_j \exp \lb - \beta \hbar \omega_{jk} \rb,
\end{gather}
%
где $\beta = \boltz T$ и $\boltz$ -- постоянная Больцмана. Кроме того, вследствие того, что внутри суммы находятся дельта-функции, центрированные на $\omega_{jk}$ , функции от $\omega$, вычисленные при частоте $\omega_{jk}$, могут быть вынесены из под знака суммы: 
\begin{align}
    -\dot{E}_\text{rad} &= \frac{\pi E_0^2}{2 \hbar} \sum_{j, k} \omega_{jk} \rho_j \lb 1 - \exp \lb - \beta \hbar \omega_{jk} \rb \rb \abs{ \bra{j} \boldsymbol{\mu} \cdot \boldsymbol{\varepsilon} \ket{k} }^2 \delta \lb \omega_{jk} - \omega \rb = \\
    &= \frac{\pi E_0^2}{2 \hbar} \omega \lb 1 - \exp \lb - \beta \hbar \omega \rb \rb \sum_{j, k} \rho_j \abs{ \bra{j} \boldsymbol{\mu} \cdot \boldsymbol{\varepsilon} \ket{k} }^2 \delta \lb \omega_{jk} - \omega \rb. 
\end{align}

Суммарный поток энергии $I$, переносимой электромагнитной волной через среду с показателем преломления $n$, равен усредненному по времени модулю вектора Пойнтинга $\langle S \rangle$ и равен \cite{mcquarrie-statistical-mechanics}
\begin{gather}
    I = \langle S \rangle = \frac{c}{8 \pi} n E_0^2,
\end{gather}
%
где $c$ -- скорость света в вакууме. Показатель поглощения среды $\alpha(\omega)$ определяют как отношение энергии, поглощаемой средой в единицу времени при частоте $\omega$, к энергии, переносимой электромагнитной волной в единицу времени \cite{mcquarrie-statistical-mechanics}
\begin{gather}
    \alpha(\omega) = \frac{-\dot{E}_\text{rad}}{I} = \frac{4 \pi^2}{\hbar c n} \omega \lb 1 - e^{- \beta \hbar \omega} \rb \sum_{j, k} \rho_j \abs{ \bra{j} \boldsymbol{\mu} \cdot \boldsymbol{\varepsilon} \ket{k}}^2 \delta \lb \omega_{jk} - \omega \rb. \label{part1-absorption-coefficient-definition}
\end{gather}

На основании выражения \eqref{part1-absorption-coefficient-definition} определяют спектральную функцию $J(\omega)$ \cite{gordon1968}
\begin{gather}
    J(\omega) = \frac{3 \hbar c n \alpha(\omega)}{4 \pi^2 \omega \lb 1 - e^{-\beta \hbar \omega} \rb} = 3 \sum_{j, k} \rho_j \abs{ \bra{j} \boldsymbol{\mu} \cdot \boldsymbol{\varepsilon} \ket{k} }^2 \delta \lb \omega_{jk} - \omega \rb. \label{part1-spectral-function-definition}
\end{gather}

Отметим, что обычно спектральную функцию определяют таким образом, чтобы ее интеграл по частотному диапазону был равен единице. Здесь принято несколько иное определение, эта функция не нормирована. Кроме того, спектральная функция может быть определена и для отрицательных частот, в этом случае она относится к испусканию излучения. \par
Альтернативную форму выражения \eqref{part1-spectral-function-definition} получают, осуществляя смену Шредингеровского представления квантовой механики на представление Гейзенберга. Состояния в представлении Гейзенберга не зависят от времени -- временная эволюция заложена в операторах. Эволюция оператора $\hat{A}(t)$ описывается оператором эволюции $\hat{U}(t)$
\begin{gather}
    \hat{A}(t) = \hat{U}^{+}(t) \hat{A}(0) \hat{U}(t) = e^{\frac{i}{\hbar} \hat{H} t} \hat{A}(0) e^{-\frac{i}{\hbar} \hat{H} t}. 
\end{gather}

Удобно перейти в выражении \eqref{part1-spectral-function-definition} к представлению Гейзенберга, представив дельта-функцию как Фурье-образ мнимой экспоненты
\begin{gather}
    \delta (\omega) = \frac{1}{2\pi} \int\limits_{-\infty}^\infty e^{i \omega t} dt,
\end{gather}
получаем
\begin{gather}
    J(\omega) = \frac{3}{2\pi} \sum_{j,k} \rho_j \bra{j} \boldsymbol{\mu} \cdot \bs{\varepsilon} \ket{k} \bra{k} \boldsymbol{\mu} \cdot \boldsymbol{\varepsilon} \ket{j} \int\limits_{-\infty}^\infty \exp \lsq \lb \frac{E_j - E_k}{\hbar} - \omega \rb t \rsq dt. \label{part1-spectral-function-2}
\end{gather}

Т.к. состояния $\ket{k}$ и $\ket{j}$ являются собственными состояниями гамильтониана $\hat{H}_0$, то
\begin{gather}
    \exp \lb -\frac{i}{\hbar} E_j t \rb \ket{j} = \exp \lb -\frac{i}{\hbar} \hat{H}_0 t \rb \ket{j}, \quad \exp \lb \frac{i}{\hbar} E_k t \rb \bra{k} = \exp \lb \frac{i}{\hbar} \hat{H}_0 t \rb \bra{k}. 
\end{gather}

Произведение матричного элемента и экспоненты с Боровской частотой, получаемое при внесении матричного элемента под интеграл в \eqref{part1-spectral-function-2}, легко переводится в представление Гейзенберга
\begin{gather}
    \exp \lb \frac{i}{\hbar} \lb E_k - E_j \rb t \rb \bra{k} \boldsymbol{\mu} \cdot \boldsymbol{\varepsilon} \ket{j} = \bra{k} \exp \lb \frac{i}{\hbar} \hat{H}_0 t \rb \boldsymbol{\mu} \cdot \boldsymbol{\varepsilon} \exp \lb -\frac{i}{\hbar} \hat{H}_0 t \rb \ket{j} = \bra{k} \boldsymbol{\mu}(t) \cdot \boldsymbol{\varepsilon} \ket{j}. \label{part1-matrix-element}
\end{gather}

Подставляя \eqref{part1-matrix-element} в \eqref{part1-spectral-function-2}, приходим к
\begin{gather}
    J(\omega) = \frac{3}{2\pi} \int\limits_{-\infty}^\infty \sum_{j,k} \rho_j \bra{j} \boldsymbol{\mu}(0) \cdot \boldsymbol{\varepsilon} \ket{k} \bra{k} \boldsymbol{\mu}(t) \cdot \boldsymbol{\varepsilon} \ket{j} e^{-i \omega t} dt.
\end{gather}

Учитывая соотношение замкнутости \eqref{app-completeness}, получим
\begin{gather}
    J(\omega) = \frac{3}{2\pi} \int\limits_{-\infty}^\infty \sum_j  \rho_{j} \bra{j} \boldsymbol{\mu}(0) \cdot \boldsymbol{\varepsilon} \cdot \boldsymbol{\mu}(t) \cdot \boldsymbol{\varepsilon} \ket{j} e^{-i\omega t} dt.
\end{gather}

Сумма в подынтегральном выражении является квантово-механическим средним по ансамблю значением оператора, которое в дальнейшем будем обозначать через $\langle \cdots \rangle$. Считая среду изотропной, проинтегрируем по всем возможным ориентациям $\boldsymbol{\varepsilon}$:
\begin{gather}
    J(\omega) = \frac{1}{2\pi} \intty \mean{ \bs{\mu}(0) \cdot \bs{\mu}(t) } e^{-i \omega t} dt. \label{litreview-spectral-function1}
\end{gather}

Итак, спектральная функция является Фурье-образом автокорреляционной функции оператора дипольного момента поглощающих молекул \cite{gordon1968}. При работе в единицах СИ в выражении \eqref{litreview-spectral-function} следует добавить переводной множитель
\begin{gather}
    J(\omega) = \frac{1}{4\pi \varepsilon_0} \frac{1}{2\pi} \intty \mean{ \bs{\mu}(0) \cdot \bs{\mu}(t) } e^{-i \omega t} dt. \label{litreview-spectral-function} 
\end{gather}
Следует подчеркнуть, что никаких приближений, касающихся характера движения диполей, в этом рассмотрении сделано не было. Движение системы полностью обусловлено уравнениями движения, определяемыми гамильтонианом системы $\hat{H}_0$. \par
Будем использовать следующее выражение для нормированного коэффициента поглощения через спектральную функцию 
\begin{gather}
    \alpha(\nu) = \frac{(2 \pi)^3 N_L^2}{3 \hbar c} \nu \lsq 1 - \exp \lb -\frac{h c \nu}{k T} \rb \rsq V J(\nu), \label{part1-absorption-coefficient-definition1}  
\end{gather}
%
где через $N_L$ обозначена постоянная Лошмидта, для частоты в обратных сантиметрах применяем обозначение $\nu$. 

\section{Теория временных функций корреляции и спектральных моментов} \label{section:correlation_functions}

Теория корреляционных функций получила широкое развитие для описания неравновесных систем \cite{zwanzig1965}, однако ее применение к равновесным системам также является эффективным. В системах, находящихся в термодинамическом равновесии, макроскопические параметры не претерпевают эволюции во времени, таким образом для них не имеет смысла вводить какой-то точки отсчета времени. Часто рассматривают условные вероятности, такие как $P(B, t_2 \vert A, t_1) dB$ -- вероятность того, что динамическая переменная $B$ примет значение в диапазоне $(B, \dots, B + dB)$ в момент времени $t_2$ при условии, что другая динамическая переменная имела значение $A$ в момент времени $t_1$ \cite{nitzan2006}. Также можно рассмотреть совместную вероятность $P(B, t_2; A, t_1) dB dA$ -- вероятность того, что переменная $A$ примет значение в диапазоне $(A, \dots, A+dA)$ в момент времени $t_1$ и переменная $B$ примет значение в диапазоне $(B, \dots B+dB)$ в момент времени $t_2$. Эти две вероятности связаны соотношением (формулой полной вероятности)
\begin{gather}
    P(B, t_2; A, t_1) = P(B, t_2 \vert A, t_1) P(A, t_1), 
\end{gather}
где $P(A, t_1) dA$ --вероятность того, что переменная $A$ примет значение в диапазоне $(A, \dots, A+dA)$ в момент времени $t_1$. В стационарных системах последняя вероятность, очевидно, не зависит от времени $P(A, t_1) = P(A)$; условная и совместная вероятности зависят только от разности времён
\begin{gather}
    P(B, t_2; A, t_1) = P(B, t_2 - t_1; A, 0), \quad P(B, t_2 \vert A, t_1) = P(B, t_2 - t_1 \vert A, 0),
\end{gather}
%
где $t = 0$ было положено произвольным образом.\par
Временную корреляционную функцию двух динамических переменных $A$ и $B$ определяют, как интеграл следующего вида 
\begin{gather}
    C_{AB}(t_1, t_2) = \mean{ A(t_1) B(t_2) } = \iint dA dB AB P(B, t_2; A, t_1). \label{part1-correlation-function-definition}
\end{gather}

В стационарных системах функция корреляции есть функция разности времен
\begin{gather}
    \mean{ A(t_1) B(t_2) } = \mean{ A(0) B(t) } = \mean{ A(-t) B(0) }, \quad t = t_2 - t_1.
\end{gather}

С точки зрения классической механики динамические переменные $A$, $B$ являются функциями координат $\mf{r}(t)$ и импульсов $\mf{p}(t)$ всех частиц системы
\begin{gather}
    A(t) = A \lb \mf{r}(t), \mf{p}(t) \rb, \quad B(t) = B \lb \mf{r}(t), \mf{p}(t) \rb.
\end{gather}

Фазовая траектория $\mf{r}(t)$, $\mf{p}(t)$ однозначно определена начальными условиями $\mf{r}(0)$, $\mf{p}(0)$. Таким образом, совместная вероятность в \eqref{part1-correlation-function-definition} определяется функцией распределения $f(\mf{r}, \mf{p})$ начальных условий для фазовых траекторий
\begin{gather}
    C_{AB}(t_1, t_2) = \int d\mf{r} \, d\mf{p} \, f \lb \mf{r}, \mf{p} \rb A\lb t_1; \mf{r}, \mf{p}, t = 0 \rb B \lb t_2; \mf{r}, \mf{p}, t = 0 \rb;
\end{gather}
%
обозначение $A(t_1; \mf{r}, \mf{p}, t = 0)$ означает, что динамическая переменная $A$ в момент времени $t_1$ вычисляется как функция координат и импульсов $A(\mf{r}(t_1), \mf{p}(t_1))$, вычисленных в момент времени $t_1$. \par
    При $t \rightarrow 0$ корреляционная функция $C_{AB}(t)$ становится средним значением произведения динамических переменных $A$ и $B$
\begin{gather}
    C_{AB}(0) = \mean{AB}.
\end{gather}
%
В другом пределе $t \rightarrow \infty$ можно предположить, что корреляция между переменными исчезает, то есть
\begin{gather}
    \lim_{t \rightarrow \infty} C_{AB}(t) = \mean{A} \mean{B}.
\end{gather}

Часто обьем системы, появляющийся в формуле \eqref{part1-absorption-coefficient-definition1}, рассматривают как часть спектральной функции $J(\omega)$ или рассматривают их произведение совместно. Поэтому в соответствии с обозначениями, используемыми в \cite{frommhold}, определим автокорреляционную функцию дипольного момента $C(t)$ как
\begin{gather}
    C(t) = \frac{1}{4 \pi \varepsilon_0} \frac{V}{2\pi} \mean{\bs{\mu}(0) \cdot \bs{\mu}(t)}. \label{litreview-autocorrelation-function-definition}
\end{gather}

Спектральная функция \eqref{litreview-spectral-function} принимает действительные значения, следовательно
\begin{gather}
    V J^+(\omega) = \intty C(t)^+ e^{i \omega t} dt = \intty C(-t)^+ e^{-i\omega t} dt = VJ(\omega), \label{litreview-time-inversion}
\end{gather}
%
где была сделана замена переменной $t \rightarrow -t$, а индекс $+$ обозначает комплексное сопряжение. Сранивая \eqref{litreview-time-inversion} с \eqref{litreview-spectral-function}, получаем
\begin{gather}
    C(t) = C(-t)^+.
\end{gather}

Если мы разложим корреляционную функцию на действительную и мнимую часть, то получим следующие соотношения 
\begin{gather}
    \text{Re} \, C(t) = \text{Re} \, C(-t), \quad \text{Im} \, C(t) = -\text{Im} \, C(t).
\end{gather}

То есть, действительная часть корреляционной функции является четной функции времени, а мнимая часть -- нечетной. Так как классические корреляционные функции являются действительными функциями, то они должны быть четными функциями времени. Квантово-механические же корреляционные функции, как правило, являются комплексными функциями, мнимая часть является исключительно квантово-механическим вкладом, отсутствующим при классическом рассмотрении.


Как уже отмечалось, корреляционные функции в равновесных системах зависят только от разности времени $t_1 - t_2$. Следовательно,
\begin{gather}
    0 = \frac{d}{ds} \mean{ A(t+s) B(s) } = \mean{ \dot{A}(t+s) B(s) } + \mean{ A(t+s) \dot{B}(s) } = \mean{ \dot{A}(t) B(0) } + \mean{ A(t) \dot{B}(0) }.
\end{gather}

Получаем следующее соотношение 
\begin{gather}
    \mean{\dot{A}(t)B(0)} = -\mean{A(t)\dot{B}(0)}, 
\end{gather}
которое для автокорреляционных функций переходит в 
\begin{gather}
    \mean{A \dot{A}} = 0. \label{litreview-temp1}
\end{gather}

Разложим автокорреляционную функцию в ряд по степеням времени $t$
\begin{gather}
    \mean{A(0) A(t)} = \Bigg\langle A(0) \lsq A(0) + t \dot{A}(0) + \frac{t^2}{2!} \ddot{A}(0) + \dots \rsq \Bigg\rangle = \mean{A(0) A(0)} + t \mean{A(0) \dot{A}(0)} + \frac{t^2}{2!} \mean{A(0) \ddot{A}(0)} + \dots \label{litreview-correlation-function-series}
\end{gather}

Отметим, что коэффициенты перед степенями $t^n$ не требуют знания динамики $A(t)$, а являются средними по ансамблю, т.к. временные производные могут быть записаны через скобку Пуассона
\begin{gather}
    \frac{dA}{dt} = \lsq A, H \rsq = \sum_k \lsq \frac{\partial A}{\partial x_k} \frac{\partial H}{\partial p_k} - \frac{\partial A}{\partial p_k} \frac{\partial H}{\partial x_k} \rsq. \label{litreview-poisson-bracket}
\end{gather}

Соотношение \eqref{litreview-temp1} соотносится с тем, что корреляционная функция является четной функцией. Из четности функции следует, что все коэффициенты перед нечетными степенями $t$ в \eqref{litreview-correlation-function-series} обращаются в нуль. В квантово-механической корреляционной функции нечетные степени не исчезают, и, более того, именно за счет них корреляционная функция обладает мнимой частью. \par
Коэффициенты в ряду по степеням времени \eqref{litreview-correlation-function-series} имеют физический смысл моментов соответствующего частотного спектра. Разрешим соотношение \eqref{litreview-spectral-function} относительно автокорреляционной функции дипольного момента и разложим комплексную экспоненту в подынтегральной функции в ряд
\begin{gather}
    C(t) = 2\pi V \intty J(\omega) e^{i \omega t} d \omega = 2 \pi V \intty J(\nu) e^{i 2\pi \nu c t} d\nu = 2 \pi \sum_{n = 0}^\infty \frac{\lb 2\pi i c t\rb^n}{n!} \intty \nu^n V J(\nu) d\nu. 
\end{gather}
%
Величины
\begin{gather}
    M_n = \intty \nu^n V J(\nu) d\nu \label{litreview-spectral-function-moments}
\end{gather}
%
называют $n$-ыми спектральными моментами. Теоретически знание всех спектральных моментов эквивалентно знанию спектральной функции (так называемая проблема моментов), однако на практике моменты выше второго находят редко. \par
Итак, спектральные моменты являются моментами спектральной функции и пропорциональны производным автокорреляционной функции дипольного момента в точке $t = 0$:
\begin{gather}
    \frac{1}{2 \pi} \frac{d^n C}{dt^n} \Bigg\vert_{t = 0} = \lb 2 \pi ic \rb^n M_n,
\end{gather}
%
что позволяет вычислить их как средние значения по фазовому пространству. Подробно рассматривать вопрос вычисления спектральных моментов не будем, приведем лишь выражения, по которым могут быть рассчитаны первые два спектральных момента:
\begin{gather}
    M_0 = 2 \pi \frac{\displaystyle \int \bs{\mu}^2 \exp \lsq -H(\mf{q}, \mf{p}) / \kb T \rsq d\mf{q} \, d\mf{p}}{\displaystyle \int \exp \lsq -H(\mf{q}, \mf{p}) / \kb T \rsq d\mf{q} \, d\mf{p}}, \label{litreview-m0-phase-space} \\
    M_2 = 2 \pi (2 \pi c)^2 \frac{\displaystyle \int \dot{\bs{\mu}}^2 \exp \lsq -H(\mf{q}, \mf{p}) / \kb T \rsq d\mf{q} \, d\mf{p}}{\displaystyle \int \exp \lsq -H(\mf{q}, \mf{p}) / \kb T \rsq d\mf{q} \, d\mf{p}}, \label{litreview-m2-phase-space}
\end{gather}
%
где производную по времени дипольного момента $\dot{\bs{\mu}}$ в выражении для второго спектрального момента следует преобразовать в скобку Пуассона \eqref{litreview-poisson-bracket}. Затем полученные значения интегралов следует привести к размерности см$^{-1} \cdot$Амага$^{-2}$ и  см$^{-3} \cdot$Амага$^{-2}$, соответственно. \par
В данной работе мы применяем спектральные моменты, полученные по формулам \eqref{litreview-m0-phase-space}, \eqref{litreview-m2-phase-space}, для контроля сходимости траекторного расчета. В траекторном расчете мы получаем спектральную функцию $J(\omega)$, моменты которой \eqref{litreview-spectral-function-moments} в пределе должны совпасть с моментами по фазовому пространству.

\chapter*{Приложение 2.A}
\addcontentsline{toc}{chapter}{Приложение 2.A. Временная теория возумещний}
{\Large\textbf{Временная теория возмущений}} \label{appendix:perturbation-theory}
\vspace{0.5cm}

Представленное ниже изложение основано на \cite{cohentanuji}. Рассмотрим физическую систему, описываемую гамильтонианом $\hat{H}_0$; пусть $E_n$ и $\ket{n}$ -- собственные значения и собственные векторы гамильтониана $\hat{H}_0$:
\begin{gather}
    \hat{H}_0 \ket{n} = E_n \ket{n}.
\end{gather}

Для простоты будем считать, что спектр гамильтониана $\hat{H}_0$ является дискретным и невырожденным. Дополнительно будем считать, что $\hat{H}_0$ не зависит явно от времени, и его собственные состояния являются стационарными. \par
В течении конечного интервала времени от $t = 0$ до $t = T$ к физической системе прикладывается возмущение, зависящее явно от времени, и гамильтониан принимает вид
\begin{gather}
    \hat{H}(t) = \hat{H}_0 + \lambda \hat{W}(t),
\end{gather}
где $\lambda$ -- малый вещественный безразмерный параметр, а $\hat{W}(t)$ -- оператор, равный нулю при $t < 0$. \par
Предполагаем, что в начальный момент времени система находится в стационарном состоянии $\ket{i}$, являющемся собственным состоянием оператора $\hat{H}_0$ с собственным значением $E_i$. В момент времени $t = 0$ приложения возмущения система начинает претерпевать эволюцию, т.к. состояние $\ket{i}$ в общем случае уже не будет собственным состоянием возмущенного гамильтониана. Нашей целью является вычисление вероятности $\mathcal{P}_{if}(t)$ найти систему в момент времени $t$ в другом собственном состоянии $\ket{f}$ гамильтониана $\hat{H_0}$. \par
Между моментами времени $0$ и $t$ система эволюционирует в соответствии с временным уравнением Шредингера:
\begin{gather}
    i \hbar \frac{d}{dt} \ket{\psi(t)} = \lsq \hat{H}_0 + \lambda \hat{W}(t) \rsq \ket{\psi(t)}, \quad \ket{\psi(t = 0)} = \ket{i}. \label{app-time-schroedinger}
\end{gather}

Искомая вероятность может быть записана в форме:
\begin{gather}
    \mathcal{P}_{if}(t) = \abs{ \braket{f | \, \psi(t)} }^2. \label{app-probability-definition} 
\end{gather}

Пусть $c_n(t)$ -- компоненты разложения кет-вектора $\ket{\psi(t)}$ в базисе $\big\{ \ket{n} \big\}$:
\begin{gather}
    \ket{\psi(t)} = \sum_n c_n(t) \ket{n}, \quad c_n(t) = \braket{n | \psi(t) }, \label{app-expansion}
\end{gather}
и $W_{nk}(t)$ -- матричные элементы оператора $\hat{W}(t)$ в том же базисе
\begin{gather}
    W_{nk}(t) \equiv \bra{n} \hat{W}(t) \ket{k}.
\end{gather}

Используя соотношение замкнутости
\begin{gather}
    \sum_k \ket{k} \bra{k} = 1, \label{app-completeness}
\end{gather}
умножим слева обе части временного уравнения Шредингера \eqref{app-time-schroedinger} на вектор состояния $\ket{n}$
\begin{gather}
    i \hbar \frac{d}{dt} c_n(t) = E_n c_n(t) + \sum_k \lambda W_{nk}(t) c_k(t). \label{app-time-dependent-eq1} 
\end{gather}

Уравнения \eqref{app-time-dependent-eq1}, записанные для разных $n$, образуют систему связанных дифференциальных уравнений, позволяющую определить компоненты $c_n(t)$ вектора $\ket{\psi(t)}$. \par
Если возмущение $\lambda \hat{W}(t)$ равно нулю, то уравнения \eqref{app-time-dependent-eq1} не связаны друг с другом, и их решение имеет форму
\begin{gather}
    c_n(t) = b_n e^{-i E_n t / \hbar}, \label{app-time-dependent-sol-nonperturb}
\end{gather}
где $b_n$ -- постоянные, зависящие от начальных условий. Это решение традиционно называют стационарным решением. \par
Если теперь рассмотреть систему с малым возмущением $\lambda \hat{W}(t) \neq 0$, то можно ожидать, что решение $c_n(t)$ уравнений \eqref{app-time-dependent-eq1} будет близким к решению \eqref{app-time-dependent-sol-nonperturb}. Таким образом, если выполнить замену функций
\begin{gather}
    c_n(t) = b_n(t) e^{-i E_n t / \hbar}, \label{app-function-change}
\end{gather}
то в случае малого возмущения мы ожидаем, что $b_n(t)$ будут медленно меняющимися функциями времени. Подставим \eqref{app-function-change} в уравнение \eqref{app-time-dependent-eq1} и получим
\begin{gather}
    i \hbar \frac{d}{dt} b_n(t) = \lambda \sum_k e^{i \omega_{nk} t} W_{nk}(t) b_k(t), \label{app-time-dependent-eq2} 
\end{gather}
где через $\omega_{nk}$ обозначены частоты Бора
\begin{gather}
    \omega_{nk} = \frac{E_n - E_k}{\hbar}.
\end{gather}

Система уравнений \eqref{app-time-dependent-eq2} абсолютно эквивалентна уравнению Шредингера \eqref{app-time-schroedinger}. Применим теорию возмущений для решения системы \eqref{app-time-dependent-eq2}. Будем искать решение в форме ряда по степеням $\lambda$
\begin{gather}
    b_n(t) = \bnp{0}(t) + \lambda \bnp{1}(t) + \lambda^2 \bnp{2}(t) + O(\lambda^3). \label{app-series}
\end{gather}

Подставив разложение \eqref{app-series} в \eqref{app-time-dependent-eq2} и приравняв коэффициенты при $\lambda^r$, находим
\begin{gather}
    \begin{aligned}
        i \hbar \frac{d}{dt} \bnp{0}(t) &= 0, \hspace{4.82cm} r = 0,  \\
        i \hbar \frac{d}{dt} \bnp{r}(t) &= \sum_k e^{i \omega_{nk} t} W_{nk}(t) b_k^{(r-1)}(t), \qquad r \neq 0.
    \end{aligned} \label{app-equations-for-coefficients}
\end{gather}

В соответствии с предположением, при $t < 0$ система находится в состоянии $\ket{i}$, следовательно, среди коэффициентов 
$b_n(t)$ отличен от нуля только $b_i(t)$
\begin{gather}
    b_n(t = 0) = \delta_{ni},
\end{gather}
и это равенство должно оставаться справедливым при любых значениях $\lambda$. Коэффициенты разложения \eqref{app-series} должны удовлетворять условиям:
\begin{gather}
    \begin{aligned}
        \bnp{0}(t = 0) &= \delta_{ni}, \\
        \bnp{r}(t = 0) &= 0, \quad r \geq 1.
    \end{aligned}  \label{app-initial-conditions-for-coefficients}
\end{gather}

Таким образом, решение нулевого порядка получается при $t > 0$:
\begin{gather}
    \bnp{0}(t) = \delta_{ni}.
\end{gather}

Этот результат позволяет переписать уравнение \eqref{app-equations-for-coefficients} для $r = 1$
\begin{gather}
    i \hbar \frac{d}{dt} \bnp{1}(t) = \sum_k e^{i \omega_{nk} t} W_{nk}(t) \delta_{ki} = W_{ni}(t) e^{i \omega_{ni} t}.
\end{gather}

С учетом начального условия \eqref{app-initial-conditions-for-coefficients} находим коэффициенты разложения первого порядка
\begin{gather}
    \bnp{1}(t) = \frac{1}{i \hbar} \int\limits_0^t W_{ni}(t^\prime) e^{i \omega_{ni} t^\prime} d t^\prime.
\end{gather}

Согласно выражению \eqref{app-probability-definition} вероятность перехода $\mathcal{P}_{if}(t)$ равна 
\begin{gather}
    \mathcal{P}_{if}(t) = \abs{ c_f(t) }^2 = \abs{ b_f(t) }^2.
\end{gather}

Допустим теперь, что состояния $\ket{i}$ и $\ket{f}$ являются различными, то есть, будем интересоваться переходами, индуцированными возмущением $\lambda \hat{W}(t)$, между двумя различными стационарными состояниями гамильтониана $\hat{H}_0$. Тогда $b_f^{(0)}(t) = 0$ и получим окончательно выражение для вероятности перехода (выполнена подстановка $\lambda = 1$) 
\begin{gather}
    \mathcal{P}_{if}(t) = \lambda^2 \abs{ b_f^{(1)}(t) }^2 = \frac{1}{\hbar^2} \abs{ \int\limits_0^t e^{i \omega_{fi} t^\prime} W_{fi}(t^\prime) dt^\prime }^2. \label{app-probability-general}
\end{gather}

Выражение \eqref{app-probability-general} показывает, что вероятность $\mathcal{P}_{if}(t)$ пропорциональна квадрату модуля преобразования Фурье матричного элемента возмущения $W_{fi}(t)$, взятого на частоте Бора, соответствующей рассматриваемому переходу. Т.к. возмущение действует в течение конечного интервала времени до $t = T$, то при $t \geq T$ коэффициент $b_m^{(1)}(t)$ становится постоянным:
\begin{gather}
    b_m^{(1)}(t) = \frac{1}{i \hbar} \int\limits_0^T W_{if}(t^\prime) e^{i \omega_{if} t^\prime} dt^\prime = \frac{1}{i \hbar} \int\limits_{-\infty}^\infty W_{if}(t^\prime) e^{i \omega_{if} t^\prime} dt^\prime.
\end{gather}

Используя Фурье преобразование матричного элемента $W_{if}(t)$
\begin{gather}
    W_{if}(\omega) = \frac{1}{2 \pi} \int\limits_{-\infty}^\infty W_{if}(t) e^{i \omega t} dt,
\end{gather}
приходим к следующему выражениям для коэффициента
\begin{gather}
    b_m^{(1)}(t) = \frac{2 \pi}{i \hbar} W_{if}(\omega_{if})
\end{gather}
и вероятности перехода
\begin{gather}
    \mathcal{P}_{if}(t) = \frac{4 \pi^2}{\hbar^2} \abs{ W_{if}(\omega_{if}) }^2, \quad t \geq T.
\end{gather}

В данном параграфе мы предполагали, что переход происходит между состояниями дискретного спектра невозмущенного оператора $\hat{H}_0$. Более того, мы предполагали, что невозмущенный оператор $\hat{H}_0$ обладает исключительно дискретным спектром. Если оператор $\hat{H}_0$ обладает и непрерывным спектром, то полный набор собственных функций состоит из
\begin{gather}
    \hat{H}_0 \ket{n} = E_n \ket{n}, \quad \hat{H}_0 \ket{\varphi, \alpha} = E(\alpha) \ket{\varphi, \alpha},
\end{gather}
где $\alpha$ -- непрерывный индекс, нумерующий состояния непрерывного спектра $\ket{\varphi, \alpha}$. Решение возмущенной задачи $\ket{\psi(t)}$ разложимо по полному набору собственных функций -- как дискретного, так и непрерывного спектра:
\begin{gather}
    \ket{\psi(t)} = \sum_n c_n(t) \ket{n} + \int c_\alpha(t) \ket{\varphi, \alpha} d\alpha.
\end{gather}

Изложенное выше рассмотрение может быть дополнено для учета непрерывной составляющей спектра невозмущенного оператора $\hat{H}_0$ \cite{greiner}. 

\iffalse
Рассмотрим гамильтониан $\hat{H}$, представимый в виде суммы разрешимого, независящего от времени гамильтониана $\hat{H}_0$ и возмущения $\hat{V}(t)$. В Шредингеровском представлении эволюция вектора состояния во времени определяется уравнением Шредингера
\begin{gather}
    i \hbar \frac{d}{dt} \ket{\alpha, t}_S = \hat{H} \ket{\alpha, t}_S,
\end{gather}
в то время как наблюдаемы стационарны во времени 
\begin{gather}
    i \hbar \frac{d}{dt} \hat{O} = 0.
\end{gather}

Исключим эволюцию состояния во времени, связанную с невозмущенным гамильтонианом $\hat{H}_0$, определив вектор состояния в \textit{представлении взаимодействия} 
\begin{gather}
    \ket{\alpha, t}_I = \exp \lb \frac{i \hat{H}_0 t}{\hbar} \rb \ket{\alpha, t}_S. 
\end{gather}

Рассморим временную эволюцию введенного состояния
\begin{gather}
    i \hbar \frac{d}{dt} \ket{\alpha, t}_I = i \hbar \frac{d}{dt} \exp \lb \frac{i \hat{H_0} t}{\hbar} \rb \ket{\alpha, t}_S = \lb i \hbar \frac{d}{dt} \exp \lb \frac{i \hat{H}_0 t}{\hbar}  \rb \rb \ket{\alpha, t}_S + \exp \lb \frac{i \hat{H}_0 t}{\hbar} \rb \lb i \hbar \frac{d}{dt} \ket{\alpha, t}_S \rb = \notag \\
    = -\hat{H}_0 \exp \lb \frac{i \hat{H}_0 t}{\hbar} \rb \ket{\alpha, t}_S + \exp \lb \frac{i \hat{H}_0 t}{\hbar} \rb \lb \hat{H}_0 + \hat{V} \rb \ket{\alpha, t}_S = \exp \lb \frac{i \hat{H_0} t}{\hbar} \rb \hat{V} \ket{\alpha, t}_S = \hat{V}_I \ket{\alpha, t}_I, \label{app1-time-evolution} 
\end{gather}
где было введено обозначение
\begin{gather}
    \hat{V}_I \equiv \exp \lb \frac{i \hat{H}_0 t}{\hbar} \rb \hat{V} \exp \lb - \frac{i \hat{H}_0 t}{\hbar} \rb.
\end{gather}

В отсутствие возмущения состояние в представлении взаимодействия является стационарным. В присутствии возмущения динамика состояния определяется уравнением \eqref{app1-time-evolution}, причем оператором временной эволюции становится $\hat{V}_I(t)$.
\fi


\chapter{Моделирование трансляционного столкновительно-индуцированного спектра смеси благородных газов} \label{chapter:two-atom}
    Наиболее простым видом столкновительно-индуцированных спектров являются трансляционные спектры, порождаемые смесью двух благородных газов при низком давлении, где доминируют бинарные столкновения. При более высоких давлениях будут случаться столкновения с участием трех и более атомов, которые будут видоизменять форму спектра поглощения. На рис. \ref{pic-two-atom-experiment} приведены примеры экспериментальных столкновительно-индуцированных спектров поглощения в дальней ИК области систем He$-$Ar, Ne$-$Ar и Ar$-$Kr \cite{frommhold}. Было экспериментально подтверждено, что интенсивность поглощения линейно зависит от произведения плотностей газов $\rho_1 \rho_2$, что говорит о том, что спектр порождается парами разных атомов. Отклонение от линейной зависимости будет говорить о том, что при данных концентрациях существенный вклад вносят многочастичные столкновения. Спектры, изображенные на рис \ref{pic-two-atom-experiment}, сняты при разных концентрациях от 60 амага (He$-$Ar) до 200 амага (Ar$-$Kr).  

\begin{figure}[H]
    \centering
    \includegraphics[width=0.7\linewidth]{./pictures/twoatom_experiment/experiment_diatom_spectra-crop.pdf}
    \label{pic-two-atom-experiment}
    \caption{Экспериментальные спектры бинарного поглощения систем гелий$-$аргон, неон$-$аргон и аргон$-$криптон при комнатной температуре \cite{frommhold}}
\end{figure}

В работе \cite{kranendonk1973} авторы разрабатывают формализм расчета столкновительно-индуцированного спектра в приближении бинарных столкновений. Авторы рассматривают систему, состояющую из молекулы $H_2$, возмущенной атомами $Ar$. Вращательное движение молекулы $H_2$ исключено из рассмотрения -- обе сталкивающихся молекулы рассматриваются как безструктурные сферически-симметричные частицы. \par
    Спектральная функция, определяющая профиль спектра поглощения, связана с функцией автокорреляции суммарного дипольного момента системы преобразованием Фурье
\begin{gather}
    J(\omega) = \intty \mean{ \bs{\mu}(0) \bs{\mu}(t) } e^{i \omega t} dt. \label{twoatom-spectral-function}
\end{gather} 
В приближении бинарных столкновений корреляционная функция суммарного дипольного момента становится 
\begin{gather}
    \mean{ \bs{\mu}(0) \bs{\mu}(t) } = N \mean{ \bs{\mu}_1(0) \bs{\mu}_1(t) },
\end{gather} 
%
где через $\bs{\mu_1}(t)$ обозначен дипольный момент индуцированный квадрупольным полем молекулы $H_2$ на атоме $Ar$, а $N$ -- количество рассматриваемых пар. Приведенную массу системы обозначают через $\mu$; вектор, соединяющий центр масс молекулы $H_2$ атомом $Ar$ -- через $\mathbf{R}$; потенциал взаимодействия -- через $V(R)$ $[$как уже говорилось, вращательное движение молекулы водорода не рассматривается, поэтому потенциал зависит только от расстояния между центрами масс $R$$]$. Автокорреляционную функцию дипольного момента приводят к виду 
\begin{gather}
    C(\tau) = \frac{N}{V} \lb \frac{\mu}{2 \pi k T} \rb^{3/2} \iint \bs{\mu}_1(\mathbf{R}) \cdot \bs{\mu}_1(\mathbf{R}(\tau)) \exp \lb -\frac{\mu \dot{\mf{R}}^2}{2 k T} \rb g_0(R) \, d \mathbf{R} \, d \dot{\mathbf{R}}, \label{kranendonk-correlation-function}
\end{gather}
%
где $\mathbf{R}(\tau)$ -- значение $\mathbf{R}$, вычисленное в момент времени $\tau$ путем расчета классической траектории, начальными условиями для которой взяты $\mf{R}$ и $\dot{\mf{R}}$, и $g_0(R)$ -- парная функция распределения (вероятность того, что между атомами расстояние $R$)
\begin{gather}
    g_0(R) = \exp \lb -\frac{V(R)}{kT} \rb.
\end{gather}

Выражение \eqref{kranendonk-correlation-function} неудобно для численного расчета, т.к. в нем имеется $R(\tau)$ для произвольного момента времени $\tau$. Для более эффективной вычислительной схемы предлагается переписать интегральное выражение \eqref{kranendonk-correlation-function} как интеграл по полным столкновительным траекториям. При этом будут рассматриваться только траектории рассеяния. В лабораторной системе отсчета энергия система может быть записана в виде
\begin{gather}
    E = \frac{1}{2} \mu \dot{\mf{R}}^2 + V(R).
\end{gather}
Траекторию рассеяния, которая имеет в момент времени $t$ радиус-вектор $\mf{R}$ и скорость $\dot{\mf{R}}$, может быть однозначно определена относительной скоростью $\mf{g}$ в момент времени $t = -\infty$, прицельным параметром $b$, углом, определяющим ориентацию плоскости столковения $\varepsilon$, и моментом времени $t_0$, в которое произошло столкновение. Применяя теорему Лиувилля
\begin{gather}
    d \mf{R} \, d\dot{\mf{R}} = g d(t - t_0) b db \, d\varepsilon \, d\mf{g},
\end{gather}
%
выражение \eqref{kranendonk-correlation-function} преобразуют к виду
\begin{gather}
    C(\tau) = \frac{N}{V} \lb \frac{\mu}{2 \pi k T} \rb^{3/2} \idotsint \bs{\mu}_1(t) \cdot \bs{\mu}_1(t + \tau) \, g \exp \lb - \frac{\mu \mf{g}^2}{2 k T} \rb b \, db \, d\varepsilon \, d \mf{g}. \label{correlation-function-kranendonk}
\end{gather}

Корреляцией двух функций $f$ и $g$, определенных на комплексной плоскости $\mathbb{C}$, называют функцию, определенную следующим интегралом
\begin{gather}
    C(\tau) = \intty f^{*}(t) g(\tau + t) dt,
\end{gather}
% 
где $*$ обозначает комплексное сопряжение. Обозначим через $F(\omega)$, $G(\omega)$ Фурье-образы функций $f(t)$, $g(t)$. Перепишем выражение для корреляции, представив функции через обратное преобразование Фурье от $F(\omega)$, $G(\omega)$, соответственно.
\begin{gather}
    C(\tau) = \intty \lsq \, \intty F^{*}(\omega) e^{-i \omega t} \frac{d \omega}{2 \pi} \rsq \lsq \, \intty G(\omega^\prime) e^{i \omega^\prime (\tau + t)} \frac{d\omega^\prime}{2 \pi} \rsq
\end{gather}

Осуществляя перестановку внутри интегрального выражения, приходим к следующему выражению 
\begin{gather}
    C(\tau) = \frac{1}{2\pi} \intty \intty F^*(\omega) G(\omega^\prime) e^{i \omega^\prime \tau} \lsq \, \intty e^{i (\omega^\prime - \omega) t} \frac{dt}{2 \pi} \rsq d\omega d\omega^\prime = \notag \\
    = \frac{1}{2\pi} \intty \inty F^*(\omega) G(\omega^\prime) e^{i \omega^\prime \tau} \delta \lb \omega^\prime - \omega \rb d\omega d\omega^\prime = \hat{F}^{-1} \Big[ F^*(\omega) G(\omega) \Big],
\end{gather}
%
где через $\hat{F}$ обозначен оператор преобразования Фурье. Если рассмотреть эту цепочку преобразований для автокорреляционной функции действительной функции $f(t)$, то приходим к теореме Винера-Хинчина \cite{frommhold}
\begin{gather}
    \hat{F} \Big[ C(\tau) \Big] = \Big\vert \hat{F}\Big[ f(t) \Big] \Big\vert^2. 
\end{gather}

Автокорреляционная функция дипольного момента распадается на сумму автокорреляционных функций его компонент
\begin{gather}
    C(\tau) = \intty \bs{\mu}_1(t) \bs{\mu}_1(t + \tau) dt = \sum_{\alpha = x, y, z} \intty \mu_1^\alpha(t) \mu_1^\alpha(t + \tau) dt = C_x(\tau) + C_y(\tau) + C_z(\tau).
\end{gather}

Следовательно, преобразование Фурье от автококорреляционной функции дипольного момента представляет собой сумму квадратов преобразований Фурье от компонент дипольного момента
\begin{gather}
    \hat{F}\Big[ C(\tau) \Big] = \sum_{\alpha = x,y,z} \hat{F} \Big[ C_\alpha(\tau) \Big] = \sum_{\alpha=x,y,z} \Bigg\vert \intty \mu_1^\alpha(t) e^{-i\omega t} dt \Bigg\vert^2,
\end{gather}
% 
что для краткости обозначают 
\begin{gather}
    \hat{F} \Big[ C(\tau) \Big] = \Bigg\vert \intty \bs{\mu}_1(t) e^{-i\omega t} dt \Bigg\vert^2. \label{correlation-theorem}
\end{gather}

Итак, преобразование Фурье от автокорреляционной функции \eqref{correlation-function-kranendonk} дает спектральную функцию  
\begin{gather}
    J(\omega) = \frac{N}{V} \lb \frac{\mu}{2 \pi k T} \rb^{3/2} \idotsint \Bigg\vert \intty \bs{\mu}_1(t) e^{-i \omega t} dt \Bigg\vert^2 \, \exp \lb -\frac{\mu g^2}{2 k T} \rb b \, db \, d \varepsilon \, 4 \pi g^3 dg.
\end{gather}

\section{Cистемы координат для описания движения двух атомов}

Рассмотрим движение двух атомов с массами $m_1$, $m_2$ с радиус - векторами $\mf{r}_1$, $\mf{r}_2$ в поле межатомного потенциала $U(\vert \mf{r}_1 - \mf{r}_2 \vert)$. Задача о движении двух взаимодействующих атомов сводится к задаче о движении виртуальной частицы с приведенной массой $\mu$,  равной 
\begin{gather}
    \mu = \frac{m_1 m_2}{m_1 + m_2}, 
\end{gather}
%
в заданном потенциальном поле $U$ \cite{landau-volume1}. Для описания движения виртуальной частицы введем несколько систем координат. Системой I будем называть декартову систему координат -- положение частицы задается вектором $\mf{r} = \mf{r}_1 - \mf{r}_2$. В этой системе координат лагранжиан и гамильтониан системы записываются как 
\begin{gather}
    \mL_\text{cartesian} = \frac{\mu \dot{\mf{r}}^2}{2} - U( \vert \mf{r} \vert ), \label{two-atom-cartesian-lagrangian} \\
    \mH_\text{cartesian} = \frac{\mf{p}^2}{2\mu} + U( \vert \mf{r} \vert ), \label{two-atom-cartesian-hamiltonian}
\end{gather}
%
где вектор импульса $\mf{p}$ равен
\begin{gather}
    \mf{p} = \frac{\partial \mL_\text{cartesian}}{\partial \dot{\mf{r}}} = \mu \, \dot{\mf{r}}. \label{two-atom-cartesian-momenta}
\end{gather}

Вектор $\mf{r}$ можно представить в сферической системе координат -- длину вектора обозначим через $r$, зенитный и азимутальный углы через $\theta$ и $\varphi$, соответственно ($\theta \in [0, \pi], \phi \in [0, 2 \pi]$. Будем называть эту координатную систему системой II. Лагранжиан и гамильтониан в ней записываются как
\begin{gather}
    \mL_\text{spherical} = \frac{1}{2} \mu \dot{r}^2 + \frac{1}{2} \mu r^2 \dot{\theta}^2 + \frac{1}{2} \mu r^2 \dot{\varphi}^2 \sin^2 \theta - U(r),  \label{two-atom-spherical-lagrangian} \\
    \mH_\text{spherical} = \frac{p_r^2}{2 \mu} + \frac{p_\theta^2}{2 \mu r^2} + \frac{p_\varphi^2}{2 \mu r^2 \sin^2 \theta} + U(r), \label{two-atom-spherical-hamiltonian}
\end{gather}
%
где обобщенные импульсы $p_r$, $p_\theta$, $p_\varphi$ связаны с обобщенными скоростями соотношениями
\begin{gather}
    p_r = \frac{\partial \mL_\text{spherical}}{\partial \dot{r}} = \mu \dot{r}, \quad \dot{r} = \frac{p_r}{mu} \label{two-atom-spherical-momenta1} \\
    p_\theta = \frac{\partial \mL_\text{spherical}}{\partial \dot{\theta}} = \mu r^2 \dot{\theta}, \quad \dot{\theta} = \frac{p_\theta}{\mu r^2} \label{two-atom-spherical-momenta2} \\
    p_\varphi = \frac{\partial \mL_\text{spherical}}{\partial \dot{\varphi}} = \mu r^2 \dot{\varphi} \sin^2 \theta \quad \dot{\phi} = \frac{p_\varphi}{\mu r^2 \sin^2 \theta} \label{two-atom-spherical-momenta3}.
\end{gather}

Декартовы координаты виртуальной частицы связаны со сферическими координатами следующими соотношениями
\begin{gather}
    \lc
    \begin{aligned}
        x &= r \cos \varphi \sin \theta \\
        y &= r \sin \varphi \sin \theta \\
        z &= r \cos \theta
    \end{aligned}
    \right. \label{two-atom-spherical-coordinates}
\end{gather}

Рассмотрим, как связаны декартовы импульсы $\mf{p}$ с импульсами $p_r$, $p_\theta$, $p_\phi$, сопряженными сферическим координатам. Для этого продифференцируем соотношения \eqref{two-atom-spherical-coordinates} по времени и умножим обе части на приведенную массу $\mu$, получив в левой части компоненты вектора $\mf{p}$ согласно \eqref{two-atom-cartesian-momenta}, а в правой части подставим выражения обобщенных скоростей $\dot{r}$, $\dot{\theta}$, $\dot{\varphi}$ через соответствующие импульсы \eqref{two-atom-spherical-momenta1}, \eqref{two-atom-spherical-momenta2}, \eqref{two-atom-spherical-momenta3}
\begin{gather}
    \lc 
    \begin{aligned}
        p_x &= p_r \cos \varphi \sin \theta + \frac{p_\theta}{r} \cos \varphi \cos \theta - \frac{p_\varphi}{r} \frac{\sin \varphi}{\sin \theta} \\ 
        p_y &= p_r \sin \varphi \sin \theta + \frac{p_\theta}{r} \sin \varphi \cos \theta + \frac{p_\varphi}{r} \frac{\cos \varphi}{\sin \theta} \\ 
        p_z &= p_r \cos \theta - \frac{p_\theta}{r} \sin \theta 
    \end{aligned}
    \right. \label{two-atom-cartesian-spherical-momenta}
\end{gather}

Разрешая линейные соотношения \eqref{two-atom-cartesian-spherical-momenta} относительно импульсов $p_r$, $p_\theta$, $p_\varphi$, находим соотношения, выражающие обратную связь импульсов.
\begin{gather}
    \lc
    \begin{aligned}
        p_r &= r \lb p_x \cos \varphi \sin \theta + p_y \sin \varphi \sin \theta + p_z \cos \theta \rb\\
        p_\varphi &= r \sin \theta \lb p_y \cos \varphi - p_x \sin \varphi \rb \\
        p_\theta &= r \lb p_x \cos \varphi \cos \theta + p_y \sin \varphi \cos \theta - p_z \sin \theta \rb  
    \end{aligned}
    \right.
\end{gather}

Выразим компоненты углового момента через координаты и импульсы системы II, пользуясь соотношениями \eqref{two-atom-cartesian-spherical-momenta}
\begin{gather}
    \mf{J} = \lsq \mf{r} \times \mf{p} \rsq = 
    \begin{bmatrix}
        -p_\theta \sin \varphi - p_\varphi \cos \varphi \cot \theta \\
        -p_\varphi \sin \varphi \cot \theta + p_\theta \cos \varphi \\
        p_\varphi
    \end{bmatrix}. \label{two-atom-angular-momenta-spherical} 
\end{gather}

Известно, что в отсутствии внешнего момента сил движение двухатомной системы происходит в плоскости, перпендикулярной вектору углового момента $\mf{J}$ \cite{goldstein}. Следовательно, движение системы можно описать при помощи полярных координат $r, \psi$, определенных в плоскости, и соответствующих обобщенных скоростей $\dot{r}$, $\dot{\psi}$. Ориентацию плоскости будем задавать при помощи пары сферических углов $\Phi$, $\Theta$, описывающих ориентацию вектора углового момента. Определим систему координат таким образом, чтобы координатные оси $OXY$ совпадали с плоскостью движения, а ось $OZ$ была сонаправлена с вектором углового момента $\mf{J}$. Будет называть эту координатную систему системой III; лагранжиан и гамильтониан в ней равны 
\begin{gather}
    \mL_\text{plane} = \frac{1}{2} \mu \dot{r}^2 + \frac{1}{2} \mu r^2 \dot{\psi}^2 - U(r), \label{two-atom-plane-lagrangian} \\
    \mH_\text{plane} = \frac{p_r^2}{2\mu} + \frac{p_\psi^2}{2 \mu r^2} + U(r), \label{two-atom-plane-hamiltonian} 
\end{gather}
%
где обобщенные импульсы $p_r$, $p_\psi$ связаны с обобщенными скоростями следующими соотношениями
\begin{gather}
    p_r = \frac{\partial \mL_\text{plane}}{\partial \dot{r}} = \mu \dot{r}, \quad \dot{r} = \frac{p_r}{\mu} \label{two-atom-plane-momenta1} \\
    p_\psi = \frac{\partial \mL_\text{plane}}{\partial \dot{\psi}} = \mu r^2 \dot{\psi}, \quad \dot{\psi} = \frac{p_\psi}{\mu r^2}.  \label{two-atom-plane-momenta2}
\end{gather}

Перевод полярных координат $r$, $\psi$ системы III в декартовы координаты $\mf{r} = \lc x, y, z \rc$ системы I можно осуществить при помощи ортогональной матрицы вращения $\bbS$, параметризованной углами $\Phi$, $\Theta$ \cite{goldstein} 
\begin{gather}
    \begin{bmatrix}
        x \\ y \\ z
    \end{bmatrix} = \bbS_\Phi^{-1} \bbS_\Theta^{-1} 
    \begin{bmatrix}
        r \cos \psi \\ r \sin \psi \\ 0
    \end{bmatrix}, \label{two-atoms-coordinate-transformation}
\end{gather}
% 
где матрицы поворота $\bbS_\Phi$, $\bbS_\Theta$ определены равны
\begin{gather}
    \bbS_\Phi = 
    \begin{bmatrix}
       -\sin \Phi & \cos \Phi & 0 \\
       -\cos \Phi & -\sin \Phi & 0 \\
      0 & 0 & 1
    \end{bmatrix}, \quad
    \bbS_\Theta = 
    \begin{bmatrix}
        1 & 0 & 0 \\
        0 & \cos \Theta & \sin \Theta \\
        0 & -\sin \Theta & \cos \Theta
    \end{bmatrix}.
\end{gather}

Раскрывая матричное выражение \eqref{two-atoms-coordinate-transformation}, получаем 
\begin{gather}
    \left\{
        \begin{aligned}
            x &= -r \cos \psi \sin \Phi - r \sin \psi \cos \Phi \cos \Theta \\
            y &= r \cos \psi \cos \Phi - r \sin \psi \sin \Phi \cos \Theta \\
            z &= r \sin \psi \sin \Theta
        \end{aligned}
    \right. \label{two-atoms-coordinates-transformation2}
\end{gather}

Продифференцируем соотношения \eqref{two-atoms-coordinates-transformation2} по времени, учитывая, что углы $\Phi$, $\Theta$ от времени не зависят.
\begin{gather}
    \begin{bmatrix} \dot{x} \\ \dot{y} \\ \dot{z} \end{bmatrix} = 
    \bbS_\Phi^{-1} \bbS_\Theta^{-1}
    \begin{bmatrix} 
        \dot{r} \cos \psi - r \dot{\psi} \sin \psi \\
        \dot{r} \sin \psi + r \dot{\psi} \cos \psi \\
        0 
    \end{bmatrix} \\
    \lc
    \begin{aligned}
        \dot{x} &= - \dot{r} \lb \cos \psi \sin \Phi + \sin \psi \cos \Phi \cos \Theta \rb + r \dot{\psi} \lb \sin \psi \sin \Phi - \cos \psi \cos \Phi \cos \Theta \rb \\ 
        \dot{y} &= \dot{r} \lb \cos \psi \cos \Phi - \sin \psi \sin \Phi \sin \Theta \rb - r \dot{\psi} \lb \sin \psi \cos \Phi - \cos \psi \sin \Phi \cos \Theta \rb \\
        \dot{z} &= \dot{r} \sin \psi \sin \Theta + r \dot{\psi} \cos \psi \sin \Theta
    \end{aligned}
\right. \label{two-atoms-coordinates-transformation3}
\end{gather}

При рассмотрении средних значений функций по фазовому пространству нам понадобятся выражения импульсов $\mf{p}$ через импульсы $p_r$, $p_\psi$. При умножении левых частей соотношений \eqref{two-atoms-coordinates-transformation3} на приведенную массу $\mu$ мы получим компоненты вектора $\mf{p}$ (согласно \eqref{two-atom-cartesian-momenta}). Подставив выражения обобщенных скоростей $\dot{r}$, $\dot{\psi}$ через импульсы $p_r$, $p_\psi$ \eqref{two-atom-plane-momenta1}, \eqref{two-atom-plane-momenta2}, получаем
\begin{gather}
    \lc
    \begin{aligned}
        p_x &= -p_r \lb \sin \psi \cos \Phi \cos \Theta + \cos \psi \sin \Phi \rb + \frac{p_\psi}{r} \lb \sin \psi \sin \Phi - \cos \psi \cos \Phi \cos \Theta \rb \\
        p_y &= p_r \lb \cos \psi \cos \Phi - \sin \Psi \sin \Phi \cos \Theta \rb - \frac{p_\psi}{r} \lb \sin \psi \cos \Phi + \cos \psi \sin \Phi \cos \Theta \rb \\
        p_z &= p_r \sin \psi \sin \Theta + \frac{p_\psi}{r} \cos \psi \sin \Theta 
    \end{aligned}
\right. \label{two-atom-momenta-transformation}
\end{gather}

Найдем координаты вектора углового момента через координаты системы III, исходя из определения вектора углового момента 
\begin{gather}
    \mf{J} = \mu \lsq \mf{r} \times \dot{\mf{r}} \rsq = 
    \begin{bmatrix}
        \mu r^2 \dot{\psi} \cos \Phi \sin \Theta \\ 
        \mu r^2 \dot{\psi} \sin \Phi \sin \Theta \\
        \mu r^2 \dot{\psi} \cos \Theta
    \end{bmatrix},
\end{gather}
или, пользуясь соотношением между скоростью $\dot{\psi}$ и импульсом $p_\psi$ \eqref{two-atom-plane-momenta2}, 
\begin{gather}
    \mf{J} = 
    \begin{bmatrix}
        p_\psi \cos \Phi \sin \Theta \\
        p_\psi \sin \Phi \sin \Theta \\
        p_\psi \cos \Theta
    \end{bmatrix}. \label{two-atom-plane-angular-momenta}
\end{gather}

Выражение \eqref{two-atom-plane-angular-momenta} подтверждает, что углы $\Phi$, $\Theta$ действительно являются сферическими углами для вектора углового момента. Кроме того, замечаем, что импульс $p_\psi$ имеет физический смысл модуля вектора углового момента. Этот факт, впрочем, может быть понят из вида гамильтониана \eqref{two-atom-plane-hamiltonian}. При составлении гамильтониана для движения в плоскости мы использовали два интеграла движения -- например, постоянство двух сферических углов, задающих ориентацию вектора углового момента. Координата $\psi$ является циклической для гамильтониана \eqref{two-atom-plane-hamiltonian} из чего следует, что $p_\psi$ является интегралом движения. Поэтому можно предположить, что импульс $p_\psi$ соответствует третьему интегралу движения -- модулю вектора углового момента, что и подтверждает приведенное рассмотрение. \par
    Соотношения \eqref{two-atoms-coordinates-transformation2}, \eqref{two-atoms-coordinates-transformation3} позволяют перейти от координат системы III к координатам системы I.    
    \color{red}{Понадобятся ли все остальные переходы?}
\color{black}{}

\section{Усреднение функций по фазовому пространству в разных системах координат} \label{section:averaging}

Рассмотрим усреднение некоторой функции $f(\mf{r}, \mf{p})$ по фазовому пространству двухатомной системы, где $\mf{r}$, $\mf{p}$ -- векторы декартовых координат и сопряженных импульсов (система I). 
\begin{gather}
    \mean{f} = \idotsint f(\mf{r}, \mf{p}) \exp \lb -\frac{\mH(\mf{r},\mf{p})}{kT} \rb d \mf{r} \, d\mf{p} \label{two-atom-mean}
\end{gather}

Целью нашего рассмотрения будет нахождение выражений, позволяющих производить усреднение функции $f$ по фазовому пространству, пользуясь координатами систем II и III. \par
Рассмотрим систему совокупную систему уравнений \eqref{two-atoms-coordinates-transformation2}, \eqref{two-atom-momenta-transformation} и найдем якобиан замены переменных $\lc x, y, z, p_x, p_y, p_z \rc$ $\rightarrow$ $\lc r, p_r, \psi, p_\psi, \Phi, \Theta \rc$. Ввиду громоздкости выкладки приводить не будем, выражение для якобиана получается следующее
\begin{gather}
    \text{Jac} = \Bigg\vert \frac{\partial \lsq x, y, z, p_x, p_y, p_z \rsq}{\partial \lsq r, p_r, \psi, p_\psi, \Phi, \Theta \rsq} \Bigg\vert = p_\psi \sin \Theta. \label{two-atom-planar-jacobian}
\end{gather}

Итак, среднее значение \eqref{two-atom-mean} в системе координат III записывается как
\begin{gather}
    \mean{ f } = \int\limits_{0}^{\infty} dr \intty dp_r \int\limits_0^{2\pi} d\psi \int\limits_0^\infty p_\psi dp_\psi \int\limits_0^\pi \sin \Theta d\Theta \int\limits_0^{2\pi} d\Phi f(r, \psi, p_r, p_\psi, \Theta, \Phi) \exp \lb -\frac{\mH_\text{plane}}{k T} \rb. \label{two-atom-mean-plane1} 
\end{gather}

Если усредняемая функция $f(r, \psi, p_r, p_\psi, \Theta, \Phi)$ не зависит от углов $\Theta$, $\Phi$, то среднее значение  \eqref{two-atom-mean-plane1} приходит к виду
\begin{gather}
    \mean{ f } = 4 \pi \int\limits_{0}^{\infty} dr \intty dp_r \int\limits_0^{2\pi} d\psi \int\limits_0^\infty p_\psi dp_\psi f(r, \psi, p_r, p_\psi, \Theta, \Phi) \exp \lb -\frac{\mH_\text{plane}}{k T} \rb. \label{two-atom-mean-plane2} 
\end{gather}

Как уже отмечалось ранее, импульс $p_\psi$ имеет физический смысл модуля углового момента, поэтому область интегрирования этого импульса составляет полуось $(0, +\infty)$, в то время как для радиального импульса -- вся прямая $(-\infty, +\infty)$. \par
Аналогично, рассмотрим совокупную систему уравнений \eqref{two-atom-spherical-coordinates}, \eqref{two-atom-cartesian-spherical-momenta} и найдем якобиан замены переменных $\lc x, y, z, p_x, p_y, p_z \rc$ $\rightarrow$ $\lc r, p_r, \varphi, p_\varphi, \theta, p_\theta \rc$. Якобиан оказывается единичным 
\begin{gather}
    \text{Jac} = \Bigg\vert \frac{\partial \lsq x, y, z, p_x, p_y, p_z \rsq}{\partial \lsq r, p_r, \varphi, p_\varphi, \theta, p_\theta \rsq} \Bigg\vert = 1. 
\end{gather}

Таким образом, среднее значение \eqref{two-atom-mean} в системе координат II записывается как
\begin{gather}
    \mean{f} = \int\limits_0^\infty dr \intty dp_r \int\limits_0^{2\pi} d\varphi \intty dp_\varphi \int\limits_0^\pi d\theta \intty dp_\theta f(r, p_r, \varphi, p_\varphi, \theta, p_\theta) \exp \lb -\frac{\mH_\text{spherical}}{k T} \rb. \label{two-atom-mean-spherical}
\end{gather}

\section{Распределения координат и импульсов в фазовом пространстве в разных системах координат в условиях канонического ансамбля}

Рассмотрим вопрос распределения координат и импульсов в фазовом пространстве в системах координат II и III в условиях канонического ансамбля. Функция распределения в фазовом пространстве в условиях канонического ансамбля задана гамильтонианом системы $\mH$ \cite{hill} 
\begin{gather}
    \rho \lb \mf{q}, \mf{p} \rb = \Gamma_0 \exp \lb -\frac{\mH \lb \mf{q}, \mf{p} \rb}{\kb T} \rb,
\end{gather}
% 
где постоянная $\Gamma_0$ определяется из условия нормировки функции распределения
\begin{gather}
    \int \rho \lb \mf{q}, \mf{p} \rb d \mf{q} \, d\mf{p} = 1.
\end{gather}

Рассмотрим распределения угловых координат $\theta, \varphi$ и импульсов $p_r$, $p_\theta$, $p_\varphi$ системы II при фиксированном большом значении межатомного расстояния $r_\text{fixed} \gg 1$. Пренебрежем значением потенциала $U(r_\text{fixed}) \approx 0$ на расстоянии $r_\text{fixed}$. Удобно представить отношение $\mH / \kb T$ в виде трех квадратичных членов $\lc \frac{1}{2} x_j^2 \rc_{j = 1 \dots 3}$
\begin{gather}
    \frac{\mH_\text{spherical}}{\kb T} = \frac{p_r^2}{2 \mu \kb T} + \frac{p_\theta^2}{2 \mu r_\text{fixed}^2 \kb T} + \frac{p_\varphi^2}{2 \mu r_\text{fixed}^2 \kb T \sin^2 \theta} = \frac{1}{2} x_1^2 + \frac{1}{2} x_2^2 + \frac{1}{2} x_3^2, \label{two-atom-spherical-hamiltonian-xs} 
\end{gather}
%
где переменные $x_j$ выражены как
\begin{gather}
    \lc
    \begin{aligned}
        x_1 &= \frac{p_r}{\sqrt{\mu \kb T}} \\
        x_2 &= \frac{p_\theta}{\sqrt{\mu r_\text{fixed}^2 \kb T}} \\
        x_3 &= \frac{p_\varphi}{\sqrt{\mu r_\text{fixed}^2 \kb T \sin^2 \theta}}
    \end{aligned}
\right. \label{two-atom-xs}
\end{gather}

Переписав гамильтониан в виде \eqref{two-atom-spherical-hamiltonian-xs}, мы видим, что вероятность нахождения системы в элементе фазового объема $d\theta d\varphi dx_1 dx_2 dx_3$ пропорциональна произведению 
\begin{gather}
    \rho \lb \theta, \varphi, x_1, x_2, x_3 \rb \propto \rho_1(x_1) \rho_1(x_2) \rho_1(x_3) \sin \theta, \label{two-atom-xs-phase-element}
\end{gather}
%
где случайные величины $x_j$ распределены по нормальному закону
\begin{gather}
    \rho_1 (x_j) = \frac{1}{\sqrt{2 \pi}} \exp \lb -\frac{x_j^2}{2} \rb. 
\end{gather}

Соотношения \eqref{two-atom-xs} позволяют установить следующие функции распределения для двух импульсов
\begin{gather}
    \lc
    \begin{aligned}
        p_r &\sim \mN \lb 0, \mu \kb T \rb \\
        p_\theta &\sim \mN \lb 0, \mu r_\text{fixed}^2 \kb T \rb 
    \end{aligned},
    \right.
\end{gather}
%
где через $\mN \lb \mu, \sigma^2 \rb$ обозначено нормальное распределение со математическим ожиданием $\mu$ и дисперсией $\sigma^2$. Импульс $p_\varphi$ представляет собой произведение двух случайных величин
\begin{gather}
    p_\varphi = x_3 \cdot \sin \Theta, \label{two-atom-pvarphi-generation}
\end{gather}
% 
где величина $x_3$ распределена по нормальному распределению $\mN \lb 0, \mu r_\text{fixed}^2 \kb T \rb$, а плотность распределения случайной величины $\Theta$ в силу \eqref{two-atom-xs-phase-element} равна
\begin{gather}
    \rho(\Theta) = \frac{1}{2} \sin \Theta.
\end{gather}

Численная генерация случайных величин $p_\varphi$ легко осуществляется по выражению \eqref{two-atom-pvarphi-generation}, однако интересно получить аналитическое выражение для плотности распределения, так как похожие распределения возникают при рассмотрении импульсов в многоатомных ван-дер-Ваальсовых комплексах. Сналчала получим плотность распределения величины $\sin \Theta$, используя то, что $\cos \Theta$ распределен равномерно на отрезке $\lsq -1, 1\rsq$. Очевидно, что в области определения зенитного угла $\lsq 0, \pi \rsq$ знак $\sin \Theta$ определен однозначно, поэтому можем выразить синус угла $\Theta$ через его косинус
\begin{gather}
    \sin \Theta = \sqrt{ 1 - \cos^2 \Theta}.
\end{gather}
Воспользуемся формулой преобразования случайной величины $Y = g(X)$
\begin{gather}
    \rho_Y(y) = \Big\vert \frac{d}{dy} g^{-1}(y) \Big\vert \cdot \rho_X(g^{-1}(y)), \label{distribution-change}
\end{gather}
%
где через $\rho_X(x)$, $\rho_Y(y)$ обозначены плотности случайных величин $X$, $Y$, соответственно. В данном случае преобразование осуществляется функцией $g(x) = \sqrt{1 - x^2}$, подставив которую в \eqref{distribution-change} приходим к следующей плотности распределения  
\begin{gather}
    \rho_{\sin \Theta}(x) = \frac{x}{\sqrt{1 - x^2}} \cdot \bbI\lsq 0, 1\rsq,
\end{gather}
% 
где через $\bbI\lsq 0, 1 \rsq$ обозначена индикаторная функция, ограничивающая носитель функции отрезком $\lsq 0, 1 \rsq$. Отметим, что полученное распределение является частным случаем распределения Кумарасвами с параметрами $a = 2$, $b = 1/2$. \par
Плотность распределения $\rho_{p_\varphi}$ может быть получена по стандартной формуле плотности случайной величины, являющейся произведением двух других случайных величин
\begin{gather}
    \rho_{p_\varphi}(z) = \intty \rho_{x_3}(z/x) \rho_{\sin \Theta}(x) \frac{dx}{\vert x \vert}.
\end{gather}
Подставив явные выражения для плотностей распределения, получаем следующий интеграл
\begin{gather}
    \rho_{p_\varphi}(z) = \frac{1}{\sqrt{2 \pi}} \int\limits_0^1 \frac{\displaystyle \exp \lb -\frac{z^2}{2x^2} \rb}{\sqrt{1 - x^2}} dx,
\end{gather}
%
разрешив который приходим к
\begin{gather}
    \rho_{p_\varphi}(z) = \frac{\pi}{8} \lb 1 - \text{sgn}(z) \erf \lb \frac{z}{\sqrt{2}} \rb \rb.
\end{gather}
Т.к. угол $\varphi$ не входит в гамильтониан $\mH_\text{spherical}$, то он распределен с равномерной плотностью на отрезке $\lsq 0, 2 \pi \rsq$.

\begin{figure}[H]
    \centering
    \includegraphics[width=0.75\linewidth]{./pictures/two_atom_distributions/pR-crop.pdf} \\
    \includegraphics[width=0.75\linewidth]{./pictures/two_atom_distributions/pPhi-crop.pdf}
    \caption{Плотности распределений импульсов $p_r$ и $p_\varphi$ при температурах от 150K до 300K для системы He$-$Ar. Межатомное расстояние $r_\text{fixed}$ взято равным $40 a_0$. Количество сгенерированых точек при каждой температуре -- $N = 5 \cdot 10^7$.}
\end{figure}

Если переходить теперь к переменным системы координат III, то легко заметить, что плотность распределения импульса $p_r$ совпадает с той, что была получена в системе координат II. Угол $\psi$ не входит в гамильтониан, следовательно распределен с равномерной плотностью. Т.к. якобиан замены декартовых координат и импульсов на координаты и импульсы системы III равен $p_\psi \sin \Theta$ (соотношение, следовательно распределен с равномерной плотностью. Т.к. якобиан замены декартовых координат и импульсов на координаты и импульсы системы III равен $p_\psi \sin \Theta$ (соотношение \eqref{two-atom-planar-jacobian}), то получаем, что плотность распределения импульса $p_\psi$ пропорциональна
\begin{gather}
    \rho(p_\psi) \propto p_\psi \exp \lb -\frac{p_\psi^2}{2 \mu r_\text{fixed}^2 \kb T} \rb, \label{two-atom-ppsi-distribution}
\end{gather}
где константа пропорциональна устанавливается из условия нормировки, оказывается равной $1/(\mu r_\text{fixed}^2 \kb T)$. Из того же якобиана замечаем, что угол $\Theta$ распределен равномерно с косинусом. \par
Распределение для импульса $p_\psi$ может быть установлено и из других соображений. Как уже отмечалось, $p_\psi$ имеет физический смысл модуля углового момента $\mf{J}$. Исходя из выражения \eqref{two-atom-angular-momenta-spherical} получаем, что квадрат модуля углового момента $J^2$ связан с импульсами $p_\varphi$, $p_\theta$ соотношением
\begin{gather}
    J^2 = p_\psi^2 = p_\theta^2 + \frac{p_\varphi^2}{\sin^2 \theta}. \label{two-atom-angular-momenta-connection} 
\end{gather}
Мы уже установили, что слагаемых в правой части \eqref{two-atom-angular-momenta-connection} распределены согласно нормальному распределению. Квадраты нормально распределенных случайных величин распределены согласно хи-квадрат распределению с одной степенью свободы $\chi_1^2$ \cite{castaneda}. А сумма двух одномерных хи-квадрат распределений $\chi_1^2$ дает двумерное хи-квадрат распределение $\chi_2^2$. Наконец, для того, чтобы получить распределение величины $p_\psi$, извлекаем корень из двумерного хи-квадрат распределения $\chi_2^2$ и получаем двумерное хи-распределение $\chi_2$, известное как распределение Рэлея, плотность которого задается  
\begin{gather}
    \rho(x; \sigma) = \frac{x}{\sigma^2} \exp \lb -\frac{x^2}{2 \sigma^2} \rb. \label{rayleigh-density}
\end{gather}

Выражение \eqref{two-atom-ppsi-distribution} является частным случаем \eqref{rayleigh-density} с $\sigma^2 = \mu r_\text{fixed}^2 \kb T$.

\begin{figure}[H]
    \centering
    \includegraphics[width=0.75\linewidth]{./pictures/two_atom_distributions/pPsi-crop.pdf}
    \caption{Плотности распределений импульса $p_\psi$ при температурах от 150К до 300К для системы He$-$Ar. Межатомное расстояние $r_\text{fixed}$ было взято равным $40a_0$. Количество сгенерированных точек при каждой температуре -- $N = 5 \cdot 10^7$.}
\end{figure}

\section{Спектральная функция при рассмотрении динамики столкновения в плоскости} \label{section:spectral_function_in_plane}

Рассмотрим спектральную функцию,связанную с функцией автокорреляции суммарного дипольного момента преобразованием Фурье
\begin{gather}
    J(\omega) = \intty \mean{ \bs{\mu}(0) \bs{\mu}(t) } e^{-i \omega t} dt.
\end{gather}

Мы будем пользоваться приближением бинарных столкновений, то есть, будем предполагать, что суммарная автокорреляционная функция распадается на сумму автокорреляционных функций индуцированных диполей пар. Для индуцированного дипольного момента пары, для простоты, мы сохраним обозначение $\bs{\mu}$. Итак, мы будем работать со следующим выражением для спектральной функции с трактовкой интеграла как интеграла по начальным условиям классических динамических траекторий, как обсуждалось в пункте \ref{section:correlation_functions} 
\begin{gather}
    J(\omega) = \intty \frac{\displaystyle \idotsint \bs{\mu}(0) \bs{\mu}(t) \, \exp \lb -\frac{\mH}{kT} \rb d\mf{q} \, d\mf{p}}{\displaystyle \idotsint \exp \lb -\frac{\mH}{kT} \rb d\mf{q} \, d\mf{p}} e^{-i \omega t} dt. \label{two-atom-spectral-function1}
\end{gather}

Вектор координат при рассмотрении в плоскости столкновений равен $\mf{q} = \lc r, \psi, \Phi, \Theta \rc$, а вектор импульсов -- $\mf{p} = \lc p_r, p_\psi \rc$. Кроме того, согласно пункту \ref{section:averaging} в интеграле появляется дополнительный весовой множитель, равный $\text{Jac} = p_\psi \sin \Theta$. Следовательно, полное интегральное выражение в этой системе координат выглядит следующим образом
\begin{gather}
    J(\omega) = \intty e^{-i\omega t} dt \frac{\displaystyle \int\limits_0^\infty dr \intty dp_r \int\limits_0^{2\pi} d\psi \int\limits_0^\infty p_\psi dp_\psi \int\limits_0^\pi \sin \Theta d\Theta \int\limits_0^{2\pi} d\Phi \bs{\mu}(0) \bs{\mu}(t) \exp \lb -\frac{\mH}{kT} \rb}{\displaystyle \int\limits_0^\infty dr \intty dp_r \int\limits_0^{2\pi} d\psi \int\limits_0^\infty p_\psi dp_\psi \int\limits_0^\pi \sin \Theta d\Theta \int\limits_0^{2\pi} d\Phi \exp \lb -\frac{\mH}{kT} \rb}. \label{two-atom-spectral-function2}
\end{gather}

Заметим, что ни дипольный момент $\bs{\mu}(t)$, ни гамильтониан $\mH$ не зависят от углов $\Phi$, $\Theta$. Проинтегрировав по ним, мы получаем фактор $4 \pi$ как в числителе, так и знаменателе, поэтому суммарно никаких дополнительных множителей не возникает.
\begin{gather}
    J(\omega) = \intty e^{-i\omega t} dt \frac{\displaystyle \int\limits_0^\infty dr \intty dp_r \int\limits_0^{2\pi} d\psi \int\limits_0^\infty p_\psi dp_\psi \bs{\mu}(0) \bs{\mu}(t) \exp \lb -\frac{\mH}{kT} \rb}{\displaystyle \int\limits_0^\infty dr \intty dp_r \int\limits_0^{2\pi} d\psi \int\limits_0^\infty p_\psi dp_\psi \exp \lb -\frac{\mH}{kT} \rb}. \label{two-atom-spectral-function3}
\end{gather}

Как известно, решение задачи о движении частицы с приведенной массой $\mu$ в центральном поле можно получить, основываясь на законах сохранения энергии и углового момента в интегральном виде \cite{landau-volume1}. Гамильтониан, записанный в полярных координатах, определенных в плоскости движения (система III), 
\begin{gather}
    \mH_\text{plane} = \frac{p_r^2}{2\mu} + \frac{p_\psi^2}{2\mu r^2} + U(r) = E,
\end{gather}
%
является интегралом движения. Как уже отмечалось, импульс $p_\psi$ имеет смысл модуля вектора углового момента, следовательно, также является интегралом движения. Решение уравнений движений в интегральном виде выглядит следующим образом \cite{landau-volume1} 
\begin{gather}
    t = \int\limits_{r_\text{нач}}^{r} \frac{dr}{\displaystyle \sqrt{\frac{2}{\mu} \lb E - U(r) \rb - \frac{p_\psi^2}{\mu^2 r^2}}}, \label{two-atom-change1} \\
    \psi = \int\limits_{r_\text{нач}}^{r} \frac{\displaystyle \frac{p_\psi}{r^2} dr}{\displaystyle \sqrt{2\mu \lb E - U(r) \rb - \frac{p_\psi^2}{r^2}}}, \label{two-atom-change2}
\end{gather}
где $r_\text{нач}$ -- начальное значение межатомного расстояния, а $r$ -- межатомное расстояние в момент времени $t$.  

Рассмотрим замену координат в интеграле в числителе \eqref{two-atom-spectral-function3} следующего вида
\begin{gather}
    \lc r, p_r, \psi, p_\psi \rc \rightarrow \lc (\rfixed), \tau, p_r^\prime, \psi^\prime, p_\psi \rc \label{two-atom-change-variables-time}.
\end{gather}

Физический смысл этой замены координат состоит в том, что вместо того, чтобы начинать классическую траекторию с произвольного межатомного расстояния $r$, мы хотим использовать фиксированное начальное расстояние $\rfixed$ ($\rfixed$ взят в скобках, потому что фиксирован для всех траекторий). Переменная $\tau$ задает время, за которое межатомное расстояние становится равным $r$. Если взять исходное $\rfixed$ бесконечно большим, то набор переменных $\lc \tau, p_r^\prime, \psi^\prime, p_\psi \rc$ опишет тот же массив свободно-разлетных траекторий, что и набор переменных $\lc r, p_r, \psi, p_\psi \rc$. Понятно, что в интеграле \eqref{two-atom-spectral-function3} нам нужно перечислить лишь те классические траектории, на которых межатомное расстояние уменьшилось до такой степени, чтобы появился значительный индуцированный дипольный момент. Поэтому, если мы положим $\rfixed$ больше некоторого расстояния, за которым мы считаем индуцированный дипольный момент равным нулю, то мы перечислим весь значимый массив траекторий (классические траектории, минимальное сближение между атомами в ходе которых больше $\rfixed$, не будут тогда учтены в интеграле, но и вклад от них равен нулю). Подходящее расстояние $\rfixed$ следует подбирать на основании радиальной зависимости индуцированного дипольного момента для каждой конкретной системы по-своему. Отметим, что импульс $p_\psi$ является интегралом движения, поэтому он сохраняется при описанной замене. \par
Для оговоренного набора траекторий замена переменных \eqref{two-atom-change-variables-time} является взаимоднозначной в силу единственности решения системы дифференциальных уравнений с начальными условиями Коши. \par
Заметим, что интегральные выражения \eqref{two-atom-change1}, \eqref{two-atom-change2} описывают ровно половину классической траектории -- от $\rfixed$ до поворотной точки $r_0$, определяемой уравнением
\begin{gather}
    \frac{2}{\mu} \lb E - V(r_0) - \frac{p_\psi^2}{2 \mu r_0^2} \rb = 0.
\end{gather}
Классические траектории столкновения двух тел являются симметричными относительно поворотной точки \cite{goldstein}, поэтому мы без ограничения общности можем рассматривать только ту половину траектории, в ходе которой происходит разлет двух тел от поворотной точки $r_0$ до некоторого выбранного значения $\rfixed$. Другими словами, будем рассматривать такие наборы начальных условий $\lc r, p_r, \psi, p_\psi \rc$,  в которых импульсы $p_r$ являются положительными и будем сопоставлять им наборы начальных условий $\lc \tau, p_r^\prime, \psi^\prime, p_\psi \rc$, в которых импульсы $p_r^\prime$ также являются положительными величинами. \par
Итак, координаты $\tau$, $p_r^\prime$, $\psi^\prime$ связаны с исходными $r$, $p_r$, $\psi$ следующими соотношениями
\begin{gather}
    \lc
    \begin{aligned}
        \tau &= \int\limits_r^{\rfixed} \frac{dr^\prime}{\displaystyle \sqrt{\frac{2}{\mu} \lb E - U(r^\prime) - \frac{p_\psi^2}{2 \mu r^{\prime 2}} \rb}}, \\
        \psi^\prime &= \psi + \int\limits_r^{\rfixed} \frac{\displaystyle \frac{p_\psi}{r^{\prime 2}} dr^\prime}{\displaystyle \sqrt{2\mu \lb E - U(r^\prime) - \frac{p_\psi^2}{2 \mu r^{\prime 2}} \rb}}, \\
        p_r^\prime &= \sqrt{2 \mu \lb E - \frac{p_\psi^2}{2 \mu \rfixed^2} - U(\rfixed) \rb},
    \end{aligned}
    \right. \label{two-atom-change3}
\end{gather}
%
где последнее соотношение получено исходя из закона сохранении энергии в форме
\begin{gather}
    E = \frac{p_r^2}{2\mu} + \frac{p_\psi^2}{2 \mu r^2} + U(r) = \frac{p_r^{\prime 2}}{2 \mu} + \frac{p_\psi^2}{2 \mu \rfixed^2} + U(\rfixed). 
\end{gather}

Учитывая в какой форме записаны соотношения \eqref{two-atom-change3}, найдем якобиан $\displaystyle \Bigg\vert \frac{\partial \lsq \tau, p_r^\prime, \psi^\prime, p_\psi \rsq}{\partial \lsq r, p_r, \psi, p_\psi \rsq} \Bigg \vert$, а затем, пользуясь тем, что якобианы обратны друг к другу
\begin{gather}
    \Bigg\vert \frac{\partial \lsq \tau, p_r^\prime, \psi^\prime, p_\psi \rsq}{\partial \lsq r, p_r, \psi, p_\psi \rsq} \Bigg \vert \cdot \Bigg\vert \frac{\partial \lsq r, p_r, \psi, p_\psi \rsq}{\partial \lsq \tau, p_r^\prime, \psi^\prime, p_\psi \rsq} \Bigg \vert = 1,
\end{gather}
%
найдем интересующий нас якобиан
\begin{gather}
    \text{Jac} = \Bigg\vert \frac{\partial \lsq r, p_r, \psi, p_\psi \rsq}{ \partial \lsq \tau, p_r^\prime, \psi^\prime, p_\psi \rsq} \Bigg\vert.
\end{gather}

Итак, матрица якобиана $\text{Jac}^{-1}$ имеет следующую структуру
\begin{gather}
    \text{Jac}^{-1} = 
    \begin{bdmatrix}
        \frac{\partial \tau}{\partial r} & \frac{\partial \tau}{\partial p_r} & \frac{\partial \tau}{\partial \psi} & \frac{\partial p_\psi}{\partial r} \\
        \frac{\partial p_r^\prime}{\partial r} & \frac{\partial p_r^\prime}{\partial p_r} & \frac{\partial p_r^\prime}{\partial \psi} & \frac{\partial p_r^\prime}{\partial p_\psi} \\
        \frac{\partial \psi^\prime}{\partial r} & \frac{\partial \psi^\prime}{\partial p_r} & \frac{\partial \psi^\prime}{\partial \psi} & \frac{\partial \psi^\prime}{\partial p_\psi} \\
        \frac{\partial p_\psi}{\partial r} & \frac{\partial p_\psi}{\partial p_r} & \frac{\partial p_\psi}{\partial \psi} & \frac{\partial p_\psi}{\partial p_\psi} 
    \end{bdmatrix} = 
    \begin{bdmatrix}
        a & b & 0 & c \\
        d & e & 0 & f \\
        g & h & 1 & k \\
        0 & 0 & 0 & 1
    \end{bdmatrix}.
\end{gather}

Все производные по $\psi$, за исключением $\partial \psi^\prime / \partial \psi$, равны 0, т.к. $\psi$ не входит в выражениях для соответствующих переменных. Производная же $\partial \psi^\prime / \partial \psi$ равна 1, потому что $\psi$ только аддитивно входит в выражение для $\psi^\prime$.
Переменная $p_\psi$ остается неизменной в результате замены, поэтому последняя строчка матрицы оказывается такой простой. \par
Вследствие особенностей структуры матрицы, получается, что детерминант матрицы якобиана $\text{Jac}^{-1}$ зависит только от 4 элементов
\begin{gather}
    \det \lc \text{Jac}^{-1} \rc = a \cdot e - b \cdot d.  
\end{gather}

Явные выражения для этих элементов матрицы выглядят следующим образом 
\begin{gather}
    a = \frac{\partial \tau}{\partial r} = -\frac{\mu}{p_r} - \frac{1}{\mu} \lb \frac{dU}{dr} - \frac{p_\psi^2}{\mu r^3} \rb \cdot I_1, \\ 
    b = \frac{\partial \tau}{\partial p_r} = - \frac{p_r}{\mu^2} I_1, \\ 
    d = \frac{\partial p_r^\prime}{\partial r} = \frac{\displaystyle \mu \lb \frac{dU}{dr} - \frac{p_\psi^2}{\mu r^3} \rb}{\displaystyle \sqrt{2\mu \lb E - \frac{p_\psi^2}{2 \mu \rfixed^2} - U(\rfixed) \rb}} = \frac{\mu}{p_r^\prime} \lb \frac{dU}{dr} - \frac{p_\psi^2}{\mu r^3} \rb, \\
    e = \frac{\partial p_r^\prime}{\partial p_r} = \frac{p_r}{\displaystyle \sqrt{2\mu \lb E - \frac{p_\psi^2}{2 \mu \rfixed^2} - U(\rfixed) \rb}} = \frac{p_r}{p_r^\prime},
\end{gather}
%
где введено обозначение 
\begin{gather}
    I_1 = \int\limits_r^{\rfixed} \lsq \frac{2}{\mu} \lb E - U(r^\prime) - \frac{p_\psi^2}{2 \mu r^{\prime 2}} \rb \rsq^{-3/2} dr^\prime. 
\end{gather}

Для полноты, представим остальные элементы матрицы якобиана
\begin{gather}
    c = \frac{\partial \tau}{\partial p_\psi} = -\frac{p_\psi}{\mu} \int\limits_r^{\rfixed} \lb \frac{1}{r^2} - \frac{1}{r^{\prime 2}} \rb \lsq \frac{2}{\mu} \lb E - U(r^\prime) - \frac{p_\psi^2}{2 \mu r^{\prime 2}} \rb \rsq^{-3/2} dr^\prime \\
f = \frac{\partial p_r^\prime}{\partial p_\psi} = \frac{\displaystyle p_\psi \lb \frac{1}{r^2} - \frac{1}{\rfixed^2} \rb}{\displaystyle \sqrt{2 \mu \lb E - \frac{p_\psi^2}{2 \mu \rfixed^2} - U(\rfixed) \rb}} = \frac{p_\psi}{p_r^\prime} \lb \frac{1}{r^2} - \frac{1}{\rfixed^2} \rb \\
    g = \frac{\partial \psi^\prime}{\partial r} = -\frac{p_\psi}{p_r r^2} - \mu p_\psi \lb \frac{dU}{dr} - \frac{p_\psi^2}{\mu r^3} \rb I_2, \\ 
    h = \frac{\partial \psi^\prime}{\partial p_r} = -p_\psi p_r \cdot I_2, \\ 
    k = \frac{\partial \psi^\prime}{\partial p_\psi} = \int\limits_r^{\rfixed} \frac{1}{r^{\prime 2}} \lsq 2 \mu \lb E - U(r^\prime) - \frac{p_\psi^2}{2 \mu r^2} \rb \rsq \lsq 2 \mu \lb E - U(r^\prime) - \frac{p_\psi^2}{2 \mu r^{\prime 2}} \rb\rsq^{-3/2} dr^\prime,
\end{gather}
%
где было введено обозначение
\begin{gather}
    I_2 = \int\limits_r^{\rfixed} \frac{dr^\prime}{r^{\prime 2}} \lsq 2 \mu \lb E - U(r^\prime) - \frac{p_\psi^2}{2 \mu r^{\prime 2}} \rb \rsq^{-3/2}.
\end{gather}

Итак, якобианы оказываются равными
\begin{gather}
    \Bigg\vert \frac{\partial \lsq \tau, p_r^\prime, \psi^\prime, p_\psi \rsq}{\partial \lsq r, p_r, \psi, p_\psi \rsq} \Bigg\vert = \frac{\mu}{p_r^\prime}, \quad \Bigg\vert \frac{\partial \lsq r, p_r, \psi, p_\psi \rsq}{\partial \lsq \tau, p_r^\prime, \psi^\prime, p_\psi \rsq} \Bigg\vert = \frac{p_r^\prime}{\mu}.
\end{gather}

Следовательно, выражение для спектральной функции \eqref{two-atom-spectral-function3} может быть переписано в виде
\begin{gather}
    J(\omega) = \frac{1}{\Gamma_0} \intty e^{-i \omega t} dt \int\limits_0^\infty d\tau \intty \frac{p_r^\prime}{\mu} dp_r^\prime \int\limits_0^{2\pi} d\psi^\prime \int\limits_0^\infty p_\psi dp_\psi \bs{\mu}(0) \bs{\mu}(\tau) \exp \lb -\frac{\mH_\text{plane}}{kT} \rb,
\end{gather}
%
где через $\Gamma_0$ обозначен интеграл, находящийся в знаменателе \eqref{two-atom-spectral-function3}
\begin{gather}
    \Gamma_0 = \int\limits_0^\infty dr \intty dp_r \int\limits_0^{2\pi} d\psi \int\limits_0^\infty p_\psi dp_\psi \exp \lb -\frac{\mH_\text{plane}}{kT} \rb.
\end{gather}

Переставив интеграл по времени $t$ c интегралами по переменным $\tau$, $p_r^\prime$, $\psi^\prime$ и $p_\psi$, воспользуемся корреляционной теоремой \eqref{correlation-theorem}
\begin{gather}
    J(\omega) = \frac{1}{\Gamma_0} \int\limits_0^\infty \frac{p_r^\prime}{\mu} dp_r^\prime \int\limits_0^{2\pi} d\psi^\prime \int\limits_0^\infty p_\psi dp_\psi \Bigg\vert \intty \bs{\mu}(t) e^{i \omega t} dt \Bigg\vert^2 \exp \lb -\frac{\mH_\text{plane}}{k T} \rb. \label{two-atom-spectral-function4}
\end{gather}

\iffalse
Integral ratio:
\begin{gather}
    \int \exp \lb -\frac{T_H}{kT} \rb d \psi d p_r dp_\psi = 2 \pi^2 \mu k T r \\
    \int \exp \lb -\frac{T_H}{kT} \rb dr d\psi dp_r dp_\psi = \pi^2 \mu k T r^2 \\
    \frac{\int \exp \lb -\frac{T_H}{kT} \rb d\psi dp_r dp_\psi}{\int \exp \lb -\frac{T_H}{k T} \rb dr d\psi dp_r dp_\psi} = \frac{2}{r}
\end{gather}
\fi

\section{Трансляционные спектры газовой смеси He$-$Ar}

Экспериментальные исследования газовых смесей инертных газов He$-$Ar и Ne$-$Ar производились в работах \cite{bosomworth1965_part1, bosomworth1965_part2} при околокомнатных температурах. Полученные в более ранней работе \cite{kiss1959} данные ограничены спектральным диапазоном от 350 до 700 см$^{-1}$ и плохо согласуются с более подробными данными, представленными в \cite{bosomworth1965_part2}, поэтому при комнатной температуре сравнение теоретического спектрального профиля мы будем производить с данными из \cite{bosomworth1965_part2}. Трансляционные спектры систем He$-$Ar и Ne$-$Ar при низких температурах экспериментально исследовались в работах \cite{bukhtoyarova1977, bukhtoyarova1977_2, ryzhov1974}. \par 
Исторически при моделировании трансляционных спектров благородных газов авторы пользовались модельными поверхностями потенциальной энергии и дипольного момента. В работе \cite{levine1967} авторы рассматривают прямолинейные классические траектории столкновения в отсутствии межатомного потенциала; в качестве модели функции дипольного момента была взята гауссова функция, которая совершенно не воспроизводит физического поведения дипольного момента, однако позволяет производить аналитические выкладки. Сделанные приближения позволили получить аналитическое выражение для коэффициента поглощения $\alpha(\omega)$ с использованием спецфункций. В результате подгонки параметров функции дипольного момента было получено согласие аналитической модели для коэффициента поглощения с экспериментальными данными. \par
В работе \cite{mcquarrie1968} представлено моделирование столкновительно-индуцированного спектра методом классических траекторий. Авторы использовали потенциал Леннарда-Джонса (6, 12) с параметрами, подогнанными под экспериментальные данные о сечениях рассеяния. Зависимость дипольного момента от расстояния аппроксимировалась экспоненциальной функцией в области малых межатомных расстояний; при больших расстояниях предполагалось, что дипольный момент отсутствует. При помощи подгонки коэффициента, определяющего скорость спада функции дипольного момента от расстояния, авторам удалось добиться неплохого согласия с экспериментальными данными \cite{bosomworth1965_part2}. \par
Работа \cite{sharma1975} посвящена моделированию столкновительно-индуцированного спектра с позиций квантового формализма. Авторы использовали функции дипольного момента, построенные на основе дальнодействующей компоненты с радиальной асимптотикой $r^{-7}$ и короткодействующей компоненты, имеющей экспоненциальный рост с уменьшением расстояния, рассчитанной методом Хартри-Фока. Качество используемых поверхностей потенциальной энергии и дипольного момента не позволило достичь высокого уровня согласия с экспериментальными данными. \par
Работа \cite{meyer1986} Фроммхольда и Мейера является одной из самых ранних работ, в которой была получена \textit{ab initio} повехность дипольного момента и получена на основе нее зависимость спектральной функции $J(\omega)$ в рамках квантового формализма. Расчеты дипольного момента производились методами Хартри-Фока и конфигурационного взаимодействия в разных базисных наборах. Полученные расчетные значениия дипольного момента были аппроксимированы простой аналитической фукнцией. Отклонение спектральной функции от экспериментальных данных не превышает 10\%. \par 
С развитием современных квантовохимических методов поверхности потенциальной энергии и дипольного момента смесей благородных газов были подробно изучены многими авторами \cite{cybulski1999, giece2003, fernandez2004}. Расчеты производились с использованием методов CCSD/CCSD(T) в корреляционно-согласованных базисных наборах, дополненные связевыми функциями, расположенными на равном расстоянии от обоих атомов. В наиболее современной работе\cite{fernandez2004} авторы использовали метод CCSD(T) и базисный набор aug-cc-pV6Z-33211 для расчета поверхности потенциальной энергии и CCSD/d-aug-cc-pVQZ-33211 для расчета поверхности дипольного момента. Для оценки точности поверхности потенциальной энергии авторы рассчитали температурную зависимость смешанного вириального коэффициента $B_{12}(T)$. Авторы отмечают, что полученные ими поверхности находятся в практически полном согласии с даными \cite{cybulski1999}. В наших расчетах мы использовали предложенные авторами аналитические разложения поверхностей потенциальной энергии и дипольного момента. \par
Нашей научной группой в 2014 году была выполнена работа \cite{buryak2014}, в которой было проведено сравнение траекторного и квантового подходов к расчету трансляционных спектров систем He$-$Ar и Ne$-$Ar. При сравнении использовались поверхности потенциальной энергии \cite{fernandez2004}, поверхности дипольного момента \cite{fernandez2004, meyer1986}. Несмотря на то, что квантовые расчеты должны давать более точные спектральные профили, результаты, полученные в траекторном расчете, оказываются в хорошем согласии как с квантовыми расчетами, так и с экспериментальными данными. Было отмечено, что некоторую проблему при классическом моделировании спектра представляет собой процедура десимметризации профиля, которую мы обговорим подробнее позже. \par

\section{Вычислительные аспекты расчета столкновительно$-$ индуцированного спектра методом классических траекторий}

Расчет спектральной функции системы из двух атомов в нашей работе мы будем производить по выражению \eqref{two-atom-spectral-function4}. Вычисление многомерного интеграла будем производить методом Монте-Карло, т.к. при рассмотрении систем с большим количеством вращательных степеней свободы мы сталкнемся с интегралами значительно более высокой размерности, вычисление которых квардратурными методами не представляется возможным. В данном случае при интегрировании квадратурами общие вычислительные затраты могут оказаться меньше, однако т.к. такая схема интегирования оказывается непереносимой на системы, в которых мономеры обладают несколькими вращательными степенями свободы, то мы отказались от ее реализации. Известно, что погрешность метода Монте-Карло асимптотически ведет себя как $N^{-1/2}$, где $N$ -- количество точек, по которым производилась оценка интеграла \cite{sobol}. Такая асимптотика ошибки не позволяет получать очень точных оценок интегралов, что в некоторых задачах оказывается неудовлетворительным. В задаче моделирования континуального спектрального профиля точность порядка $\sim 0.5\%$ оказываеся приемлимой, что достижимо с использованием метода Монте-Карло. Более подробно вопрос о точности получающегося спектрального профиля в нашем подходе мы обсудим после описания вычислительной схемы. \par
Интегрирование в \eqref{two-atom-spectral-function4} мы будем производить методом Монте-Карло с весовой функцией $p_\xi(\bxi) = p_\psi \exp \lb -\mH_\text{plane}(\bxi) / \kb T \rb / \Gamma_1$, где $\bxi = \lc p_R^\prime, \psi^\prime, p_\psi \rc$, а $\Gamma_1$ -- нормировочный множитель, равный
\begin{gather}
    \Gamma_1 = \int\limits_0^\infty dp_r^\prime \intty d\psi^\prime \intty p_\psi \exp \lb -\frac{\mHplane}{\kb T} \rb dp_\psi.
\end{gather}
Выражение \eqref{two-atom-spectral-function4} может быть рассмотрено как математическое ожидание квадрата преобразования Фурье на распределении $\boldsymbol{\xi}$, заданном весовой функции
\begin{gather}
    J(\omega) = \frac{\Gamma_1}{\Gamma_0} \sum_{k = 1}^{\infty} \frac{p_r^\prime(\bxi_k)}{\mu} \Big\vert \intty \bs{\mu}(t; \bxi_k) e^{i \omega t} dt \Big\vert^2, \label{two-atom-spectral-function-mc}
\end{gather}
%
где обозначение $p_r^\prime(\bxi_k)$ подразумевает, что импульс, сопряженный радиальной координате, взят из вектора $\bxi$, реализующего распределение с плотностью $p_\xi$. \par
Основываясь на выкладках, сделанных в параграфе \ref{section:averaging}, несложно получить аналитические выражения для интегралов $\Gamma_0, \Gamma_1$
\begin{gather}
    \Gamma_0 = \int\limits_0^\infty dr \intty dp_r \int\limits_0^{2\pi} d\psi \int\limits_0^\infty p_\psi \exp \lb -\frac{\mHplane}{\kb T} \rb d p_\psi = \frac{4}{3} \pi r_\text{fixed}^3 \lb 2 \pi \mu \kb T \rb^{3/2}, \\
    \Gamma_1 = \int\limits_0^\infty dp_r \int\limits_0^{2\pi} d\psi \int\limits_0^\infty p_\psi \exp \lb -\frac{\mHplane}{\kb T} \rb dp_\psi  = 2 \pi r_\text{fixed}^2 \lb 2 \pi \mu \kb T \rb^{3/2}.
\end{gather}
Таким образом, конечное выражение для спектральной функции, как среднее значение на распределении с плотностью $p_\xi(\bxi)$, приходит к виду
\begin{gather}
    J(\omega) = 2 \pi \rfixed^2 \sum_{k = 1}^\infty \frac{p_r^\prime(\bxi_k)}{\mu} \Big\vert \intty \bs{\mu}(t; \bxi_k) e^{i \omega t} dt \Big\vert^2. \label{two-atom-spectral-function-mc-final}
\end{gather}

Вопрос генерации реализаций случайной величины $\bxi$ с плотностью распределения $p_\xi$ был обсужден в параграфе \ref{section:averaging}. Рассмотрим вопрос расчет классических траекторий рассеяния. Несмотря на то, что для динамических переменных могут быть выписаны квадратурные выражения, которыми мы пользовались в параграфе \ref{section:spectral_function_in_plane}, вычисления по ним производить не очень удобно. В первую очередь неудобство связано с тем, что для вычисления усредняемого выражения в \eqref{two-atom-spectral-function-mc-final} нам потребуется вычислять преобразование Фурье от дипольного момента вдоль траектории, которое эффективно реализуется в виде дискретного преобразования Фурье. Для этого нам потребуются значения дипольного момента $\bs{\mu}(t)$ с равноверным шагом по времени, что вызовет некоторые затруднения при вычисления по квадратурной формуле. Кроме того, квадратуры для всех динамических переменных являются несобственными интегралами, имеющими особенность в поворотной точке. Поэтому траектории рассеяния рассчитывались интегрированием трех уравнений Гамильтона
\begin{gather}
    \lc
    \begin{aligned}
        \dot{r} &= \frac{p_r}{\mu} \\
        \dot{\psi} &= \frac{p_\psi^2}{\mu r^3} - \frac{d U}{d r} \\
        \dot{p_\psi} &= \frac{p_\psi}{\mu r^2}
    \end{aligned}
    \right. \label{two-atom-hamilton-equations}
\end{gather}

Была разработана программа на языке С++, реализующая классический траекторный расчет по описанной схеме. Система уравнений \eqref{two-atom-hamilton-equations} численно интегрировалась при помощи пакета процедур для решения дифференциальных уравнений SUNDIALS \cite{sundials}. Использовались процедуры, реализующие BDF формулы переменного порядка. Решение нелинейных уравнений, возникающих в результате применения BDF формул, происходит с использованием стандартного метода Ньютона. Вычисления производились с двойной точностью со значением параметра, определяющего относительную ошибку решения, равным $10^{-15}$. Так как каждая траектория может быть рассчитана в независимости от остальных, то легко может быть написана программа для расчета массива траекторий в параллельном режиме. В коде была использована библиотека MPI \cite{mpi}, стандартизующая передачу сообщений между узлами параллельного приложения. Приложение реализовано в модели взаимодействия \enquote{ведущий-ведомый}, в которой \enquote{ведущий} процесс отвечает за распределение начальных условий между \enquote{ведомыми} процессами, осуществляющими расчет траекторий. Получив начальное условие, \enquote{ведомый} процесс рассчитывает траекторию рассеяния с внутренним переменным шагом по времени, при этом с фиксированным шагом по времени $\Delta t = 200$ атомных единиц времени (около 5 фс) вдоль траектории вычисляется значение индуцированного дипольного момента и собирается в заранее подготовленный массив, содержащий $L = 2^{15}$ ячеек. Расчет траектории заканчивается в тот момент, когда межатомное расстояние вновь достигает начального значения или количество вычислений дипольного момента превышает $L$. Затем в рамках \enquote{ведомого} процесса осуществляется дискретное преобразование Фурье временной зависимости индуцированного дипольного момента. Выбор длины массива $L$ обусловлен тем, что наиболее эффективно дискретное преобразование Фурье выполняется для массива, длина которого есть некоторая степень двух (так называемое быстрое преобразование Фурье). Согласно выражению \eqref{two-atom-spectral-function-mc-final} вычисляется квадрат преобразования Фурье (трактовка вычисления квадрата обсуждалась в самом начале главы \ref{chapter:two-atom}), умножается на отношение радиального импульса $p_r$ в начальный момент времени к приведенной массе $\mu$ и полученный массив передается на \enquote{ведущий} процесс. \enquote{Ведущий} процесc накапливает результаты \enquote{ведомых} процессов и после усреднения получает спектральную функцию.
    Спектральный профиль, полученный в результате траекторного расчета, представлен на рис. \ref{fig:two-atom-desymmetrizations}.
\begin{figure}
    \centering
    \includegraphics[width=0.7\linewidth]{./pictures/two_atom_spectra/desymmetrizations_295K-crop.pdf}
    \caption{Сравнение классического и квантового спектральных профилей с профилями, полученными в результате применения различных процедур десимметризации. Пунктиром обозначен классический профиль, сплошной толстой линией -- квантовый профиль \cite{buryak2014}, точками с пунктиром -- профиль, полученный в результате применения процедуры D1, точечной линией -- профиль, полученный в результате применения процедуры D2, сплошной линией -- профиль, полученный в результате применения процедуры D3. Квадратами обозначены экспериментальные данные \cite{bosomworth1965_part2}.}
    \label{fig:two-atom-desymmetrizations}
\end{figure}

Как известно, классические траектории обратимы во времени. Это свойство классических траекторий приводит к тому, что автокорреляционная функция дипольного момента $C(t)$ является симметричной функцией от времени \cite{frommhold}
\begin{gather}
    C(t) = C(-t).
\end{gather}

Классическая спектральная функция, получающаяся в результате применения преобразования Фурье к автокорреляционной функции, также оказывается симметричной функции частоты 
\begin{gather}
    J_\text{class.}(\omega) = J_\text{class.}(-\omega). \label{classical-detailed-balance}
\end{gather}

Однако квантовая спектральная функция удовлетворяет так называемому условию детального баланса \cite{frommhold} 
\begin{gather}
    J(-\omega) = J(\omega) \exp \lb -\frac{\hbar \omega}{\kb T} \rb, \label{detailed-balance}
\end{gather}
%
которое может быть получено из выражения \eqref{part1-spectral-function-definition} перестановкой индексов $j, k$. Условие детального баланса \eqref{detailed-balance} отражает разницу в заселенностях исходного и конечного состояний. Ввиду этого, соотношение \eqref{classical-detailed-balance} называют классическим условием детального баланса. \par
Считается, что симметрийная разница между классической и квантовой спектральной функциями может быть устранена при помощи так называемой процедуры десимметризации. Предполагается, что классическую спектральную функцию можно определенным образом \enquotes{десимметризовать} так, что функция, получившаяся в результате, будет удовлетворять квантовому условию детального баланса \cite{borysow1985phenomena}. Однако процедура десимметризации не может быть однозначна задан условием детального баланса -- можно построить бесконечное количество процедур, которые будут приводить к спектральной функции, которая будет удовлетворять квантовому условию детального баланса \cite{frommhold}. Поэтому от процедуры десимметризации дополнительно требуют совпадения десимметрованного классического профиля с квантовым профилем в особенности в области дальних крыльев. \par
В литературе представлен набор различных процедур десимметризации \cite{borysow1985}. Некоторые из них устроены следующим образом:
\begin{gather}
    J_{D1}(\omega) = \frac{2}{1 + \exp \lb -\hbar \omega / \kb T \rb} J_\text{class}(\omega)  \\
    J_{D2}(\omega) = \frac{\hbar \omega}{\kb T} \frac{1}{1 - \exp \lb -\hbar \omega / \kb T \rb} J_\text{class}(\omega) \\
    J_{D3}(\omega) = \exp \lb \frac{\hbar \omega}{2 \kb T} \rb J_\text{class}(\omega)
\end{gather}
В литературе также используется преобразование Эгельстаффа \cite{egelstaff1962}; нами она практически не использовалась, поэтому результаты применения этой процедуры не приводятся. 

\begin{figure}
    \centering
    \includegraphics[width=0.7\linewidth]{./pictures/two_atom_spectra/spectral_function_desymmetrizations-crop.pdf}
    \caption{Сравнение крыльев классической, квантовой и десимметризованных спектральных функций системы He$-$Ar при температуре 295 K. Сплошной толстой линией обозначена квантовая спектральная функция \cite{buryak2014}; пунктиром -- классическая спектральная функция; пунктиром с точками -- спектральная функция, полученная в результате процедуры десимметризации D1; точечной линией -- спектральная функция, полученная в результате процедуры десимметризации D2; тонкой сплошной линией -- спектральная функция, полученная в результате процедуры десимметризации D3.}
    \label{fig:desymmetrisation-spectral-functions}
\end{figure}

Несложно убедиться, что приведенные процедуры десимметризации D1-D3 приводят к спектральным функциям, удовлетворяющим квантовому условию детального баланса. Продемонстрируем это на примере десимметризации D3, используя классическое правило детального баланса для $J_\text{class}(\omega)$:
\begin{gather}
    J_{D3}(\omega) = \exp \lb \frac{\hbar \omega}{2 \kb T} \rb J_\text{class}(\omega) \quad \implies \quad J_\text{class}(\omega) = \exp \lb -\frac{\hbar \omega}{2 \kb T} \rb J_{D3}(\omega), \\
    J_{D3}(-\omega) = \exp \lb -\frac{\hbar \omega}{2 \kb T} \rb J_\text{class}(-\omega) = \exp \lb -\frac{\hbar \omega}{2 \kb T} \rb J_\text{class}(\omega), \\
    J_{D3}(-\omega) = \exp \lb -\frac{\hbar \omega}{\kb T} \rb J_\text{D3}(\omega).
\end{gather}

Основываясь на рисунке \ref{fig:desymmetrisation-spectral-functions} замечаем, что при малых частотах все спектральные функции ведут себя примерно примерно одинаковым образом, однако с увеличением частоты различие между ними становится все больше. Одной из наиболее популярной в литературе является процедура D3. Спектральная функция, полученная в результате процедуры D3, переоценивает квантовую спектральную функции в области крыла, однако считается, что около максимума спектра она дает наилучшую оценку. Сравнение при других температурах мы будем производить именно со спектром, полученным в результате процедуры D3. \par
Сравнение с экспериментальными данными на \ref{fig:two-atom-spectra} показывает, что несмотря на проблему выбора подходящей процедуры десимметризации, общее совпадение с экспериментальными данными оказывается неплохим. Отклонения между спектрами, полученными в результате применения различных процедур десимметризации, не превышает отклонения между квантовым профилем и экспериментальными данными.  

\begin{figure}[H]
    \centering
    \includegraphics[width=0.49\linewidth]{./pictures/two_atom_spectra/spectral_function_d3-crop.pdf}
    \includegraphics[width=0.49\linewidth]{./pictures/two_atom_spectra/spectrum_effect_d3-crop.pdf}
    \caption{Влияние десимметризации на спектральную функцию и спектральный профиль с учетом отрицательных частот. Пунктиром обозначены классические спектральная функция и профиль; сплошной линией -- полученные из классических в результате процедуры D3.}
\end{figure}


\begin{figure}[H]
    \centering
    \includegraphics[width=0.49\linewidth]{./pictures/two_atom_spectra/alpha_165K-crop.pdf} 
    \includegraphics[width=0.49\linewidth]{./pictures/two_atom_spectra/alpha_200K-crop.pdf} \\
    \includegraphics[width=0.49\linewidth]{./pictures/two_atom_spectra/alpha_240K-crop.pdf}
    \includegraphics[width=0.49\linewidth]{./pictures/two_atom_spectra/alpha_295K-crop.pdf}
    \caption{Трансляционные спектры системы He$-$Ar при температурах 165K, 200K, 240K и 295K. Черными кружками обозначены экспериментальные данные из \cite{bukhtoyarova1977, bukhtoyarova1977_2, ryzhov1974}, квадратами -- из \cite{bosomworth1965_part2}.}
    \label{fig:two-atom-spectra}
\end{figure}



%\section{Квантовый подход к моделированию спектра}
%\section{Классический подход в лабораторной системе координат}
%\section{Вычислительные аспекты моделирования классических траекторий}

\chapter{Моделирование рототрансляционного столновительно-индуцированного спектра систем с вращательными степенями свободы}
\section{Колебательно-вращательные гамильтонианы в молекулярной системе координат}

При выводе молекулярного гамильтониана полиатомной системы часто предполагается, что амплитуда колебаний мала. Гамильтониан, полученный Вильсоном и Говардом, для системы с колебаниями малой амплитуды широко используется и часто служит базисом для дальнейших уточнений. \par
Однако колебания большой амплитуды, проявляющиеся в в слабосвязанных комплексах не могут быть описаны при помощи того же подхода. Для описания колебаний большой амплитуды часто используются координаты Якоби, координаты Радау, гиперсферические координаты и др. \par 
Рассмотрим систему из $N$ частиц с массами $\lb m_1, \dots m_N \rb$ и координатами $\lb \mf{r}_1, \dots, \mf{r}_N \rb$ в лабораторной системе координат. Лагранжева кинетическая энергия в лабораторной системе координат записывается как
\begin{gather}
    T_\text{tot} = \frac{1}{2} \sum_{k = 1}^N m_k \dot{\mf{r}}_k^2.
\end{gather}

Чтобы отделить трансляционные степени свободы введем координаты Якоби \cite{greiner, littlejohn1995}, являющиеся обобщением координат, используемых в двухатомных системах. Переход к трансляционно-инвариантному набору координат может быть осуществлен следующим образом \cite{greiner}
\begin{gather}  
    \lc
    \begin{aligned}
        \bs{\rho}_1 &= \frac{m_1 \mf{r}_1}{m_1} - \mf{r}_2 = \mf{r}_1 - \mf{r}_2, \\
        \bs{\rho}_2 &= \frac{m_1 \mf{r}_1 + m_2 \mf{r}_2}{m_1 + m_2}, \\
        \mf{\rho}_j &= \frac{\displaystyle \sum_{k = 1}^j m_k \mf{r}_k}{\displaystyle \sum_{k = 1}^j m_k} - \mf{r}_{j + 1}, \quad 2 < j < N \\
        \mf{\rho}_N &= \frac{1}{M} \sum_{k = 1}^N m_k \mf{r}_k,
    \end{aligned}
    \right.
\end{gather}
%
где через $M$ обозначена суммарная масса системы частицы. \par
Первый вектор Якоби $\bs{\rho}_1$ соединяет частицы 1 и 2. Второй вектор Якоби соединяет центр масс первых двух частиц и третью частицу. Третий вектор Якоби соеднияет центр масс первых трех частиц и четвертую частицу и т.д. Последний вектор Якоби суть вектор, направленный в центр масс системы. Пример векторов Якоби для системы, соcтоящей из трех частиц, приведен на Рис. \ref{fig:jacobi_coordinates}. 

\begin{figure}
    \centering
    \includegraphics[width=0.5\linewidth]{pictures/jacobi_coordinates.pdf}
    \caption{Координаты Якоби  для системы из 3 частиц, пронумерованных 1, 2, 3. Через $\mf{R}_{12}$ бозначен вектор, направленный в центр масс пары частиц 1 и 2.}
    \label{fig:jacobi_coordinates}
\end{figure}

Можно показать, что кинетическая энергия в лагранжевой форме, выраженная через векторы Якоби записывается как \cite{greiner} 
\begin{gather}
    T_\text{tot} = \frac{1}{2} M \dot{\bs{\rho}}_N^2 + \frac{1}{2} \sum_{k = 1}^{N - 1} \mu_k \dot{\bs{\rho}}_k^2,
\end{gather}
%
где приведенные массы $\mu_j$ связаны с исходными массами $m_j$ следующими соотношениями
\begin{gather}
    \frac{1}{\mu_j} = \frac{1}{M_j} + \frac{1}{m_{j+1}}, \quad M_j = \sum_{k = 1}^j m_k, \quad j = 1 \dots N - 1. \label{polyatom-jacobi-masses}
\end{gather}

Данная последовательность введения векторов Якоби не является единственно возможной. Выбор векторов Якоби в каждом отдельном случае обусловлен структурой рассматриваемой системы. Описанная последовательность является общим случаем, в котором получена кинетическая энергия для системы из $N$ частиц. При другом выборе векторов Якоби общая форма кинетической энергии сохранится, однако изменятся приведенные массы $\mu_j$. \par
Отделяя центр масс, мы приходим к следующей форме кинетической энергии
\begin{gather}
    \Tl = \frac{1}{2} \sum_{k = 1}^{N - 1} \mu_k \dot{\bs{\rho}}_k^2.
\end{gather}

Для отделения вращательных степеней свободы введем подвижную систему координат. Лабораторная и подвижная системы координат связаны друг с другом матрицей ортогонального преобразования $\bbS$ \cite{goldstein}. Обозначим через $\mf{R}_j$ координаты векторов Якоби в подвижной системе координат. Введенные векторы $\mf{R}_j$ связаны с векторами в лабораторной системе линейным преобразованием
\begin{gather}
    \boldsymbol{\rho}_j = \bbS \mf{R}_j.
\end{gather}
Будем рассматривать параметризацию матрицы ортогонального преобразования $\bbS$ тройкой углов Эйлера $\Phi$, $\Theta$, $\Psi$ \cite{goldstein}
\begin{gather}
    \bbS = 
    \begin{bmatrix}
        \cos \Psi \cos \Phi - \cos \Theta \sin \Phi \sin \Psi & -\sin \Psi \cos \Phi - \cos \Theta \sin \Phi \cos \Psi & \sin \Theta \sin \Phi \\ 
        \cos \Psi \sin \Phi + \cos \Theta \cos \Phi \sin \Psi & -\sin \Psi \sin \Phi + \cos \Theta \cos \Phi \cos \Psi & - \sin \Theta \cos \Phi \\
        \sin \Theta \sin \Psi & \sin \Theta \cos \Psi & \cos \Theta
    \end{bmatrix}.
\end{gather}

Лагранжева кинетическая энергия в подвижной системе отсчета может быть записана как \cite{landau-volume1}
\begin{gather}
    \Tl = \frac{1}{2} \sum_{i = 1}^{N-1} \mu_i \dot{\mf{R}}_i^2 + \frac{1}{2} \sum_{i = 1}^{N - 1} \mu_i \lsq \boldsymbol{\Omega} \times \mf{R}_i \rsq^2 + \boldsymbol{\Omega}^{+} \sum_{i = 1}^{N - 1} \mu_i \lsq \mf{R}_i \times \dot{\mf{R}}_i \rsq,
\end{gather}
%
где $\boldsymbol{\Omega}$ -- вектор угловой скорости в проекции на подвижную систему координат. Вектор угловой скорости $\boldsymbol{\Omega}$ связан с углами Эйлера и эйлеровыми скоростями следующим соотношением
\begin{gather}
    \boldsymbol{\Omega} = \bbV \mf{v} = 
    \begin{bmatrix}
        \sin \Theta \sin \Psi & \cos \Psi & 0 \\
        \sin \Theta \cos \Psi & -\sin \Psi & 0 \\
        \cos \Theta & 0 & 1 
    \end{bmatrix}
    \begin{bmatrix}
        \dot{\Phi} \\ \dot{\Theta} \\ \dot{\Psi}
    \end{bmatrix}. \label{polyatom-matrixV}
\end{gather}

Введем набор внутренних координат $\mf{q} = \lb q_1, \dots q_s \rb$ и, используя связь координат Якоби с введенными координатами $\mf{R}_j = \mf{R}_j(\mf{q})$, перепишем выражение для кинетической энергии в форме \cite{petrov2015}
\begin{gather}
    \Tl = \frac{1}{2} \dot{\mf{q}}^+ \bba \dot{\mf{q}} + \bOmega^+ \bbA \mf{q} + \frac{1}{2} \bOmega^+ \bbI \, \bOmega, \label{body-fixed-lagrange-energy} 
\end{gather}
%
где через $\bba, \bbA, \bbI$ обозначены матрица относительной кинетической энергии, кориолисова матрица и матрица тензора инерции, соответственно. Элементы матриц относительной кинетической энергии и кориолисова взаимодействия заданы следующими выражениями
\begin{gather}
    \bba_{jk} = \sum_{i = 1}^{N - 1} \mu_i \frac{\partial \mf{R}_i}{\partial q_j} \frac{\partial \mf{R}_i}{\partial q_k}, \quad \bbA_{jk} = \sum_{i = 1}^{N - 1} \mu_i \lsq \mf{R}_i \times \frac{\partial \mf{R}_i}{\partial q_k} \rsq_j. \label{polyatom-kinetic-energy-matrices}
\end{gather}

Выражение для кинетической энергии \eqref{body-fixed-lagrange-energy} для перехода к Гамильтоновой форме удобно переписать в матричном виде
\begin{gather}
    \Tl = \frac{1}{2} \begin{bmatrix} \bOmega^+ & \dot{\mf{q}}^+ \end{bmatrix} \bbB \begin{bmatrix} \bOmega \\ \dot{\mf{q}} \end{bmatrix}, \label{polyatom-block-matrix-kin-energy}
\end{gather}
%
где через $\bbB$ обозначена блочная матрица со следующими элементами 
\begin{gather}
    \bbB = \begin{bmatrix} 
    \bbI & \bbA \\ \bbA^+ & \bba 
    \end{bmatrix}.
\end{gather}

Можно показать, что если ввести величину $\mf{J}$ как производную кинетической энергии в лагранжевой форме $\Tl$ по вектору угловой скорости $\bOmega$, то $\mf{J}$ суть вектор углового момента в подвижной системе координат (приложение \ref{appendix:angular-momentum-body-fixed}). Обобщенные импульсы $\mf{p}$, сопряженные координатам $\mf{q}$, по определению ранвы производным кинетической энергии в лагранжевой форме $\Tl$ по обобщенным скоростям $\dot{\mf{q}}$: 
\begin{gather}
    \mf{J} = \frac{\partial \Tl}{\partial \bOmega} = \bbI \bOmega + \bbA \dot{\mf{q}}, \label{polyatom-angmom1} \\
\mf{p} = \frac{\partial \Tl}{\partial \dot{\mf{q}}} = \bbA^+ \bOmega + \bba \dot{\mf{q}} \label{polyatom-gen-momenta}.
\end{gather}

Заметим, что выражения \eqref{polyatom-angmom1}, \eqref{polyatom-gen-momenta} могли быть получены дифференцированием выражения  \eqref{polyatom-block-matrix-kin-energy} по блочному вектору с компонентами $\bOmega$ и $\dot{\mf{q}}$. Выражения для углового момента и обобщенных импульсов объединим в один блочный вектор
\begin{gather}
    \begin{bmatrix} \mf{J} \\ \mf{p} \end{bmatrix} = \bbB \begin{bmatrix} \bOmega \\ \dot{\mf{q}} \end{bmatrix} = 
    \begin{bmatrix} \bbI & \bbA \\ \bbA^+ & \bba \end{bmatrix} \begin{bmatrix} \bOmega \\ \dot{\mf{q}} \end{bmatrix}.
\end{gather}

Для того, чтобы выразить лагранжевы переменные из полученного выражения, нам необходимо обратить блочную матрицу $\bbB$. Это обращение удобно выполнить при помощи формул Фробениуса \cite{petrov2015, gantmaher}. Через $\bbG$ обозначают обратную матрицу к $\bbB$, ее матричные компоненты выражаются как
\begin{gather}
    \begin{aligned}
        \bbG_{11} &= \lb \bbI - \bbA \bba^{-1} \bbA^+ \rb^{-1} \\
        \bbG_{12} &= -\bbI^{-1} \bbA \bbG_{22} = -\bbG_{11} \bbA \bba^{-1} \\
        \bbG_{21} &= -\bba^{-1} \bbA^+ \bbG_{11} = \bbG_{22} \bbA^+ \bbI^{-1} \\
        \bbG_{22} &= \lb \bba - \bbA^+ \bbI^{-1} \bbA \rb^{-1}
    \end{aligned} \label{polyatom-frobenius}
\end{gather}

Переход от кинетической энергии в форме Лагранжа к кинетической энергии в форме Гамильтона осуществляем при помощи стандартной процедуры \cite{goldstein}
\begin{gather}
    \Th = \begin{bmatrix} \bOmega^+ & \dot{\mf{q}}^+ \end{bmatrix} \begin{bmatrix} \mf{J} \\ \mf{p} \end{bmatrix} - \Tl = \frac{1}{2} \begin{bmatrix} \mf{J}^+ & \mf{p}^+ \end{bmatrix} \bbG \begin{bmatrix} \mf{J} \\ \mf{p} \end{bmatrix} = \frac{1}{2} \mf{J}^+ \bbG_{11} \mf{J} + \mf{J}^+ \bbG_{12} \mf{p} + \frac{1}{2} \mf{p}^+ \bbG_{22} \mf{p}, \label{general-rovibrational-kin-energy} 
\end{gather}
а полный колебательно-вращательный гамильтониан получается в результате добавления потенциальной энергии
\begin{gather}
    H = \frac{1}{2} \mf{J}^+ \bbG_{11} \mf{J} + \mf{J}^+ \bbG_{12} \mf{p} + \frac{1}{2} \mf{p}^+ \bbG_{22} \mf{p} + U(\mf{q}). \label{general-rovibrational-hamiltonian} 
\end{gather}

В данной работе мы будем рассматривать системы, состоящие из жесткой линейной молекулы атома и из двух жестких линейных молекул. \par
В случае системы линейная молекула$-$атом построим вектор Якоби вдоль линейной молекулы, тогда второй вектор Якоби соединяет центр масс линейной молекулы с атомом. Определим подвижную систему координат таким образом, чтобы построенные вектора Якоби $\mf{R}_1$, $\mf{R}_2$ задавали плоскость $OXZ$ подвижной системы. Дополнительно потребуем, чтобы вектор $\mf{R}_2$, соединяющий центр масс линейной молекулы с атомом, лежал вдоль оси $OZ$ (см. рис. \ref{fig:body-fixed-linear-atom}). Такое определение системы координат известно как $R$-вложение \cite{tennyson1986}. Обозначим длину линейной молекулы через $l$.
    
\begin{figure}[H]
    \centering
    %\begin{minipage}{0.49\linewidth}
    \includegraphics[width=0.5\linewidth]{pictures/triatom_coordinates.pdf}
    %\end{minipage}
    %\begin{minipage}{0.49\linewidth}
    %    Взять картинку из будущей статьи с N$_2$-N$_2$
    %\end{minipage}
    \caption{Молекулярная система координат для системы линейная молекула-атом}
    \label{fig:body-fixed-linear-atom}
\end{figure}

В качестве обобщенных координат выберем $R$ -- длину вектора Якоби $\mf{R}_2$ или, что эквивалентно, расстояние от атома до центра масс линейной молекулы, и угол $\theta$ между векторами Якоби $\mf{R}_1$ и $\mf{R}_2$. Выпишем координаты векторов в подвижной системе через обобщенные координаты $\mf{q} = \lb R, \theta \rb$: 
\begin{gather}
    \begin{aligned}
        X_1 &= l \sin \theta \\
        Y_1 &= 0 \\
        Z_1 &= l \cos \theta
    \end{aligned} \qquad
    \begin{aligned}
        X_2 &= 0 \\ 
        Y_2 &= 0 \\
        Z_2 &= R 
    \end{aligned}. \label{linear-molecule-atom-jacobi-coords}
\end{gather}

Вывод дальнейших выражений реализовывался в системе компьютерной алгебры Maple. Координаты векторов Якоби \eqref{linear-molecule-atom-jacobi-coords} использовались для расчета компонент матриц относительной кинетической энергии, кориолисова взаимодействия и тензора инерции по выражениям \eqref{polyatom-kinetic-energy-matrices}. В случае линейная молекула$-$атом блоки матрицы $\bbG$ могут быть получены в компактном виде. Уже для случая двух жестких линейных молекул выражения для блоков матрицы $\bbG$ оказываются слишком громоздкими, поэтому была разработана схема реализации траекторного расчета, избегающая аналитической работы с компонентами этих матриц, переносимая на системы с произвольным количеством вращательных степеней свободы. \par
В качестве примера системы линейная молекула$-$атом мы выбрали CO$_2-$Ar. Внутримолекулярные колебания молекулы CO$_2$ происходят существенно быстрее межмолекулярных движений комплекса с атомом аргона, поэтому предполагается, что взаимодействие между внутри- и межмолекулярными колебательными модами в данном случае компенсируется. \par
Приведенные массы частиц Якоби согласно \eqref{polyatom-jacobi-masses} равны
\begin{gather}
    \mu_1 = \frac{m_1}{2}, \quad \mu_2 = \frac{m_2 \lb 2 m_1 + m_3 \rb}{2 m_1 + m_2 + m_3},
\end{gather}
% 
где через $m_1$ обозначена масса атома кислорода, через $m_2$ -- масса атома аргона, через $m_3$ -- масса атома углерода. \par
При помощи системы компьютерной алгебры были получены следующие выражения для матриц кинетической энергии в форме Лагранжа
\begin{gather}
	\bba =
	\begin{bmatrix}
		\mu_2 & 0 \\
		0 & \mu_1 l^2
	\end{bmatrix}, \quad 
	\bbA = 
	\begin{bmatrix}
		0 & 0 \\
		0 & \mu_1 l^2 \\
		0 & 0 
	\end{bmatrix}, \quad
	\bbI = 
	\begin{bmatrix}
		\mu_1 l^2 \cos^2 \theta + \mu_2 R^2 & 0 & -\mu_1 l^2 \sin \theta \cos \theta \\
		0 & \mu_1 l^2 + \mu_2 R^2 & 0 \\
		- \mu_1 l^2 \sin \theta \cos \theta & 0 & \mu_1 l^2 \sin^2 \theta
	\end{bmatrix}. \notag
\end{gather}

Подставив полученные выражения для матриц в формулы Фробениуса \eqref{polyatom-frobenius}, приходим к следующим выражениям для матриц, определяющим кинетическую энергию в форме Гамильтона 
\begin{gather}
	\bbG_{11} =
	\begin{bmatrix}
		\dfrac{1}{\mu_2 R^2} & 0 & \dfrac{\ctg \theta}{\mu_2 R^2} \\
		0 & \dfrac{1}{\mu_2 R^2} & 0 \\
		\dfrac{\ctg \theta}{\mu_2 R^2} & 0 & \dfrac{\ctg^2 \theta}{\mu_2 R^2} + \dfrac{1}{\mu_1 l^2 \sin^2 \theta}
	\end{bmatrix}, \quad
	\bbG_{12} =
	\begin{bmatrix}
		0 & 0 \\
		0 & - \dfrac{1}{\mu_2 R^2} \\
		0 & 0
	\end{bmatrix}, \quad 
	\bbG_{22} = 
	\begin{bmatrix}
		\dfrac{1}{\mu_2} & 0 \\
		0 & \dfrac{1}{\mu_2 R^2} + \dfrac{1}{\mu_1 l^2}
	\end{bmatrix}. \notag
\end{gather}

Итак, кинетическая энергия в форме Гамильтона для системы CO$_2-$Ar в выбранной нами молекулярной системе отсчета получается следующей
\begin{gather}
\Th = \frac{1}{2 \mu_2} p_R^2 + \lb \frac{1}{2 \mu_2 R^2} + \frac{1}{2 \mu_1 l^2} \rb p_\theta^2 - \frac{1}{\mu_2 R^2} p_\theta \Jy + \frac{1}{2 \mu_2 R^2} \Jy^2 + \frac{1}{2 \mu_2 R^2} \Jx^2 + \frac{1}{2 \sin^2 \theta} \lb \frac{\cos^2 \theta}{\mu_2 R^2} + \frac{1}{\mu_1 l^2} \rb \Jz^2 + \notag \\
+ \frac{\ctg \theta}{\mu_2 R^2} \Jx \Jz. 
\end{gather}

В случае системы, состоящей из двух жестких линейных молекул, удобно построить векторы Якоби вдоль каждой из линейных молекул, тогда третий вектор соединяет их центры масс. Определим подвижную систему координат таким образом, чтобы вектор, соединяющий центры масс линейных молекул, лежал на оси $OZ$ подвижной системы. Потребуем, чтобы вектор, определяющий одну из линейных молекул, лежал в плоскости $OXZ$ подвижной системы (Рис. \ref{fig:two-linear-molecules-body-fixed}). Вторая молекула при этом может выходить из плоскости подвижной системы -- для того, чтобы описать ее ориентацию необходимо ввести два угла, тогда как для описания ориентации первой молекулы достаточно одного полярного угла. Введя таким образом систему координат мы заведомо сделали линейные молекулы неравноправными, так что соответствующие им координаты будут входить в кинетическую энергию неодинаковым образом. Обозначим длины линейных молекул через $l_1$ и $l_2$, соответственно. В качестве обобщенных координат выберем $R$ -- расстояние между центрами масс линейных молекул, $\theta_1$, $\theta_2$ -- полярные углы линейных молекул в плоскости $OXZ$, $\varphi$ -- угол между плоскостью $OXZ$ и плоскостью, заданной молекулой, выходящей из плоскости подвижной системы, и центром масс второй молекулы.

\begin{figure}[H]
    \centering
    \includegraphics[width=0.5\linewidth]{pictures/n2n2_coordinate_frame.png}
    \caption{Молекулярная система координат для системы из двух линейных молекул; взять картинку из драфта статьи?}
    \label{fig:two-linear-molecules-body-fixed}
\end{figure}

Выпишем координаты векторы Якоби в подвижной системе через обобщенные координаты $\mf{q} = \lb R, \theta_1, \varphi, \theta_2 \rb$:
\begin{gather}
    \begin{aligned}
        X_1 &= l_1 \sin \theta_1 \\
        Y_1 &= 0 \\
        Z_1 &= l_1 \cos \theta_1 
    \end{aligned} \qquad
    \begin{aligned}
        X_2 &= l_2 \cos \varphi \sin \theta_2 \\
        Y_2 &= l_2 \sin \varphi \sin \theta_2 \\
        Z_2 &= l_2 \cos \theta_2
    \end{aligned} \qquad
    \begin{aligned}
        X_3 &= 0 \\
        Y_3 &= 0 \\
        Z_3 &= R
    \end{aligned} \label{polyatom-linear-linear-coordinates}
\end{gather}

В качестве примера системы, состоящей из двух линейных молекул, выбрана N$_2-$N$_2$. Приведенные масс частиц Якоби равны
\begin{gather}
    \mu_1 = \frac{m}{2}, \quad \mu_2 = \frac{m}{2}, \quad \mu_3 = 2m, \label{polyatom-n2n2-jacobi-masses}
\end{gather}
%
где через $m$ обозначена масса атома азота. Длины линейных молекул в дальнейшем обозначаются одинаково $l = l_1 = l_2$.

Выражения для матриц кинетической энергии в форме Лагранжа были получены из \eqref{polyatom-linear-linear-coordinates} при помощи системы компьютерной алгебры. Введенный набор координат является ортогональным, т.к. матрица относительной кинетической энергии оказывается диагональной. Приводим выражения для случая двух произвольных линейных молекул через массы частиц Якоби; выражения для конкретного случая N$_2-$N$_2$ получаются в результате подстановки масc \eqref{polyatom-n2n2-jacobi-masses}.  
\begin{gather}
\bba = 
\begin{bmatrix}
\mu_3 & 0 & 0 & 0 \\
0 & \mu_2 l_2^2 \sin \theta_2 & 0 & 0 \\
0 & 0 & \mu_1 l_1^2 & 0 \\
0 & 0 & 0 & \mu_2 l_2^2 \\ 
\end{bmatrix}, \quad  
\bbA = 
\begin{bmatrix}
0 & -\mu_2 l_2^2 \cos \varphi \sin \theta_2 \cos \theta_2 & 0 & -\mu_2 l_2^2 \sin \varphi  \\
0 & -\mu_2 l_2^2 \sin \varphi \sin \theta_2 \cos \theta_2 & \mu_1 l_1^2 & \mu_2 l_2^2 \cos \varphi \\
0 &  \mu_2 l_2^2 \sin^2 \theta_2 & 0 & 0
\end{bmatrix} \notag 
\end{gather}

Компоненты матрицы тензора инерции в подвижной системе отсчета получаются следующие
\begin{gather}
    \begin{aligned}
        I_{XX} &= \mu_1 l_1^2 \cos^2 \theta_1 + \mu_2 l_2^2 (\sin^2 \varphi \sin^2 \theta_2 + \cos^2 \theta_2) + \mu_3 R^2, \\ 
        I_{XY} &= -\mu_2 l_2^2 \sin \varphi \cos \varphi \sin^2 \theta_2, \\ 
        I_{XZ} &= -\mu_1 l_1^2 \sin \theta_1 \cos \theta_1 - \mu_2 l_2^2 \cos \varphi \sin \theta_2 \cos \theta_2, \\
        I_{YY} &= \mu_1 l_1^2 + \mu_2 l_2^2 (\cos^2 \varphi \sin^2 \theta_2 + \cos^2 \theta_2) + \mu_3 R^2, \\
        I_{YZ} &= -\mu_2 l_2^2 \sin \varphi \sin \theta_2 \cos \theta_2, \\
        I_{ZZ} &= \mu_1 l_1^2 \sin^2 \theta_1 + \mu_2 l_2^2 \sin^2 \theta_2.  
    \end{aligned} 
\end{gather}

Как уже было сказано, выражения для блоков матрицы $\bbG$ в этом случае получаются слишком громоздкими для работы, поэтому не приводятся. Для реализации траекторного расчета нет необходимости получать аналитические выражения для этих матриц. 

\section{Уравнения вращательной динамики}

В отсутствии внешних сил угловой момент $\mf{j}$ в лабораторной системе координат является векторным интегралом движения. Компоненты вектора углового момента в подвижной системе отсчета связаны с компонентами в лабораторной системе матрицей $\bbS^{-1}$
\begin{gather}
    \mf{J} = \bbS^{-1} \mf{j}. \label{polyatom-j-connection}
\end{gather}

Производная вектора углового момента в подвижной системе удовлетворяет следующему соотношению \cite{goldstein}
\begin{gather}
    \dot{\mf{J}} + \lsq \bOmega \times \mf{J} \rsq = 0.
\end{gather}

Используя теорему Донкина можно показать, ??? \cite{petrov2015}
\begin{gather}
    \bOmega = \frac{\partial \Th}{\partial \mf{J}}. 
\end{gather}

Таким образом, векторы угловой скорости и углового момента, не будучи сами динамическими переменными, связаны такими же соотношениями как канонически сопряженные переменные (так называемые \textit{псевдо-канонические переменные}):
\begin{gather}
    \mf{J} = \frac{\partial \Tl}{\partial \bOmega}, \qquad \bOmega = \frac{\partial \Th}{\partial \mf{J}}.
\end{gather}

Закон сохранения углового момента в подвижной системе координат преобразуется к уравнениям, называемым обобщенными уравнениями Эйлера \cite{petrov2015}
\begin{gather}
    \dot{\mf{J}} + \Big[ \frac{\partial \Th}{\partial \mf{J}} \times \mf{J} \Big] = 0. \label{generalized-euler-equations}
\end{gather}

Понятно, что модуль углового момента в подвижной системе отсчета является интегралом движения (в силу ортогональности преобразования в \eqref{polyatom-j-connection}). Следовательно, уравнения \eqref{generalized-euler-equations} можно преобразовать таким образом, чтобы учитывать сохранение модуля углового момента. Будем описывать ориентацию вектора углового момента в подвижной системе при помощи пары сферических углов $\alpha$ и $\beta$
\begin{gather}
    \begin{aligned}
        \Jx &= J \cos \alpha \sin \beta, \\
        \Jy &= J \sin \alpha \sin \beta, \\
        \Jz &= J \cos \beta.
    \end{aligned} \label{polyatom-angmom-spherical-coordinates}
\end{gather}

Получим соотношения между производными сферических координат и производными декартовых координат вектора углового момента. Для этого продифференцируем соотношения \eqref{polyatom-angmom-spherical-coordinates} по времени 
\begin{gather}
    \begin{aligned}
        \dot{\Jx} &= \dot{J} \cos \alpha \sin \beta - J \dot{\alpha} \sin \alpha \sin \beta + J \dot{\beta} \cos \alpha \cos \beta, \\
        \dot{\Jy} &= \dot{J} \sin \alpha \sin \beta + J \dot{\alpha} \cos \alpha \sin \beta + J \dot{\beta} \sin \alpha \cos \beta, \\
        \dot{\Jz} &= \dot{J} \cos \beta - J \dot{\beta} \sin \beta,
    \end{aligned}
\end{gather}

и разрешим их относительно сферических координат
\begin{gather}
    \begin{aligned}
        \dot{J} &= \dJx \cos \alpha \sin \beta + \dJy \sin \alpha \sin \beta + \dJz \cos \beta, \\
        \dot{\alpha} &= -\frac{\dJx \sin \alpha + \dJy \cos \alpha}{J \sin \beta}, \\
        \dot{\beta} &= \frac{\dJx \cos \alpha \cos \beta + \dJy \sin \alpha \cos \beta  - \dJz \sin \beta}{J}.
    \end{aligned} \label{polyatom-temp1}
\end{gather}

Подставив производные декартовых координат из \eqref{generalized-euler-equations} в соотношения \eqref{polyatom-temp1}, получим
\begin{gather}
    \begin{aligned}
        \dot{J} &= 0, \\
        \dot{\alpha} &= \lb \frac{\partial H}{\partial \Jx} \cos \alpha + \frac{\partial H}{\partial \Jy} \sin \alpha \rb \ctg \beta - \frac{\partial H}{\partial \Jz}, \\
        \dot{\beta} &= \frac{\partial H}{\partial \Jx} \sin \alpha - \frac{\partial H}{\partial \Jy} \cos \alpha.
    \end{aligned} \label{polyatom-temp2}
\end{gather}

Первое из уравнений \eqref{polyatom-temp2} говорит нам о сохранении модуля вектора углового момента в подвижной системе координат. Это уравнение мы отбросим, так как оно не дает нам ничего нового. Мы получили динамические уравнения для углов $\alpha$, $\beta$, но они содержат производные гамильтониана по декартовым компонентам углового момента. Чтобы получить замкнутые уравнения относительно сферических углов, выразим при помощи цепного правила производные гамильтониана по декартовым координаты через производные по сферическим координатам:    
\begin{gather}
    \begin{aligned}
        \frac{\partial H}{\partial \Jx} &= \sum_{\gamma = J, \alpha, \beta} \frac{\partial H}{\partial \gamma} \frac{\partial \gamma}{\partial \Jx}, \\
        \frac{\partial H}{\partial \Jy} &= \sum_{\gamma = J, \alpha, \beta} \frac{\partial H}{\partial \gamma} \frac{\partial \gamma}{\partial \Jy}, \\
        \frac{\partial H}{\partial \Jz} &= \sum_{\gamma = J, \alpha, \beta} \frac{\partial H}{\partial \gamma} \frac{\partial J_\gamma}{\partial \Jz}.
    \end{aligned} \label{polyatom-temp3}
\end{gather}

Соотношения \eqref{polyatom-temp3} удобнее представить в матричном виде. Вычислив производные сферических координат по декартовым, приходим к следующим соотношениям
\begin{gather}
    \begin{bdmatrix}
        \frac{\partial H}{\partial \Jx} \\
        \frac{\partial H}{\partial \Jy} \\
        \frac{\partial H}{\partial \Jz}
    \end{bdmatrix} = \bbF 
    \begin{bdmatrix}
        \frac{\partial H}{\partial J} \\
        \frac{\partial H}{\partial \alpha} \\
        \frac{\partial H}{\partial \beta}
    \end{bdmatrix}, \qquad 
    \bbF = 
    \begin{bdmatrix}
        \cos \alpha \sin \beta & - \frac{1}{J} \frac{\sin \alpha}{\sin \beta} & \frac{1}{J} \cos \alpha\ \cos \beta \\ 
        \sin \alpha \sin \beta & \frac{1}{J} \frac{\cos \alpha}{\sin \beta} & \frac{1}{J} \sin \alpha \cos \beta \\
        \cos \beta & 0 & -\frac{1}{J} \sin \beta
    \end{bdmatrix}. \label{polyatom-temp4}
\end{gather}

Подставив уравнения \eqref{polyatom-temp4} в динамические уравнения \eqref{polyatom-temp2}, приходим к динамическим уравнениям, замкнутым относительно сферических переменных:
\begin{gather}
    \begin{aligned}
        \dot{\alpha} &= \frac{1}{J \sin \beta} \frac{\partial H}{\partial \beta}, \\
        \dot{\beta} &= - \frac{1}{J \sin \beta} \frac{\partial H}{\partial \alpha}.
    \end{aligned} \label{generalized-euler-equations-angles}
\end{gather}

Кроме того, уравнения вращательной динамики могут быть написаны относительно углов Эйлера $\bUpsilon = \lb \Phi, \Theta, \Psi \rb$ и сопряженных к ним импульсов $\pe = \lb p_\Phi, p_\Theta, p_\Psi \rb$. Так как они являются набором канонически сопряженных переменных, то в этом случае уравнения вращательной динамики будут представлять собой два векторных уравнения Гамильтона
\begin{gather}
    \begin{aligned}
        \dot{\boldsymbol{\Upsilon}}_e &= \frac{\partial \Th}{\partial \pe}, \\
        \dot{\mathbf{p}}_e &= -\frac{\partial \Th}{\partial \bUpsilon}.
    \end{aligned} \label{euler-hamilton-equations}
\end{gather}

Итак, для описания вращательной динамики у нас имеется три разных системы динамических уравнений:
\begin{enumerate}
    \item Обобщенные уравнения Эйлера, записанные в компонентах углового момента \eqref{generalized-euler-equations}
    \item Преобразованные обобщенные уравнения Эйлера, в которых учтено сохранение модуля вектора углового момента в молекулярной системе координат \eqref{generalized-euler-equations-angles}
    \item Уравнения, записанные в эйлеровых углах и сопряженных к ним импульсах \eqref{euler-hamilton-equations}
\end{enumerate}

Система уравнений \eqref{generalized-euler-equations-angles} содержит наименьшее возможное количество уравнений для описания вращения. Однако введение сферической системы вводит в систему уравнений два полюса -- $\beta = 0$ и $\beta = \pi$. При этих значениях полярного угла, правые части уравнений принимают бесконечные значения, хотя они отвечают физически реализуемым положениям полного углового момента вдоль положительного или отрицательного направлений оси $OZ$ подвижной системы. Эту особенность сферической системы несложно преодолеть при построении вычислительной процедуре. Уравнения \eqref{generalized-euler-equations-angles} выписаны таким образом, что выделенной осью является ось $OZ$. Не представляет труда выписать аналогичную систему уравнений таким образом, чтобы выделенной осью была ось $OX$ или $OY$. Пусть в таком случае вращательными переменными будут $\alpha^\prime$ и $\beta^\prime$. Положение углового момента вдоль оси $OZ$ для таким образом построенной сферической системы не является особым. При численном интегрировании уравнений \eqref{generalized-euler-equations-angles} в случае приближения решения к одному из полюсов следует пересчитать угловые переменные $\alpha$, $\beta$ в переменные $\alpha^\prime$, $\beta^\prime$ и продолжить решение уже системы уравнений, построенной с другой выделенной осью. При прохождении полюса можно пересчитать вращательные переменные обратно и продолжить интегрирование в переменных $\alpha$, $\beta$. \par
Среди двух векторных уравнений \eqref{euler-hamilton-equations} содержится одно тривиальное уравнение. Рассмотрим выражение для эйлеровых импульсов $\pe$
\begin{gather}
    \pe = \frac{\partial \Tl}{\partial \dot{\boldsymbol{\Upsilon}}_e} = \frac{\partial \bOmega}{\partial \dot{\boldsymbol{\Upsilon}}_e} \frac{\partial \Tl}{\partial \bOmega} = \bbV^+ \mf{J}.
\end{gather}

Следовательно, вектор углового момента связан с вектором эйлеровых импульсов $\pe$ соотношением
\begin{gather}
    \mf{J} = \bbW \pe,
\end{gather}
%
где через $\bbW$ обозначена матрица $\lb \bbV^{+} \rb^{-1}$. Выражения для компонент матрицы $\bbV$ приведены в формуле \eqref{polyatom-matrixV}. Заметим, что эйлеров угол $\Phi$ не содержится в выражениях для компонент матрицы $\bbV$, следовательно, матрица $\bbW$ также не зависит от угла $\Phi$. Таким образом, угол $\Phi$ не содержится в общем колебательно-вращательном гамильтониане \eqref{general-rovibrational-hamiltonian}, откуда следует, что сопряженный импульс $p_\Phi$ является интегралом движения. \par

\section{Полные системы динамических уравнений и обращение классических траекторий}

В предыдущем параграфе мы рассматривали вопрос общего вида уравнений, описывающих вращательное поведение системы. Для получения полной системы динамических уравнений эти уравнения должны быть дополнены гамильтоновыми уравнениями по переменным $\mf{q}$ и $\mf{p}$
\begin{gather}
    \begin{aligned}
        \dot{\mf{q}} &= \frac{\partial H}{\partial \mf{p}}, \\
        \dot{\mf{p}} &= -\frac{\partial H}{\partial \mf{q}}.
    \end{aligned} \label{hamilton-equations}
\end{gather}

Рассмотрим структуру гамильтоновых уравнений \eqref{hamilton-equations} с Гамильтонианом в форме \eqref{general-rovibrational-hamiltonian}. Отметим, что блоки матрицы $\bbG$ являются функциями обобщенных координат $\mf{q}$:
\begin{gather}
    \begin{aligned}
        \frac{\partial H}{\partial \mf{p}} &= \bbG_{22} \mf{p} + \bbG_{12}^{+} \mf{J}, \\
        \frac{\partial H}{\partial \mf{q}} &= \frac{1}{2} \mf{p}^+ \frac{\partial \bbG_{22}}{\partial \mf{q}} \mf{p} + \mf{J}^+ \frac{\partial \bbG_{12}}{\partial \mf{q}} \mf{p} + \frac{1}{2} \mf{J}^+ \frac{\partial \bbG_{11}}{\partial \mf{q}} \mf{J} + \frac{\partial U}{\partial \mf{q}}.
    \end{aligned} \label{hamilton-equations1}
\end{gather}

Для того, чтобы построить вычислительную процедуру для численного интегрирования уравнений \eqref{hamilton-equations1} мы должны получить выражения, по которым можно осуществлять расчет правых частей в произвольной точке фазового пространства $\lb \mf{q}_0, \mf{p}_0, \mf{J}_0 \rb$. Заметим, что формулы Фробениуса \eqref{polyatom-frobenius}, представляющие сложность при получении аналитических выражений для блоков матрицы $\bbG$, оказываются очень удобными, если матрицы $\bba$, $\bbA$, $\bbI$ оказываются числовыми матрицами. Для систем линейная молекула$-$атом и пара линейных молекул выражения для матриц лагранжевой кинетической энергии оказываются в достаточной степени простыми, и их вывод может быть в существенной степени автоматизирован при помощи системы компьютерной алгебры. Числовые значения компонент этих матриц определяются лишь значениями обобщенных координат $\mf{q}_0$. Вычислив по полученным выражениям численные значения компонент матрицы $\bba(\mf{q}_0)$, $\bbA(\mf{q}_0)$, $\bbI(\mf{q}_0)$, получаем значения матриц $\bbG_{11}(\mf{q}_0)$, $\bbG_{12}(\mf{q}_0)$, $\bbG_{22}(\mf{q}_0)$ по формулам Фробениуса. Таким образом, вычисление вектора производных $\partial H / \partial \mf{p}$ в произвольной точке не представляет сложности. \par
Рассмотрим вопрос дифференцирования блоков матрицы $\bbG$ по обобщенным координатам. Пусть $\bbM(\mf{q})$ -- дифференцируемая, обратимая матричная функция векторного аргумента $\mf{q}$. Производная обратной матрицы $\bbM^{-1}(\mf{q})$ по векторной переменной $\mf{q}$ связана с производной $\bbM^\prime(\mf{q})$ следующим соотношением
\begin{gather}
\frac{\partial}{\partial \mf{q}} \bbM^{-1}(\mf{q}) = - \bbM^{-1}(\mf{q}) \bbM^\prime(\mf{q}) \bbM^{-1}(\mf{q}).
\end{gather}


\begin{subappendices}
    \section{Вектор углового момента в подвижной системе отсчета} \label{appendix:angular-momentum-body-fixed}

    Рассмотрим производную кинетической энергии в лагранжевой форме $\Tl$ по вектору угловой скорости $\bOmega$, компонентны которого выражены в подвижной системе отсчета. Продифференцировав выражение \eqref{body-fixed-lagrange-energy} по вектору угловой скорости, получаем 
    \begin{gather}
        \frac{\partial \Tl}{\partial \bOmega} = \bbA \dot{\mf{q}} + \bbI \bOmega. \label{appendix-angular-momentum1}
    \end{gather}

    Несложно показать, что вектор углового момента $\mf{j}$ в лабораторной системе отсчета может быть записан через векторы Якоби как
    \begin{gather}
        \mf{j} = \sum_{i = 1}^{N - 1} \mu_i \lsq \bs{\rho}_i \times \dot{\bs{\rho}}_i \rsq. \label{appendix-angular-momentum-jacobi-vectors}
    \end{gather}

    Выразим производная вектора $\bs{\rho}_i$ в лабораторной системе координат через производную в подвижной системе координат \cite{goldstein}
    \begin{gather}
        \dot{\bs{\rho}}_i = \bbS \lb \dot{\mf{R}}_i + \lsq \bs{\Omega} \times \mf{R}_i \rsq \rb. \label{appendix-jacobi-vector-derivative} 
    \end{gather}

    Подставив выражение \eqref{appendix-jacobi-vector-derivative} в выражение углового момента \eqref{appendix-angular-momentum-jacobi-vectors} и осуществив несложные алгебраические преобразования, приходим к 
    \begin{gather}
        \mf{j} = \sum_{i = 1}^{N-1} \mu_i \lsq \bs{\rho}_i \times \bbS \lb \dot{\mf{R}}_i + \Big[ \bOmega \times \mf{R}_i \Big] \rb \rsq = \bbS \sum_{i = 1}^{N-1} \mu_i \Big[ \mf{R}_i \times \dot{\mf{R}}_i \Big] + \bbS \sum_{i = 1}^{N-1} \mu_i \Big[ \mf{R}_i \times \lsq \bOmega \times \mf{R}_i \rsq \Big] = \bbS \bbA \dot{\mf{q}} + \bbS \bbI \bOmega.
    \end{gather}

    Умножая обе части на матрицу $\bbS^{-1}$, получаем в правой части выражение \eqref{appendix-angular-momentum1}
    \begin{gather}
        \bbS^{-1} \mf{j} = \bbA \dot{\mf{q}} + \bbI \bOmega.
    \end{gather}

    Согласно введенному определению матрицы $\bbS$, выражение в левой части суть вектор углового момента в подвижной системе отсчета. Таким образом, мы показали, что производная кинетической энергии в лагранжевой форме по вектору угловой скорости в подвижной системе равна вектору углового момента в подвижной системе
    \begin{gather}
        \mf{J} = \frac{\partial \Tl}{\partial \bOmega}.
    \end{gather}
    
\end{subappendices}

%\section{Существующие методы моделирования столкновительно-индуцированных спектров}
%\section{Координаты Якоби}
%\section{Гамильтониан в подвижной системе отсчета}
%\section{Сравнение динамических систем уравнений}
%\section{Классический подход в подвижной системе координат}
%\section{Генерация начальных условий}
%\section{Сравнение с экспериментальными данными}

\chapter{Выводы}

% можно попытаться докрутить свой bst-file, пока лень
%\bibliographystyle{mybst}
\bibliographystyle{unsrt}

\bibliography{biblio}

\end{document}
