\documentclass[12pt]{article}

\usepackage[T1]{fontenc}
\usepackage[utf8]{inputenc}
\usepackage[russian]{babel}

% page margin
\usepackage[top=2cm, bottom=2cm, left=2cm, right=2cm]{geometry}

\usepackage{graphicx}

% AMS packages
\usepackage{amsmath}
\usepackage{amssymb}
\usepackage{amsfonts}
\usepackage{amsthm}

% blackboar lettering
\usepackage{dsfont}
\usepackage{bbm}

\usepackage{fancyhdr}
\pagestyle{fancy}
% modifying page layout using fancyhdr
\fancyhf{}
\renewcommand{\sectionmark}[1]{\markright{\thesection\ #1}}
\renewcommand{\subsectionmark}[1]{\markright{\thesubsection\ #1}}

\rhead{\fancyplain{}{\rightmark }}
\cfoot{\fancyplain{}{\thepage }}

\newcommand{\lb}{\left(}
\newcommand{\rb}{\right)}

\begin{document}
\textbf{Слайд 1}. Начало активному изучению явления столкновительно-индуцированного поглощения (СИП) в газовых смесях положили исследования, относящиеся к середине прошлого века. Авторы [1] изучали поглощение азота при высоких давлениях и наблюдали полосу поглощения в области фундаментального перехода (2330 cм$^{-1}$). Была обнаружена квадратичная зависимость интегральной интенсивности полосы от давления, что свидетельствует о том, что поглощение вызвано парными межмолекулярными взаимодействиями, в ходе которых происходит индукция дипольного момента. Столкновительные комплексы обладают малыми временами жизни и низкой симметрией, а следовательно, и дипольным моментом. \par 
В этой работе мы рассматриваем столкновительно-индуцированное поглощение на примере молекулярных комплексов, образованных бездипольными мономерами. Парные состояния слабо связанных систем классифицируют на связанные, метастабильные и свободные. Связанные состояния мы не рассматриваем в нашей работе. Интегральный вклад связанных состояний при околокомнатных температурах невелик для многих систем по сравнению с вкладами свободных и метастабильных состояний. \par 
\textbf{Слайд 2}. Явление столкновительно-индуцированного поглощения в инфракрасной области вносит существенный вклад в радиационный баланс плотных планетных атмосфер. Его учитывают в так называемых окнах прозрачности -- спектральных диапазонах, в к которых отсутствуют дипольно-разрешенные полосы поглощения атмосферных молекул. Если говорить о планетах Солнечной системы, то можно выделить три основных типа атмосфер. Первые тип -- водородо-гелиевые атмосферы планет-гигантов. Второй тип -- преимущественно азотные атмосферы Титана и Земли. К третьему типу относятся атмосферы Венеры и Марса, состоящие преимущественно из диоксида углерода. В качестве примесных компонентов в атмосферах часто встречаются метан, благородные газы, аммиак, вода и др. Экспериментальные исследования набора молекулярных систем проводились при отдельных температурах, однако для приложений требуются данные в широком дипапазоне температур и частот, вследствие чего необходимо теоретически моделировать спектры СИП. \par 
\textbf{Слайд 5}. В силу того, что и спектральная и корреляционная функция -- действительны, они симметричные функции частоты и времени, соответственно. Существует ряд полуклассических процедур, поправляющих классическую спектральную функцию таким образом, чтобы она удовлетворяла квантовому детальному балансу. Процедура десиммеризации задана неоднозначно. Расчет спектральной функции как преобразования Фурье автокорреляционной функции дипольного момента для свободных и метастабильных состояний вычислительно неэффективен. Поэтому было выполнена определенная замена переменных в интеграле, подробно описанная в тексте работы для двухатомной системы, приводящая к выражению, по которому можно производить эффективное усреднение по ансамблю траекторий рассеяния. \par
\textbf{Слайд 9}. Как известно, классические траектории обратимы по времени, однако в результате накопления вычислительных ошибок на практике это не всегда так. На слайде представлены ...

\end{document}

