\documentclass[12pt]{article}

\usepackage[T1]{fontenc}
\usepackage[utf8]{inputenc}
\usepackage[russian]{babel}

% page margin
\usepackage[top=2cm, bottom=2cm, left=2cm, right=2cm]{geometry}

\usepackage{graphicx}

% AMS packages
\usepackage{amsmath}
\usepackage{amssymb}
\usepackage{amsfonts}
\usepackage{amsthm}

% blackboar lettering
\usepackage{dsfont}
\usepackage{bbm}

\usepackage{fancyhdr}
\pagestyle{fancy}
% modifying page layout using fancyhdr
\fancyhf{}
\renewcommand{\sectionmark}[1]{\markright{\thesection\ #1}}
\renewcommand{\subsectionmark}[1]{\markright{\thesubsection\ #1}}

\rhead{\fancyplain{}{\rightmark }}
\cfoot{\fancyplain{}{\thepage }}

\newcommand{\lb}{\left(}
\newcommand{\rb}{\right)}

\begin{document}
\textbf{Слайд 1}. Активное изучение явления столкновительно-индуцированного поглощения в газовых смесях происходит с середины прошлого века. В одной из самых ранних работ, посвященной этой тематике, Кроуфорд с соавторами изучали поглощение азота в инфракрасной области при высоких давлениях. Они наблюдали полосу поглощения в области фундаментального перехода (2330 cм$^{-1}$). Была обнаружена квадратичная зависимость интегральной интенсивности полосы по давлению, что свидетельствует о том, что за поглощение ответственны парные комплексы. В результате межмолекулярных взаимодействий в столкновительных комплексах происходит индукция дипольного момента. Так как они обладают малыми временами жизни, спектры оказываются протяженными по частоте. \par 
В этой работе мы рассматриваем столкновительно-индуцированное поглощение молекулярными комплексами, образованными бездипольными мономерами. Парные состояния слабо связанных систем классифицируют на связанные, метастабильные и свободные. Связанные состояния мы не рассматриваем в нашей работе. Их интегральный вклад при околокомнатных температурах невелик для многих систем по сравнению с вкладами свободных и метастабильных состояний. \par 
\textbf{Слайд 2}. Явление столкновительно-индуцированного поглощения в инфракрасной области вносит существенный вклад в радиационный баланс плотных планетных атмосфер. Его учитывают в так называемых окнах прозрачности -- спектральных диапазонах, в к которых отсутствуют дипольно-разрешенные полосы поглощения атмосферных молекул. Если говорить о планетах Солнечной системы, то можно выделить три основных типа атмосфер. К первому типу относятся водородо-гелиевые атмосферы планет-гигантов. Второй тип -- преимущественно азотные атмосферы Титана и Земли. К третьему типу относятся атмосферы Венеры и Марса, основу которых составляет диоксид углерода. В качестве примесных компонентов в атмосферах часто встречаются метан, благородные газы, аммиак, вода и др. Экспериментальные исследования набора молекулярных систем проводились при отдельных температурах, однако для приложений требуются данные в широком дипапазоне температур и частот, вследствие чего необходимо теоретически моделировать спектры столкновительно-индуцированного поглощения. \par 
\textbf{Слайд 5}. В силу того, что и спектральная и корреляционная функция -- действительны, они симметричные функции частоты и времени, соответственно. Существует ряд полуклассических процедур, поправляющих классическую спектральную функцию таким образом, чтобы она удовлетворяла квантовому детальному балансу. Процедура десиммеризации задана неоднозначно. Расчет спектральной функции как преобразования Фурье автокорреляционной функции дипольного момента для свободных и метастабильных состояний вычислительно неэффективен. Поэтому было выполнена определенная замена переменных в интеграле, подробно описанная в тексте работы для двухатомной системы, приводящая к выражению, по которому можно производить эффективное усреднение по ансамблю траекторий рассеяния. \par
\textbf{Слайд 9}. Как известно, классические траектории обратимы по времени, однако в результате накопления вычислительных ошибок на практике это не всегда так. На слайде представлены ...

\end{document}

