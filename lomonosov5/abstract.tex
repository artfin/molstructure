\documentclass[14pt]{extarticle}

\usepackage[T1]{fontenc}
\usepackage[utf8]{inputenc}
\usepackage[russian]{babel}

% page margin
\usepackage[top=2cm, bottom=2cm, left=0.5cm, right=0.5cm]{geometry}

% AMS packages
\usepackage{amsmath, array}
\usepackage{amssymb}
\usepackage{amsfonts}
\usepackage{amsthm}

\usepackage{graphicx}
\usepackage{rotating}

\usepackage{fancyhdr}
\pagestyle{fancy}
% modifying page layout using fancyhdr
\fancyhf{}
\renewcommand{\sectionmark}[1]{\markright{\thesection\ #1}}
\renewcommand{\subsectionmark}[1]{\markright{\thesubsection\ #1}}

\rhead{\fancyplain{}{\rightmark }}
\cfoot{\fancyplain{}{\thepage }}

\usepackage{titlesec}
\titleformat{\section}{\bfseries}{\thesection.}{1em}{}
\titleformat{\subsection}{\normalfont\itshape\bfseries}{\thesubsection.}{0.5em}{}

\begin{document}

\title{Расчет классических статсумм истинных димеров слабосвязанных пар с использованием многомерной поверхности потенциальной энергии}
\date{}
\maketitle

В нашей работе мы представляем формулы расчета точных классических статсумм истинных димеров слабосвязанных пар в парном приближении. В работе [1] представлен систематический подход к рассмотрению равновесию мономер-димер в рамках классической механики. Однако в в этой работе были использованы некоторые допущения, которые парадоксальным образом привели к абсолютно точным выражениям, что мы и демонстрируем в этой работе.   

Квантово-механический подход к расчету статсумм требует знания колебательно-вращательной структуры уровней в парном потенциале, сложность и точность определения которой растет с увеличением количества степеней свободы. 

\end{document}
