%we don't want ae.sty
\expandafter\def\csname ver@ae.sty\endcsname{}

\documentclass[
  14pt,
  a1paper,
  portrait, 
  margin=0mm,
  innermargin=15mm,
  blockverticalspace=0mm,
  colspace=0mm,
  subcolspace=0mm
]{tikzposter}

\usepackage[T1]{fontenc}
\usepackage[utf8]{inputenc}
\usepackage[russian]{babel}

% page margin
\usepackage[top=2cm, bottom=2cm, left=2cm, right=2cm]{geometry}

% AMS packages
\usepackage{amsmath}
\usepackage{amssymb}
\usepackage{amsfonts}
\usepackage{amsthm}

% blackboard lettering
\usepackage{dsfont}
\usepackage{bbm}

\usepackage{fancyhdr}
\pagestyle{fancy}
% modifying page layout using fancyhdr
\fancyhf{}
\renewcommand{\sectionmark}[1]{\markright{\thesection\ #1}} % adding number to section name
\renewcommand{\subsectionmark}[1]{\markright{\thesubsection\ #1}} % adding number to subsection name

\rhead{\fancyplain{}{\rightmark }} % placing the section/subsection name in the right corner of the header
\cfoot{\fancyplain{}{\thepage}}  % placing a page number in the center of the footer   

% table packages
\usepackage{tabularx, ragged2e, booktabs, caption}
% includegraphics package
\usepackage{graphicx}

\usepackage{subfigure}

% commands
\newcommand{\lb}{\left(}
\newcommand{\rb}{\right)}
\newcommand{\mH}{\mathcal{H}}
\newcommand{\mL}{\mathcal{L}}
\newcommand{\mf}{\mathbf}

\newcommand{\bbB}{\mathbb{B}}
\newcommand{\bbG}{\mathbb{G}}
\newcommand{\bbS}{\mathbb{S}}
\newcommand{\bbI}{\mathbb{I}}
\newcommand{\bbA}{\mathbb{A}}
\newcommand{\bba}{\mathbbm{a}}

% positioning commands
\newcommand{\vpravo}{\hspace{0.63cm}}
\newcommand{\vverh}{\vspace*{-0.1cm}}

% table cell centering
\newcolumntype{C}[1]{>{\Centering}m{#1}}
\renewcommand\tabularxcolumn[1]{C{#1}}

\makeatletter
\providecommand\barcirc{\mathpalette\@barred\circ}%
\def\@barred#1#2{\ooalign{\hfil$#1-$\hfil\cr\hfil$#1#2$\hfil\cr}}%
\newcommand\stst{^{\protect\barcirc}}%
\makeatother

% tikz packages
\usepackage{tikz}
\usetikzlibrary{shapes.geometric, arrows, positioning, decorations.markings}
\usetikzlibrary{fit}
\usepackage{microtype}
\usepackage{framed}
\usetikzlibrary{decorations.pathmorphing,calc,backgrounds}

\usepackage[nottoc]{tocbibind}



\makeatletter
\def\TP@titlegraphictotitledistance{-9cm}
\settitle{ \centering \vbox{
    \@titlegraphic \\ [\TP@titlegraphictotitledistance] 
    \centering
    \color{titlefgcolor} {\bfseries \fontsize{1.5cm}{1cm}\selectfont \@title \par}
    \vspace*{1em}
    {\bfseries \fontsize{1.2cm}{1cm}\selectfont \@author \par} 
    \vspace*{1em} 
    {\bfseries \fontsize{1.2cm}{1cm}\selectfont \@institute}
}}
\makeatother

\newcommand{\lb}{\left(}
\newcommand{\rb}{\right)}
\newcommand{\lsq}{\left[}
\newcommand{\rsq}{\right]}

\newcommand{\mL}{\mathcal{L}}
\newcommand{\mH}{\mathcal{H}}

\newcommand{\bbI}{\mathbb{I}}
\newcommand{\bOmega}{\boldsymbol{\Omega}}

\title{\parbox{0.75\linewidth}{\centering Использование методов статистической механики для расчета термодинамических и спектральных характеристик слабосвязанных молекулярных пар}}

\author{Чистиков Д., Финенко А.}
\institute{МГУ имени М.В. Ломоносова, Химический факультет}

\titlegraphic{
    \includegraphics[width=6.5cm,height=5cm]{pictures/logo.jpg}
    \hfill 
    \includegraphics[width=6.5cm,height=5cm]{pictures/logo.jpg}
    \vspace{5cm}
}

\usetheme{Simple}

\usepackage{amsmath,amssymb}

\newcommand{\vverh}{\vspace*{-0.1cm}}

\begin{document}
\maketitle[width=1.0\linewidth, titletotopverticalspace=0.1cm, titletoblockverticalspace=0.3cm]

\vspace*{-5cm}

\begin{columns}

\column{0.5}
\block[titleoffsety=1cm,bodyoffsety=2.5cm]{Введение}
{
	Интерес к теоретическому изучению слабосвязанных молекулярных пар в газах существует уже с серидины прошлого века, однако подобные исследования привлекают все больше внимания в связи с проблемами климатического моделирования в планетных атмосферах [1]. В последние годы было надежно установлено, что для построения достоверных климатических моделей планетных атмосфер необходим детальный учет эффектов слабого индуцированного поглощения в тепловой области спектра [1, 3]. Теория столкновительно-индуцированных спектров поглощения в газах, состоящих из молекул, не имеющих постоянного дипольного момента, на протяжении многих лет ограничивалась использованием сферически-симметричных потенциалов взаимодействия и модельных функций дипольного момента [2]. Расчет спектрального профиля является трудоемкой задачей, однако обратившись к методам статистической механики можно вычислить некоторые его характеристики, такие как спектральные моменты [2]. С использованием аналогичных методов могут быть получены другие характеристики [4], примерами которых служат второй вириальный коэффициент и константа равновесия образования ван-дер ваальсова комплекса. Все эти свойства оказываются полезными при моделировании спектров. Знание первых спектральных моментов позволяет сделать оценку спектрального профиля бинарного поглощения (см., например, [3]), а константа равновесия образования димеров позволяет оценить вклад димеров в интегральную интенсивность полос индуцированного поглощения (см., например, [5]). 
}

\block[titleoffsety=1.5cm,bodyoffsety=3cm]{Общее рассмотрение задачи}
{
	Классификация связанных, метастабильных и свободных состояний? 
	Классическая сумма по состояниям связанного димера представляет собой следующий фазовый интеграл
	\begin{gather}
			Q_{\textup{bound}}(T) = \frac{1}{s_b h^{(l + 6)}} \int\limits_{H - E_{c.m.} < 0} \exp \left( -\frac{H}{kT} \right) dx^{\textup{c.m.}} \, dy^\textup{c.m.} \, dz^\textup{c.m.} \, dp_x^\textup{c.m.} \, dp_y^\textup{c.m.} \, dp_z^\textup{c.m.} \, dq_i \, dp_i, \label{1.1}
	\end{gather}
	где $h$, $k$ -- постоянные Планка и Больцмана, $s_b$ -- число симметрии молекулярной пары, $l$ -- количество независимых координат $q_i$ ($1 \leqslant i \leqslant l$); компоненты векторов координат и импульсов центра масс молекулярной пары обозначены $\mathbf{R} = \lb x^\textup{c.m.}, y^\textup{c.m.}, z^\textup{c.m.} \rb$, $\mathbf{P} = \lb p_x^\textup{c.m.}, p_y^\textup{c.m.}, p_z^\textup{c.m.} \rb$. Обозначим через $\mH$ гамильтониан, получающийся при отделении энергии центра масс $\mH = H - \mathbf{P}^2/2M$, где $M$ -- общая масса пары. Интегрирование по переменным центра масс дает трансляционную сумму по состояниям
	\begin{gather}
			Q_\textup{bound} = \frac{1}{s_b h^l} \lb \frac{2 \pi M k T}{h^2} \rb^{3/2} V \int\limits_{\mH < 0} \exp \lb - \frac{\mH}{kT} \rb dq_i \, dp_i, \label{1.2}
	\end{gather}
	где через $V$ обозначен объем, в котором заключена рассматриваемая молекулярная пара. \par
	Дальнейшее рассмотрение статистической суммы $\eqref{1.2}$ будем вести в специальной системе координат, которую будем называть ''трансляционной'', а угловые координаты и сопряженные к ним импульсы будем отмечать верхним индексом ''t''. Будем рассматривать общий случай взаимодействия двух жестких мономеров, выпишем соответствующий этому случаю гамильтониан и проследим за выражениями для связанной статистической суммы. 
	
	\vspace*{-0.5cm}
	\subsection*{Транслированная система}
	Поместим начало системы отсчета в центр масс системы как целого; параметризуем ось, соединяющую центры масс мономеров, сферическими углами $\Theta$, $\Phi$; обозначим вектор, направленный из центра масс первого мономера в центр масс второго, через $\mathbf{R}$. Для описания ориентации каждого из мономеров транслируем систему отсчета в их центры масс, введем тройки углов Эйлера $\lb \varphi_1^t, \theta_1^t, \psi_1^t \rb$, $\lb \varphi_2^t, \theta_2^t, \psi_2^t \rb$. Кинетическая энергия в форме Лагранжа в определенной системе имеет следующий вид
	\begin{gather}
		T_\mL = \frac{\mu}{2} \lsq \dot{R}^2 + R^2 \dot{\Theta}^2 + R^2 \dot{\Phi}^2 \sin^2 \Theta \rsq + \frac{1}{2} \bOmega_1^\top \bbI_1 \bOmega_1 + \frac{1}{2} \bOmega_2^\top \bbI_2 \bOmega_2, \qquad \mu = \frac{m_{\textup{mon}_1} m_{\textup{mon}_2}}{m_{\textup{mon}_1} + m_{\textup{mon}_2}}, \label{1.3} 
	\end{gather}
	где через $m_{\textup{mon}_1}$ и $m_{\textup{mon}_2}$ обозначены массы мономеров, через $\mu$ -- приведенная масса молекулярной пары, через $\bbI_1 = \textup{diag} \lb I_\alpha^1, I_\beta^1, I_\gamma^1 \rb $, $\bbI_2 = \textup{diag} \lb I_\alpha^2, I_\beta^2, I_\gamma^2 \rb $ -- матрицы тензоров инерции мономеров в главных осях. Выражаем вектора угловой скорости $\bOmega_1, \bOmega_2$ через соответствующие вектора эйлеровых скоростей $\dot{\mathbf{e}}_1^t$, $\dot{\mathbf{e}}_2^t$; применяем стандартную процедуру перехода к гамильтоновой форме кинетической энергии, в ходе которой переходим от эйлеровых скоростей к эйлеровым импульсам $\mathbf{p}_1^t = \lb p_1^\varphi, p_1^\theta, p_1^\psi \rb$ и $\mathbf{p}_2^t = \lb p_2^\varphi, p_2^\theta, p_2^\psi \rb$.   
	\begin{gather}
		T_\mH = \frac{p_R^2}{2\mu} + \frac{p_\Theta^2}{2 \mu R^2} + \frac{p_\Phi^2}{2 \mu R^2 \sin^2 \Theta} + \frac{1}{2 I_\alpha^2 \sin^2 \theta_2^t} \lsq \lb p_2^\varphi - p_2^\psi \cos \theta_2^t \rb \sin \psi_2^t + p_2^\theta \sin \theta_2^t \cos \psi_2^t \rsq^2 + \frac{1}{2 I_\beta^2 \sin^2 \theta_2^t} \times \notag \\
		\times \lsq \lb p_2^\varphi - p_2^\psi \cos \theta_2^t \rb \cos \psi_2^t  - p_2^\theta \sin \theta_2^t \sin \psi_2^t \rsq^2 + \frac{1}{2 I_\gamma^2} \lb p_2^\psi \rb^2 + \frac{1}{2 I_\alpha^1 \sin^2 \theta_1^t} \lsq \lb p_1^\varphi - p_1^\psi \cos \theta_1^t \rb \sin \psi_1^t + \right. \notag \\ 
		+ \left. p_1^\theta \sin \theta_1^t \cos \psi_1^t \rsq^2 + \frac{1}{2 I_\beta^1 \sin^2 \theta_1^t} \lsq \lb p_1^\varphi - p_1^\psi \cos \theta_1^t \rb \cos \psi_1^t - p_1^\theta \sin \theta_1^t \sin \psi_1^t \rsq^2 + \frac{1}{2 I_\gamma^1} \lb p_1^\psi \rb^2 \label{1.4}   
	\end{gather}

	Заметим, что потенциальная энергия в ''транслированной'' координатной системе будет зависеть от всех координат $U = U \lb R, \Theta, \Phi, \varphi_1^t, \theta_1^t, \psi_1^t, \theta_2^t, \theta_2^t, \psi_2^t \rb$. Дальнейшие выкладки покажут, как можно заменить переменные внутри интегрального выражения, чтобы потенциальная энергия зависела от пяти внутренних координат $R, \varphi_1, \varphi_2, \theta_1, \theta_2, \psi_1$ (см. рис. 1).

	\vspace*{-0.5cm}
	\subsection*{Преобразование угловых переменных в интегральном выражении}
	Интегрирование в выражении $\eqref{1.2}$ ведется по области $\mH = T_\mH + U < 0$, где $T_\mH$ представлен формулой $\eqref{1.4}$. Осуществим тривиальную замену переменных, сводяющую кинетическую часть $\mH / kT$ к сумме квадратов
	\begin{gather}
		\frac{\mH}{kT} = x_1^2 + \dots + x_9^2 + \frac{U}{kT}, \quad \lsq \, \textup{Jac} \rsq = 2^{9 / 2} \mu^{3/2} R^2 \sqrt{I_\alpha^1 I_\beta^1 I_\gamma^1 I_\alpha^2 I_\beta^2 I_\gamma^2} \sin \Theta \sin \theta_1 \sin \theta_2. \label{1.5} 
	\end{gather}
	Представляя интеграл $\eqref{1.2}$ в виде повторного интеграла, в котором интегрирование во внешнем интеграле ведется по координатам, а интегрирование во внутренним интеграле -- по сопряженным им импульсам. Сделанная замена позволяет аналитически произвести интегрирование по импульсной части 
	\begin{gather}
			Q_\textup{bound} = \frac{1}{s_b h^9} \lb \frac{2 \pi M k T}{h^2} \rb^{3/2} \int\limits_{U < 0} \frac{\displaystyle \gamma \lb \frac{9}{2}, - \frac{U}{k T} \rb}{\displaystyle \Gamma \lb \frac{9}{2} \rb} \lsq Jac \rsq dR \, d\Theta \, d\Phi \, d\varphi_1^t \, d\theta_1^t \, d\psi_1^t \, d\varphi_2^t \, d\theta_2^t \, d\psi_2^t, \label{1.5}
	\end{gather}
	где через $\gamma( \cdot, \cdot)$ и $\Gamma(\cdot)$ обозначены неполная и полная гамма-функции, соответственно. Используя формализм теории представлений группы вращений $SO(3)$ было показано, что внутри интеграла \eqref{1.5} можно перейти к переменным $\lb \varphi_1, \theta_1, \psi_1 \rb$, $\lb \varphi_2, \theta_2, \psi_2 \rb$, определяющим ориентацию мономеров относительно подвижной системы отсчета (см. рис. 1), причем само интегральное выражение остается неизменным (единичный якобиан преобразования).  
}
\column{0.5}
\block[titleoffsety=1cm,bodyoffsety=1.5cm]{}
{
	Проводя алгебраические преобразования в формуле \eqref{1.5} приходим к следующему интегральному выражению для константы равновесия во внутренних переменных:
	\begin{gather}
			K_\textup{P}^\textup{bound} = \frac{1}{P} \frac{Q_\textup{bound}}{Q_{\textup{mon}_1} Q_{\textup{mon}_2}} = \notag \\ 
			= \lb R T \rb^{-1} \frac{N_0}{8 \pi^2 s_b} \idotsint\limits_{U(R, \varphi_1, \theta_1, \psi_1, \varphi_2, \theta_2) < 0} \frac{\gamma \lb 9/2, - U / k T \rb}{\Gamma \lb 9 / 2 \rb} \exp \lb - \frac{U}{kT} \rb R^2 dR \, \sin \theta_1 d \theta_1 \sin \theta_2 d\theta_2 d \varphi_1 d \varphi_2 d \psi_1, \label{1.6}
	\end{gather}
	где через $Q_{\textup{mon}_1}$, $Q_{\textup{mon}_2}$ были обозначены статистические суммы мономеров, через $N_0$, $R$ -- число Авогадро и универсальная газовая постоянная, соответственно. Полученное выражение может быть легко упрощено для частных случаев взаимодействия мономеров с меньшим числом вращательных степеней свободы. 
}
\block[titleoffsety=1cm, bodyoffsety=1.5cm]{Температурные зависимости связанных статистических сумм и вклада связанных димеров в нулевой спектральный момент}
{
	Полученные точные классические выражения могут быть использованы для усреднения величин, зависящих от внутренних координат, по фазовому пространству молекулярной пары. Использованный метод ползволяет выписать для таких средних значений интегральное выражение по области в конфигурационном пространстве, значительно упрощая вычислительную процедуру, необходимую для такого расчета. В качестве примера такой величины рассмотрим квадрат индуцированного дипольного момента, среднее значение которого по фазовому пространству равен нулевому моменту рототрансляцинной полосы (?).
}

\end{columns}

\end{document}
